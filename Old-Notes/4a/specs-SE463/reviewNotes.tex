% @Author: Lara Janecka
% @Date:   2016-08-03 11:16:03
% @Last Modified by:   Lara Janecka
% @Last Modified time: 2016-08-03 12:51:11

\documentclass[12pt]{article}
\usepackage{parskip}
\usepackage{amsmath}
\usepackage{pdfpages}
\usepackage[margin=.6in]{geometry}

\begin{document}
\section*{Scoping}
\label{sec:scoping}
\paragraph{Project Purpose} 
\label{par:project_purpose}
Why the project is wanted

\paragraph{Constraints} 
\label{par:constraints}	
Restrictions on the scope or style of the system

\paragraph{Requirements} 
\label{par:requriements}
condition or capability that must be achieved in terms of environment
	
\paragraph{Work vs System} 
\label{par:work_vs_system}
Work is the area under study and the system is what you built to solve the problem

\paragraph{Stakeholders} 
\label{par:stakeholders}
\begin{itemize}
	\item owner, ultimate steakholder
	\item customer, person who buys it
	\item users
	\item domain experts
	\item technology experts
	\item inspectors
	\item market researchers
	\item laywers
	\item industry standards
	\item experts on adjacent systems
	\item negative steakholder
\end{itemize}


\paragraph{Context Diagram} 
\label{par:context_diagram}
A graphical model of the context(environment) in the work exists, phenomena are relevant descriptions of the requirements. Basically a circle containing the work in the middle and connect external effects to it. 

\paragraph{Use Case Diagram} 
\label{par:use_case_diagram}
Show some end to end functionality, captures a complete response to a triggering event. 


\section*{Elicitation}
\label{sec:elicitation}
\paragraph{Challenges of Requirements Elicitation} 
\label{par:challenges_of_requirements_elicitation}

\paragraph{Techniques} 
\label{par:techniques}
\begin{itemize}
	\item Artifact based (learn what we can before asking anyone): document studies, similar companies, norms, domain analysis, requirements taxonomies
	\item Model based (reexpress requirements): modelling, analysis patterns, mockups/prototyping, pilots
	\item Stakeholder based (ask stakeholder): stakeholder analysis, questionnaires, observation, interviews, task demo, ask suppliers, domain workshop, personas
	\item Creativity based: systematic thinking, brainsorm, workshops
\end{itemize}

\paragraph{Lightweight Models} 
\label{par:lightweight_models}
Mind maps, Scenarios (a full execution path through a use case), Acitivity Diagrams

\paragraph{Quality Requirements} 
\label{par:quality_requirements}
desired characteristics of the software

\paragraph{Fit Criteria} 
\label{par:fit_criteria}
The extent to which a quality requirement must be met

\paragraph{Rational Behind User Stories} 
\label{par:rational_behind_user_stories}
should fit on an index card, have lots of conversation, get confirmation
\begin{itemize}
	\item easy for stakeholders
	\item shift focus to written documentation
	\item encourages iterative development
	\item delay the elicitation of requriements until just before development
	\item support participatory elicitation
\end{itemize}

\paragraph{Activity Diagram} 
\label{par:activity_diagram}
Basicallly a flow chart of interactions between actors
\begin{itemize}
	\item dots = entrance, exit (if circle around it)
	\item diamond = decision, label with conditions
	\item bar = synchronization/fork
\end{itemize}

\paragraph{Process Model} 
\label{par:process_model}
a functional decompositions of the work
\begin{itemize}
	\item squares = actors
	\item circle = process (actor on either side)
	\item arrows = flow of data (label tells what data)
	\item bars = data store
\end{itemize}

\paragraph{User Stores} 
\label{par:user_stores}
light-weight approach to managing requirements

\paragraph{Devise Requirements for System} 
\label{par:devise_requirements_for_system}

\section*{Requirements Analysis}
\label{sec:requirements_analysis}
\paragraph{Benefits/Challenges in prioritizing} 
\label{par:genefites_challenges_in_prioritizing}
Challenges
\begin{itemize}
	\item all requirements deemed to be essential
	\item large number of requirements to prioritize
	\item conflicting priorities
	\item changing priorities
	\item stakeholder and developer collaboration
	\item subjective prioritization
\end{itemize}
Benefits
\begin{itemize}
	\item improves customer satisfaction
	\item helps to determine how to prioritize allocation of resources
	\item encourages stakeholders to consider all requirements0
\end{itemize}


\paragraph{Prioritization Techniques} 
\label{par:prioritization_techniques}
\begin{itemize}
	\item Kano Model: basic, performance, excitement, indifferent, reverse
	\item 100 Dollar: give users 100 points to assign
	\item Analytic Hierarchy Process: pariwise comparison
\end{itemize}


\paragraph{Estimation vs Target vs Commitments} 
\label{par:estimation_vs_target_vs_commitments}
An estimate is a prediction, a target is a statement of the desirable value, a commitment is a promise to deliver


\paragraph{Software Estimation Techniques} 
\label{par:software_estimation_techniques}
\begin{itemize}
	\item Estimation by analogy: use previous values to figure out a ratio (between use case, interfaces, graphs, or reports, etc) and extrapolate project size
	\item funciton point analysis: estimate number of function points, estimate code size, estimate resources required, can multiply by some value based on type
	\begin{itemize}
		\item external inputs
		\item external outputs
		\item internal logical files
		\item external interfaces
	\end{itemize}
	\item expert judgment
	\item estimation tools
\end{itemize}


\paragraph{Influences on Software Size and Cost} 
\label{par:influences_on_software_size_and_cost}


\paragraph{Risk Analysis} 
\label{par:risk_analysis}


\section*{Requirements and Specifications}
\label{sec:requirements_and_specifications}
Requirements vs. specification
The environment of a system vs. interface phenomena
Atomic requirements, conditions of satisfaction
The vocabulary used to express requirements vs. the vocabulary
used to express specifications
Devising atomic requirements or conditions of satisfaction
Deriving specifications from requirements
Identifying assumptions that need to hold in order for a system that meets the specification will also meet the requirements











\end{document}