\documentclass[12pt]{article}
\usepackage{parskip}
\usepackage{amsmath}
\usepackage{pdfpages}
\usepackage[margin=.6in]{geometry}


\begin{document}
When we do searches we want to allow some threashhold so that we don't search exhaustively. This may result in suboptimal solutions but that is ok because it greatly decreases search time. 

\section{Genetic Algorithms} % (fold)
  \label{sec:genetic_algorithms}
  Basically we mutate our children to see if it improves. If it does we keep it else we dump it. Works just like evolution, hence the name.


  Mutation is to explore in the vicinity and cross over is to combine two things.



GO BACK AND TAKE BETTER NOTES GODDAMMIT


  % section genetic_algorithms (end)  




\section{Parallel GAS} % (fold)
\label{sec:parallel_gas}
\textbf{Master Slave} - A master does all the selection and mating an the slaves do all the fitness calculations

\textbf{Fine Grained} - selection and mating is limited to local neighborhoods even though it may overlap

\textbf{Coarse Grained} - let populations evolve in parallel where mating and selection are done internally, every once in a while they exchange data

There are tons of different factors to consider when we get multiple populations going. Topology is a huge factor. In dense topology a good solution will spread very quickly, but in a sparse topology there will be more diversity. We need to balance these two.

Migration policies:
\begin{itemize}
  \item best worst - give me your best and I will swap it for my worst
  \item best random - give me your best and I will swap it for some random
  \item random random - give me some randome and I will swap it for my random
  \item radom worst - give me some random and I will swap it for my worst
\end{itemize}

When do we migrate:
\begin{itemize}
  \item synchronized - every n generation
  \item asynchronous - after some event has occurred
\end{itemize}

If we are dong synchronous migration we need to fine tune how many generations to migrate after. Doing it too early or often will result in suboptimal solutions and high communication costs. But doing it too little will result in low cooperation.

Now how do we determine how much to migrate at a time? This is called the \textbf{migration rate}. If you migrate too few the demes remain fairly isolated and the migrated genes will not have much effect, but if we migrate too many causes fast convergence onto a suboptimal solution.

Need to figure out what \textbf{trap functions} are. They are somehow disceptive but no clue what they do.


% section parallel_gas (end)

\section{Cooperative GAS} % (fold)
\label{sec:cooperative_gas}
Basically the fitness of an individual is determined by the populations around it. With this we can have different populations optimizing different elements of the problem so that the optimal solution is found through the combination of multiple solutions.



















% section cooperative_gas (end)


























\end{document}