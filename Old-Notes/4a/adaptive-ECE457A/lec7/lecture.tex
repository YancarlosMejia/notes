\documentclass[12pt]{article}
\usepackage{parskip}
\usepackage{amsmath}
\usepackage{pdfpages}
\usepackage[margin=.6in]{geometry}

\begin{document}
\section{ACO} % (fold)
\label{sec:aco}
As ants walk they leave behind a trail of pheremones. This can be used by other ants to figure out which trail is the most direct (the pheremones fade over time). We can think of this as collective memory.

To simulate this we assign pheromone counters to paths that are intially equal. Run m ants through it and let them apply a probability to when they chose a path. The probably is the current pheromone level of the path divided by the total pheromone level. We then apply evaporation over time as determined by some ratio. As the ant goes over the path it leaves some more pheromone equal to one over the length of the edge. 

We can add the ant denisty to the model by including a constant Q as the amount the pheromone is increased so that it is proportional to the number of ants crossing it and not its length. You can also consider the ant quantity by adding Q divided by the length of the edge when increasing pheremone.

Another approach is the track the ant back home from the solution, updating the pheromone as it goes called the \textbf{online delayed} or \textbf{ant cycle}.  With this, after evaporation, we add Q divided by the length of the path found by the ant. 

The \textbf{Ant System} algorithm is an implementaion of this using online delayed. 

The \textbf{Min Max Ant System} keeps track of the best solution in this iteration to be added during evaporation. This puts bounds around the evaporation rate which can allow high exporation in the beginning and more intensification later.

The \textbf{Ant Colony System Algorithm} we use AS but have a different transition rule based on elitist strategy. We introduce a random number and if it is higher than some threshold we go down that path, else we do some crazy formula to update the value (see slides, its stupidly complicated). This system also updates how evaporation is done, see slides for equation.

We can use ACS to solve the traveling salesman problem. Run a bunch of ants through these steps
\begin{enumerate}
	\item assign random pheromone values to edges
	\item create a bunch of solutions (this keeps a tabu list for each ant as it goes and choses based on probability)
	\item apply the local search algorithm (these actions are actually optional, )
	\item update pheromones
\end{enumerate}

Another application of ACS is \textbf{assembly line balancing}. This is the problem of having a set of tasks that need to be done that have a time taken to finish, a space requirement, and a precedence relationship. You need to schedule sets of tasks to be done in a cycle for a certain amount of time within the space we have. Within this problem there could be very different goals you are trying to accomplish.



% section aco (end)    

\section{Cooperative} % (fold)
\label{sec:cooperative}

% section cooperative (end)































\end{document}