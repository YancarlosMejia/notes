\documentclass[12pt]{article}
\usepackage{parskip}
\usepackage{amsmath}
\usepackage{pdfpages}
\usepackage{listings}
\usepackage{color}
\usepackage[margin=.6in]{geometry}

\definecolor{dkgreen}{rgb}{0,0.6,0}
\definecolor{gray}{rgb}{0.5,0.5,0.5}
\definecolor{mauve}{rgb}{0.58,0,0.82}

\lstset{frame=tb,
  language=C++,
  aboveskip=3mm,
  belowskip=3mm,
  showstringspaces=false,
  columns=flexible,
  basicstyle={\small\ttfamily},
  numbers=none,
  numberstyle=\tiny\color{gray},
  keywordstyle=\color{blue},
  commentstyle=\color{dkgreen},
  stringstyle=\color{mauve},
  breaklines=true,
  breakatwhitespace=true,
  tabsize=3
}

\begin{document}
This formula is very important:
\begin{align}
  P = e^{-\frac{\delta c}{t}}
\end{align}

$\deta c$ is the cost increase of this solution from the current solution and $t$ is some control variable called the temperature

We use this formula if we hit a solution that is not better (or we get stuck) to jump randomly. Basically we keep lowering the value of t to decrease the probability that we will accept a solution so that we narrow down on a good solution in a more random sense than just exploring a good neighborhood. We are more likely to find a good solution this way.

Similar to search, annealing has two modes \textbf{explore} and \textbf{exploit}. We head the material until it reaches an annealing temperature. We model the acceptance probability with some function. If the acceptance criteria is not fulfilled we play the lottery and try to go to a random place, else we try to leverage being in a good neighborhood. 

When we get to a good neighborhood we keep increasing our threashold so that there is a lower probability of selecting a bad solution. As we bring the temperature down the probability of accepting is lower. 

Over iterations we changed the temperature using some schedule that we decide. It can be linear or geometric. We want to iterate a few time at each temperature to give the system time to stabilize. The number of iteration at a temperature can be constant or vary depending on what you want.


\paragraph{Cooperative SA} % (fold)
\label{par:cooperative_sa}
Basically you create a set of solutions concurrently. To make a new solution you take what your solution currently is and grab a random other solution from the set of current solution. Between these two solution you find which of your neighbors is closer to the random solution. From this set of closer solution you select randomly, if the set is empty you chose one of your neighbors randomly.The new temperature the then calculated based on of difference between the mean fitness of the new and old sets of solutions.
% paragraph cooperative_sa (end)






  
\end{document}