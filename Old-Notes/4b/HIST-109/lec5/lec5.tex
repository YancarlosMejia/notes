\documentclass{article}
\usepackage{parskip}
\usepackage{pdfpages}
\usepackage[margin=.6in]{geometry}
\begin{document}
\title{French Revolution}
\section*{Spelling}
\label{sec:spelling}
\begin{itemize}
	\item Arsenal
	\item de Launay
	\item Hotel de Ville
	\item cockade
	\item porlement
	\item remonstrance
	\item cahiers de doleances
	\item Jacques Necker
\end{itemize} 

\maketitle
People were revolting all over france. They tried to take military complexes to gain armaments.

The bastille was a huge armory in france. The city it was in was walled. The bastille had been a very important military fortification for many different wars. After the main era of combat it had become a prison. Paris had grown to encompass the bastille. It was actualy a rather cushy prison. By the time it was stormed it was rather old and run down. To maintain it was very expensive so the number of prisoners had declined. When it was stormed it had only 7 prisoners.

It was a prison, but more than that it was a symbol of royal military and authority. It had only 32 swiss guard and 82 pensioned troupes.

When the rioting in the streets got bad they locked it up and raised the draw bridge. Thousands had surrounded the bastille as the protestors demanded that they open it and give it to the citizens militia. This is not just a crazy mob, it had the city government negotiating for it. Basically it was the city of paris fighting the french government for the gunpowder there to prevent it from being used on the crowd.

Two soldiers showed up from the royal military brought a cannon to help brake down the gates. Important to note, that these are royal troups and royal guns joining the people in the streets. The commander of the bastille eventually capitulates with honor. He expects to be allowed to leave the bastille with honor. He doesnt get a response but he assumes that his offer has been accepted so he lowers the door. This was not true, the mob arrests everyone there and kills many in the march to the jail. The leader was beheaded.

Rumor has it that the king was planning on sending troups to subdue france. The king goes himself to the city and is welcomed by the new governor. They invite the king to wear the revolutionary cockade. This was the symbol of the revolution so asking him to wear it was a major slap in the face. If he had refuse the crowd would have torn him apart.

\section*{Causes}
\label{sec:causes}
By the 1780's the crown was broke due to a gradual accumulation of debt. The myth was the cost of maintaining the palace of versaille and all the people that lived there. The biggest debts were the wars faught of the last few decades including the 7 years war and the american revolution. The 7 years war involved most of europe. France lost her colonies in north america to britain. It was very expensive war that they lost. Later they wanted to help the american colonies to stick it to britain. They sent a ton of money there.

By this point they were broke and they couldn't really borrow any. Usually they go to other governments or banks or even corporate bodies (like the church). All these sources basically were dry. They were reluctant to lend because france was in such a bad state, or they raised their interest rates super high so the crown didnt go with it. As an alternative to this, they wanted to tax the nobility. The other classes were paying all they could, but the nobility were not paying their full share.

The problem with this is that the nobility had special exemptions on certain taxes which they had gotten in exchange for their loyalty to the crown. When the crown was comming up they made these deals for peace. The nobility viewed these as rights, they were irrevocable. They were not at all happy with this plan. The finance ministers wanted a tax based on how much assets a person had. The nobility refused to pay these taxes.

The king needed the parliament to agree to his decree make it law. The parliament is populated with nobility so naturally this didnt go well. Taxes were considered laws so it was nearly impossible. The nobility in parliament basically started hassling the king. They would frequently take their time and change his decrees. They also would publish remonstrances, their responses to the king explaining their reasoning.

The remostrances used the language of the enlightenment. This was an intellectual movement that encouraged a new way of looking at society and government. Before the englightenment you were born into a class and you stayed there until you died. You didnt question your place or the hierarchy and instead focused on the after life. The king was gods representative on earth accountable to only god, you could not question him as that was heracy. The enlightenment questioned all of this and blaimed the institutions that put into place the class system. This showed that progress was possible, the government was flawed, and gave everyone was language to argue with.

Important to note, the nobles are using the enlightenment to justify their objecgtions, they just dont want to pay taxes. They argued that the taxes were heavy and arbitrary. They argued that it was exhausting on the people. The taxes persecuted the nobility and removed from them their wealth as they are due and removed from the poor the substance they needed to survive. The citizens were anyone who paid taxes, and they do so voluntarily. They volunteered to do so because they are given a voice on how that money is spent. The people were represented by the estates general.

The orders of estate:
\begin{itemize}
	\item clergy - ensured the safety of people's souls
	\item nobility - protect the country from threats outside and in (military and police)
	\item rest of the population - supported the other estates
\end{itemize}

This system was supposed to be cooperative. The nobles argued that that had broken down. They argued that the king was not giving them enough power in government.

The king of the time Lois XVI called the assembly of the notables. These were people chosen by the crown. He hoped to have friendly ears, but the assembly was made up of nobles. They told him to call the estates general and give up on the tax reform. This is a crazy suggestion because the estates general had not been called for a hundred years. He does call it.

Record keeping at the time was not the best. The structure however is well known. There were three chambers, one for each estate. Each chamber had 300 deputies and these would together make one vote. The king gave the third estate 600 chairs. Everyone over the age of 16 got a vote. The king gave them the ability to make a list of grievances from each estate (called cahiers de doleances).

They mainly wanted electoral reform. They came up with two solutions vote by order (one per estate) or vote by head. The second one would put tons of power into the third estate since it had 600 deputies. The king decided not to decide which way to vote and wanted to let them decide. This caused the war to be between the estates instead of a war between the king and the nation.

The third estate now started to assert itself. Most of its deputies were lawyers that were well versed in the languages of the enlightenment.

If the citizens are someone who pays taxes and this is what gives them a right to make decisions, then the third estate (who pays the most taxes) should get at least half the power. This meant that they should not only get 600 heads, but a vote by head.

When they convene they reach a deadlock. The third estate refuses to swear in their deputies because they want them sworn in as a block which would imply that they all only get one vote. This means that the meeting cannot continue. This deadlock holds for a month. The third estate just decides to start swearing in people separately. They eventually call themselves the national assembly and claim that they are the only ones that represent france. The king does basically nothing. The king calls a royal meeting, but they exclude the third estate. In preparation for the kings visit they close the hall and don't tell the deputies. When the deputies arrived the next day they found it closed and surrounded with guards.

They assumed that the king was planning to dessolve the estates assembly. They moved to a nearby tennis court where they took an oath to never stop until a new consitution is written (called the tennis court oath). Here they say that they are france, not the king or nobility. This leads them to conclude that they are the sovereign power. The king doesn't acknowledge this and goes on with his assembly. There he says that if the third assembly doesnt cooperate he will dissolve the assembly.

The third estate just ignores him and continues to meet. The king had to handle it with force. To do so he declares bankrupcy and dissolves the estates general. In response he starts to gather his troups which has the  national assembly worried because they have no military power.

The people of france had a lot of complaints. Most were living in poverty and starving. There had been many bad harvests. In normal years the average person spent 45-60\% on bread. Now it was 90\%. This is no good. No one can afford anything else so the manufacturers are all out of work which results in the deaths of their families. People never blaimed the weather, they blaimed hoarders that were storing bread to drive up the price of bread to make more money. The people wanted the king and the deputies to do something about this.

The peasents had suffered from the abuse of the aristocracy for decades so they expected them to fuck this up. As things got held up and troups gathered rumors spread that the deputies were hostages and the king was going to ransack the cities. Then the king canned his newest finance minister Jacques Necker who was rather beloved of the people.  This was seen as aristocratic plot to dismiss the third estate and drive up the price of bread. Lots of people went to pallet variale (spelling) to stand on tables and gain news about what was going on. Shit started being thrown and eventually violence broke out.

















\end{document}
