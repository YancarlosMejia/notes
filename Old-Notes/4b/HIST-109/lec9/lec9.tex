\documentclass{article}
\usepackage{parskip}
\usepackage{pdfpages}
\usepackage[margin=.6in]{geometry}
\begin{document}
Michel Foucault `invented' homosexuality. Why we have two lectures on homosexuality I dont know.

He had a bunch of impact on how we study history. A historiographer studies the writing of history. This is different from regular history. There was fierce debate about his influence weather it was positive or negative. He was undoubtedly prolific.

He died early and this shook the world. Another shaking fact about this is that he died of aids. It was not well known at the time. They noticed that a lot of young men were dieing of something pnemonia like. There was no known cure and the mortality rate was high which is why a famous person dying was a big deal. It also happened during the growing gay liberation movement.

During the 60's people were not fans of homosexuality. This was also the era of the stonewall riots and government crack downs on homosexual clubs. Homosexuality was discussed in medical or legal terms.

The late 70's and 80's was a bit of a liberation where people pushed towards a better view of homosexuality. This was also a time where people loosened up about sexuality.

Foucault was gay and he was a proponent for gay rights.

The gay community rallied for safe sex education and fought against social stigma.

Foucault presented a paper before shortly before his death and as always he was critiqued on playing loose with the facts.

Foucault was required to teach open lectures as part of his role as chair of history at some university. These became massively popular and they needed amphitheaters to hold it.











\end{document}
