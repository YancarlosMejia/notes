\documentclass{article}
\usepackage{parskip}
\usepackage[margin=.6in]{geometry}
\begin{document}
\section*{Good Design}
\label{sec:good_design}

Ten Commandments:
\begin{enumerate}
	\item innovative
	\item make a product useful
	\item aesthetic
	\item makes a product understandable
	\item unobtrusive
	\item honest
	\item long lasting
	\item thorough down to the last detail
	\item environmentally friendly
	\item as little design as possible
\end{enumerate}

\section*{Rational Design}
\label{sec:rational_design}

\paragraph{Science vs Engineering}
\label{par:science_vs_engineering}
science concerns natural system, engineering concerns artificial systems

\section*{Rational Design}
\label{sec:rational_design}

\paragraph{Inner Environment}
\label{par:inner_environment}
The set of plan alternatives

\paragraph{Outer Environment}
\label{par:outer_environment}
The parameters within which alternatives would operate

Successful design involves relating three sorts of parameters
\begin{itemize}
	\item means: basic components and their quantities
	\item laws: unalterable facts about the problem situation
	\item ends: define what counts as a satisfactory solution
\end{itemize}

\textbf{Obstacles to Rational Design}
\begin{itemize}
	\item out knowledge of inner and outer environments is not perfect
	\item we lack the computing capacity to make accurate predictions
	\item reasonable people may differ in their methods of finding solutions to a problem and their representation of a problem and its solutions
\end{itemize}

\paragraph{Bounded Rationality}
\label{par:bounded_rationality}
When knowledge in incomplete we solve problems as well as we know how

\section*{Social Psychology}
\label{sec:social_psychology}
\paragraph{Technotonicity}
\label{par:technotonicity}
Designs may provoke emotionally positive or negative reactions.

A technotonic design:
\begin{itemize}
	\item feelings of control
	\item reflects high degree of skill
	\item evokes aesthetic pleasure
	\item evokes pleasant associations
\end{itemize}

A technostressing design:
\begin{itemize}
	\item removes feelings of control
	\item demonstrates lack of skill
	\item ugly
	\item evokes unpleasant associations
\end{itemize}

A socially technotonic design:
\begin{itemize}
	\item promotes harmonious social interactions
	\item enhance the user's social standing
	\item increases the user's attachment to their social group
\end{itemize}

A socially technostressing:
\begin{itemize}
	\item promotes hostile or embarassing social interactions
	\item discredits the user
	\item detaches the user from their social group
\end{itemize}

Levels of trust:
\begin{itemize}
	\item appropriate
	\item over-trust
	\item under-trusT
\end{itemize}

\section*{Style}
\label{sec:style}

Three views:
\begin{itemize}
	\item reductionist: good style is a matter of good function
	\item structuralist: good style is a matter of social signaling
	\item commercialist: good style is a matter of marketing
\end{itemize}

An honest design:
\begin{itemize}
	\item does not disguise what it is
	\item exhibits what it is
\end{itemize}

Social signing is what you use the design says about you. A design has decorum when it:
\begin{itemize}
	\item appears appropriate for its kind
	\item belongs in its setting
	\item reflects its user's place in the social order
\end{itemize}

Branding is association of form with producer:
\begin{itemize}
	\item identifiability
	\item differentiability
\end{itemize}

\paragraph{Fashion}
\label{par:fashion}
involves imitation of another brand in form, confusion invites transfers transfer of positive associations from original to imitation

\section*{Culture}
\label{sec:culture}
\paragraph{Social Groups}
\label{par:social_groups}
responses to designs vary with the social context so groups form through membership values.

Three values of culture:
\begin{itemize}
	\item modernism: culture is inessential to good design
	\item contextualism: fit of designs with cultural expectations is important
	\item progressivism: designs should challenge the cultural status
\end{itemize}

\paragraph{MAYA}
\label{par:maya}
most advanced yet acceptable: good designs fits culture but also challenges the status quo
\begin{itemize}
	\item people tend to want progress and innovation
	\item people tend to resist change
\end{itemize}

Technology transfers from developed to developing nations.
\begin{itemize}
	\item small scale
	\item energy efficient
	\item environmentally friendly
	\item labor intensive
	\item controlled by local community
	\item simple enough to be maintained with local expertise
\end{itemize}

\section*{Social Contract}
\label{sec:social_contract}

Two kinds of good:
\begin{itemize}
	\item rational: good designs are ones that achieve their goals in excellent ways
	\item moral: good designs are ones that help achieve excellent goals
\end{itemize}

 Basic rights:
 \begin{itemize}
  	\item life
  	\item liberty
  	\item property
  \end{itemize}

Everyone is in competition with everyone else but we limit competition in basic ways. Cooperation can be helpful so we observe an extended set of rights called \textbf{social contract}.

On the scale of manners to morals:
\begin{itemize}
	\item etiquette
	\item cultural norms
	\item laws
	\item the social contract
\end{itemize}

\section*{Samples}
\label{sec:samples}
Below is a model example for this design assessment:

"The cheap waterless toilet (nanomembrane), is to serve the need of users in developing countries where water, proper hygiene methods, safety and essential utilities are lacking. It will have the specifications necessary to discard waste into usable byproducts such as energy, water and ash. This toilet will be analyzed based on the aspects of Style and Culture.

When considering Style, the reductionist approach states that the quality of a product is highly dependent on the usefulness, such that only well executed designs are aesthetically pleasing. With the waterless toilet, this holds true as the structure is simple and its style is not exaggerated to appear highly advanced even though it does allow for the separation and isolation of clean water, incineration of solid waste and production of energy. This is all hidden under what seems to be just a shiny toilet. On this same aspect of hiding its incredible functionality under nothing extraordinary, the design is honest in that it does not disguise what it is. The shape is not deceiving and it is immediately identified as a toilet.  This is especially important as it is being rolled out in developing areas where there may be lower literacy levels and introducing a device that may require reading manuals or carrying out high end procedures to meet the advanced functionality would not be effective. The toilet’s operation is based on lifting and lowering the lid, and collecting ash and water; these activities are easily understandable and can even be said to be instinctive. The only aspect of style this toilet may contradict with is decorum, for it will have intricate and modern specifications while it will be utilized in a setting where other facilities such as standalone showers or kitchens are not found. This will make it slightly incongruent with its setting.

The nanomembrane toilet has gone above and beyond the expectations of a standard toilet in that it is not only being used to conserve water as other toilets are presently being designed for, it is also producing water while taking none in, and giving out free energy. This clearly defines a progressive view to what good design is about. The design challenges the cultural status quo for we are all of the idea that toilets take in waste and get rid of it, but this takes in waste, doesn’t get rid of it but converts it to products that could be used domestically in farming, irrigation or even small device power supply. It could be argued that this toilet gives out more than it takes in.

In conclusion, the nanomembrane toilet will prove to be a vital addition to the societies in which it is to be implemented. The toilet not only meets its basic functionalities but is especially a good design for its style is honest and not overdesigned. This will make it more user friendly; the only downside is that it will be lacking some decorum due to the rural settings it may end up being installed in, though this does not really affect its operation. Its progressive nature has proved that we can get a lot more from devices which we would naturally confine to specific uses."

What makes this a model example:

For more general points, please have a look at the Gryffindor Feedback, as all of the same points are applicable here as well.
The assessment specifically describes what reductionism and progressivism mean: the author mentions that reductionism "states that the quality of a product is highly dependent on the usefulness, such that only well executed designs are aesthetically pleasing" and that progressivism "challenges the cultural status quo." Some other assessments did not fully explain what these (or other concepts they mentioned) actually mean. The assessment also clearly indicates how the chosen concepts specifically apply to this design.
The assessment does not confuse what honest design is. Some assessments accuse this design of dishonesty for hiding its mechanisms or for doing more than what a normal toilet does. Not being able to see the inner workings does not constitute dishonesty: most designs in fact are not transparent. Innovation is also not dishonesty: just because a design does something different from what might be expected does not make it dishonest, necessarily. Dishonest designs are ones that either hide what they really are, or that promise things that are not realistic or are misleading. This toilet does not hide the fact that it is a toilet, and it is very clear from the article what its intended function is, which is also made clear to those using it. It does not attempt to deceive the user into thinking it is anything more than what it is.
One other point that is not really covered by this example, but was a source of some confusion, is the difference between culture and environment. The toilet design is aimed at addressing the environmental conditions in places such as Ghana, which is not the same as being designed for the culture. This most often arose in connection with contextualism. When discussing contextualism, you have to be clear that you are describing how the design fits in with the culture, rather than just the environment. As with the "lucky fish" example, some designs that address issues common in certain areas (anemia in Cambodia) are rejected on a cultural basis, even thought they are designed to solve the environmental problem very effectively.
Lastly, the assessment covers the assigned concepts. It is important to focus on the topics that are assigned in the assignment schedule (so in this case Style and Culture). While some discussion of previous concepts is fine, and some concepts certainly overlap, for example Good Design and Style, the emphasis should be on the assigned concepts.

Below is a model example for this design assessment:

"The Bird of Prey bike is a new bike design where the rider lays prone on a leather pad seat rather than sitting upright as in conventional bike design. The bike contains many of the same main components as a conventional bike, including the handlebars and pedals. However, the pedals have been moved to the back of the bike rather than the middle, because of the new riding posture. The design has both positive and negative aspects associated with it and will be evaluated using the concepts of Good Design and Social Psychology.



The concepts of Good Design recommend that a design should follow the principles of minimalism, and put function over aesthetic appeal. With regards to minimalism, the Bird of Prey bike does not contain any features or parts that are not directly related to the function of the bike. For example, the bike contains the same components as a conventional bike including a seat, handlebar, pedals, chain and gears. Since minimalism calls for a design to be simple yet functional, I believe that this design adheres to that principle. I think that the bike has two main functions, one is to get the rider to their destination safely, as well as quickly. I think that the Bird of Prey design is innovative enough that it accomplishes both of these functions better than a conventional bike. Since the rider is prone, this design adds an additional safety factor since the rider can no longer flip over their handlebars in the event of rapid braking. The rider is also in a more aerodynamic position with better leg extension to help them get to their destination faster. Therefore, the new design should be able to achieve both functions better than a conventional bike without the addition of any extraneous or unneeded components.



While the design does adhere to the concepts of Good Design learned in class, from the perspective of Social Psychology, I think that this design is rather technostressing. By using this bike, the rider is excluded from the well-established social group of regular bike riders. However, if this bike design were to become popular enough, the rider could now join a new social group of “elite” bike riders, all of which ride this design and pride themselves on getting to destinations faster than the other conventional bike riders. I personally find this bike design rather unappealing, and I think that most people would at first dislike this design because it is so strange compared to the conventional design which has been around for hundreds of years. In addition to the rider being excluded from the regular, large social group of regular bike riders, this bike is also not compatible for any function besides getting from one destination to another as quickly as possible. Most bike baskets which can be used for carrying groceries are designed for regular conventional bikes, and I feel like lights and bells attached to the handlebars might slightly obstruct the vision of the rider.



In conclusion, the new bike design is innovative and complies to the principles of Good Design as defined by Dieter Rams. However, the design is technostressing since it isolates the user from the well-established social norm, is unappealing since it is so unconventional and may not be compatible with common marketplace add-ons such as grocery baskets."

What makes this a model example:

The assessment follows all of the instructions. Some assessments did not include a reflection or conclusion on what the author learned from the assignment.
The assessment clearly outlines all the concepts used. Common weaknesses in other assignments were a lack of clarity on what the key concepts actually mean, for example, saying that the design follows minimalism without clearly indicating what minimalism actually is.
The assessment clearly explains why the author believes the key concepts mentioned apply to the design. Other assessments may have been lacking in this explanation, for example, it is not enough to say that the design is technostressing. You also need to show clearly why you believe the design fits in with the concepts you are using in order to fully contextualize your assessment.
The assessment makes clear, unambiguous statements. Some assessments were more vague, for example, saying "this may show (concept)" without coming to a conclusion either way. It is fine to indicate that there can be diverse opinions on the design, but you should reach your own conclusions as well in order to provide persuasive arguments.
The assessment does not unnecessarily re-describe the design. While description of relevant features of the design is helpful, particularly in the context of why certain elements fit in with the concepts you are discussing, it is not of any real value to simply re-describe the entire design. Focus on describing only what parts of the design are connected to your discussion and clearly explain why they are relevant.
The assessment is clear and focused. Lack of clarity in both spelling and construction of sentences or paragraphs can lead to it being difficult for readers to fully understand or follow your arguments. Having someone proofread your assessment before submission can help to make sure that your points are coming across clearly to readers.


\end{document}
