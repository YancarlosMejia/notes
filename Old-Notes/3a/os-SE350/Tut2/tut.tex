\documentclass[12pt]{article}
\usepackage{parskip}
\usepackage{amsmath}
\usepackage{pdfpages}
\usepackage[margin=.6in]{geometry}
\begin{document}

\section{Kernel}
Also nown as the nucleus. This is is in main memory and includes the most frequently used portions of the software. Run in privileged mode.

\section{Multiprogramming}
The ability to switch between multiple programs and increases cpu utilization and throughput. minimize response time for time sharing systems.

\section{Process}
A program execution. Scheduled and controlled by OS. Consists of program code, associated data, and context.

\section{Execution Context}
internal data buy which the os is able to supervise and control th process.

\section{Problems}
\subsection{2.1}
Suppose we have multiprogrammed computer. Computation time T for a job half the time is spend in IO and other half in processor. Each job runs for a total of N periods. Using round robin scheduling. Only processing cycles cant overlap.

For 1, 2, and 4 simultaneous jobs find:
\begin{itemize}
    \item turnaround time (actual time to complete the job):
    \item throughput (average number of jobs completed per time period T)
    \item Processor utilization (percentage of time the processor is active)
\end{itemize}
Assume first half of T is for IO and second is for processor.

1 Job: The squence just alternates between IO and PRocessor, turnaround time is NT, throughput is $\frac{1}{N}$. Utilization is 50\%

2 Jobs: Since both IO's can run at the same time they each just execute away but since the processor can only run one program at a time they must take turns using it. So turanound time is NT, throughput is $\frac{2}{N}$, utilization is 100\%.

4 Jobs: Is an alternating sequence of 2 job sets so its turnaround is (2N-1)T, throughput is $\frac{4}{2N-1}$, and utilization is 100\%

\subsection{2.1b}
Repeate 2.1a but assume that the processor is for the first and last quarters ... see answer in slides


\subsection{2.3}
Show the difference between scheduling policies you might use to optimize time sharing system or a multiprogrammed batch system.

Timesharing: minimize turn around so use dtime slicing to give all processes
FUUUUCKKKKK POST THE GODDAMN SLIDES BEFORE CLASS YOU ASSHOLE


\subsection{2.4}
What is a system call and how do they relate to dual mode operation?

System call (a program to invoke functions from the OS) is used to switch between modes.

\section{Instruction Trace}
This spits out the series of instructions that have happened in the program so far.

\section{Process Creation}
Creates a data struction for the OS and allocates memory for it. Caused by a new job, interactive log on, by the os to do stuff, created by the existing procress.

\section{Process Pre-emption}
The OS needs to stop the currently running process (probably for a more important one).

\section{Modes of Operation}
User mode: restrictions on what they can do and what memory can be access to keep the user from fucking shit up.

Kernel mode: system has full access to processor, instructions, registers, etc.

\section{Process Model}
New (process is ready ) is admited (memory is allocated) into ready (ready to run). Ready is dispatched (started when processor is ready) to running. Running is then sent to blocked when there is a event wait where it stays until a the event occurs where it goes into ready. Running can go back to ready when there is a timeout. If neither of these things happen the process reaches exit by releasing the processor.
\end{document}
