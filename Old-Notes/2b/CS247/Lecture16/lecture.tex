\documentclass[12pt]{article}
\usepackage{parskip}
\usepackage[margin=.6in]{geometry}
\usepackage{listings}
\usepackage{color}

\definecolor{dkgreen}{rgb}{0,0.6,0}
\definecolor{gray}{rgb}{0.5,0.5,0.5}
\definecolor{mauve}{rgb}{0.58,0,0.82}

\lstset{frame=tb,
  language=C++,
  aboveskip=3mm,
  belowskip=3mm,
  showstringspaces=false,
  columns=flexible,
  basicstyle={\small\ttfamily},
  numbers=none,
  numberstyle=\tiny\color{gray},
  keywordstyle=\color{blue},
  commentstyle=\color{dkgreen},
  stringstyle=\color{mauve},
  breaklines=true,
  breakatwhitespace=true,
  tabsize=3
}

\begin{document}

\subsection*{Stl}
This is a collection of useful typesafe, generic containers that
\begin{itemize}
  \item know almost nothing about their elements
  \item focus mostly on membership (insert, erase)
  \item know nothing about algorithms
  \item can define their own iterators
\end{itemize}

There are lots of algorithms and ways to extend and personalize them to your particular solution.

\subsection*{Overview}
Most stl algorithms process a sequence of data elements
\begin{itemize}
  \item traverse a sequence of elements bounded by two iterators
  \item access elements through the iterators
  \item operate on each element during traversal
\end{itemize}

\subsection*{Non modifying algorithms}
This is a group of algorithms that read but never wrtie to the elments in their input ranges
\begin{lstlisting}
  template<class Input Iterator, class T>
  InputIterator find (InputIterator first, Input Iterator last, const T& val) {
    while(first != last) {
      if(*first = val) {
        return first;
      }
      ++first
    }
  }
\end{lstlisting}

Usage code:
\begin{lstlisting}
  #include<algorithm>
  vector<int> vec;
  //add stuff into it
  vector<int>::iterator result = find(vec.begin(), vec.end(), 42);
  *result == 42 //the find returned a iterator to the 42 element
  result == vec.end() // the element was not in vec so find retuned the end
\end{lstlisting}


\subsection*{Algorithms over two sequences}
Algorithms that operaate over two sequences of data specity the full range over the firs swquence and only the start of the second sequance

\begin{lstlisting}
  tempate <class Iterator1, class Iterator2>
  bool equal (Iterator1 first1, Iterator1 last1, Iterator2 first2, Iterator2 last2) {
    while(first1 != last1) {
      if(1(*first1==*first2)) {
        return false
      }
      ++first1;
      ++first2;
    }
  }
\end{lstlisting}

The above algorithm as some problems:
\begin{itemize}
  \item sequences are of different length
  \begin{itemize}
    \item STL does not catch when it iterates past the end of the container resulting in unpredictable results or an error
  \end{itemize}
  \item containers are not the same type
  \begin{itemize}
    \item similarly STL does not care about the type of the container (it doesnt know what its being applied to)
    \item this goes undetected, programmer must spot this
  \end{itemize}
  \item elements are not the same type
  \begin{itemize}
    \item this will be caught by the compiler
  \end{itemize}
  \item first and last do not point to the same container
  \begin{itemize}
    \item algorithm will never finish
  \end{itemize}
  \item first is after last
  \begin{itemize}
    \item another infinite loop
  \end{itemize}
\end{itemize}

Moral of the story algorithms dont check shit so be careful when you use them.

EX:
\begin{lstlisting}
  #include iostream algorithm vector
  int main(){
    int myints[] = shit
    vector<int> myvector = more shit
    if(equal ( myvector.begin, myvector.end(), *myints[0], *myints[myints.size()] ))
    print these are equal
  }
\end{lstlisting}

\subsection*{Modifying algorithms}
Some algorithms overwrite element values in existing containters. We need to make sure that the destination sequence is large enough for the number of elements being written

\begin{lstlisting}
  tempate <class Iterator1, class Out>
  bool copy (Iterator1 first1, Iterator1 last1, Out result) {
    while(first != last) {
      *result = *first;
      ++result, ++first;
    }
    return results
  }
\end{lstlisting}
For this example the ability to go past the end of the container is a big security breach, watch for it.

\subsection*{Overwriting verses Inserting}
The default behavior is to write to a destination sequence, overwriting what is there, but you can pass it a inserter destination to tell it to go there.
\begin{lstlisting}
//This example will insert the first 5 elements of myints to the start of myvector
copy(myints + 1, myints + 6, inserter(myvector, myvector.begin()))
\end{lstlisting}


\subsection*{Removing elements}
Algorithms never directly change the size of the container so instead they rearrange the elements putting the elements to be removed at the end of the container and returning a pointer to the start of those elements.

\begin{lstlisting}
  template<class Iterator, class T>
  Iterator remove (Iterator first, Tterator last, const T& val)

  //Usage
  vector<int>::iterator end = remove(vec.begin(), vec.end(), 42);
  vec.erase(end, vec.end());
\end{lstlisting}

\subsection*{Algorithms that apply operations}
A number of algorithms apply operations to the elements in the input range. Some Stl algorithms accept a predicate which is applied to all elements in the iteration and is used to restrict the set of data elements that are operated on.

\begin{lstlisting}
  bool gt20(int x) { return 20<x; }
  bool gt10(int x) { return 10<x; }

  int a[] = {20, 15, 10};
  int b[10];

  remove_copy_if(a, a+3, b, gt20); //b[] == {25}
  cout<< count_if(a, a+3, gt10); // Prints 2

\end{lstlisting}

\subsection*{Function Objects}
If we need a funtion that refers to data other than the iterate elements we need to define a funciton object called a functor. This is a class that
\begin{itemize}
  \item overloads the operator() to allow a object to be used with function call syntax
  \item can be used in other algorithms
\end{itemize}
\begin{lstlisting}
class gt_n {
  int val;
public:
  gt_n(int val) : val_(val){}
  bool operator()(int n { return n > val_});
};

int main() {
  gt_n gt4(4); //constructor
  cout< gt4(4);//prints 0 for false
  cout<< gt4(5);  //prints 1 for true
  cout<<count_if(a, a + size, gt4) //this allows us to count only things greater than 4 through this unary predicate
}
\end{lstlisting}

Function objects are declared as classes but override the () operator. They work like a single function, but can be passed like an object.

Another example:
\begin{lstlisting}
  class inc {
  public:
    inc(int amt) : increment_(amt){}
    int operator()(int x) { return x + incrememnt_};
  private:
    incrememnt_
  };

transform(v.begin(), v.end(), )
\end{lstlisting}


Recap:
\begin{itemize}
  \item processes: transform, search, sort, filter, copy
  \item data sequences: sequential data structure
  \item of any type
  \begin{itemize}
    \item operates over iterators into data structure
    \item type of data element is almost irrelevant
  \end{itemize}
  \item in a type-safe-manner: compiler type checks, reports errors
\end{itemize}

There is a second form of transform.
\begin{lstlisting}
  transform<first1, last1, first2, result, binaryFuntion>

  class sum_inc{
    int amt_;
  public:
    sum_inc(jnt amt): amt_(amt){}
    int operator()(int x, int y) { return x + y + amt_; }
  }

  //usage
  transform(v.begin(), v.end(), l.begin(), v.begin(), sum_inc(100));
\end{lstlisting}

\subsection*{Classification of STL Function Objects}
\begin{itemize}
  \item Gernerator: a type of function object that takes not arguments and returns a value of an arbitrary type
  \item unaary function: a type of function object that takes a single argument a of any type and returns a value that may be of a different type or may be void
  \item binary function: a type of function object that takes 2 parameters of any, possibly distinct types, and returns any type
  \item unary predicate: unary function that returns a bool
  \item binary predicate: binary function that returns a bool
\end{itemize}

\subsection*{Predefined Function Objects}
The header <functional> provides a number of useful generic function objects (most of the general ones you'd think of). These make using function objects easier since we don't have to instantiate them ourselves.

Ex:
\begin{lstlisting}
  #include
#include
#include
#include
<iostream>
<algorithm>
<vector>
<functional>


int op_increase (int i) { return ++i; }

int main () {
  std::vector<int> foo;
  std::vector<int> bar;
  // set some values:

  for (int i=1; i<6; i++)
    foo.push_back (i*11); // foo: 11 22 33 44 55

  bar.resize(foo.size()); // allocate space
  std::transform (foo.begin(), foo.end(), bar.begin(), op_increase); // bar: 12 23 34 45 56

  // std::plus adds together its two arguments:
  std::transform (foo.begin(), foo.end(), bar.begin(), foo.begin(), std::plus<int>());

  // foo: 23 45 66 87 111
  std::cout << "foo contains:";

  for (std::vector<int>::iterator it=foo.begin(); it!=foo.end(); ++it)
    std::cout << ' ' << *it;
  std::cout << '\n';

  return 0;
}
\end{lstlisting}

\subsection*{Predefined Funtion Object Adapters}
\begin{itemize}
\item <functional> also defines a number of useful generic adaptors to modify the interface of a function object.
\item bind1st - convert a binary function object to a unary function object (by fixing the value of the first operand)
\item bind2nd - convert a binary function object to a unary function object (by fixing the value of the second operand)
\item mem_fun - convert a member function into a function object (when member function is called on pointers to objects)
\item mem_fun_ref - convert a member function into a function object (when member function is called on objects)
\item not1 - a function adaptor that reverses the truth value of a unary function object
\item not2 - a function adaptor that reverses the truth value of a binary function object
\item ptr_fun - convert a function pointer to a function object so that a generic adaptor can be applied -- otherwise, can simply use function pointer
\end{itemize}

Example: Convert a binary function object to a unary function object

\begin{lstlisting}
bind2nd( greater<int>(), 15 ); // x > 15
bind1st( minus<int>(), 100);
// 100 – x
\end{lstlisting}
Example: Convert a member function into a function object
\begin{lstlisting}
vector<string> strings;
vector<Shape*> shapes;
transform( strings.begin(), strings.end(), dest,
mem_fun_ref( &string::length ) );
transform( shapes.begin(), shapes.end(), dest2,
mem_fun( &Shape::size ) );
\end{lstlisting}
Example:
\begin{lstlisting}
  gt15: bool

  furnction object gt_n // gt10(x), gt15(x)

  greater<T>

  count_if(v.begin(), v.end(), bind2nd(greater<int>(), 15));
\end{lstlisting}

\subsection*{Adaptable Function Objects}
Tbe adaptable a function object class must provide nested type definitiond for its arguments
\begin{lstlisting}
typedef T1 argument_type;
typedef T2 result_type;
typedef T3 first_argument_type;
typedef T4 second_argument_type;
\end{lstlisting}

The functional header provides a unary_function type and binary_function type that you can inherit from.
Ex: Creating the greater than function object
\begin{lstlisting}
class gt_n : public unary_function<int, bool> {
  int value;
public:
  gt_n(int val) : value(val) {}
  bool operator()(int n) {
    return n > value;
  }
};
\end{lstlisting}










\end{document}
