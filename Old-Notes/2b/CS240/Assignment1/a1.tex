\documentclass[12pt]{article}
\usepackage{amsmath}
\usepackage{algo,fullpage,url,amssymb,epsfig,color,xspace,enumerate}
\usepackage[pdftitle={CS 240 Assignment 1},%
pdfsubject={University of Waterloo, CS 240, Spring 2014},%
pdfauthor={Romain Lebreton}]{hyperref}

\renewcommand{\thesubsection}{Problem \arabic{subsection}}

\begin{document}

\begin{center}
{\Large\bf University of Waterloo}\\
\vspace{3mm}
{\Large\bf CS240 - Spring 2014}\\
\vspace{2mm}
{\Large\bf Assignment 1}\\
\vspace{3mm}
\textbf{Due Date: Wednesday May 21 at 09:15am}
\end{center}

\definecolor{care}{rgb}{0,0,0}
\def\question#1{\item[\bf #1.]}
\def\part#1{\item[\bf #1)]}
\newcommand{\pc}[1]{\mbox{\textbf{#1}}} % pseudocode

Please read
\url{http://www.student.cs.uwaterloo.ca/~cs240/s14/guidelines.pdf}
for guidelines on submission.  
Problems 1 -- 6 are written
problems; submit your solutions electronically as a PDF with file name {\tt a01wp.pdf} using MarkUs. We will also accept individual question files named {\tt a01q1w.pdf}, {\tt a01q2w.pdf}$, \dots,$ {\tt a01q6w.pdf} if you wish to submit questions as you complete them.

There are 78 marks available; the assignment will be marked out of 74.

%%%%%%%%%%%%%%%%%%%%%%%%%%%%%%%%%%%%%%%%%%%%%%%%%%%%%%%%%%%%%
\subsection{[4+4+4+4+4=20 marks]}
Provide a complete proof of the following statements from first principles
(i.e., using the original definitions of order notation).  All logarithms
are natural logarithms: $\log = \ln$.
\begin{enumerate}[(a)]
\item $ (n + 10)^3 \in \Theta(n^3) $
\begin{align*}
0\leq n^3 - 21n^2 +100 &\leq cn^3\\
n^3 - 21n^2 \leq n^3 + 100 &\leq cn^3\\ 
n^3 + 100 &\leq cn^3\\
n^3 &\leq n^3 \rightarrow n_{0} \geq 1\\
100 &\leq n^3 \rightarrow n_{0} \geq \sqrt[3]{100}\\
n^3 + 100 &\leq 2n^3 \rightarrow c = 2
\end{align*}
$\therefore$ there exists a value c (2) and $n_{0} = \sqrt[3]{10}$ such that $0\leq n^3 - 21n^2 +100 \leq cn^3$ for $\forall n \geq n_{0}$
\item $ (n + 10)^3 \in \Theta(n^3) \implies $
$0\leq c_{1}n^3 \leq n^3 + 30n^2 + 300n + 1000 \leq c_{2}n^3$ \\
\begin{align*}
c_{1}n^3 &\leq (n+10)^3 \\
n &\leq n + 10 \rightarrow n_{0} > 0 \text{ and } c_{1} = 1 \\
n^3 + 30n^2 + 300n + 1000 &\leq c_{2}n^3\\
n^3 &\leq n^3 \rightarrow n_{0} \geq 1\\
30n^2 &\leq n^3 \rightarrow n_{0} \geq 30\\
300n &\leq n^3 \rightarrow n_{0} \geq \sqrt[2]{300}\\
1000 &\leq n^3 \rightarrow n_{0} \geq 10\\
n^3 + 30n^2 + 300n + 1000 &\leq c_{2}n^3 \rightarrow c_{2} = 4 \text{ and } n_{0}>30
\end{align*}
$\therefore$ there exists $c_{1} = 1 \text{ and } c_{2} = 4$ such that $(n+10)^3 \in \Theta (n^3)$ for $\forall n \geq n_{0}$
\item $1000n \in o(n \log n)$
\begin{align*}
0 \leq 1000n &\leq cn \ln n \\
1000 &\leq c\ln n \\
e^{1000} &\leq cn \rightarrow n_{0} \geq e^{1000} \enspace \forall c > 0
\end{align*}
$\therefore \enspace \forall c > 0$ there exists a $n_{0} = e^{1000}$ such that $1000n \leq cn \ln n$ for  $\forall n \geq n_{0}$
\item $n ! \in o(n^{n})$ where $n ! = \prod_{i=1}^n i$
\begin{align*}
n! &\leq cn^n\\
n*(n-1)*(n-2)* \dots *2*1 &\leq cn*n*\dots *n*n\\
(n-1)*(n-2)* \dots *2*1 &\leq cn*\dots *n*n\\ 
(n-1)&\leq n\\
(n-2)&\leq n\\
\dots &\\
2 &\leq n\\
1 &\leq n\\
\end{align*}
$\therefore$ $\enspace \forall c > 0$ there exists a $n_{0} >1 $ such that $n ! \leq cn^{n}$ for  $\forall n \geq n_{0}$
\item Bonus question : Let $H(n) = \sum_{i=1}^n 1/i$. 
  Prove that $H(n) \in \omega (1) $.
\begin{align*}
H(n)\in \Theta (\log n)&\\
&\implies c_{1}\log n \leq H(n) \leq c_{2}\log n\\
&\implies c\log n \leq H(n) \\
\log n > 1 \enspace \forall n > 0 &\implies 1 \leq c\log n\\
1\leq c\log n &\leq H(n) \enspace \forall c > 0\\
H(n) &\in \omega(1)
\end{align*}
\end{enumerate}
%%%%%%%%%%%%%%%%%%%%%%%%%%%%%%%%%%%%%%%%%%%%%%%%%%%%%%%%%%%%%
\subsection{[4+4+4=12 marks]}
For each pair of the following functions, fill in the correct asymptotic
notation among $\Theta$, $o$, and $\omega$ in the statement $f(n)\in
\sqcup(g(n))$.  Provide a brief justification of your answers.  In your
justification you may use any relationship or technique that is described
in class.
\begin{enumerate}[(a)]
\item $f(n) = n^3(5+2\cos{2n})$ versus $g(n)=3n^2+4n^3+5n$
\begin{align*}
\displaystyle\lim_{n \to \infty}\frac{n^3(5 +2\cos{2n})}{3n^2+4n^3+5n}&=\displaystyle\lim_{n \to \infty}\frac{5 + 2\cos{2n}}{3n^{-1}+4+5n^{-2}}\\
&= \frac{5 + 2\cos{2n}}{4}\\
&= \frac{5 \pm 2}{4}\\
&= \frac{7}{4} \text{ or } \frac{3}{4}\\
\end{align*}
$\therefore f(n) \in \Theta (g(n))$\\
\item $f(n) = n (\log n)^{3}$ versus $g(n) = n^{2}$. \emph{Hint:} Use L'Hopital's rule.
\begin{align*}
\displaystyle\lim_{n \to \infty} \frac{n(\log n)^3}{n^2} &= \displaystyle\lim_{n \to \infty} \frac{(\log n)^3}{n}\\
&= \displaystyle\lim_{n \to \infty} \frac{3(\log n)^2}{n}\\
&= \displaystyle\lim_{n \to \infty} \frac{6\log n}{n}\\
&= \displaystyle\lim_{n \to \infty} \frac{6}{2}\\
&= 0
\end{align*}
$\therefore \enspace f(n)\in o(g(n))$  
\item $f(n) = n^{0.01}$ versus $g(n) = (\log n)^{2}$
\begin{align*}
\displaystyle\lim_{n \to \infty} \frac{n^{0.01}}{(\log n)^2} &= \displaystyle\lim_{n \to \infty}\frac{0.01n}{n^{0.99}2\log n}\\
&=\displaystyle\lim_{n \to \infty}\frac{0.005}{0.99n^{-0.01}\log n + n^{-0.01}}\\
&= \displaystyle\lim_{n \to \infty}\frac{0.005n^{0.01}}{0.99\log n + 1}\\
&= \displaystyle\lim_{n \to \infty}\frac{0.00005n^{0.01}}{0.99}\\
&= \infty
\end{align*}
$\therefore \enspace f(n) \in \omega (g(n))$
\end{enumerate}
\newpage
%%%%%%%%%%%%%%%%%%%%%%%%%%%%%%%%%%%%%%%%%%%%%%%%%%%%%%%%%%%%%
\subsection{[4+4+4+4+4=20 marks]}
Prove or disprove each of the following statements.  To prove a
statement, you should provide a formal proof that is based on the
definitions of the order notations.  To disprove a statement, you can
either provide a counter example and explain it or provide a formal proof.
All functions are positive functions.
\begin{enumerate}[(a)]
\item $f(n) \not \in \omega(g(n)) \Rightarrow f(n) \in O(g(n))$
\begin{align*}
f(n) \not\in \omega(g(n)) & \implies \not\exists c > 0 \mid cg(n) \leq f(n) \\
&\implies f(n) < g(n) \enspace \forall c > 0\\
&\implies f(n) < g(n) \enspace \exists c >0 \\
&\implies f(n) \in O(g(n))
\end{align*}
\item $f(n) \in O(g(n)) \Rightarrow \exists c>0 \ \forall n\in \mathbb{N}, f(n) < c g(n)$
\begin{align*}
f(n) &\leq c_1 g(n) \enspace \forall n > n_{0} \\
f(n_0) &\leq g(n_0)\\
f(n) &< f(n_0) \enspace \forall n < n_0\\
g(n) &> g(0) \enspace \forall n > 0\text{ Because g is a positive function }\\
\frac{g(n)}{g(0)} &> 1\\
f(n_0) \leq g(n_0) \land \frac{g(n)}{g(0)} > 1 &\implies
f(n_0) < g(n_0)\frac{g(n)}{g(0)}\enspace \forall n > 0\\
f(n) < f(n_0) &< g(n)\frac{g(n_0)}{g(0)}\enspace \forall n > 0\\
\therefore & \exists c = c_1 \frac{g(n_0)}{g(0)} \mid  f(n) \leq g(n) \enspace \forall n > 0 = n \geq 1\\
\end{align*}
\item $f(n)\in \Theta(g(n))\Rightarrow 2^{f(n)} \in \Theta(2^{g(n)})$
\begin{align*}
f(n) \leq cg(n)
&\implies 2^{f(n)} \leq 2^{cg(n)} \enspace \forall n > n_0\\
&\implies 2^{f(n)} \leq 2^{d+g(n)} \text { for } d(n) = g(n)(c-1)\\
&\implies 2^{f(n)} \leq 2^{d(n)}2^{g(n)}\\
&\implies 2^{f(n)} \leq c_{2}2^{g(n)} \text{ where } c_2 = 2^{d(n_0)}\\
\therefore f(n) \in O(g(n))&\\
f(n) \geq c(g(n))
&\implies 2^{f(n)} \geq 2^{cg(n)} \enspace \forall n > n_0\\
&\implies 2^{f(n)} \geq 2^{d+g(n)} \text { for } d(n) = g(n)(c-1)\\
&\implies 2^{f(n)} \geq 2^{d(n)}2^{cg(n)} \text{ where } c_2 = 2^{d(n_0)}\\
&\implies 2^{f(n)} \geq c_{3}2^{cg(n)}\\
\therefore f(n) \in \Omega (g(n))\\
\end{align*}
$\therefore f(n) \Theta (g(n))$\\
\item $f(n)\in \Theta(g(n))$ and 
         $h(n)\in \Theta(g(n)) \Rightarrow \frac{f(n)}{h(n)}\in \Theta(1)$ 
\begin{align*}
f(n)\in \Theta (g(n)) \implies c_{1}g(n) &\leq f(n) \leq c_{2}g(n)\\
h(n)\in \Theta (g(n)) \implies c_{3}g(n) &\leq h(n) \leq c_{4}g(n)\\
\implies c_{5} &\leq \frac{f(n)}{h(n)} \leq c_{6}\\
\end{align*}         
$\therefore \frac{f(n)}{h(n)}\in\Theta(1)$
\item $\min(f(n),g(n)) \in \Theta\left (\frac{f(n)g(n)}{f(n)+g(n)}\right)$  
\begin{align*}
Case: f(n) < g(n)\\
\min(f(n), g(n)) & \leq c \frac{f(n)g(n)}{f(n)+g(n)}\\
f(n) &\leq c\frac{f(n)g(n)}{f(n)+g(n)}\\
f(n)^2 + f(n)g(n) &\leq cf(n)g(n)\\
2f(n)^2 &\leq cg(n)^2 \enspace \forall c > 2\\
Case: g(n) < f(n)\\
\min(f(n), g(n)) & \leq c \frac{f(n)g(n)}{f(n)+g(n)}\\
g(n) & \leq c \frac{f(n)g(n)}{f(n)+g(n)}\\
2g(n)^2 &\leq cf(n)^2 \enspace \forall c>2\\
\end{align*}
$\therefore \min(f(n),g(n)) \in O\left (\frac{f(n)g(n)}{f(n)+g(n)}\right)$
\begin{align*}
Case: f(n) < g(n)\\
c \frac{f(n)g(n)}{f(n)+g(n)} &\leq min(f(n),g(n))\\
c \frac{f(n)g(n)}{f(n)+g(n)} &\leq g(n)\\
c \frac{g(n)^2}{2g(n)} &\leq g(n)\\
c \frac{g(n)}{2} &\leq g(n) \enspace \forall c >0\\
Case: f(n) < g(n)\\
c \frac{f(n)g(n)}{f(n)+g(n)} &\leq min(f(n),g(n))\\
c \frac{f(n)g(n)}{f(n)+g(n)} &\leq f(n)\\
c \frac{f(n)^2}{2f(n)} &\leq g(n)\\
c \frac{f(n)}{2} &\leq f(n) \enspace \forall c >0\\
\end{align*}
$\therefore \min(f(n),g(n)) \in \Omega\left (\frac{f(n)g(n)}{f(n)+g(n)}\right)$
$\therefore \min(f(n),g(n)) \in \Theta\left (\frac{f(n)g(n)}{f(n)+g(n)}\right)$
\end{enumerate}
%%%%%%%%%%%%%%%%%%%%%%%%%%%%%%%%%%%%%%%%%%%%%%%%%%%%%%%%%%%%%
\subsection{[4+4+2=10 marks]}
%% Suppose $n$ is a power of two and 
Let $\theta$ be either $2$ or $3$. 
%a parameter in the range $2 \leq \theta \leq 3$.  
Consider the sum
$$f(n) := \sum_{i=0}^{\log_2 n} 4^i \left \lfloor \frac{n}{2^i} \right \rfloor ^{\theta}.$$
\begin{enumerate}[(a)]
\item Assume that $n$ is a power of two. 
Give an exact closed form for $f(n)$ in terms of $n$ and $\theta$.  {\em Hint:} Re-write the formula as a geometric series, and treat $\theta=2$ and $\theta=3$ as separate cases.
\begin{align*}
f(n) &= \sum_{i=0}^{\log_2 n} 4^i \left \lfloor \frac{n}{2^i} \right \rfloor ^{\theta}\\
&= \sum_{i=0}^{\log_2 n} 2^{2i} \left \lfloor \frac{n}{2^i} \right \rfloor ^{\theta}\\
n = 2^k \Rightarrow \frac{n}{2^i} \in \mathbb{Z} \Rightarrow f(n) &=\sum_{i=0}^{\log_2 n} 2^{2i}\bigg (\frac{n}{2^i}\bigg )^{\theta}\\
&=\sum_{i=0}^{\log_2 n} 2^{2i-\theta i} n^{\theta}\\
\text{Case: } \theta = 2 &\\
f(n) &= \sum_{i=0}^{\log_2 n} n^{\theta}\\
&= n^{\theta}(1+ \log_2 n)\\
\text{Case: } \theta = 3\\
f(n)&= \sum_{i=0}^{\log_2 n} 2^{2i-\theta i} n^{\theta}\\
&= n^{\theta} \sum_{i=0}^{\log_2 n} \bigg (2^{2-\theta}\bigg )^i\\
&= n^{\theta}\frac{1-\bigg (2^{2-\theta}\bigg )^{log_2 n + 1}}{1-2^{2-\theta}}
\end{align*}
\item Now consider the function $g(n) := f(2^{\lceil \log_2 n\rceil})$, 
defined for all positive $n$. Give simple bounds for $g(n)$
using $\Theta$-notation. You should have two $\Theta$ bounds: 
one for $\theta=2$ and one for $\theta = 3$.\\
Since n is 2 to the power of a whole number we can sub it into the above equations\\
\begin{align*}
g(n) &= f(2^{\lceil \log_2 n \rceil})\\
\text{Case }\theta = 2:&\\
g(n)&= \bigg(2^{\lceil \log_2 n \rceil}\bigg)^{\theta}(1+ \log_2 \bigg(2^{\lceil \log_2 n \rceil}\bigg))\\
&= \bigg(2^{\lceil \log_2 n \rceil}\bigg)^{\theta}(1+ \lceil \log_2 n \rceil)\\
\text{Case }\theta = 3:&\\
g(n) &= n^{\theta}\frac{1-\bigg (2^{2-\theta}\bigg )^{log_2 ((2^{\lceil \log_2 n\rceil}) + 1)}}{1-2^{2-\theta}}
\end{align*}
\item Deduce simple bounds on $f(n)$ using $\Theta$-notation.
\begin{align*}
\text{Case: }\theta = 2 &\\
f(n) &\in \Theta(\displaystyle\sum_{i=0}^{\log_2 n} 2^{2i-\theta i} n^{\theta})\\
\text{Case: } \theta = 2&\\
f(n)&\in \Theta(n^{\theta}\frac{1-\bigg (2^{2-\theta}\bigg )^{log_2 n + 1}}{1-2^{2-\theta}})
\end{align*}
\end{enumerate}
%%%%%%%%%%%%%%%%%%%%%%%%%%%%%%%%%%%%%%%%%%%%%%%%%%%%%%%%%%%%%
\subsection{[4+4=8 marks]}
\begin{enumerate}[(a)]
\item Prove that the following code fragment will always terminate.
\begin{verbatim}
s := n  // n is an integer
while (s>1)
   if (s is even)
      s := s/2
   else
      s := s+1
\end{verbatim}
Since this function will never execute s = s+1 multiple times in a row (after one incrementation it become even and therefore is divided), but can have s = s/2 multiple times in a row (if you chance upon a power of 2) we can say that s/2 will happen more often than s+1 we can say that the function describing the inside of the loop is a decreasing function. And since this function is decreasing we can say that s will eventually reach a value below or equal to 1 terminating the loop.
\item Prove that its running time is $O(\log n)$.\\
$\displaystyle\sum_{i=1}^{\log_2 n}c = c\log_2 n \enspace \in O(\log n)$\\
\end{enumerate}

%%%%%%%%%%%%%%%%%%%%%%%%%%%%%%%%%%%%%%%%%%%%%%%%%%%%%%%%%%%%%
\subsection{[4+4=8 marks]}
Consider the following (not necessarily the best) implementation of an algorithm that finds the largest element in the array. Give the best case (4 marks)
and worst case (4 marks) running time of the function, using $\Theta$-notation. Justify your answer.
\begin{verbatim}
function max-element(A[1..n])
    if n = 1 do
        return A[1]
    else if A[1] > max-element(A[2..n]) do
        return A[1]
    else return max-element(A[2..n])
\end{verbatim}
\begin{enumerate}[(a)]
\item Best case: $\Theta(n)$
If the array is already has the largest element at its start is returns that, then compares it to each following element which gives:\\
\begin{align*}
\displaystyle\sum_{i=1}^n c&=cn\\
cn &\in \Theta(n)\\ 
\end{align*}
\item Worst case: $\Theta(n^2)$
If the array is arranged smallest to largest elements the program will compare this to the next element and since its smaller than the next element recalls the function on the array minus that element resulting:\\
\begin{align*}
\displaystyle\sum_{i=0}^n\sum_{j=i}^n c&= \displaystyle\sum_{i=0}^n (n-i+1)c\\
&= \displaystyle\sum_{i=0}^n nc - \displaystyle\sum_{i=0}^n ic + \displaystyle\sum_{i=0}^n c\\
&= n(n+1)c - c(n+1)\frac{n}{2} + (n+1)c\\
&= n(n+1)c\\
&=cn^2 + cn\\
&= cn^2\\
cn^2 &\in \Theta(n^2)
\end{align*}
\end{enumerate}




\end{document}
