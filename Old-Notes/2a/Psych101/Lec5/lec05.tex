\documentclass[12pt]{article}
\usepackage{parskip}
\usepackage[margin=.6in]{geometry}
\title{Consciousness and Dreams}
\author{Dan Reynolds}
\begin{document}
\maketitle
\section*{The Brain and Consciousness}
\textbf{Consciousness} is the awareness of ourselves and our environment
\subsection*{Cognitive Neuroscience}
the interdisciplinary study of the brain activity linked with our mental process. Takes the first small step by relating specific brain states to conscious experiences. 

\subsection*{Dual Processing}
Growing evidence says that we have two kinds, each supported by its own neural equipment. 

Conscious right brain and more intuitive left brain. Perception, memory, thinking, etc all operate on two levels - conscious, deliberate high road and an unconscious, automatic low road. This is called \textbf{dual processing}.

A \textit{visual perception track} enables us to create the mental furniture that allows us to think about the world.

A \textit{visual action track} guides our moment-to-moment actions. 

\subsection*{Selective Attention}
Conscious awareness focuses, like a flashlight beam, on only a very limited aspect of all that we experience. If two people talking into each ear and you have to say what the person in the left says as they say it, you cannot recall what person on right said.

\textbf*{Inattentional blindness} is not noticing something as a result of focus elsewhere.

People also often exhibit a blindness to change. Clothing colour, objects may change, but viewers do not notice. This is \textbf{change blindness.}

\textbf{Change deafness} also occurs, as well as \textbf{choice blindness}.

\section*{Sleep and Dreams}
\subsection*{Biological Rhythms and Sleep}
Our bodies roughly synchronize with the 24-hour cycle of day and night through a biological clock called the \textbf{ciradian rhythm}. Body temp rises as morning approaches, peaks during the day and dips in the early afternoon. 

Bright light tweaks the clock by activating light-sensitive retinal proteins. These proteins control the clock by triggering the brain's \textit{suprachiasmatic nucleus} or SCN, a pair of 20000 clusers in the hypothalamus.

This decreases production of the sleep-inducing hormone melatonin.

\subsection{Sleep Stages}
About every 90 minutes, we pass through a cycle of five distinct sleep stages. Stage 1 may involve fantastic images, resembling \textbf{hallucinations}, sensory experiences that occur without a sensory stimulus. May have a sense of falling or of floating weightlessly.

These \textbf{hypnagogic}, of the period immediately preceding sleep, accompanied by drowsiness, sensations may later be incorporated into memories. 

Stage 3, then 4 is deep sleep, in this stage brain emits \textbf{delta waves. } These two slow-wave stages last for 30-min, during which it is hard to be woken. 

\subsection*{REM Sleep}
Rather than continuing in deep slumber, returning through stage 3 and 2, enter REM sleep. Brain waves become rapid, sharp, heart rate rises, breathing becomes rapid and irregular, every half-minute eyes dart around in a momentary burst of activity. 

Except during scary dreams, genitals become aroused during REM sleep. Hard to awaken, sometimes calld paradoxical sleep, with the body aroused but externally calm.

Sleep cycle repeats every 90min. 

\subsection{Effects of Sleep Loss}
Loss of sleep decreases productivity and increases the tendency to make mistakes, as well as irritability and fatigue. A large sleep debt can make you dumb, and fat as a result of the increase in the hunger-arousing hormone ghrelin and a decrease in its hunger supressing partner, leptin. 

Increases the stress hormone cortisol. 

\subsection*{Sleep Theories}
Sleep protects, species' sleep pattern tends to suit its ecological niche. Sleep helps us recuperate. It restores brain tissue, restores and rebuilds fading memories of the day's experience. Sleep increases creative thinking. 

To think smart and see connections, it is often beneficial to sleep on it. 

\subsection*{Sleep Disorders}
10 percent of people complain of insomnia. 

\textbf{Narcolepsy} sufferers experience periodic, overwhelming sleepiness. \textbf{Sleep apnea} occurs when a person stops breathing during sleep. After an airless minute, decreased blood oxygen arouses tthem and they wake up enough to snort in air for a few seconds.

\subsection*{Dreams}
REM dreams, hallucinations of the sleeping mind, are vivid. The dreams of REM sleep are so vivid they may be confused with reality.

The storyline, or \textbf{manifest content} of the dream, incorporates traces of previous days' nonsexual experiences and preoccupations.

\subsection*{Why we dream}
To satify our own wishes, Freud believed dreams provide a psychic safety valve that discharges otherwise unacceptable feelings. According to Freud, a dream's manifest content is a censored, symbolic version of its \textbf{latent content}, which consists of unconscious drives and wishes that would be threatening if expressed directly.

Although most dreams have no overt sexual imagery, Freud believed that most adult dreams can be tracked back by analysis to erotic wishes. 

Dreams file away memories, sifting, sorting and fixing the day's experiences in our memory. 

Dreams develop and preserve neural pathways. Some researchers speculate that dreams may also serve a physiological function, providing the sleeping brain with periodic stimulation. 

To make sense of neural static. Dreams erupt from neural activity spreading upward from the brainstem. Dreams are the attempt to make sense of random neural activity.

To reflect cognitive development, some dream researchers dispute both the Freudian and activation-synth theories, seeing dreams as part of brain maturation and cognitive development.

\subsection*{Dream Theories:}
\begin{enumerate}
\item Freud's wish-fulfillment: Dreams provide a psychic safety valve, expressing otherwise unacceptable feelings; contain manifest, remembered, content and a deeper layer of latent content - hidden meaning. 

Lacks any scientific support, dreams may be interpreted in many ways.

\item Information-processing: Dreams help us sort out the days's events and consolidate our memories. 

But how then can we dream about things we have not experienced?

\item Physiological function: Regular brain stimulation from REM sleep may help develop and preserve neural pathways. This may be true, but it does not explain why we experience meaningful dreams.

\item Activation-synthesis: REM sleep triggers neural activity that evokes random visual memories, which our sleeping brain weaves into stories. The individual brain is weaving the stories, which still tell us something about the dreamer.

\item Cognitive Development: Dream content reflects dreamers' cognitive development - their knowledge and understanding. Does not address the neuroscience of dreams.

\end{enumerate}

Although sleep researchers debate dreams' function, they agree we need REM sleep. When finally allowed to sleep undisturbed, people sleep like babies, called the \textbf{REM rebound}. 

We are reminded of a basic principle, \textit{biological and psychological explanations of behaviour are partners, not competitors.}

\section*{Hypnosis}
After several minutes of \textit{hypnotic induction}, you may experience hypnosis. Power resides in the subject's openness to suggestion. Anyone who can turn attention inward and imagine is able to experience some degree of hypnosis, as that is what hypnosis is. An authoritative person in a legitimate context can induce people - hypnotized or not, to perform unlikely acts.

\subsection*{Therapeutic Properties}
Hypnotherapists try to help patients harness their own healing powers. \textbf{Posthypnotic suggestions} have helped alleviate headaches, asthma, and stress-related skin disorders. 

Hypnosis has helped in the aid of obesity, and several other ailments. It can alleviate pain.

Hilgard believed that hypnosis not only involved social influence, but a special state of \textit{dissociation }, a split between different levels of consciousness. Viewed hypnotic dissociation as a vivid form of everday mind splits. Similar to doodling while listening to a lecture. 

Hypnotic pain relief may also result from another form of dual processing, \textit{selective attention}, as when an injured athlete feels nothing until after the game.

It may not block sensory input, but block attention to stimuli. 

Two major views, \textbf{social influence} and \textbf{divided consciousness}.

\section*{Drugs and Consciousness}
There is little dispute that some drugs alter consciousness. \textbf{Psychoactive drugs} are chemicals that change perceptions and moods through their actions at neutral synapses. 

\subsection*{Dependence and Addiction}
Continued use of alcohol and other psychoactive drugs produces a tolerance. As the user's brain adapits its chemistry to offset the drug effect, called a \textit{neuroadaptation}, the user requires larger and larger doses to experience the same effect. 

Users who stop taking psychoactive drugs may experience the undesirable side effects of withdrawal. The user feelds pain and cravings indicating chemical dependence, and even psychological dependence. 

An \textbf{addiction} is a compulsive craving for a substance despite adverse consequence and often with physical symptoms such as aches, nausea and distress following sudden withdrawal. 

\subsection*{Psychoactive Drugs}
Alcohol:
\begin{enumerate}
\item Lowers inhibitions, slows neural processing, disrupts memory formation and reduces self-awareness. 

\item Disinhibition: It is an equal-opportunity drug, it increases harmful tendencies as when angered people become aggressive, increases helpful tendencies like when tipsy restaurant owners leave large tips.

\item Slowed Neural Processing: Low doses of alcohol relax the drinker by slowing sympathetic nervous system activity. Potent sedative if sleep deprived. 

\item Memory Disruption: Alcohol can also disrupt the processing of recent experiences into long-term memories. 

\item Reduced Self-Awareness and Self-Control:

Alcohol also reduces self-awareness. People want to supress their awareness of failures or shortcomings by drinking. 

\item Expectancy Effects:

As with other psychoactive drugs, alcohol's behavioural effects stem from its alteration of brain chemistry but also from the user's expectations. When people believe that alcohol affects their behaviour, it more likely will.

\item Alcohol + Sex = Perfect Storm, alcohol often leads to sexual situations. 

\end{enumerate}

\subsection*{Barbiturates}
The barbiturate drugs, or \textit{tranquilizers,} mimic the effects of alcohol. Because they depress nervous system activity, barbiturates such as Amytal are sometimes prescribed to induce sleep or reduce anxiety.

\subsection*{Opiates}
The \textbf{opiates}, opium and its derivatives, morphine and heroin also depress neural functioning. Pupils constrict, breathing slows, and lethargy ensues. 

For this short-term pleasure, the user may pay the price of long-term cravings and a need for progressively larger doses. 

Brain stops producing its own endorphins. 

\subsection*{Stimulants}

Stimulants, such as caffeine and nicotine temporarily excite neural activity. People use these to stay awake, lose weight, or boost mood or athletic performance. 

This category of drugs also includes amphetamines, and ecstacy, or methamphetamine. Can be addictive resulting in fatigue, headaches, irritability, and depression.

\subsection*{Methamphetamine}
Can include extended periods of heightened energy and euphoria. The drug triggers the release of the neurotransmitter dopamine, stimulating the brainc cells that enhance energy and mood.

May reduce baseline dopamine levels, resulting in increased irritability, insomnia, depression and violent outbursts. 

\subsection*{Caffeine}

Used regularly, stimulating effects lessen, and discontinuing heavy caffeine intake often produces withdrawal symptoms.

\subsection*{Nicotine}
Adolescents often take up smoking, often having friends who smoke. As addictive as heroin or cocaine, smoker becomes dependent. Triggers release of epinephrine and norepinephrine, which diminishes apptetite and boosts alertness and mental efficiency. 

Stimulates the Central Nervous System to release neurotransmitters that calm anxiety and reduce sensitivity to pain. Stimulates the release of dopamine and opioids. 

\subsection*{Cocaine}
Cocaine use offers a fast track from euphoria to crash. Gives a rush of euphoria that depletes the brain's supply of dopamine, serotonin and norepinephrine. Wears off quickly, leading to a crash of agitated depression.

\subsection*{Ectasy}
Officially MDMA, a stimulant and mild hallucinogen. As an amphetamine derivative, triggers dopamine release. Major effect is releasing stored serotonin and blocking its reabsorption. 

Produces feelings of intimacy, destroys serotonin producing neurons and permanently deflating mood and impairing memory. 

\subsection*{Hallucinogens}
Distort perceptions and evoke sensory images in teh absence of sensory input. LSD is similar to a subtype of serotonin, the emotions of an LSD trip vary from euphoria to detachment to panic.

The user's current mood and expectations color the emotional experience. 

\subsection*{Marijuana}
Major component is THC, relaxes, disinhibits, and may produce a euphoric high. Amplifies sensitivity to colors, sounds, tastes, and smells. 

Produces a lingering effect, regular users need less than occasional as it lingers in the system. 

\subsection*{Near Death Experiences}
Relay of old memories, out-of-body sensations, visions of tunnles or funnels and bright lights or beings of light. 

Oxygen deprivation can produce such hallucinations, complete with tunnel vision. As oxygen deprivation turns off the brain's inhibitory cells, neural activity increases in the visual cortex. 

In the oxygen-starved brain, the result is a growing patch of light, which looks much like what you would see as moved through a tunnel. 

Summary of Psychoactive Drugs:

\begin{enumerate}
\item Alcohol: Depressant: Initial high followed by relaxation and disinhibition. Results in depression, memory loss, organ damage, impaired reactions
\item Heroin: Depressant: Rush of euphoria, relief from pain. Depressed physiology, agonizing withdrawal. 
\item Caffeine: Stimulant: Increased alertness and wakefulness. Anxiety, restlessness, and insomnia in high doses; uncomfortable withdrawal.
\item Methamphetamine: Stimulant: Euphoria, alertness, energy. Irritability, insomnia, hypertension, seizures.
\item Cocaine: Stimulant: Rush of euphoria, confidence, energy. Cardiovascular stress, suspiciousness, depressive crash. 
\item Nicotine: Stimulant: Arousal and relaxation, sense of well-being. Heart disease, cancer. 
\item Ectasy: Stimulant, mild hallucinagen: Emotional elevation, disinhibition. Dehydration, overheating, depressed mood, impaired cognitive and immune functioning. 
\item Marijuana: Mild hallucinagen: Enhanced sensation, relief of pain, distortion of time, relaxation. Impaired learning and memory, increased risk of psychological disorders, lung damage from smoke.

\end{enumerate}

\subsection{Biological Influences}
Heredity influences can are present in some aspects of abuse problems. 

\section*{Sensation and Perception}
\textit{Prosopagnosia} - face blindness. 

Sensation and perception blend into one continuous process. \textbf{Bottom up processing} begins at the sensory receptors and works up to higher levels of processing in the brain.

\textbf{Top-down processing} is processing beginning with our experience and expectations, and work down to the senses. 

\subsection{Thresholds}
Our senses are very limited compared to those of some other animals and all the possible things we could detect. \textbf{Psychophysics} examines the effect what we can detect has on our psychological experience.

The \textbf{absolute threshold} is the minimum stimulation necessary to detect a particular stimuli, be it light, sound, pressure, etc. Vary with person and age, generally inversely proportional to age.

\subsection*{Signal Detection}
Detecting a weak stimulus depends not only on signal's strength, but on experience, expectations, motivation, alertness.

\textbf{Signal Detection Theory} predicts when we will detect weak signals. Signal detection theorists determine why people react differently to the same stimuli.

\subsection*{Subliminal Stimulation}
Subliminal literally means below threshold stimuli, stimuli that is only detected some of the time, but is below the absolute threshold.

Under certain conditions, weak stimuli can affect us without our notice. An invisible image or word can \textbf{prime} a response to a later question.

Typically the primer is very quick, replaced with a \textit{masking stimulus} that interrupts the brain's processing before conscious perception. 

Sometimes we feel what we do not know and cannot describe. Ultimately, much of our info processing occurs automatically, outside the scope of our conscious mind.

\subsection{Difference Thresholds}
Also called the \textit{just noticeable difference,} is the minimum difference that a person can detect between any two stimuli half the time. Such as the ability to detect a change of 1 ounce to a 10 ounce weight. 

\textbf{Weber's Law:} For their difference to be perceptible, two stimuli must differ by a constant proportion, not a constant amount. If can detect change of 1 ounce in 10 ounces, can detecct 10 ounces in 100 ounces, not 1 ounce in 100 ounces.

\subsection*{Sensory Adaptation}
Our diminishing sensitivity to unchanging stimuli. After constant exposure to a stimulus, nerve cells fire less frequently. 

We perceive the world not exactly as it is,but as it is useful for us to perceive it. 

\section*{Classical Conditioning}
Although \textbf{associative learning} has been around for a while, was exemplified by \textbf{Ivan Pavlov}. He explored \textbf{classical conditioning} and laid the foundation for John B. Watson's ideas. 

In search for laws underlying learning, Watson urged colleagues to discard inner thoughts, feelings, and motives. The science of psychology should instead study how organisms respond to stimuli in their environments.

Introspection should form no essential part of its methods, rather behaviour should be examined. Psychology should be an objective science based on observable behaviour. This view was called \textbf{behaviourism}. 

Watson and Pavlov both shared a disdain for mentalistic concepts such as consciousness.

\subsection*{Pavlov's Experiments}
Driven by a lifelong passion for research, received a medical degree and spent two decades studying digestive system. 

\subsection*{Dogs Salivate}
He realized that putting food in a dog's mouth caused the animal to salivate. Moreover, the dog began not only at taste, but sight, or sound. 

After placing food in dog's mouth, played a tone. After pairings, dog would salivate with the sound. 

Because salivation in response to food in the mouth was unlearned, he called it an \textbf{unconditional response (UR)} to the food, which was the \textbf{unconditional stimulus (US}. Salivation in response to tone was conditional upon the dog's learning the association, and was therefore called a \textbf{conditional response (CR)}. The previously neutral tone stimulus now triggered the conditional salivation called the \textbf{conditional stimulus. (CS)}. 

\subsection*{Acquisition}
To understand the \textbf{acquisition}, or initial learning, conducted additional experiments. US must be preceded by CS. 

\textbf{higher-order conditioning}, a new neutral stimulus, NS, can become a new conditioned stimulus, CS. All that is required is for it to become associated with a previously conditioned stimulus. If a tone regularly signals food and produces salivation, a light that becomes associated with a tone can as well.

\subsection*{Extinction and Spontaneous Recovery}
If the CS occurs repeatedly without the US, \textbf{extinction} takes place and the CS no longer elicits a CR. \textbf{Spontaneous Recovery} occurs when there is a reappearance of the CR after a pause, suggesting that extinction supresses the CR rather than eliminating it.

\subsection*{Generalization}
Pavlov and students noticed that a dog conditioned to the sound of one tone also responded somewhat to the sound of a different tone that had never been paired. This tendency to respond to a stimuli similar to the CS is called \textbf{generalization}.

\subsection{Discrimination}
Pavlov's dogs also learned to respond to the sound of a particular tone and not other tones. Discrimination is the learned ability to distinguish between a conditioned stimulus and other irrelevant stimuli.

\subsection*{Cognitive Processes}
Early behaviourists believed that rats and dogs learned behaviours could be reduced to mindless mechanisms. Robert Rescorla and Allan Wagner showed that an animal can learn the \textit{predictability} of an event. 

The animal learns an expectancy, an awareness of how likely it is that the US will occur.

\subsection*{Biological Predispositions}
Pavlov and Watson believed that basic laws of learning were universal. This was not true, however, and an animal's capacity for conditioning is constrained by its biology. 

John Garcia was among those who challenged the idea that animals learned equally well. He noticed that rats avoided drinking water if they consequently became sick, but using NS such as light could not become a CS despite the same illness US. 

If humans become ill after eating, can developed an aversion to certain foods, but music or sights that were also present are still tolerated.

Organisms learn associations that help them adapt. 

Support Darwin's principle that natural selection favors traits that aid survival. 

\subsection*{Pavlov's Legacy}
Classical conditioning is a basic form of learning. Many responses to many stimuli can be classically conditioned in many organisms. It is one way which virtually all organisms learn to adapt to their environment.

Pavlov's principles of classical conditioning are now being used to improve human health and well-being. Former drug users are advised to not be exposed to stimulus associated with their former addiction. When a particular taste accompanies a drug that influences immune response, the taste itself may come to produce an immune response.

\subsection*{John Watson and Little Albert}
John Watson, a behaviourist, taught Little Albert to fear rats by banging a metal sound each time he saw one. Albert would cry at the sight of the rat, as well as similar objects.

It is possible that we are each a walking repository of conditioned emotions. 

\section*{Operant Conditioning}
A type of associative learning, different to classical conditioning. While classical conditioning forms associations between stimuli, a CS and the US it signals, it also involves \textbf{respondent behaviour} - actions that are automatic responses to a stimulus. 

In \textbf{operant conditioning}, organisms associate their own actions with consequences. Actions followed by reinforcers increase those followed by punishers decrease. Behaviour that \textbf{operates} on the environment to produce rewarding or punishing stimuli is called operant behaviour. 

Learning association between events it cannot control = \textbf{Classical Conditioning}

Learning association between its behaviour and resulting events = \textbf{Operant Conditioning}

\section*{Skinner's Experiments}
B.F. Skinner was a college English major and aspiring writer who entered psych. Became behaviourism's most influential and controversial figure. Elaborated Thorndike's \textbf{law of effect:} Rewarded behaviour is likely to recur.

Skinner developed a behavioural technology that revealed principles of behavior control. Taught pigeons to play ping-pong.

Designed an \textbf{operant chamber}, or Skinner Box, that allowed animal to press or peck to release food and a device that records these responses. 

\subsection*{Shaping Behaviour}
Skinner used \textit{shaping}, a procedure in which reinforcers such as food gradually guide an animal's actions toward a desired behaviour. Using successive approximations, can get an animal to get progress closer to desired behaviour through successive rewards.

Can train them to discriminate between faces, etc depending on the rewards.

\subsection*{Types of Reinforcers}
A reinforcer is any event that strengthens a preceding response. Can be a tangible reward, such as food or money, or praise or attention. 

Reinforcers are person specific and do not always have to be positive. \textbf{Negative reinforcement} strengthens a response by reducing or removing something undesirable. Negative reinforcement is not punishment, but the removal of a negative versus an addition of a positive. 

\textbf{Primary Reinforcers} - getting food when hungry or having a headache go away are unlearned. \textbf{Conditional Reinforcers}, or secondary reinforcers, get their power through learned association with primary reinforcers. If a rat in a Skinner box learns tha a light reliably signals food, the rat will work to turn on the light.

\textbf{Immediate and Delayed Reinforcers:} Typically for animals, a reward must immediately follow an action in order for that action to become associated with the reward. For humans, reinforcers can be delayed, such as a university degree, or a mutual fund.

\subsection*{Reinforcement Schedules}
Most examples have assumed \textbf{continous reinforcement:} reinforcing the desired response every time it occurs. Results in rapid adoption and extinction. Partial or \textbf{intermittent reinforcement} schedules, in wich responses are sometimes reinforced and other times not, produces greater \textit{resistance to extinction} than is found with continuous reinforcement.

Hope is eternal with intermittent reinforcement, such as occassionally giving in to a child's tantrums. 

Four different types of reinforcement schedules: 
\begin{enumerate}
\item \textbf{Fixed-ratio Schedules:} reinforce behaviour after a set of numerical reponses. Once conditioned, animal will pause only briefly to receive reward and then return to high rate of responding.

\item \textbf{Variable-rate Schedules:} provide reinforcers after an unpredictable number of responses. Includes gambling, fly-fishing, like the fixed ratio schedule, the variable-ratio schedule produces high rates of responding, because reinforcers increase as the number of responses increases. 

\item \textbf{Fixed-interval schedules} reinforce the first response after a fixed period of time. Like people checking more frequently for the mail, pigeons on a fixed- interval schedule peck a key more frequently when the anticipated time for reward draws near.

Not steady anymore, but start stop pattern.

\item \textbf{Variable-interval Schedules:} reinforce the first repsonse after varying time intervals. There is no knowing when the waiting may be over.
\end{enumerate}

\subsection*{Punishment}
Reinforcement increases a behaviour, punishment does the opposite. A punishment decreases the frequency of a preceding behaviour.

Positive Punishment: administer an aversive stimulus, a parking ticket.

Negative Punishment: withdraw a desirable stimulus, withold sex.

Physical punishment, especially on children, has negative effects:

\begin{enumerate}
\item Punished behaviour is supressed, not forgotten. 
\item Punishment teaches discrimination. It becomes okay to misbehave elsewhere, just not by parent
\item Punishment can teach fear. The child may associate fear not only with the undesirable behaviour but also with the person who delivered the punishment or the place it occurred. Fear a teacher, fear school.
\item Physical punishment may increase aggressiveness by modeling aggression as way to cope with problems. Many aggressive delinquents and abusive parents come from abusive families. 
\end{enumerate}

\subsection*{Extending Skinner's Understanding}
Skinner died resisting the growing belief that cognitive processes, thoughts, perceptions and expectations, have a necessary place in the science of psychology and in understanding conditioning. 

Rats exploring a maze develop a \textbf{cognitive map}, a mental representaion of the maze. During their explorations, the rats have experienced \textbf{latent learning}, learning that becomes apparent only when there is some incentive to demonstrate it. 

There is more to learning than associating a response with a consequence; there is also cognition. 

\subsection*{Intrinsic Motivation}
Promising people a reward for a task they already enjoy can backfire. If they must be paid for it, the work is probably not doing for its own sake and it becomes less enjoyable.

Excessive rewards undermine \textbf{intrinsic motivation}, the desire to perform a behaviour based solely for its own sake. \textbf{Extrinsic motivation} is the desire to behave in a certain way to receive external rewards or avoid threatened punishment.

Doing these readings is extrinsically motivated, we must do so or we will do poorly. If we liked the reading becomes we found them interesting then we would be intrinsically motivated. 

\subsection*{Biological Predispositions}
An animal's natural predispositins constrain its capacity for operant conditioning. Using food as a reinforcer, can easily tell a pigeon to peck for food, or flap to get away, but pecking to get away or flapping for food is not easily done.

Biological constraints predispose organisms to learn associations that are naturally adaptive.

Skinner insisted external influences shaped behaviour. Urged people to use operant principles to influence behaviour. Skinner's critics said he dehumanized people by neglecting their freedom and by seeking to control their actions.

Comparison of Classical vs Operant Conditioning:

Classical:

\begin{enumerate}
\item Basic Idea: Organism learns associations between events it doesn't control.
\item Response: Involuntary, automatic
\item Acquisition: Associating events, CS announces US
\item Extinction: CR decreases when CS is repeatedly presented alone
\item Spontaneous Recovery: The reappearance, after a rest period, of an extinguished CR
\item Generalization: The tendency to respond to stimuli similar to CS
\item Discrimination: The learned ability to distinguish between a CS and other stimuli that do not signal a US
\item Cognitive Processes: Organisms develop expectation that CS signals the arrival of US
\item Biological Predispositions: Natural predispositions constrain what stimuli and responses can be easily associated

\end{enumerate}

Operant:

\begin{enumerate}
\item Basic Idea: Organism learns associations between its behaviour and resulting events. 
\item Response: Voluntary, operates on environment
\item Acquisition: Associating response with consequence
\item Extinction: Responding decreases when reinforcement stops
\item Spontaneous Recovery: The reappearance after a resting  period
\item Generalization: Organism's response to similar stimuli is reinforced
\item Discrimination: Organisms learn that certain responses will be reinforced
\item Cognitive Processes: Organisms develop expectation that a response will be reinforced or punished; they also exhibit latent learning without reinforcement
\item Biological Predispositions: Organisms best learn behaviours similar to their natural behaviours, unnatural behaviours instinctively drift backward towards natural ones
\end{enumerate}

\section*{Learning By Observation}
Higher animals, especially humans, can learn without direct experience, through \textbf{observational learning}. A child sees a parent do something and does it for him or herself.

Learning by observation and imitation is called \textbf{modeling}. 

\subsection*{Mirrors in the Brain}
When humans went to eat infront of a money, it triggered a response in the monkey as if it was eating. Found a new type of neuron, \textbf{mirror neurons}. When a monkey grasps, holds or tears something, or sees another monkey doing so, the mirror neurons fire.

As humans observe another's actions, the brain generates an inner simulation, enabling us to experience the other's experience within ourselves. Give rise to children's empathy and their ability to infer another's mental state, an ability known as \textit{theory of mind.} 

Even fiction or movies may simulate in our brains the experience described.

\subsection*{Bandura's Experiment}
A preschooler works on a drawing, then watches an adult beat on a toy. When angered, the child is left alone with several toys, but is likely to pick the toy the adult chose and express its anger in the same way. 

Observing the adult's outburst lowered the child's inhibitions. The child inmitated the very acts he/she had observed and used the very words they had heard.

Prosocial, positive and helpful models can also be beneficial. 

Antisocial effects can be passed through genes, or environment. Violence on television does lead to aggressive behaviour by children and teenagers who watch the programs. Watching cruelty fosters indifference.



\end{document} 
