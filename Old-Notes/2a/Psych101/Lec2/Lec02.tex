\documentclass[12pt]{article}
\usepackage{parskip}
\usepackage[margin=.6in]{geometry}
\title{Developmental Psychology}
\author{Dan Reynolds}
\begin{document}
\maketitle
\section*{Industrial Revolution}
He talked about the setting, times were bad, nothing testable blah blah.
\section*{Affective Development: Attachment Theory}
Two Theories of Attachment:
\subsection*{Psychodynmaic:}
Primary biological need, \textit{sex}. Secondary Emotional need, \textit{attachment.} Freud supported this, he believed it was about sexual gratification and attachment followed from that.
\subsection*{Behaviourist:}
Primary biological need, \textit{survival}. Secondary emotional need, \textit{attachment.} The child gets conditioned to associate a person with good things. Now there is a learned preference for a certain individual. It began as a survival mechanism and developed into emotional love. It is however not love, it is a condition.
\subsection*{Harry Harlow and His Monkeys, 1905 - 1981}
A behaviourist, he worked with animals. Mother's did not do a very good job of mothering in captivity. They tried taking the animals from the mother immediately after birth, bottle feeding them. The animals fell in love with whatever they had, blankets, etc. They had nothing but these objects and became attached to them. Believed they were pre-wired to form a relationship with the closest thing they have. An infant is predisposed to form an emotional bond with a caregiver. What if \textit{attachment} was not a secondary need but a \textit{primary} need.

He conducts one of the most famous studies in history. He has two surrogate mothers provided for the animals. He divides them into two grooups, half with a cloth mother, and half with a wire mother. The wire mother gives milk, therefore Freud would say they should prefer the wired mother. The behaviourists would predict it is the wire mother giving sustenance, even heat through the wires, so it should prefer the wire mother from this perspective as well.

The infants are then given both mothers and choose. It prefers the cloth mother, regardless of the situation. They wanted the cloth mother, there was something about the softness, clingability that allowed it to become a surrogate. We have a primary need to form emotional bonds with members of our species. 

They don't do well when not allowed to be with others of their own species. It is not that they just don't do well, the actually die. It is a primary need to form an emotional connection with members of our own species. It is true that love is necessary. 

This love reaction is often triggered by an infant's cry, triggering a response in the lymbic system in the adult, this \textbf{separation cry} almost forces the adult to act.

\subsection*{Mary Ainsworth, 1913 - 1999}
She wanted to know if the same results would be triggered by work on humans. Mary creates the scenario called the \textbf{strange situation.} There are two chairs in a room, one for the mother and one for the stranger. There are toys in the room, mother is told to remain in the chair. Humans behaved similarly, tightening their grip on mother. Most children begin to get down from mother cautiously. 

 When stranger enters, child goes right to mother. When mother leaves, baby cries or goes to chair of mother. The most important part of love is trust. In order to have true attachment, the child has to trust. 

\subsection*{Attachment Patterns}
\begin{itemize}
\item Secure
\item Insecure/Anxious
\item Ambivalent (Resistant) - infant wants love but is not confident that it is going to occur. 
\item Avoidant - usually goes right to toys, minimal interaction between mother, even when stranger is present.
\item Disorganized (Disoriented) 

\end{itemize}

Research shows it does not have to be the mother, it is whoever is nearby and comes forward and cares for the children. An infant can have differing attachments with different adults, such as  mother, father or sibling. Some children are very affectionate, kind, others are less engaging to adults, changing the nature of the attachment. 

Trust forms, a connection is established. The child is then free to explore, knowing that they are cared for. 

\subsection*{Conditions:}
\begin{enumerate}
\item Positive and loved - Secure, everyone is okay,
\item Unloved and rejected - Avoidant, i'm okay, you're not.
\item Angry and Confused - Ambivalent, you're okay, i'm not.
\end{enumerate}

The results show that in North America, most children feel secure, as well as in Japan, Germany is avoidant. In German culture, it is a preferred attachment style. Many Germanic cultures emphasize the independence and autonomy of a child, to instill that they can handle whatever comes along. What appears to be an attachment style is not mistrust, it is just a different culture.

In Asian cultures such as Japan, there is no avoidant style. Members of the family immediately fill in gaps in attachment in the life of the child. 

\subsection*{Patterns and Peers}
\begin{enumerate}
\item Secure/Secure - Smooth and Reciprocal
\item Secure/Ambivalent - Smooth, secure, tolerant, caring
\item Secure/Avoidant -  Aggressive, secure, intolerant
\item Ambivalent/Ambivalent - Hot and Cold
\item Ambivalent/Avoidant - Dominant/Submissive
\item Avoidant/Avoidant Power struggle, mistrust
\end{enumerate}

As NA people age, we all gravitate towards a secure relationship.

\section*{Albert Bandura, 1925 -present}
Studies behaviourism, goes to Stanford, comes to Waterloo, widely cited. One way to learn is by observing associations. Trial and error learning. Children aquire a massive repertoire of info very quickly. A lot of our learning is social in nature, we learn from eachother's experiences as well as our own. We don't have to learn everything from scratch ourselves, this is called imitative or vicarious learning as well as model learning. 

\subsection*{Attention}
\begin{itemize}
\item Characteristics of observer
\item Characteristics of model
\item Characteristics of event
\end{itemize}

\subsection*{Retention}
\begin{itemize}
\item Cognitive Maturity
\end{itemize}

\subsection*{Motoric Reproduction}
\begin{itemize}
\item Physical Maturity
\item Prerequisite SKills
\end{itemize}

\subsection*{Motivation}
\begin{itemize}
\item Outcome Expecation
\item Efficacy expectations
\item Incentive Value
\end{itemize}

\section*{Modules 13-15}
\subsection*{Developmental Psychology}
\begin{enumerate}
\item Nature/nurture: how does genetic inheritance and experience influence our dvelopment
\item Continuity/stages: is developmental a gradual, continuous process or is it a sequence of separate stages
\item Stability/change: do early personality traits persist through life or do we become a different person from aging
\end{enumerate}

\textbf{Habituation} is a decrease in response with repeated stimulation. The more often stimulation is presented, the less frequently it is used. The seeming boredom with stimuli was used to evaluate what infants see, hear, feel, etc. This was called the \textbf{novelty-preference procedure}, introducing new things intermixed with many old.

We are born preferring sights and sounds that facilitate social responsiveness.

Ultimately, developmental psychologists study the physical, mental, and social changes throughout the life span, beginning at conception. 

\section*{Brain Development}

The developing brain overproduces neurons, after birth the neural networks that enable walking, talk, etc grow wildly. Fiber pathways supporting language and agility proliferate into puberty until a \textit{pruning process} shuts down excess connections, strengthening others. 

\subsection*{Maturation}
Deprivation or abuse can retart development, ample parental experiences of talking adn reading will sculpt neural connections. Maturation sets the basic course of development to be adjusted by experience.

\subsection*{Motor Development}

The developing brain enables physical coordination. Genes play a major role in motor development. Identical twins sit up at nearly the same day. Maturation including the rapid development of cerebellum allows us to walk at age 1. 

\subsection*{Maturation and Infant Memory}
\textit{Infantile amnesia} is the condition where children beneath the age of 3 are unable to recall otherwise memorable occurrences. Earliest conscious memory is 3.5 years. 

The nervous system is capable of remembering what the brain cannot, children who could not recall classmates mentally exhibited increased body perspiration. 

\section*{Cognitive Development}
\subsection*{Cognitive Development}
\textbf{Cognition} refers to all the mental activities associated with thinking, knowing, remembering and communicating.

Developmental psychologist \textit{Jean Piaget} was intrigued by children's wrong answers, eventually showing that children reason differently than adults and that there mind is fundamentally different from their elders. 

Concluded that a child's mind develops through a period of stages. The maturing brain builds \textit{schemas}, concepts or mental molds into which we pour our experiences. By adulthood we have built countless schemas ranging from animals to love.

Piaget proposed two concepts. Firstly, we \textit{assimilate} new experiences, interpreting them in terms of our current experiences/built schemas. As we adjust with the world, however, we adjust and \textbf{accomodate} our schemas to incorporate info provided by new experiences. Broad views are quickly changed by narrowing categories and developing new schemas as necessary.

He believed that as children construct their understanding of the world, they experience spurts of change, followed by cognitive stability plateaus. These plateaus were forming stages and are listed below:

\subsection*{Stages of Cognitive Development}
\begin{itemize}
\item Birth to 2: \textit{Sensorimotor}, experience the world through senses and actions, looking, hearing, touching, etc. Developmental Phenomena: Object Permanence, Stranger Anxiety.

Live in present, out of sight, out of mind. Don't understand object permanence, the awareness that objects exist when not perceived.

\item 2 to 7: \textit{Preoperational}, represent things in words and images; using intuition rather than logical reasoning. Developmental Phenomena: Pretend Play, Egocentrism.

In the preoperational stage, children are too young to perform mental operations. Lack principles such as \textbf{conservation}, the principle that quantity remains the same despite changes in shape. 

Piaget contented that preschool children are \textbf{egocentric}: they have difficulty perceiving things from another's point of view. 

\item 7 to 11: \textit{Concrete Operational}, thinking logically about concrete events; grasping concrete analogies and performing arithmetical operations. Developmental Phenomena: Conservation, Mathematical, Transformations

Enter at age 6-7, begin to grasp conservation. Piaget believed that at this time children fully gain the mental ability to comprehend math. 

\item 12 through adulthood: \textit{Formal Operational} Abstract reasoning, Developmental Phenomena: Abstract logic, potential for mature moral reasoning. 

\textit{if this then that} mathematical reasoning is now possible. 
\end{itemize}

\subsection*{Theory of Mind}
Children begin to develop the ability to infer others' mental states when they form a \textbf{theory of mind}, coined by David Premack and Guy Woodruff. As children's ability to take another's perspective develops, they seek to understand what made someone feel a certain away. They tease, empathize, persuade. From 3.5-4.5 they begin to realize that others may hold false beliefs.

Children with autism lack the theory of mind described here.

Piaget ultimately identified significant cognitive milestones and stimulated worldwide interest in how the mind develops. Cared more about sequence of milestones than age at which they occurred. He was ultimately proved relatively correct.

\section*{Social Development}

At 8 months children develop \textbf{stranger anxiety}. They greet strangers by crying and reaching for familiar figures. Children have schemas for faces.

\subsection*{Origins of Attachment}

The \textbf{attachment} bond is a survival impulse that keeps infants close to caregivers. Developmental psychology reasoned that infants became attached to those who satisfied their need for nourishment, but this was ultimately proved to not be the case.

From Harry and Margaret Harlow's experiment with monkeys, it was determined that they became attached to whatever was close to them, and comfortable. Rather than nourishment, it was the comfort and familiarity of the blanket that interested the monkeys more than the wire nourishing mother.

\subsection*{Familiarity}
Contact is one key to attachment. Another is familiarity. In many animals, attachments based on familiarity form during a \textbf{critical period} - an optimal period when certain events must take place to facilitate proper development.

Konrad Lorenz explored the rigid attachment process called \textbf{imprinting}. Discovered that baby birds would imprint other species, even objects, and once formed, this attachment is difficult to reverse.

Children, however, \textit{do not imprint}, they become attached to what they've known. Familiarity is a safety signal, familiarity breeds content.

\section*{Attachment Differences}
Placed in a \textit{strange situation}, 0.6 of infants display \textit{secure attachment}. When mother present, they are fine, they are distressed otherwise and welcome her return. Other infants avoid attachment or show \textit{insecure attachment}. They cling to their mother, when she leaves they cry and remain upset, severely distressed. These situations were developed by Mary Ainsworth. 

Children have inborn differences, making there behaviour dependent on more than just parenting, rather temperament has a role as well. While our capacity for love grows, and our pleasure at touching and holding those we love never ceases, early attachment does gradually relax, and our view widens.

Erik Erikson concluded that securely attached children approach life with a sense of \textbf{basic trust}, a sense that the world is predictable and reliable. He attributed trust not to environment or inborn temperment but to early parenting. Infants with loving caregivers form a lifelong attitude of trust rather thanf fear.

Many researchers believe that our early attachments form the foundation for our adult relationships and our comfort with affection and intimacy. Adult styles of romantic love tend to exhibit secure, trusting attachment; insecure, anxious attachment; or the avoidance of attachment. These styles affect relationships with children, as avoidant people find parenting stressful and unsatisfying.

\section*{Deprivation of Attachment}
If secure attachments nurture social competence, children deprived of love and nurturing can be withdrawn, frightened, even speechless. 

Without any figure, adult animals would be frightened or angry by the presence of others and were incapable of mating. If artifically impregnated, females were neglectful, abusive and even murderous to their children.

Extreme trauma seems to leave footprints on the brain. Changes in the brain chemical serotonin, which calms aggressive impulses, are noted. Stress can set off a ripple of hormonal changes that permanently wire a childs brain to cope with a malevolent world. 

Abuse victims are at considerable risk for depression if they carry a gene variation that spurs stress-hormone production. Once again it is evident that behaviour and emotion arise from a particular environment interacting with particular genes.

\subsection*{Disruption of Attachment}
In studies of children who have been separated from caregivers, while they show difficulty initially, they typically recover from separation distress. Detaching is difficult for adults as well, detaching is a process not an event, progressing through sadness, emotional detachment and ultimately a return to normality.

Daycare generally does not have a negative impact on children. Regardless of the care, all children generally need is a consistent, warm relationship with people whom they can learn to trust.

\section*{Self Concept}
Infancy's major social achievement is attachment. \textit{Childhood's} major achievement is a positive sense of \textbf{self.} By age 12, most children have developed a \textbf{self-concept}, an understanding and assessment of who they are.

\section*{Parenting Styles}
\begin{enumerate}
  \item {\bf Authoritarian}: parents impose rules and expect obedience. 
  \item {\bf Permissive}: parents submit to their children's desires. They make few demands and use little punishment.
  \item {\bf Authoritative}: parents are both demanding and responsive. They exert control by setting rules and enforcing them, but they also explain the reasons for rules. They encourage open discussion when making rules and permitting exceptions.  
\end{enumerate} 
Children with the highest self-esteem, self-reliance, and social competence come from authoritative parents. Authoritarian is insecure and low self-esteem, permissive is aggressive and immature.

\section*{Adolescence}
Development is lifelong, not just childhood. Starts with physical beginnings of sexual maturity and ends with the social achievement of independent adult status. 

Lots of things we already know about puberty and such. 

\section*{Cognitive Development}
\subsection*{Developing Reasoning Power}
Reasoning is initially self-focused. Eventually they reach the intellectual summit Jean Piaget calles \textit{formal operations}, and they become more capable of abstract reasoning. Adolescents debate human nature, truth and justice.

\subsection*{Developing Morality}
Two crucial tasks of childhood and adolescents are discerning right from wrong. Much our morality is gut-level, from which we rationalize. Lawrence Kohlberg sought to describe the development of \textit{moral reasoning}, the thinking that occurs as we consider right and wrong.

Kohlberg posed moral dilemmas, asking children and adults and recording the results, ultimately leading to the following beliefs:

\begin{enumerate}
\item \textbf{Preconvential Morality}: Before the age 9, most children's morality focuses on self-interest. They obey rules to avoid punishment or to gain concrete rewards.
\item \textbf{Conventional Morality:} By early adolescence, focuses on caring for others and upholding laws and social rules, simply because they are laws and rules.

\item {\bf Postconventional Morality} With the abstract reasoning of formal operational thought, people may reach a third moral level. Actions are judged \textbf{right} because they flow from people's rights or from self-defined, basic ethical principles. 

\end{enumerate}

\subsection*{Moral Feeling}
The mind makes moral judgement quickly and automatically. In Jonathan Haidt's \textit{social intuitionist} account of morality, moral feelings precede moral reasoning. Could human morality be run by moral emotions? The social intuitionist explanation of morality finds support from a study of moral paradoxes.

These results showed that moral judgement is more than thinking, it is a gut-level feeling. The gut feeling that drives our moral judgements are widely shared, it is hardwired. 

\subsection*{Moral Action}
Many Nazi concentration camp guards were ordinary, moral people who were corrupted by an evil situation.

As our thinking matures, our behaviour becomes less selfish and more caring. Children are taught empathy for other's feelings, and the self-discipline to restrain their own impulses. Moral action feeds moral attitudes.

\section*{Social Development}
Each stage of life has its own \textit{psychosocial} task, a crisis that needs resolution (Erik Erikson aka worst name ever). Young children wrestle with issues of \textit{trust, autonomy, initiative}. School children strive for competence, feeling able to be productive.

\subsection*{Erikson's Stages of Psychosocial Development}
\begin{enumerate}
  \item Infancy to 1: trust vs mistrust, needs are dependably met, develop sense of basic trust
  \item Todder 1 to 3: autonomy vs shame and doubt, learn to exercise their will and do for themselves, otherwise doubt their ability
  \item Preschool 3 to 6: initiative vs guilt, learn to begin tasks or feel guilty about efforts to be independent.
  \item Elementary school: Industry vs inferiority, learn the pleasure of applying themselves to task, or feel inferior
  \item Adolescence: Identity vs role confusion, refine sense of self, test roles and integrate them to form a single identity,or become confused about who they are.
  \item Young adulthood: Intimacy vs isolation, struggle to form close relationships and gain the capcity for intimate love, or feel socially isolated
  \item Middle adulthood: Generativity vs. stagnation, people discover a sense of contributing to the world, usually through family and work, or feel that they lack purpose. 
  \item Late adulthood: Integrity vs. despair, reflecting on his or her life, an older adult may feel a sense of satisfaction or failure. 
\end{enumerate}

\end{document}
