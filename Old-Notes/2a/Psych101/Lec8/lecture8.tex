\documentclass{article}
\usepackage{parskip}
\usepackage[margin=.6in]{geometry}
\title{Lecture 8: Personality}
\begin{document}
\section*{Sigmund Freud}
Conversion hysteria, women who had these crazy symptoms with no physical causes. Most doctors thought they were just crazy and not worth fixing. Fred was the only one who actually believed this was a thing, so many of the women with conversion hysteria come to him for help. He uses them to help attempt to create a unifying theory of psychology. Freud explores many different ways to get into people's mind, he learns hypnosis, and learns with Joseph Brewer the talking cure. This is when they just wait for you to talk. This is what leads to the whole bed thing, he also gives his patients coco cola which contained cocaine. He ended this after accidentally getting his friend addicted to cocaine. His real road to the subconscious is through dreams. He releases his results in a book about dreams. It is met with a lot of critisism for not being very scientific in his observations. He also get critisized for gleaning information from women who were veiwed as lesser humans. This causes him to alter some of his theories to be more male bias. In the US a man named Johnson translates it and it becomes very popular there. This makes him very famous which makes him arrogant. One of the critism is that his book wasnt consistent in his phrasing of things, the american translation fixes this becuase Johnson had to come up with words to match. Funnily enough this copy is translated into many versions including back into german where it sold much better. The rise of anti-semitism forced him to flee to england
\subsection*{Developing Personality}
We start life with a personality made up of the id, striving impulsively to meet basic needs, living by the pleasure principle. This is part of our evolutionary past, very animistic. \\
In the toddler stage and ego developes, a self that has thoughts, judgements, and memories following a reality principle, though still self focused. It notices that there are some customs that must be followed to get your main desires. Ego can reign in the Id a lot, but always loses.\\
Around age 4-5, the child developes the Superego, a conscious internalized from parents and society, following the ideals of a morality principle. Basically the angel on your shoulder. This is where the ideal self theory comes from. The embodiment of socialization. 
\subsection*{Layers of Personality}
Memeory: The top layer is the conscious(what is currently being processed), the preconscious (mental material that can be easily accessed), then bottom layer is the subconscious (not readily available). \\
Personality: Developes when we result conflicts between id and superego. The unconscious is a reservior of thoughts wishes feelings and memories that are hidden from awareness because they feel unacceptable. \\
In the iceburg the unconscious is under water exerting pressure up onto the other two layers. This is basically the id.
\subsection*{Routes to the Unconscious}
The talking method on the couch was his main method.\\
Tried to get themes to be projected into the conscious world through hypnosis, cocaine, free association, and dream analysis\\
Projective tests are structured, systematic exposure to a standardized set of ambigusou prompts, desinged to reveal inner dynamics\\
Projection is judgeing things based on subconscious conflicts. Rorscharch blot test.\\
Thematic Apperception Test - this is a projection test in which he shows people pictures and ask them what situation it displays. This is a great way to measure the achievement motivation. 
\subsection*{Psychosexual Stages of Personality Development}
Fight between erotos and phanatos impulse. This is fight between good and bad, life and death, creation and destruction.
\begin{itemize}
\item oral(0-18 months) - pleasure centres on the mouth
\begin{itemize}
\item erogenous zone - mouth, key development task is weaning
\item oral dependant personality - fixation:excessive gratifications; passive, dependent, gullible (when the id is always satisfied)
\item oral aggressive personality - fixation:excessive frustrations; argumentative, cynical, exploitative, cruel, sarcastic
\item expressions - gum chewing, nail-biting, smoking, kissing, eating disorders, alcoholism
\end{itemize}
\item anal(18-36 months) - pleasure focuses on bowel and bladder elimination
\begin{itemize}
\item erogenous zone - anus; key developmental task is toilet training
\item anal expulsive personality - Fixation: Excessive gratification; leads to sloppy careless, messy, disorderly
\item anal retentive personality - Fixation: excessive frustration; leads to compulsive cleanliness, orderly, rigid, stubborn
\item expressions - Neil Simon's "The Odd Couple"
\end{itemize}
\item phallic(3-6 years) - pleasure zone is in the genitals, coping with incestuous feelings
\begin{itemize}
\item erogenous zone - genitals; key developmental task: sexual identity
\item Boy's Oedipal Complex - Fixation: Incestuous attraction to mother; leads to castration anxiety (due to rival daddy castrating him, resolved through mimicking father)
\item Girl's Electra Complex - Fixation:Incestuous attraction to father; leads to penis envy (girl wants to have father's penis to make her a complete human) - this was altered due to social pressures, was originally reverse oedipal complex
\item expressions - overly macho men, or overly flirty (daddy issues) or dominant (more masculine) girl, homosexuality (basically anything of a sexual nature problem)
\end{itemize}
\item latency(6-puberty) - a phase of dominant sexual feelings
\begin{itemize}
\item erogenous zone - quite psychosexual stage; key developmental task - gender roles (peer pressure is big)
\end{itemize}
\item genital(puberty on) - masturbation and sexual interests 
\begin{itemize}
\item erogenous zone - genitalia; key developmental task - intimacy and procreation
\item personality - caring responsibility, mutual gratification; or intimacy issues and sexual dysfunctions
\item Neurosis and Anxiety - dependent on extent of earlier fixations; repression or unresolved conflicts
\item expressions - defence mechanisms, intimacy issues and sexual dysfunctions
\end{itemize}
\end{itemize}
\subsection*{Is Freud Dead}
\begin{itemize}
\item medical assault - evidence of biological determinants, AMA abandons psychoanalysis for pharmacology (to be a psychologist you needed a medical degree)
\item economic assault - exorbitant cost of psychoanalysis, insurers abandon psychoanalysis for cheaper treatments
\item legal assault - indefensible claims of repressed sexual abuse, defenders abandon psychoanalysis for scientific evidence (psychoanalysis are no longer credible expert witnesses)
\item scientific assault
\begin{itemize}
\item unfalsifiable - he developed theories that are hard to prove or disprove, can we test to see if there is an id
\item unrepresentable sampling - he did not build his theories on a broad sample of observations; he described all humanity based on people with unusual psychological problems
\item biased observations - he based theories on his patients zas djhdfglaf  8pbyia thgzvb[98yoihv	k5WFSkJSDBV;<ISDB ;i<bcxB ;IUh jbh aiufhvb AWOEFHA;IVG;AUDBJVASDIFU	
\item post facto - avnaoifha;ibgkjabd asdkljbasdiughweifhb lma lkjafladg;lkjg  kjagowiej
\item he skipped his shit
\end{itemize} 
\end{itemize}

\section*{Personality}
\subsection*{Introduction}
an individuals characteristic patterns of thoughts, feelings, and behaviors that persist over time and across situations.\\
State shyness - is when the environment triggers shyness
Train shyness - is how inheritantly shy a person is
\paragraph*{State Shyness} 
\begin{itemize}
\item most of population
\item shy in certain situations only (often elicits sympathy) 
\item minimal negative consequences
\end{itemize}
\paragraph*{Trait Shyness}
\begin{itemize}
\item 15-25\% of population
\item pervasive shyness across situations and times
\item severe consequences
\begin{itemize}
\item avoid social events, meeting new people
\item makes others uncomfortable
\item inhibits communication and assertiveness
\item excessive self-consciousness
\item prone to low self-esteem, loneliness, stress, depression
\end{itemize}
\end{itemize}

\section*{Trait Theory}
Gordon Allport decided that Freud overvalued unconscious motives and undervalues out real, observable personality styles/traits. Myers and Briggs wanted to study individual behaviors and statements to find how people differed in personality: having different traits.\textbf{Trait} is and enduring quality that makes a person tend to act a certain way. \textbf{The trait theory of personality:} that we are ,ade up of a collection of traits, behavioral.\\
some personality tests
\begin{itemize}
\item Minnesota Multiphasic personality inventory - largely designed for clinical populations (spotting abnormal personalities)
\item Eysenck personality inventory - 
\item 16 personality factor questionnaire
\item California psychological inventory
\item Myers-Briggs type inventory
\end{itemize}
The big five personality factors:
\item concientiousness
\item agreeableness
\item neuroticism
\item openness
\item extraversion
\end{itemize}
\paragraph*{16 PF} This test scores you on scales of 16 variables (including the big five) and gives a numeric code for the pattern formed by these values. There is a manual with pages of information about people with this certain personality code. 
\end{document}