\documentclass[12pt]{article}
\usepackage{parskip}
\usepackage[margin=.6in]{geometry}
\title{waiting}
\author{Dan Reynolds}
\begin{document}
\maketitle
\section*{Types of ESP}
\subsection*{Precognition}
The accurate prediction of future events. Knowing the future.
\subsection*{Clairvoyance}
The direct mental perception of a state of physical affairs. Seeing remote events.
\subsection*{Telepathy}
The direct communication between one mind and another through the use of psi. Reading messages from other minds.

\subsection*{The Evidence}
\begin{enumerate}
  \item Anecdotal evidence, you want to call someone and they call, etc. 
  \item Biased reporting
  \item No replication
  \item No identified energy sources or sensory processes
  \item The will to believe, we want to believe it
  \item No scientific support despite 150 years of research
\end{enumerate}
\section*{States of Consciousness: Dreams}
\subsection*{Sleep Onset}
Physiological:
\begin{enumerate}
  \item HR slows
  \item breathing more irregular
  \item muscles relax ( sometimes with sudden twitch or jerk )
  \item sensory equipnment closes down ( vision first, then hearing and others )
  \item hynogogic jerk and myolonic kick, may have experienced it if you've slept with someone
\end{enumerate}
Neurological:
\begin{enumerate}
\item Electrical voltage increases with more diffuse firings throughout brain.
\end{enumerate}
Psychological:
\begin{enumerate}
\item Awareness of environment and time slips away
\item Control of thought and imagery decreases
\item Hypnagogic hallucinations
\begin{enumerate}
\item Brief, weird unusual experiences just before nodding off
\item Tend to be surprising, involve movement, and have red as the dominant colour
\item Most common hallucination involves falling or stepping out into space
\end{enumerate}
\end{enumerate}
Dreams occur during periods with \textbf{REM} sleep. During REM sleep, heart rate rises and breathing becomes rapid. Sleep paralysis occurs when the brainstem blocks the motor cortex's messages and the muscles don't move. This is sometimes known as paradoxical sleep. The brain is active but the body is immobile.

Genitals are aroused, although not as a result of sexual stimulation.

The length of REM sleep increases the longer you remain asleep. With age, there are more awakenings and less deep sleep.

\section*{Physiology of Dreams}
REM in infants and all mammals. RAS become active prior to and during REM, excite motor neurons in eyes.

Serotonin and norepinephrine levels drop and acetylcholine increases. ACI stimulates diffuse areas of brain in unpredictable manner.

\section*{Content of Dreams}
\begin{enumerate}
\item Experienced as real, ABC's and all sense.
\item A coherent bizarre storyline.
\item Usually mundane, everyday content. 50 percent of content linked to recent experiences.
\item Can be influenced by external stimuli.
\item Most dream in colour
\item Fleeting story that requires immediate recall upon waking or it is forgotten
\item If frightening then called a nightmare
\item Everyone dreams
\end{enumerate}
\section*{Theories about function of dreams}
\subsection*{Wish Fulfillment psycho-analytical theory}
Dreams provide a psychic safety valve, they often express otherwise unacceptable feelings and contain both manifest remembered content and a latent content hidden meaning
\subsection*{Information Processing}
Dreams help us sort out the day's events and consolidate our memories.
\subsection*{Physiological function}
Regular brain stimulation from REM sleep may help develop and preserve neural pathways.
\subsection*{Activation Synthesis}
REM sleep triggers impulses that evoke random visual memories which our sleeping brain weaves into stories
\subsection*{Cognitive Developmental Theory}
Dream content reflects the dreamers' cognitive development - his or her knowledge and understanding

\section*{The Evidence}
REM sleep occus in all mammals but not reptiles and decreases with age. Peak is 30-day old fetus who spends entire 24hrs in REM. Corresponds to greatest period of neural development. 

\section*{Modules}

\subsection*{Vision}
Eyes \textbf{transduce} or transform light into neural messages that our brain can process.

\begin{enumerate}
  \item cornea: bends like to provide focus, also provides protection
  \item pupil: small, adjustable opening surrounded by the iris
  \item iris: a coloured muscle that adjusts light intake, dilates or constricts in response to light intesity and even inner emotions. 
  \item lens: behind pupil, focuses incoming light into an image on the retina
  \item retina: a multilayered tissue on the eyeball's sensitive inner surface.
\end{enumerate}

\subsection*{The Retina}
Light enteres the retina's outer layer of cells and proceeds to its buried receptor cells, the rods and cones. Light energy triggers chemical changes that would spark neural signals, activating neighbouring bipolar cells. 

Bipolar cells activate neighbouring ganglion cells, converging, like the strands of a rope, to form the optic nerve that carries info to your brain, where the thalamus distributes the info. 

Where the optic nerve leaves the eye there are no receptor cells, creating a blind spot.

Cones cluster in and around the \textbf{fovea}, the retina's area of central focus. Many cones have their own hotline to the brain, bipolar cells that help relay the cone's individual message to the visual cortex, which deveotes a large area to input from the fovea.

Cones are better able to detect fine detail. Example, stare at one word, words to the left and right appear blurred because their image strikes the more peripheral region of your retina, where rods predominate. 

Cones enable perception of colour, becoming ineffectual in dim light. Rods enable black and white vision, remain sensitive in dim light. 

\section{Visual Information Processing}
At entry level, retina processes info before routing it via the thalamus to the brain's cortex. The retina's neural layers, which are actually brain tissue that migrates to the eye during early fetal development, don't pass just along the electrical impulses, they also help to encode and analyze the sensory info. 

After processing receptors and cones, information travels to your bipolar cels, then to ganglion, through the axons making up the optic nerve, to the brain.

Any given retinal area relays its info back to a corresponding location in the visual cortex, in the occipital lobe at the back of your brain. 

The same sensitivity that enables retinal cells to fire messages can lead lead them to misfire as well.

\subsection*{Feature Detection}
Nobel prize winners David Hubel and Torsten Wiesel demonstrated that neurons in the occipital lobe's visual cortex receive information from individual ganglion cells in the retina. These \textbf{feature detector} cells derive their names from their ability to respond to a scene's specific features, to particular edges, lines, angles and movements.

Feature detectors in the visual cortex pass info to other cortical areas where teams of cells respond to more complex patterns. 

\subsection*{Parallel Processing}
The brain divides a visual scene into subdimensions, such as colour, movement, form and depth. To recognize a face, brain integrates information the retina projects to several visual cortex areas, comapres it and enables images. 

\textbf{Blindsight} is a localized area of blindness in part of their field of vision. While these people could not see any sticks in these areas during an experiment, when asked to guess they routinely succeeded. There appears to be a second mind, a parallel processing system.

\section*{Color Vision}
Thomas Young and Hermann von Helmholtz made the\textbf{trichromatic theory}, that the brain has three colour receptors, red, green and blue.

Colour blind people lack functioning red or green sensitive cones, sometimes both. If this is true, how can those blind to red and green see yellow? Ewald Hering, a physiologist, found a clue in the occurrence of \textit{afterimages}. 

When you stare at a green square and then white, you see red. Stare at a yellow square and you will see later see blue. Hering determined that there must be two additional colour processes, one responsible for red versus green and one for blue versus yellow.

This was confirmed as \textbf{opponent processing theory}. As visual info leaves the receptor cells, we analyze it in terms of three sets of opponent colours: \textit{red-green, yellow-blue, white-black.}

In the retina and the thalamus, where impulses from the retina are relayed en route to the visual cortex, neurons are turned on by red but turned off by green. 

When we stare at a white after green, only the red part of the green-red pairing fires normally, the green part is tired.

Colour processing occurrs in two stages, the retina's red, green and blue cones respond in varying degrees to different color stimuli, as the Young-Helmholtz trichromatic theory suggested.

Their signals are then processed by the nervous system's opponent-process cells, en route to the visual cortex.

\section*{Other Senses}
\subsection*{Touch}
Skin sensations are variations of the basic 4:
\begin{enumerate}
  \item pressure: such as stroking pressure spots to tickle
  \item warmth: stimulating cold and warm spots produces the sensation of hot
  \item cold: touching adjacent cold triggers a sense of wetness, like cold metal
  \item pain: repeated stroking of a pain spot creates an itch
\end{enumerate}

\textbf{Kinesthesis}: a person's sense of position and movement of his or her body parts.

\textbf{Vestibular sense} monitors one's head's and thus one's body's position and movement. The biological gyroscopes for this sense of equilibrium reside in one's inner ear. The semicircular canals and the \textit{vestibular sacs} which connect the canals with the cochlea contain fluid that moves when one rotates their head or tilts.

The movement stimulates hair-like receptors which send back messages to the cerebellum at the back of the brain, enabling one to maintain body position and balance.

If one spins around, neither the fluid in the canals or the kinesthetic receptors immediately return to their neural state. The dizzy aftereffect fools the brain with the sensation that one is spinning.

\subsection*{Pain}
No one type of stimulus triggers pain, instead there are different \textit{nociceptors} detect hurtful temperatures, pressure or chemicals. 

Ronald Melzack and biologist Patrick Wall's \textbf{gate-control theory} attempts to explain pain. The spinal cord contains small nerve fibers that conduct most pain signals and larger fibers that conduct most other sensory signals.

They theorized that the spinal cord contains a neurological gate. When tissue is injured, the small fibers activate and open the gate and you feel pain. Large-fiber activity closes the gate, blocking pain signals and preventing them from reaching the brain.

Therefore one way to block pain signals is to stimulate by massage, etc, gate closing activity in the large neural fibers. 

When we are distracted from pain, and soothed by endorphins, our experience of pain may be greatly diminished. 

\subsection*{Psychological Influences}
People tend to record pain's peak moment, and how much pain they felt at the end. We tend to perceive more pain when others also seem to experience it. 

\subsection*{Taste}
Five taste receptors:
\begin{enumerate}
  \item sweet
  \item sour
  \item salty
  \item bitter
  \item umami - the flavor enhancer monosodium glutamage
\end{enumerate}
\subsection*{Taste Aversion}
Nauseation can be paired with a food and it becomes inedible. 

\subsection*{Cognitive Learning}
Refers to acquiring new behaviours and information mentally rather than by direct experience. 

\section*{Smell}
We smell when molecules of a substance carried in the ari reach a tiny cluster of 5 million or more receptor cells at the top of each nasal cavity. 350 or so receptor proteins that recognize particular odor molecules.

Smell can evoke feelings and memories, a hotline runs between the brain area receiving info from the nose and the brain's ancient limbic centers associated with memory and emotion. 

\section*{Perceptual Organization}
German psychologists noticed that when given a cluster of sensations, people organize them into a \textbf{gestalt}, a form or whole.

The whole may exceed the sum of its parts, our brain makes inferences.

\subsection*{Form Perception}
The \textbf{figure-ground} is the organization of the visual field into objects, the figures, that stand out from their surroundings, the ground.

\subsection*{Grouping}
To bring order and form to the sensations of color, movement, light and dark contrast, our minds follow certain rules for grouping stimuli.

\begin{enumerate}
  \item Proximity: We group nearby figures together
  \item Similarity: We group simiilar figures together
  \item Continuity: We perceive smooth, continuous patterns rather than discontinuous ones
  \item Connectedness: Perceive object overlayed as one unit
\end{enumerate}

\textbf{Closure}, we fill in gaps to create a complete, whole object.

\section*{Depth Perception}
Ability to see objects in three dimensions, enables us to estimate distance. Infants innately knew to not approach a \textbf{visual cliff}. 

\subsection*{Binocular Cues}
Two eyes are better able to judge distance than one. Because of their separation, their \textbf{retinal disparity} provides an important binocular cue to the relative distance of different objects.

\subsection*{Monocular Cues}
When looking straight ahead, these cues help us tell whether a person is close or far away. Influence our everyday perceptions.

\subsection*{Motion Perception}
We perceive motion as shrinking objects are retreating and enlarging objects are approaching. The brain perceives a continuous movement in a rapid series of slightly varying images, called \textit{stroboscopic movement}.

Marquees and holiday lights create another illusion of movement using the \textbf{phi phenomenon}. When two adjacent stationary lights blink on and off in quick succession, we perceive a single light moving back and forth between them.

\subsection*{Perceptual Constancy}
The ability to recognize objects without being deceived by changes in their shape, size, brightness, or color is an ability called \textbf{perceptual constancy}

Shape constancy is the ability to form familiar objects as constant even while our retinal image of it changes.

Size constancy is the ability to perceive objects as having a constant size in the same way. 

\subsection*{Lightness Constancy}
We perceive an object as having a constant lightness even while its illumination varies. Perceived lightness depends on relative luminance, the amount of light an object reflects relative to its surroundings. 

\subsection*{Color Constancy}
If you view only part of a red apple, its color will seem to change with light. If you view the whole apple, its color will remain roughly the same, a phenomenon called color constancy.

\section*{Perceptual Interpretation}
\subsection*{Sensory Deprivation and Restored Vision}
Writing to John Locke, William Molyneux wondered whether "a man born blind, and now adult, taught his touch to distingiuish between a cube and a sphere, could, if made to see, visually distinguish the two. Locke's answer was no, as the man would never have learned the difference.

\subsection*{Perceptual Adaptation}
Our \textbf{perceptual adaptation} to changed visual input makes the world seem normal again. If we saw everything upside down, we would adapt and get good at it.

\subsection*{Perceptual Set}
Our expectations give us a mental predisposition that greatly influences what we perceive. People perceive an adult-childd pair as looking more alike when they are parent and child.

Once we have formed a wrong idea about reality, we have more difficulty seeing the truth. Perceptual sets include not liking something when told it will take awful, versus if it was said to be expensive and of high quality.

\subsection*{Context Effects}
Imagine hearing a noise interrupted by the words "eel on a wagon" you would perceive the first word to be wheel. This phenomenon suggests that the brain can work backward in time to allow a later stimulus to determine how we perceive an earlier one. 

\subsection*{Emotion and Motivation}
Perceptions are influenced not only by our expectations and by the context, but also by emotions. Walking destinations look farther away to those who have been fatigued. Hills look steeper to those wearing a backpack or listening to sad music.

Motives also matter, if seeing one outcome results in reward, it is more likely to be seen.

\subsection*{Perception and the Human Factor}
\textbf{Human factor psychologists} determine whether designs make sense to the average person. 

Curse of knowledge: When you know a thing, it is very difficult to mentally stimulate what it is like to not know it. 

\subsection*{Extrasensory Perception}
Nearly half of Americans believe we are capable of \textbf{extrasensory perception.} 
\begin{enumerate}
\item Telepathy: mind-to-mind communication, one person sending thoughts to another or perceiving another's thoughts.
\item Clairvoyance perceiving remote events, such as sensing that a friend's house is on fire
\item Precognition, perceiving future events, such as a political leader's death
\item Psychokinesis, or mind over matter, such as levitating a tble or influencing the roll of a die
\end{enumerate} 





\end{document}
