\documentclass[12pt]{article}
\usepackage{parskip}
\usepackage[margin=.6in]{geometry}
\title{Memory}
\author{Dan Reynolds}
\begin{document}
\maketitle
\section*{Stage Model of Memory}
\section*{Stage Model: Encoding and Storage}
\begin{enumerate}
  \item Iconic and echoic memory 
  \item Internal and external triggers
\end{enumerate}
\section*{Short-term Memory}
\begin{enumerate}
  \item Working Memory
  \item Evaluation of Information
\end{enumerate}
\section*{Long-term Memory}
\begin{enumerate}
  \item Permanence and capacity
\end{enumerate}
\subsection*{Sensory Memory}
\begin{itemize}
  \item Temporarily store sensory info
  \item Capacity, high
  \item Duration, less than 1 second, vision, or few seconds, hearing
\end{itemize}
\subsection*{Short-term Memory}
\begin{itemize}
  \item Brief storage of info currently being used
  \item Capacity, unlimited
  \item 
\end{itemize}
\subsection*{Forgetting}
Forgetting can occur at any memory stage. As we process info, we filter, alter, or lose much of it. 

Our memory is accessible depending on interference, retrieval cues, moods, motives, some things don't get retrieved. 

\section*{Intelligence and IQ Testing}
\subsection*{The History of Intelligence Testing}
\begin{enumerate}
  \item Francis Galton and Eugenics
  \item Afred Binet and the First Intelligence Test
  \item Charles Spearman and the Two-Factor Theory
  \item William Stern and the Q in IQ
  \item Lewis Terman and the Dark Ages
  \item David Weschler and the Return of Binet
\end{enumerate}
\subsection*{Modern Uses of Intelligence Testing}
\subsection*{Ironies of I.Q. Testing}

\section*{The Long Reach of IQ}
IQ score is one of the best predictors of socio-economic status, earnings, GPA.

If an employer were to use only intelligence tests and select the highest scoring applicant for each job, training results would be predicted well regardless of the job, and overall performance form the employees selected would be maximized. - Random quote.

\subsection*{Intelligence and Success}
Success in life is impossible to define, however, wealth tends to be related to intelligence test scores. \textbf{Intelligence tests} are a series of questions and other exercises which attempt to assess a people's mental abilities in a way that generates a numerical score, so that one person can be compared to another. 

\textbf{Intelligence} can be defined as whatever intelligence tests measure. 
\subsection*{Francis Galton and Eugenics}
Successful in many areas, Francis was a founder of correlation. He was an explorer. He did fingerprinting, density of beautiful women in GB. Showed what topics caused people to fall asleep during lectures. 

He realized that success seemed to go in families. Achievement was hereditary. He believed that intelligence was inherited. Wrote a book \textbf{heriditary genius} in 1869, the \textbf{First Intelligence Test} in 1884.

His family stock had evolved more than that family more replete with failures. That is the product of selection, and it is evident on a world scale. The British people were surely more evolved than those of less developed nations. 

\subsection*{Intelligence according to Galton}
\begin{itemize}
  \item Biological definition
  \item Powerful and efficient brain and nervous system
  \item Determined by heredity
  \item Eugenics
\end{itemize}
\textbf{Eugenics }is the notion that the world's resources should be disproportionally distributed to those who can make the most use of them. Don't waste assets or education on them, put all resources on those who are superior. 

He invented the questionnaire, about family quality and based on that he would predict scores on intelligence tests.

\subsection*{Alfred Binet}
Comes from a poorer family, never did that well in school. Wants to study psych, but gets embarassed. Has to learn legitimate psych. Learns from Simon, who is trying to test intelligence. 

According to Binet, Intelligence is the tendency to take and maintaina  definite direction, the capacity to make adaptations for the purpose of attaining a desired end, and the power of auto-criticism.

This was a flexible and pragmatic definition. Intelligence is not fixed but grows over time. It is a loose collection of varioius mental abilities, tied together by common sense. 

Where Dalton was arguing that intelligence is inborn, Binet is arguing that nature has an effect on a person's intelligence. 

All children are suddenly given schooling in France. They have a problem with students of various ranges and so they call Binet. Binet gets the idea that he will use nurture to find a solution. Intelligence is going to be based on the relationship between one's mental age and chronological age. A 10 year old who has not done school would begin with a 6 year old, both in same year with no previous schooling.

He believes that even though neither has been in school, the 10 probably has some acquired knowledge. An uneducated 10 year old might not have a mental age of 10, rather 8. Go according to the mental age rather than the chronological age.

Intelligent person is someone whose mental age is at least as high as their chronological age. Now give children items until they start to fail and where they fail is where their mental age is, then calculate where they are chronologically.

FIrst tests in 1905.

\subsection{Two-Year Old Items}
\begin{itemize}
  \item Followed a lighted match with eyes
  \item Unwrap and eat a piece of candy ......
\end{itemize}

If children were within 2 years of their mental age, they could adapt. If they were over 2 years behind, they were retarded, or behind due to a lack of experience. 

\subsection*{Charles Spearman}
British mathematician, had \textbf{Two-Factor Theory}

\begin{enumerate}
  \item G-Factor: general intelligence, an overall mental ability common to all intellectual task, biologically inherited. Galton.
  \item S-Factor: specific intelligence, mental abilities applicable to unique intelligence tasks. Environmental, learned. Binet.
\end{enumerate}

\subsection*{William Stern}
turned intelligence into intelligence quotient.

IQ = Mental Age/Chronological Age * 100

Armies were becoming specialized so needed an intelligence ranking. Louis Termin is a supported of eugenics and those of American Caucasian ancestry. He believed that lesser races be eliminated through forced sterilization or genocide. Eugenics gets support from wealthy classes. His conclusions were drawn off of tests on a pair of children from Mexico and Native children from Arizona based on his science.

The American government limited immigration, allowing people only from certain countries. They limited immigration from countries whose immigrants typically scored lower on IQ tests. If they did not score at least a certain level, they were sent back. They instituted quotas on the number of people who would be allowed in. They eventually decided it was stupid. 

If you scored low, Binet believed that you could be made smart through environmental influence. To Termin, if this happened, their genes should be removed. 

\subsection*{David Wechsler}
Creates an intelligence scale for children, WISC and adults WAIS. Still used today in some form, updated and adapted.

According to Weschler, intelligence is the aggregate or global capacity of the individual to act purposefully, to think rationally and to be resourceful to cope with life's challenges. 

\section*{Introduction to Memory}
To a psychologist, \textbf{memory} is learning that has persisted over time, information that has been restored and may be retrieved. 

To remember any event, we must get info into our brain \textbf{encoding}, retain information \textbf{storage}, and later get it back \textbf{retrieval}.

Richard Atkinson and Richard Shiffrin proposed we form memories in three stages"

\begin{itemize}
  \item We first record to-be-remembered info as a fleeting \textbf{sensory memory}.
  \item From there, we process information into a \textbf{short-term memory} bin, where we encode it through rehearsal.
  \item Finally information moves into \textbf{long-term memory} for later retrieval/.
\end{itemize}

Although historically important and helpfully simple, this three-step process is limited and fallible. 

\textbf{Working memory} - a newer understanding of Atkinson's second stage, concentrates on active processing of info in this intermediate stage.

We process incoming stimuli, along with info we retrieve from long-term memory, in temporary working memory. Working memory associates new and old info and solves problems.

People's working memory capacity differs.



\end{document}
