\begin{enumerate}
  \item 
\end{enumerate}\documentclass[12pt]{article}
\usepackage{parskip}
\usepackage[margin=.6in]{geometry}
\title{Developmental Psychology}
\author{Dan Reynolds}
\begin{document}
\maketitle
\section*{Neuropsychology}
\section*{Two Requirements for Interacting with the Environment}
\textbf{Person} and \textbf{Environment}. A process that involves bringing something inward is an \textit{afferent} process. To detect the external world is not enough for survival, need to be able to respond back to the environment effectively depending on the demands of that world. Processes that are outward bound are called \textit{efferent} processes. 

\subsection*{The Neuron}
\begin{itemize}
  \item Dendrites: receive messages from other cells
  \item Axon: passes message away from the cell body to other neurons, muscles or glands.
  \item Myelin Sheath: covers the axon of some neurons and helps speed neural impulses
  \item Cell body: the cell's life support center
\end{itemize}

Types of Neurons:

\begin{enumerate}
  \item Sensory neurons carry messages in from the body's sensory receptors to the CNS for processing. Afferent direction from outside CNS to inside CNS. Quantity 2-3 million
  \item Motor neurons carry instructions OUT from the CNS to the body's muscles and glands. Efferent direction from inside CNS to outside. 2-3 million.
  \item Interneuons in brain/spinal cord process information between the sensory input and motor output. Within CNS Neuron to neuron. Quantity, 10-100 billion.
\end{enumerate}

\section*{Neuronal Connection}
Luigi Galvani, famous for work with electrocuting animals. 

Otto Loewi was an Austrian, believed that it was chemical that sent messages in the brain, whereas electrical impulses would arc and not be capable of such precision. It takes him 17 years before he comes up with the experiment, which came to him in a dream. Heart sitting in beaker of water. He puts a second heard in another beaker, but just takes the liquid from first and pours it over the second. The liquid alone caused the second heart to beat, therefore there must have been some chemical reaction. This was the fist neurotransmitter.

\section*{The Hodgkin-Huxley Model}
Presented modern "electro-chemical" theory in 1952. Received Nobel Prize in Physiology. Chemical process between neurons, electrical within a neuron. Neurotransmitters accumulate on the dendrites as they pass. The dendrites check for the correct molecular structure to excite or inhibit. 

\subsection*{Agonist and Antagonist Molecule}
A \textbf{agonist} molecule flls the receptor site and actives it, acting like the neurotransmitter. 
An \textbf{antagonist} molecule fills the lock so that the neurotreansmitter cannot get in and activate the receptor site. 

After synaptic transmission:
Decomposed by certain enzymes
Reuptake - after neurotransmitters stimulate receptors on the receiving neuron, the chemicals are take back up into the sending neuron to be used again.
Continues binding - they release then bind right back again and keep binding and binding.

\section*{Neurotransmitters}
\subsection*{Acetylcholine}
Involved in muscle action, learning and memory. Deterioration of ACH neurons implicated in Alzheimer's disease. 

\subsection*{Dopamine}
Influence movement, learning, attention and emotion
Oversupply linked to schizophrenia
  he he El Dopa
Undersupply to Parkinson's Disease

Lo --------- Normal ------------ Hi
                                 Schizophrenia
Parkinson's

\subsection*{Serotonin}
Infuences mood, hunger, sleep and arousal
Undersupply linked to depression

SSRI - selective serotonin reuptake inhibitors

\subsection*{Epinephrine Norepinephrine}
Classic example of antagonistics transmitters
Influence alertness, arousal, mood
Same chemical used in endocrine system (adrenaline and noradrenaline)

\subsection*{Endorphins}
Known as natures painkiller, natural morphine
Also involved in various emotions in the limbic system

\subsection{Cocaine and amphetamines}
Stimulants that increase release of norepinephrine and block reuptake of dopamine. 
Perceived as pleasurable and associated behaviours are reinforced. 

\subsection*{Opiates - opium, morphine, heroine, codeine}
Agonists that mimic endorphins by attaching to their binding sites

\subsection*{Alcohol}
Generally depresses neural acticity throughout brain. 

\section*{The Neurology of Addiction}
Brain maintains homeostasis, increases or decreases neurotransmitters to compensate effect of addictive substance. 
Need increased dosage to experience same effect
Brain learns to anticipate consumption. Makes pre-emptive adjustment, overdose potential.

\section*{Modules}
\subsection*{Neural Communication}
Biological psychs examine sleep, dreams, sex, depression, stress, etc.

We are biopsychosocial systems, and to understand our behavior, we study how these systems work and interact.

The body's information system handling all these tasks is built from interconnected cells called neurons. 

\subsection*{Neurons}
Body's neural information system is built on neurons, or nerve cells. \textbf{Sensory neurons} carry messages from the body's tissues and sensory organs inward to the brain and spinal cord. Brain and spinal cord send instructions out to the body's tissues via \textbf{motor neurons.}

Information is processed in the brain's internal communication system via \textbf{interneurons}. 

\begin{itemize}
  \item Dendrites: receive messages from other cells
  \item Terminal branches of axon: form junctions with other cells
  \item Axon: passes messages away from the cell body to other neurons, muscles, or glands
  \item Neural Impulse: electrical signal traveling down the axon
  \item Cell Body: the cell's life-support center
\end{itemize}

The \textbf{dendrite fibers} receive information and conduct it toward the cell body. From there the cell's axon passes the message to other neurons or to muscles or glands. Axons speak. Dendrites listen. \textbf{Axons} can be up to feet long within the body. The \textbf{myelin sheath} insulates the axons of some neurons and help speed their impulses. Degeneration leads to ms.

Neurons transmit messages when stimulated by signals from our senses or when triggered by chemical signals from neighbouring neurons. At such times, a neuron fires an impulse called the \textbf{action potential} - a brief electrical charge that travels down the axon.

Neurons generate electricity from chemical events. The chemistry-to-electricity process involves the exchange of ions. The fluid interior of a resting axon has an excess of negatively charged ions, while the fluid outside the axon membrane has positively charged ions. This positive-outside/negative-inside state is called the \textbf{resting potential.}

Axon's surface is selectively permeable. A resting axon has gates that block positive sodium ions. When opened, flood membrane, depolarizing that section of the axon and causing the next section to open.

Each neuron is a decision making device performing complex calculations as it receives signals from other neurons. Most of these signals are excitatory, others are inhibitory. If excitatory minus inhibitory signals exceed a minimum intensity threshold, the combined signal trigger an action potential.

\section*{How Neurons Communicate}
Meeting point between neurons is called a \textbf{synapse}. Axon terminal of one neuron is separated from the receiving neuron by a \textit{synaptic gap}. When an action potential reaches the terminals at an axon's end, it triggers the release of chemical messengers, called \textbf{neurotransmitters.}

The neurotransmitter unlocks tiny channels at the receiving site, and electrically charged atoms flows in, exciting or inhibiting the receiving neuron's readiness to fire. Then in a process called \textbf{reuptake}, the sending neuron reabsorbs the excess neurotransmitters.

\subsection*{How Neurotransmitters Influence Us}
\textbf{Acetylcholine} is one of the best understood neurotransmitters. In addition to its role in learning and memory, is the messenger at every junction between a motor neuron and skeletal muscle. 

Candace Pert and Solomon Snyder discovered that the brain creates natural occurring opiates. \textbf{Endorphins} help explain good feelings such as the runner's high Dan's a big poopy pants, the painkilling effect of acupuncture and the indifference to pain in some severely injured people. 

\begin{enumerate}
\item Acetylcholine: enables muscle action, learning and memory. With Alzheimers, Ach neurons deteriorate.
\item Dopamine: influences movement, learning, attention and emotion. Excess dopamine activity is linked to schizophrenia. Starved of dopamine, the brain produces tremors and decreased mobility of Parkinson's disease. 
\item Serotonin: affects mood, hunger, sleep, arousal. Undersupply linked to depression.
\item Norepinephrine: helps control alert-ness and arousal. Undersupply can depress mood. 
\item GABA: a major inhibitory neurotransmitter. Undersupply linked to seizures, tremors, and insomnia. 
\item Glutamate: a major excitatory neurotransmitter, involved in memory. Oversupply can overstimulate the brain, producing migraines and seizures. 
\end{enumerate}

\subsection*{How Drugs and Other Chemicals Alter Neurotransmission}
When flooded with artificial opiates, such as heroine and morphine, the brain stops producing its own. Brain then becomes deprived and uncomfortable.

Drugs and other chemicals affect brain chemistry at synapses, often by amplifying or blocking a neurotransmitter's activity. An \textit{agonist} molecule can mimic the effects of a neurotransmitter.

\textit{Antagonists} block a neurotransmitter's functioning. 

\section*{The Nervous System}
Our \textbf{nervous system} is the body's electrochemical communication network. The brain and spinal cord form the \textbf{central nervous system,} which communicates with the body's sense receptors, muscles and glands via the \textbf{peripheral nervous system.} 

The PNS information travels through axons that are bundled into electrical cables known as \textbf{nerves.} Information travels in the nervous system through sensory neurons, motor neurons, and interneurons.

\subsection*{Peripheral Nervous System (PNS)}
Our peripheral nervous system has  two main components - somatic and autonomic. Our \textbf{somatic nervous system} enables voluntary control of our skeletal muscles.

\subsection*{Nervous System:}
\begin{enumerate}
  \item Nervous System
  \begin{enumerate}
    \item Peripheral
    \begin {enumerate}
      \item Autonomic (controls self-regulated action of internal organs and glands)
      \begin{enumerate}
        \item Sympathetic (arousing)
        \item Parasympathetic (calming)
      \end{enumerate}
      \item Somatic (controls voluntary movements of skeletal muscles)
    \end{enumerate}
    \item Central (brain and spinal cord)
  \end{enumerate}
\end{enumerate}
    
Our \textbf{autonomic nervous system} controls our glans and muscles of our internal organs, influencing functions such as glandular activity, heartbeat, and digestion. 

The \textbf{sympathetic nervous system} arouses and expends energy. If something alarms, enrages, or challenges you, the sympathetic system will accelerate your heartbeat, raise your blood pressure, slow your digestion, etc.

When stress subsides, your \textbf{parasympathetic nervous system} produces opposite effects. It conserves energy as it calms you by decreasing heartbeat, lowering blood sugar, etc. 

\subsection*{The Central Nervous System}
The \textit{spinal cord} is an information highway connecting peripheral nervous systems to the brain. Ascending neural fibers send up sensory information and descending fibers send back motor-control info. 

The neuropathways governing our \textbf{reflexes}, our automatic responses to stimuli, illustrate the spinal cord's work. Hand jerks away from hot surface before feel pain because pain reflect pathway runs through spinal cord, takes longer to get to brain.

\section*{Endocrine System}
Interconnected with nervous system is endocrine system, the glands of which secrete \textbf{hormones.} 

Some hormones are chemically identical to neurotransmitters. While nervous system mesages move quickly, endocrine system take several seconds or more. The messages linger, however, such as in the example when a person is upset.

In moments of danger, \textbf{adrenal glands} release epinephrine, and norepinephrine, or adrenaline, and noradrenaline. These hormones increase heart rate, blood pressure, etc, providing us with a surge of energy.

Most influential endocrine gland is the \textbf{pituitary gland.} Controlled by the hypothalamus, the gland releases hormones that influence growth, release of hormones. Master gland whcih triggers sex glands to release sex hormones. 

\section*{Brain's Electrical Activity}
An electroencephalogram (EEG) is an amplified read-out of the electrical activity in the brain's billions of neuron sweeps. It identifies the electrical wave evoked by stimulus.

\subsection*{Neuroimaging Techniques}
the \textbf{PET - Positron Emission Tomography scan} depicts brain activity by showing each brain area's consumption of sugar glucose. After consuming radioactive glucose, tracks the glucose throughout the brain. Shows which areas of brain are most active a a person performs mathematical calculations, looks at images of faces, or daydreams. 

\textbf{MRI - Magnetic Resonance Imaging} puts the head in a strong magnetic field, aligning the spinning atoms of brain molecules. A radio wave disorients the atoms, and when they return to their normal spin, signal are released which outline the brain's soft tissue.

\subsection*{The Brainstem}
\textbf{Brainstem} is the brain's oldest and innermost region. It begins where the spinal cord swells, called the \textbf{medulla}. Controls heartbeat and breathing. Just above the medulla is the \textit{pons}, which helps to coordinate movement. 

Brainstem is a crossover point where nerves from each side of the brain cconnect with the body's opposite side. Inside the brainstem, between ears, is the \textbf{reticular formation}, a finger-shaped network of neurons that extends from the spinal cord right up to the thalamus. 

The reticular formation is involved in arousal, if severed produces coma, if stimulated, produces alertness.

\subsection*{Thalamus}
Sits at top of brainstem, acts as brain's sensory switchboard. Receives information from all senses except smell and routes it to the higher brain regions. Directs higher brain replies, sending them to medulla and cerebellum.

\subsection*{Cerebellum}
Extends from the rear of the brainstem, means little brain. Enables one type of nonverbal learning and memory. Helps us judge time, modulate our emotions, discriminate sounds and textures. 

These functions all occur without conscious effort, our brain processes most information outside of our body's awareness. 

\section*{The Limbic System}
At the border between the brain's older parts and the cerebral hemispheres is the limbic system. One limbic system is the \textit{hippocampus}, which processes memory. 

\subsection{The Amygdala}
Influences aggression and fear.

\subsection{The Hypothalamus}
below (hypo) thalamus, hypothalamus is an important link in the chain of command governing bodily maintenance. Some clusters influence hunger, others thirst, body temperature, sexual behaviour.

Hypothalamus monitors blood chemistry and takes orders from other parts of the brain. For example, thinking about sex could cause hormones to be released, illustrating how the brain influences the endocrine system, which in turn influences the brain.

Stimulating the hypothalamus of animals can trigger their reward system, triggering the release of dopamine, and specific centers associated with eating, drinking, sex.

\section*{Cerebral Cortex}
A thin surfance of interconnected neural cells, it is the brain's thinking crown, the ultimate control and information processing center. 

The larger the cortex, the greater the capacity for learning and thinking, enabling them to be more adaptable. What makes us distinctively human mostly arises from complex functions of our cerebral cortex.

\subsection*{Structure of the Cortex}
Supporting the billions of nerve cells in the cortex are nine times as many spidery \textbf{glial cells}, or glue cells. Glials work for neurons, providing nutrients and insulating myelin, mopping up ions and neurotransmitters.

Einstein had more glial cells than the average person.

Each hemisphere of the cortex is divided into four lobes, geographic regions separated by prominent fissures. There are the \textbf{frontal lobes}, behind the forehead, \textbf{parietal lobes}, at the top and to the rear, and the \textbf{occipital lobes}, at the back. Just above ears, find \textbf{temporal lobes}. 

Arch-shaped region at the back of the frontal lobe, running form year-t-year across the top of the brain is the \textbf{motor cortex.} Body areas requiring precise control require greatest amount of cortical space.

\subsection*{Sensory Functions}

If motor cortex sends messages out to the body, the area at the forn of the parietal lobes, parallel and just behind motor cortex is called the \textbf{sensory cortex.} The more sensitive the body region, the larger the sensory cortex area devoted to it. 

Also have auditory cortex in temporal lobes, and visual cortex in occipital lobes. 

\subsection*{Association Areas}
Three quarters of the thin, wrinkled layer, are a part of the association areas, whose neurons integrate information. They link sensory inputs with stored memories.

We don't only use a tenth of our brains, rather these association areas interpret and act on info processed by the other areas.

Association areas are found in all 4 lobes, enabling jdugment, planning and processing of new memories in the frontal lobes. With ruptured forntal lobes, become less inhibited, and more judgments seem unrestrained by normal emotions. 

In the parietal lobes, enable mathematical and spacial reasoning. An area on the underside of the right temporal lobe enables us to recognize faces. 

\section*{The Brain's Plasticity}
Brains are sculpted by our experiences. Neurons do not usually regenerate when severed, rather they can reorganize in response to damage.

\textit{Constraint-induced therapy} forces the brain to rewire to use a bad limb when the good limb is made unavailable.

Damaged brain functions can migrate to other rain regions, as in the case of those who have lost one of their senses. If a blind person uses one finger to read Braille, the brain area dedicatd to that finger expands as the sense of touch invades the visual cortex that normally helps people see. 

Adjacent regions in the brain, such as the arm and hand invade one another if one is lost. Touching an arm when a person has lost a hand may cause them to feel as though their hand has been touched as well. The arm area had invaded the space vacated by the hand. 

\textbf{Neurogenesis} is the process of generating new neurons. Natural promoters of neurogenesis include exercise, sleep and nonstressful but stimulating environmnents.

\section*{Our Divided Brain}
In 1960, believed that left hemisphere is the dominant or major hemisphere, right is subordinate or minor. 

In 1961, neurosurgeons speculated that epileptic seizures were causes by an amplification of abnormal brain activity bouncing back and forth between the two cerebral hemispheres. They tried to end this by severing the \textbf{corpus callosum},the wide band of axon fibers connecting the hemispheres and carrying messages between them.

After severed, people were surprisingly normal, operating fine with these \textbf{split brains.} With a split brain, both hemispheres can comprehend and follow an instruction simultaneously.

These studies revealed that the left hemisphere is more active when a person deliberates over decisions. Right hemisphere understands simple requests, easily perceives objects, and is more engaged when quick, intuitive responses are needed. 

Right side is skilled at perceiving emotion and portraying emotions through the more expressive left side.  

\section*{Right/Left Differences in the Intact Brain}
Perceptual tasks increase glucose consumption in the right side. Speaking or calculating increases activity in the left side. 

If left side disabled, lose control of right arm and speech. Sign language, like hearing using the left to process speech, also uses left hemisphere for sign language. Language is language to the brain, whether spoken or signed.

While left is adept at making quick, literal interpretations of language, right excels in making inferences. 

Given word \textit{foot}, the left quickly associate it with \textit{heel.} But given \textit{foot, cry, glass}, the right more quickly recognizes another word, \textit{cut}. 

Right hemisphere also orchestrates sense of self. In right brain damage, have difficulty perceiving who other people are in relation to oneself. People had difficulty recognizing themselves in photos with the right brain disabled.

\section*{Brain Organization and Handedness}
Almost all right-handed people process speech primarily in the left hemisphere, which tends to be larger. 

\section*{Behaviour Genetics and Evolutionary Psychology}
\textbf{behaviour geneticists} study our differences and weigh the effects and interplay of herdity and environment.

\textbf{Identical Twins} who develop from a single fertilized egg that splits in two are genetically identical. 

\textbf{Fraternal twins develop from seperate fertilized eggs.} They share a fetal environment but they are no genetically similar than any two siblings.

\subsection*{Separated Twins}
Two guys were wombmates and then separated and met 38 years later and learned they were living like the exact same lifestyles. True story bro.

Despite criticisms as to why this could result, the striking twin-study results helped shift scientific thinking toward a greater apprecition of genetic influences.

\subsection*{Biological versus Adoptive Relatives}
Adoption creates two groups:
\textit{genetic relatives} - biological parents and siblings
\textit{environmental relatives} - adoptive parents and siblings

The finding from studies of adoptive families show that people who grow up together, whether biologically related or not, do not resemble one another much in personality. 

In traits such as extraversion and agreeableness, adoptees are more similar to their biological parenst than to their caregiving adoptive parents. 

\textbf{Reiterating:} the environment shared by a family's children has virtually \textbf{no} discernible impact on their personalities. Two adopted children are no more likely to share personalities with each other than with the children down the street.

While genetics may limit the family's environment's influence on personality, parents do influence their children's attitudes, values, manners, faith, and politics. \textbf{Parenting does matter.}

\section*{Temperament and Heredity}
Infants' temperaments are their emotional excitability, whether reactive, intense, and fidgety or easygoing, quiet and placid. Temperament differences tend to persist. 

Heredity predisposes temperament differences, identical twins have more similar personalities than fraternal twins.

One form of a gene that regulates the neurotransmitter serotonin predisposes a fearful temperament in children. Concludes that biologically rooted temperament helps form our enduring personality. 

\section*{Heritability}
Using twin and adoption studies, behaviour geneticists can mathematically estimate the \textbf{heritability} of a trait, the extent to which variation among individuals can be attributed to differing genes. 

If the heritability intelligence is half, does \textbf{not} mean that your intelligence is 50 percent genetic. Rather it means that genetic influence explains 50 percent of observed behaviour among people. We can \textbf{never} say what percentage of an individual's personality or intelligence is inherited. Heritability refers instead to the extent to which \textit{differences among people} are attributable to genes.

As environments become more similar, heredity as a source of differences becomes more important and apparent. 

If all people were in the same environment, heritability would increase. At the other extreme, if people had similar heredities but were raised in widely different environments, heritability would be much lower.

\subsection*{Group Differences}
If genetic influences helps explain individual diversity, the same cannot be said of group differences between men and women or between races. Individual height and weight are highly heritable, yet nutrition influenced these factors more than a century worth of genes.

Heritable individual differences do not imply heritable group differences. If some individuals are genetically disposed to be more aggressive than others, that does not explain why some groups are more aggressive than others. 

\subsection*{Nature vs. Nurture} 
Among our similarities, the most important is our enormous adaptive capacity. Some human traits, such as having two eyes, develop in virtually every environment.

However, go barefoot for a summer and you develop callused feet, a biological adaption. 

Our shared biology enables our developed diversity through adaptation.

\section*{Gene Environment Interaction}
To say that genes and experience are both important is true. But more precisely, they \textbf{interact.} Environments trigger gene activity. Biological appearances have social consequences and our future environments are the result of our inherent personality, thus think not nature versus nurture, rather nature via nurture.

\section*{The New Frontier: Molecular Genetics}
Behaviour geneticists draws on "bottom-up" \textbf{molecular genetics} as it seeks to identify \textit{specific genes} influencing behaviour.
avl of molecular behaviour genetics is to find some of the many genes that influence nromal human traits, such as body weight, sexual orientation, extraversion, etc.

Genetic tests can now reveal at-risk populations for many diseases.

\section*{Evolutionary Psychology: Understanding Human Nature}
Geneticists explore the geneti and environmental roots of human differences. \textbf{Evolutionary psychologists} instead focus on what makes us so much alike to humans. They use Darwin's principle of \textbf{natural selection} to understand the root of behavior and mental processes.

Natural selection works as follows:

\begin{enumerate}
\item Organisms' varied offspring compete for survival.
\item Certain biological and behavioural variations increase their reproductive and survival chances in their environment. 
\item Offspring that survive are more likely to pass their genes to ensuing generations.
\item Thus, over time, population characteristics may change.
\end{enumerate}

\section*{Natural Selection and Adaptation}

Nature has selected advantageous variations from among the \textbf{mutations} (random error in gene replication) and from the new gene combinations produced at conception. Genes and experience together wire the brain. Our adaptive flexibility in responding to different environments contributes to our \textit{fitness}, our ability to survive and reproduce.

\section*{Evolutionary Success Helps Explain Similarities}
Our shared human traits across cultures were shaped by natural selection acting over the course of human nature.

Our behaviour and biological similarities arise from our shared human genome. No more than 5 percent of the genetic differences among humans arise from population group differences. 

\subsection*{Outdated Tendencies}
We are predisposed to behave in ways that promoted our ancestors' surviving and reproducing. We love the taste of sweets and fats, which were hard to come by. 

\subsection*{An Evolutionary Explanation of Human Sexuality}
Men like sex. But no, nature selects behaviours that increase the likelihood of sending one's genes into the future. Social expectations also shape gender differences in mate preferences.










\end{document}
