\documentclass[12pt]{article}
\usepackage{parskip}
\usepackage[margin=.6in]{geometry}
\title{Lecture 7: Motivation}
\begin{document}
\section*{Characteristics}
This is the question of why we behave certain ways. Its an inferred construct presumed to be an intervening variable essential for performance. 
\subsection*{Measuring Motivation}
\begin{itemize}
\item intensity
\begin{itemize}
\item energy, enthusiasm, degree of effort exerted
\item measured by perceived level of effort or by physiological arousal
\end{itemize}
\item direction/choices
\begin{itemize}
\item colitional approach or avoidance of alternative activites
\item selection of outcomes worthy of effort
\item degree of task difficulty selected
\end{itemize}
\item persistence
\begin{itemize}
\item commitment to choices
\item continued effort following frustration (goal-blockage)
\end{itemize}
\end{itemize}
\section*{Mechanistic Approach}
\subsection*{Internal Forces}
We respond very predictably to external stimulus. 
\paragraph*{instinct} is a fixed pattern of behavior that is not acquired by learning and is likely to be rooted in genes and the body. It is most often found all members of the species. The early notion of instinct has been replaced by a different name called Evolutionary Perspective after being adopted by evolutionary theory. For instance the brood defence is the instinct to protect our young. Another well researched one is the rooting reflex which causes babies to suck on their mothers, it is automatic, not learned. Over time this instinct becomes more elaborate. 
\paragraph*{Needs and Drives} is another theory to explain things. Clark Hull was trying to explain psychological behavior in mathematics. Konrad Lorenz studied animal behavior and found the similarities in the behavior of humans, most famous for a book on aggression. He founded the need, drive and catharsis thought about behavior. We have a need to aggress, when it isnt met it causes a drive which causes a change in behavior, once this is fulfilled we dont feel the need to aggress anymore in a cathardic release. He uses the analogy of a boiling pot. Sigmund Freud popularizes this notion in his own theory. These are called \textbf{drive reduction theories} which refers to the idea that humans are motivated to reduce drives and restore homoeostasis. A \textbf{drive} is an aroused/tese state related to a physical need such as hunger or thirst. There are a lot of cases, especially social, where this falls apart. Freud elaborates on these theories in the description of intrapsychic conflicts between the unconscious drives and needs and what we consciously want. All of behavior is caused by a conflict between what we subconsciously want and what we consciously want. We are trying to reduce these inner drives and when we cant it causes anxiety and problems. 

\subsection*{External Forces}
This comes during the behaviorism movement. Needs and drives do not come from the individual, but the evironement. We behave in a way that will more likely cause a desirable outcome. When we find that certain behavior will lead to desirable outcomes we will keep doing that. This is called incentive value. 

\section*{Humanistic Approach}
The cognitive revolution happens and spawns this guy. People Abraham Maslow and Carl Rogers simultaneously, but independantly come up with the idea that humans, unlike animals, have a motovational quest for self actualization (become all you can).

\subsection*{Hierarchy of Needs}
Maslow found the organization of needs. In order to move up from the bottom you need to fulfill the lower items on the pyramid. The problem with this is that no research was done to come up with it. What research there is counteracts this. It has been proven very wrong. The people who reach the top of the pyramid tended to be asses. 
\begin{enumerate}
\item self transcendence - need to find meaning and identity beyond the self
\item self actualization - need to live up to our fullest and uniqe potential
\item esteem - need for self esteem, achievement, compentance, and independance; need for recognition and respect
\item belongingness and love - need to love and be loved, to belong and be accepted; need to avoid lonliness and separation
\item safety - need to feel that the world is organized and predictable; need to feel safe
\item physiological - need to satisfy hunger and thirst
\end{enumerate}

\subsection*{Discrepancy of Theory}
This is Roger's theory. Incongruent selves is where there is a discrepancey between our real self and our ideal self. This causes problems with self-actualization and is very demotivating. Congruent selfs is when those two selves overlap making self-actualization more attainable and is motivating. When we have incongruent selves we can be motivated to makes those into congruent selves or we can become very depressed and give up. We continuously change our ideal self so that we continue to have a discrepancy so we are continuously trying to improve

\section*{Expectancy Approach}
\subsection*{Expectancy-Value Theory}
Henry Murray thought motivation is a product of the outcome expectations and the incentive value of the outcome. Motivation = E * V. We weight the outcome's worth and its probability when deciding on what to do. We find that around the 50\% probability people are very likely to act upon things. This is a sweet zone that is used in all source of a lot of decision making models.
\subsection*{Self-Efficany Theory}
Albert Bandura expands on Murrays theory by adding efficacy expectations and outcome expectations. This accounts for when people are avoidant instead of approach oriented when they face an event that should have high expected value. Efficacy expectations is the connection between the person and their ability to execute the behavior. If we dont feel that we cannot enact the behavior it kills the expected value theory. Confidence is key. Outcome expectations is the connection between behaviors and outcomes. If we dont think that our behavior will lead to the outcomes we wanted we wont enact them. \\
\textbf{Outcome Expectations} are the belief that particular behaviors or behavior will produce certain outcomes.\\
\textbf{Efficacy Expectations} are based on the belief that your possess the skills, experience, strategy, etc to enact the behavior and the beief that you can execute the behavior in the given situation.\\
\subsection*{Recovering Intrinsic}







\end{document}