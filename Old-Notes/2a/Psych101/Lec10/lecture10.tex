\documentclass[12pt]{article}
\usepackage{parskip}
\usepackage[margin=.2in]{geometry}
\title{Dating / Depression}
\begin{document}
\section*{The Dating Marketplace}
Men seek: an attractive, younger, fertile woman.\\
Women seek: masculinity, maturity, affluence.\\
This seemed to hint to dating as a social marketplace. An explanation is that this is an evolved trait, males bring resources and women raise children.
\subsection*{Inclusive Fitness}
the basic motive of evolution is to pass on our genes. Often called a biological imperative. To satisfy this you need to survive to sexual maturity (why teenagers are selfish douches), procreate (why young adults are horney bastards), and then ensure that children reach maturity (why parents are overprotective weirdos).
\paragraph*{Optimal Strategy for Women} 
\begin{itemize}
\item Genetic packet - eggs
\item availability - 1 per month
\item life supply - approximately 400, minus 9 per success
\item optimal strategy - quality - be very selective that you get the best possible partner for your limited supply of eggs
\end{itemize}
\paragraph*{Optimal Strategy for Men}
\begin{itemize}
\item genetic packet - sperm
\item availability - millions per minute
\item life supply - infinite
\item optimal strategy - quantity - have as many as possible
\end{itemize}
\paragraph*{The Family Option}
Men abandon the oppertunistic strategy for paternal investment. Women provide sexual exlusivity (men gain paternal certainty for their investment). Men want healthy, young mate for maximum reproduction opertunities. Women want a stron intelligent mate for protection, security and resources.
\paragraph*{Dating Rituals}
\begin{itemize}
\item women determine "shopping list" of seirable attributes (the genetic pressure is on the women. 
\item women create male competition
\item men demonstrate suitability as a husband and father
\item women select "winners"
\end{itemize}
\paragraph*{Beauty Bias}
People are biased toward beautiful. People are actually more selfconscious if they are beautiful. In one experiment people filled out a survey and received feedback, half beleived they had been watched by the marker and the other half brought it to a man down the hall. Those that thought they had been watched feel that most of the positive feed back was a lie.

\part*{CHUNK OF MISSED LECTURE GET LATER}

\section*{Psychoanalysis}
Freud fpimd tjat imisia; symptoms of patients sometimes improve when repressed inner conflicts and feeling were brought to consciousness. Thanatonic (death) side of personality is taking control. Some techniques:
\begin{itemize}
\item free associations: the patient speaks freely about memories, dreams, and feelings
\item interpretation: the therapist suggests unconscious meanings and underlying wished to help the client gain insight and release tension
\end{itemize}
\section*{Humanistic}
Maslow and Rogers, emphasizes the human potential for growth,m self-actualization and personal fulfilment. Therapy helps to support personal growth by helping people gain self-awareness and self-acceptance. Trying to help people see that the discrepancy between the real and ideal self is natural and not to stress about it. Rogers developed client centred therapy.
\paragraph*{Client based therapy} Rogers didn't like that many of the treatments treated the client as lesser, leaving all of the judgements up to the therapist:
\begin{itemize}
\item being non-directive - the therapist is a guide to let the client find the answers themselves, let insight and goals come from the client, rather than dictating interpretations
\item being genuine - be yourself and be truthful; don't put on a facade
\item being accepting and showing unconditional positive regard - help the client to learn to accept themselves despite any weaknesses, don't tell the client they were wrong
\item being empathetic - demonstrate careful attention to the clients' feelings, partly by reflecting what you hear the client saying. This is done through active listening
\begin{itemize}
\item paraphrase what people say, even echo it back on them
\item ask for elaboration, get them to explore their own statements
\item reflect feelings (guess at them and get the client to confirm)
\end{itemize}
\end{itemize}
\section*{Cognitive Behavior Approach}
The proff's preferred method, also supported by statistics. Focuses on what people are thinking. Helps people alter the negative thinking and behavior that worsens depression. The depression is caused by faulty thought processes. 
\paragraph*{Cognitive Triad} Negative thoughts about self (low self esteem); negative thoughts about ongoing experiences (pessimism); negative thoughts about the future hopelessness).

Depression is associated with:
\begin{itemize}
\item low self esteem - discounting positive information and assuming the worst about self, situation, future
\item learned helplessness - self defeating beliefs such as assuming that one is unable to cope, improve, achieve or be happy
\item rumination - stuck focusing on what is bad
\item depressive explanitory style  - attributing negative outcomes to self and positive to others, which tends to be the exact opposite
\end{itemize}

\section*{Social-Cultural}
Often used together with other forms of treatment. 
\begin{itemize}
\item family therapy - having the family involved in sessions, allows therapist to work on the family system
\item group therapy - a group of people with the same issues working together (costs less, and gets more feedback and support)
\item self help group - people with the same problems helping each other, usually to support people leaving treatment
\end{itemize}












\end{document} 