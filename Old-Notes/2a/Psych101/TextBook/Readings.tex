\documentclass[12pt]{article}
\usepackage{parskip}
\usepackage[margin=.6in]{geometry}
\title{Textbook Reading, after Midterm}
\begin{document}

\section*{Module 26: Introduction to Memory}
Memory is learning that has persisted over time. We cant really remember that well, but we are killer at recognizing things that we had seen before or were familiar to us.
\section*{Information Processing Models}
To remember an event we first \textbf{encode} it to our brain, then \textbf{store} is and later on \textbf{retreive} it.

Connectionism is a model of information processing that veiws memories as emerging from interconnected neural networks. This is hard to imagine so we use an older model.

The three stage processing model was created by Richard Atkinson and Richard Shiffrin:
\begin{enumerate}
\item record data on the fleeting \textbf{sensory memory}
\item pass data to \textbf{short term memory} where it is encoded through rehersal
\item pass it to \textbf{long term memory} for later retreival
\end{enumerate}
This method has its limits so we modified it a little bit. Within short term memory is working memory. This is the active processing of information for decisions. 



\part*{Module 27: Encoding}
Some events are processed as sensory input and others are recorded subconsciously. The ones we are conscious of are encoded, processed into working memory. These are processed further by rehersal and encoded into our long term memory. 
\section*{Automatic Processing}
Your brain is awesome, particularly at multitasking. While you are doing whatever the fuck you do there is a ton of unconscious activity. You \textbf{automatically process}:
\begin{itemize}
\item space - recall the place of something to remember its contents
\item time - recall the sequence of events to find what happened
\item frequency - easy to remember how often something happened
\item well-learned information - easy to recall things that are familiar to you
\end{itemize}
The brain is also awesome at adapting. Even when the letters are flipped or the entire sentence is backwards your brain can very quickly adapt to understand this.
\section*{Effortful Processing}
\textbf{Effortful processing} often produces durable and accessible memories. This is when we learn new data through \textbf{rehersal}. The pioneering researcher of verbal memory	Herman Ebbinghaus learned about how non-verbal memory worked by studying his own. He created a series on all possible nonsense syllables. Then he tried to recall them. Obviously he sucked pretty bad at it, but the more he repeated the sylables, the more he remembered. This proved that the amount learned is proportional to the time spent learning. You remember better by rehersing over a long period of time, if you learn quickly you forget quickly. This effect is called the \textbf{spacing effect} which basically means the longer between practice sessions, the better you learn. The \textbf{testing effect} says that repeated testing on the material also helps memory retention. The final effect on processing is the \textbf{serial position effect} which is that you remember things in a certain order better. This results in us usually remembering the first and last items only.
\section*{Levels of Processing}
When we encode the meaning not the actual content. The three levels of encoding are visual, acoustic, and semantic. Each helps for different situations. One experiment found that people remembered more of a paragraph when given the meaning of it this is called \textbf{semantic encoding}. This is a much deeper form of processing and produces more lasting memories. An odd side effect is we remember particularly well when the things we need to remember are about ourselves. 
\section*{Visual Encoding}
Some of our earliest memories are probably visual. We remember concrete words (that can be mapped to images) much better than we do abstract concepts because they are remembered visually and semantically. We also tend to remember the most vivid events, the highest highs and lowest lows due to the \textbf{rosy retrospection} phenomenon. We relate visual encoding with acoustic encoding through \textbf{mnemonics} which is where we map objects that have similar sounding names to each other. An example is a peg-word system (look it up, this shit is cool) which allows you to map numbers to items and thus remember lists through images. 
\section*{Organizing Information for Encoding}
\subsection*{Chunking} allows us to more easily memorize information by grouping into familiar, manageable chunks. This is often coupled with mnemonics by chunking a set of items and forming a world from their first letters called an acronym.
\subsection*{Hierarchies} are composed of a few broad concepts divided and subdivided into narrower concepts and facts. We like words mapped in a tree shape so we can just follow a path to the end goal. 

\part*{Module 28: Storage}
\section*{Sensory Memory}
If asked to recall a 3x3 grid of letters after a split second people cant do it, but if a tone in sounded to tell them the location of the letter after the image flashed, they got it right almost all the time (this proved that they could see and comprehend the letters in that short amount of time). He found that we have a fleeting photographic memory called \textbf{ionic memory}. For a very short amount of time we have an exact representation of a scene. We have a similar one for auditory stimuli called echoic memory. 
\section*{Working/Short-Term Memory}
Short term memories usually disappear after 10 seconds without active processing. Our max us usually around 5-9 pieces of information. We are better at remembering number than letters, sounds than scenes. Usually we can remember about the same number of words we can speak in 2 seconds. If we chunk the information we can only get around 4 chunks, even less if we separate them with 'the'.
\section*{Long-Term Memory}
Long term memory is like a dynamically allocated array, the more things you add to it the more it can hold. We can see this with the people who can memorize insane amounts of material.
\section*{Storing Memory}
Our brain can store a metric fuck tonne of data, but it doesnt store everything, and it doesnt store it specific physical spots (you can rearrange a rat brain and they will still remember at least some things).
\subsection*{Synaptic Changes} 
Researchers have been trying to find a memory trance and this led them to the study of synaptic connections. The main research on this is done using a slug called the Aplysia who has only 20 000 nerve cells that are large enough to be observed. They classically trained the slug to withdraw its gills when squirted with water. The found that the change before and after conditioning is the slug releases more serotonin at certain synapses. This makes them better at transmitting signals. They found that this happens anytime a area is rapidly stimulated. This heightened sensitivity can last for a while. This prolonged strengthening of potential neural firing is called \textbf{long-term potentiation}. This is the basis for all learning. Many companies want to come up with drugs to increase LTP. Some due this by boosting CREB (a protein that switches genes on and off) so that this leads to increased production of proteins that will help reshape synapses and consolidate a short term memory into a long term memory. Another approach is a drug that boosts glutamate which is a neurotransmitter that enhances LTP. After LTP does happen electric current wont disrupt this memory. It does wipe out short term memory. This is also why people who get knocked out don’t remember what led to it, the short term memory doesnt have time to consolidate.
\subsection*{Stress Hormone and Memory} 
When we are excited or stressed hormones make more glucose energy to fuel the bring causing an increase in activity. The amygdala boosts available proteins for the memory forming areas of the brain. This results in arousal searing certain events into the brain. Very traumatic events create indelible memories. Certain drugs can block the effects of stress hormones causing traumatized people to be able to forget the cause of their trauma. Some scientists call this act of instantly and permanently remembering a arousing even \textbf{flashbulb memory}. Overtime and retelling these memories while still vivid can start to skew the facts, but we still have the utmost confidence in them. When stress is continuous, like in cases of abuse, stress hormones act like acid and corrode neural pathways. Also in moments of stress the hormone can overwrite older memories, like when you forget a speech in front of a crowd.
\subsection*{Storing Implicit and Explicit Memories} Memory is stored in different places based on what kind of memory it is. This is aparent in cases of amnesia, the most famous example of this is patient H. M. who had part of his brain removed surgically which prevented him from creating new memories. He could learn new skills, but the experiences seemed new to him. Another example is Jimmy who had no memories, and thus no sense of time. He freaked the fuck out when he thought he was 19 but the mirror showed a much older man. People with no short term memory can still learn many new skills, but they do not have awareness that they know these things, like being able to navigate a building with being able to give directions.This shows that we have two different memory systems :
\begin{itemize}
\item implicit memory - skills that are learned unconsciously
\item explicit memory - things that we know we know and can describe to others
\end{itemize}
\subsection*{The Hippocampus} This is the part of the brain that lays down new memories. People with amnesia have damage to this. Damage to the left causes problems with verbal memories, and damage to the right causes problems with visual memories. There are different subregions of the hippocampus that take care of different kinds of memories, some for facial memories, some for spacial memories, etc. During sleep there is little activity here since memories are being processed and not formed, but during the day stronger memories are born from high hippocampus activity. Memories are not stored here, just cached and then moved to the cortex for more permanent storage. During sleep these two structures have simultaneous activity rhythms.
\subsection*{Cerebellum} This plays a key role in forming implicit memories. Damage to this prevents classical conditioning.

\part*{Module 29: Retrieval} 
Psychologists categorize something as being memorized if recognizing or relearning has been enhanced. 
\section*{Retrieval Cues} 
Memories are stored in a web of associations, the things that a memory is associated with (the memory of where it happened, who it involved, how you felt, etc) are called retrieval cues. The more of these you have, the better chance you have of remembering it. Mnemonics try to take advantage of these, but the best are ones we encode naturally. When we activate these strings to retreive a specific memory it is called \textbf{priming}. This often happens without us knowing it. 
\subsection*{Context Effects} This is how the setting of a memory influences how it is formed. When you walk into a room and forget why it is because the context you are currently in has changed so the memory, which was associated with the previous setting, is forgotten. The opposite of this is deja vu which is the feeling of something having been repeated in the same contex. This happens more to well-educated, imaginative youths when tired or stressed. It is possibly caused by the situation being laoded with cues that unconsciously retrieve an earlier similar experience. Or it could be caused by being a situation which cues are similar to past cues so you associate the two. Something that deja vu is simply caused by lag in our brain making one signal arrive slightly before the other.
\subsection*{Moods and Memories} Emotions can also be used as retrieval cues. This is also seen in \textbf{state dependant memory} which is when we learn something in one state (drunk for instance) we can remember it much better when we are in that same state again. Memories are also \textbf{mood-congruent} in that being in a certain mood will only remind you of other memories that triggered that same mood (being mad makes you remember things that make you mad). It also primes positive or negative associations (being in a bad mood makes you remember certain events negatively). 

\part*{Forgetting, Memory Construction, and Improving Memory} 
\section*{Forgetting}
If we always remembered everything (there are some people who do) it would clutter our minds and make abstraction much harder. Researchers have come up with 7 ways our memory fails us :
\begin{itemize}
\item 3 sins of Forgetting
\begin{itemize}
\item Absent Mindedness - inattention to details leading to encoding failure
\item Transience - storage decay over time
\item Blocking - inaccessibility of stored information
\end{itemize}
\item 3 sins of Distortion
\begin{itemize}
\item Misattribution - confusing the source of information
\item Suggestibility - the lingering effects of misinformation
\item Bias - belief colored recollections
\end{itemize}
\item 1 sin of Intrusion
\begin{itemize}
\item Persistence - unwanted memories
\end{itemize}
\end{itemize}
\subsection*{Encoding Failure} If we never encode something it doesnt become a memory.As we get older we suck more at encoding. Somethings we encode instantaneously others take effort to encode.
\subsection*{Storage Decay} A german scientist Herman Ebbinghaus came up with a series of nonsensical symbols and used them to test people's ability to remember and found the Ebbinhaus forgetting curve. Initially forgetting is very rapid, but it quickly levels off. There currently isnt much of an explanation for this, but scientists are focusing on the physical storage process.
\subsection*{Retrieval Failure} Some memories are irretrievable because we don’t have enough associations to find it. The more retrieval cues we get the easier it is to find. 
\paragraph*{Interference} This is when learning things interferes with retrieving other. \textbf{Proactive interference} is when something you learned earlier disrupts your recall of something you experience later. The more information you have upstairs the more convoluted the web of associations is. To lighten this up we often forget things that arent pertinent. \textbf{Retroactive interference} is when new information makes it harder to recall something you learned earlier. This is blocked for information learned in the hour before sleep because the opportunity for interfering events in minimized. When people were given those nonsensical syllables right before bed they remembered significantly better. The few seconds before sleep are terrible though, noting gets remembered. On the other hand \textbf{positive transfer} is when old memories help us learn new ones.
\subsection*{Motivated Forgetting} People unknowingly revise their memories as they go. Freud argued that we repress painful memories to protect our self-concept and to minimize anxiety. This is counteracted by the fact that emotional memories, even painful ones, are often remembered the best and mundane ones the easiest forgotten.
\section*{Memory Construction} 
We often construct our memories as we encode them and even alter it.
\subsection*{Misinformation and Imagination Effects} Eyewitnesses could be influenced by the wording of a question in one experiment. The memories are remembered incorrectly. We can even fabricate memories by showing people childhood photos that have been edited. This can also be done by vivid descriptions or continuously imagining something. 
\subsection*{Source Amnesia} We often mistake the source of a memory. Source amnesia is when we remember the event, but not the context in which it was acquired. 
\subsection*{Discerning True and False Memories} The gist of a memory is much more durable and less susceptible to priming which is why psychologists often ask for the gist of a memory rather than certain details. Overtime our memories alter to match our current sentiments, why hypnotically convince ourselves of certain things. This results in many bad eyewitness testimonies, another cause of this is leading questions.
\subsection*{Children Eyewitness Recall} Children are especially prone to priming and leading questions. Researchers can easily plan false memories in the children and have them believe that these events happened to them. They often flush these stories out themselves to make them more convincing. This makes them very sincere and talented liars. They are more accurate if the first person to question them is a neutral party. 
\section*{Repressed or Constructed Memories of Abuse}
Many therapists claim to be able to retrieve memories of childhood abuse, but more often then not they ask such leading questions that the memories are just fabricated. The other side to this is that memories of abuse are often repressed. Its possible that these people were really abused and we are discounting their memories. A committee was gathered and they came up with some rules for protecting abused children and falsely accused adults:
\begin{itemize}
\item Sexual abuse happens - and happens more often then we thought, but there is no survivor symptoms, it can result in many different conditions that can also be cause by other things
\item Injustice happens - some innocent people are found guilty and some guilty people found innocent
\item Forgetting happens - everyone forgets, many abused children didn’t understand what was happening at the time
\item Recovered memories are commonplace - everyone suddenly remembers forgotten things, its whether memories are repressed and can be retrieved that is under question. Memories that surface naturally are more likely to be true.
\item Memories of things happening before age 3 are unreliable - infantile amnesia, the older a child and the more severe the abuse, the more likely it is to be true
\item Memories recovered under hypnosis or the influence of drugs are unreliable
\item memories can be emotionally upsetting - sometimes just remembering or accusing someone causes trauma of its own.
\end{itemize} 
Elizabeth Loftus experimented with planting false memories in children. She had a trusted family member recall for a teenager three real memories and a false one, after a few days they were recounting their emotions during that time, and after a few more days they could even picture details of the memory that the hadn’t even been told. This is the same way that people remember alien abduction and other things. 
\section*{Improving Memory}
SQ3R is a good method to improve memory:
\begin{itemize}
\item Study Repeatedly- study in small chunks rather than one giant session. One method of memorizing is the rehearse the topic over and over with increasing intervals between each rehearsal. 
\item Make the material meaningful - build a network of retrieval cues. take notes in your own words, apply the concepts to your life, relate material to stuff you already know, etc.
\item Activate retrieval cues - to remember, mentally recreate the situation you were in, and look for retrieval cues
\item Use mnemonic devices - make up a story to associate items with memorable images, vivid images and rhymes can act as great retrieval cues, chunk information into acronyms
\item Minimize interference - study before sleeping, don’t study two conflicting subjects near each other
\item sleep more - sleep is where memory is made
\item test your own knowledge
\end{itemize}
\part*{Module 31: Thinking}
When thinking of things we form mental groupings of similar objects. We then sort these conecots into category hierarchies. Objects that more closely match to a concept are recognized more quickly. When something doesnt fit our concept prototypes we have a much harder time recognizing or identifying it.
\section*{Solving Problems}
Some problem are soved with trial and error, most though we solve with algorithms or heuristics.  Often a problem solution just comes to us in a flash of insight. This often results in good feelings, like getting the punch line of a joke.
\subsection*{Confirmation Bias} This is caused by seeking evidence that proves our theories more eagerly than evidence that refutes it. 
\subsection*{Fixation} Once we come up with a solution to an answer, even when thats wrong, its hard to scrap it and come up with something new. Mental fixation is our tendancy to approach a problems using the same theories as has worked for us previously. Functional fixation is that we think in familiar functions for objects (using a coin to turn a screw is odd and such).
\section*{Making Decisions and Forming Judgements}
\subsection*{Using and Misusing Heuristics} Heuristics are mental shortcuts to finding a solution. These result in quick but bad judgements. Reresentativeness Hueristic is judgeing the likeliness of things in terms of how well they represent particular schemas ignorning actual numbers and facts. Availability Heuristic is judging based on how mentally available information is, how vividly we remember things influences how often we think they will happen.
\subsection*{Overconfidence} We tend to overestimate the accuracy of our knowledge and judgements. No matter how often these predictions are wrong, we still tend to feel very confident of them. People with more confidence tend to be happier and make decisions easier. 
\subsection*{The Belief Perseverance Phenomenon} This is a weird tendency to stand by our belief despite all sorts of evidence to contrary. One study found that introducing new evidence for both sides of an argument actually makes either side more sure of its decision. The way to erradiacte this is to consider both sides of the argument first (it works better than asking people to be unbias).\\
\textbf{The Perils and Powers of Intuition} \\
Intuition's Dozen Deadly Sins
\begin{itemize}
\item Hindsight bias - we claim to have known all along things that we only just now know
\item Illusory correlation - finding relationships where there are none
\item Representativeness and availability heuristics - using heuristcs to gain speed but lose some logic when making judgements
\item Memory construction - we can be influenced to create false memories based on our moods and misinformation
\item Over confidence - we assess our knowledge as more accurate than it actaully is]
\item belief perseverence and confirmation bias - our beliefs are very hard to get rid of
\item framing - our judgement changes based on how the information on it is presented
\item interveiwer illusion - confidence is increased based on the interveiwer
\item mispredicting our own feelings - wrong estimation the intensity or duration of emotions 
\item self-serving bias - we assess ourselves as better than we actually are
\item fundamental attribution error - discounting situational forces when assessing the cause of someone's disposition
\item mispredicing our own behaviour - we are always wrong here
\end{itemize}
Evidence of Intuition's Powers
\begin{itemize}
\item blindsight - people with brain damage's bodies react to things unconscious
\item right brain thinking - people with split brains knowing things they cannot explain
\item infants intuitive learning of language and physics
\item divided attention and priming - information processed without us being conscious of it
\item everyday perception - parallel processing and integration of conscious streams
\item automatic processing - our autopilot taking care of many things
\item implicit memory - knowing how to do something without knowing when we learned it
\item heuristics - short cuts to make judgements
\item intuitive expertise - unconscious learning
\item creativity - the spontaneous appearance of ideas
\item social and emotional intelligence - unconscious understanding to how to behave in social situations (including how to perceive emotions)
\item the wisdom of the body - instant responses that pass the brain, hunches
\item thin slices - getting information from very brief slices of behaviour
\item dual attitude system - gut level and rational responses
\end{itemize}
\subsection*{The Effect of Framing} The way that we present information effects how people evaluate it. People find statistics involving numbers far more alarming than those involving percentages. People will more likely chose the option that requires less work from them, make people opt in or out of the option you don't want them to take.

\part*{Module 33: Intro to Intelligence}
\section*{Is Intelligence One General Ability or Several Specific Abilities}
Charles Spearman thought that we had one \textbf{general intelligence (g)} and our special abilities are not part of that intelligence. He came to this conclusion by developing \textbf{factor analysis} which showed people who scored high in one area scored high in the others as well. Another psychologist L.L.Thurstone gave people 56 different tests and mathematically found that people scored well in 7 clusters. He wanted to prove Spearman wrong, but people found that if you did well in one cluster you tended to do well in the others. 
\section*{Theories of Multiple Intelligences}
Scientists now believe that people who do well on one test casuing them to do well on the others is caused by the fact athat different abilities interact and feed on another.
\subsection*{Gardner's Eight Intelligences}
Howard Gardner viewed intelligence as multiple abilities that came in packages. Which explains why people with brain damage or idiot savants can do very badly in some areas and still do well in others. An example of this is Kim Peek who had fantastic memory, but couldn't understand abstract concepts. With this as his evidence Gardner identified 8 intelligences : 
\begin{itemize}
\item linguistic
\item logical-mathematical
\item musical
\item spatial
\item kinesthetic
\item intrapersonal
\item interpersonal
\item naturalist
\end{itemize}
\subsection*{Sterneberg's Three Intelligence}
\begin{itemize}
\item  Analytical (academic problem solving) - assessed by intelligence tests, well defined problems with a correct answer, predict school grades and vocational success
\item Creative - reacting adaptively novel situations and generating novel ideas
\item Practical - sort of msc everyday skills
\end{itemize}
Gardener gave a mass survery to university students testing these three areas of intelligence and found that they more accuratly predict grades and helped eleminate racial bias. 
\subsection*{Intelligence and Creativity}
\textbf{Creativity} is the ability to produce new and valuable ideas. Creativity isnt well measured by IQ tests. This is because intelligence tests require convergent thinking, comming to one answer, and creativity is divergent thinking, comming up with many different answers. Convergent thinking is based around the left parietal lobe and divergent thinking around the frontal lobes. Sternberg identified 5 components of creativity:
\begin{itemize}
\item Expertise - a good base of knowledge to jump off of
\item Imaginative thinking skills - the ability to see things in a new way, recognize patterns and make connections
\item A Veruresome personality - someone who seeks new experiences, tolerates ambiguity and risk, is very persistant
\item Intrinsic Motivation - being driven internally, focus on pleasure of the tasks rather than extrinsic rewards
\item A Creative Environment - having things around to challenge them and peers to consult
\end{itemize}  
\subsection*{Emotional Intelligence}
This is the ability to function in a social environment. Its four main components are :
\begin{itemize}
\item perceive emotions - recognize them in faces, music, and stories
\item understand emotions - predict them and how the change and blend
\item manage emotions - how to exprese them in situations
\item use emotions - manipulate emotions to enable adaptive or creative thinking
\end{itemize}
People with higher emotional intelligence do better socially and rarely get depressed. They tend to look for long term rewards over impulsive decisions. People who are brain damaged to lose emotional intelligence tend to lose the ability to express themselves. One man Elliot became unable to feel emotions. Some scientists find that we stretch the definition of emotional intelligence too far.
\section*{Is Intelligence Neurologically Measurable}
\subsection*{Brain Size and Complexity}
Famous geniuses of the past have been found to have larger or wrinklier than normal brains leading some to believe that brain size can influence intelligence. This is countered by evidence to the contrary. We now know that the size of certain areas of the brain (parietal and frontal lobes) can be indicators of intelligence. People with higher intelligence also end up with more synapses (more learning equals more connections). This can also be seen in that children with higher intelligences had a thinner cortex that thickened slower than their average counterparts. 
\subsection*{Brain Function} 
The frontal lobe of the brain is much more active while contemplating the solution to a problem. People who well on intelligence tests tended to be faster and verbal recognition, speed was indicative of intelligence. 

\part*{Module 34: Assessing Intelligence}
\section*{The Origins of Intelligence Testing}
Charles Darwin's cousin, and English scientists, Francis Galton wanted to measure natural ability to encourage those of high aptitude. He assessed 10000 visitors at a London Exposition with tests based on reaction time, sensory acuity, muscular power, and body proportions. The results were completely unfounded and his theory disproven, but it sparked a debate.
\subsection*{Alfred Binet: Predicting School Achievement}
This all started when France first institutionalized school so they now had students of all ages and skills entering the school system. The school system didnt want teachers judging students' intelligence because they might mistake intelligence for prior learning or simply be biased against the poorer students. Binet started by theorizing that all children develop through the same mental stages and that intelligence should be measured in terms of mental age. They created skill tests and compared the averages for each age, using those to set the bar for what should be expected from children of that age. Binet felt that people could improve their mental age through "mental orthopedics". His test was designed to find french school children who needed special attention. but worried that these labels would limit them.
\subsection*{Lewis Terman: The Innate IQ}
Terman took Binet's work and adapted it for california school children. He shifted the age norms and altered they types of questions to create the \textbf{Standford-Binet} test. This test led to the \textbf{intelligence quotient} this is the mental age divided by chronological age. This relationship only really worked for children. Now IQ is the relation between the results of the individual as compared to the average scores. Terman worked hard to propogate this test as a way to assess a child's "vocational fitness" which alied well with the eugenics movement. Due to this the government started using intelligence tests on immigrants and army recruits. People who didnt speak english well tended to do very badly leding to a large ethnic divide between english speakers believing themselves superior and the rest. 
\section*{Modern Tests of Mental Abilities}
Tests that reflect what you have learned are called acheivement or aptitude tests usually correlate fairly well to intelligence scores. Acheivement tests current ability and aptitude measured future ability. One test is the \textbf{Wechsler Adult Intelligence Scale} which consisted of 11 subtests to test specific kinds of intelligence. 
\section*{Principles of Test Construction}
\subsection*{Standardization}
When testing intelligence we need the scores of a group in order to compare against. We usually find a normal distribution of scores. The middle point set to 100 and the tests are periodically restandardized. The \textbf{flynn effect} shows that over time the average intelligence test scores are increasing as time goes on. No one knows why this happens (many theories, none proven). 
\subsection*{Reliability}
A intelligence test must yeild consistant scores. To check this researchers retest the same people by  giving them the same test twice or splitting it in half and giving them two independant tests. The major intelligence tests used today have a reliability of +.9.
\subsection*{Validity}
This is the extent to which the test actually measures what it set out to. \textbf{Content Validity} is when the test actually tests the pertinent behavior and \textbf{predictive validity} which is the prediction of future performance. As people age these tests lose a lot of their predictive ability. This usually happens due to a narrowing of samples.
\section*{The Dynamics of Intelligence}
\subsection*{Stability or Change}
Psychologists wanted to know if intelligence is constant throughout life. Intelligence tests dont really gain any real weight until a child turns 4. Early reading is also a indicator of future intelligence. Intelligence test scores start to stabilize around 7 years old. SAT and GRE tests have a .86 correlation despite being 5 years apart. During one long term study all children in Ireland were given a intelligence test at 11 years old, nearly 60 years later they found that people who scored high on that initial test were less likely to have altzheimer's disease and tended to live longer. 
\subsection*{Extremes of Intelligence}
\paragraph*{The Low Extreme} These are people whose intelligence scores below 70, they are considered disabled. For a long time people with disabilities were pretty much put in a corner and ignored, but around the turn of the century we started to fix that. Many things such as social welfare cheques, and even getting the death penalty can be dependant on that test score.
\paragraph*{High Extreme} People with high IQ's do not have autism. They are normal people, just smarter. Most go on to have normal successful careers and fit in to society fine. Many school systems have gifted programs to help these kids get a more challenging education. Some people fear this will cause greater segregation. 
\part*{Module 35: Genetic and Environmental Influences on Intelligence}
\section*{Twin and Adoption Studies}
Genes do play a role in intelligence. Identical twins raised together are basically the same, while fraternal twins were not, also identical twins raised apart had very similar scores. Environment also plays a role in intelligence with adoption raising the intelligence of abused children
\section*{Heritability}
Heritability is the measure of deviation from your biologial parents' intelligences. Genes and environment work together to determine intelligence.
\section*{Environmental Influences}
\subsection*{Early Environmental Influences}
Infants that are deprived show extreme drops in intelligence, if they have to control over their environment they feel no need to be intelligent since their environment doesnt respond to them. A psychologist (Hunt) taught caregivers at an Iranian orphanage, to engage the children in games and saw a huge jump in intelligence, which resulted in a raise in adoption rates. Malnutrition and bad teachers also can cause lowered intelligence. Children should be allowed to experience many different things in order to gain intelligence. Basically not lowering your child's intelligence through neglect is the best you can do, there is no real way to increase it above that. 
\section*{Group Differences in Intelligence Test Scores}
\begin{itemize}
\item Spelling - females are better at spelling
\item Verbal ability - females have better verbal ability - remembering words and facts
\item Nonverbal Memory - Females are better at nonverbal memory - finding lost objects, picture associations
\item Sensitivity- females are more sensitive (physically) 
\item Emotion detecting  - females are better at picking up on emotional cues, an experiment showed pictures of emotional faces and asked people to guess what that person was discussing
\item Math and spatial - females are better at computation, males are better at reasoning, most high scorers on SAT were men, etc, 	
\end{itemize}
\subsection*{Ethnic Similarities and Differences}
African Americans tend to score worse than Hispanics who tend to score worse than Caucasians. These scores dont really reflect on the individual, but the group. Language plays a role in that people taking intelligences in a language other than their first language will score much lower. 
\section*{The Question of Bias}
\subsection*{Two Meanings of Bias}
A bias is found when tests are altered by cultural differences. For a while the US required immigrants to take intelligences in English despite their not speaking the language. Sometimes intelligence tests required prior knowledge in order to make associations. A test can be biased by only accurately predicting for one group and not the other, each group must have the same probability for passing.
\subsection*{Test-Takers' Expectations}
When people are taking a test their results can be directly effected by their environment. the \textbf{stereotype threat} is when a person does better on a test when they feel threatened. An example of this is when a minority receives benefits implying them not succeeding causes them to not do as well. African Americans tend to have a sharp drop off in grades during high school, when they are told the most that they wont succeed. 

\part*{Module 26: Introduction to Motivation}
\section*{Instincts and Evolutionary Psychology}
An instinct is any complex behaviour that is consistent through out a species without being learned. Instinct theory doesnt explain humans so well because we learn so much from our environment.
\section*{Drives and Incentives}
\textbf{Drive Reduction Theory} replaced instinct theory with the idea that when we have a psychological need we enter an aroused state that drives the organism to get what is needed. We want to get homoeostasis. 
\section*{Optimum Arousal}
When our needs are met we can also be aroused simply for the sake of being aroused, in this case we seek optimum levels of arousal (think of curiosity).
\section*{A Hierarchy of Motives}
Abraham Maslow came up with a hierarchy of needs, each layer needs to be fulfilled before you can move up the ladder:
\begin{enumerate}
\item Self-transcendence - need to find meaning and identity beyond the self
\item Self-actualization - need to live up to our fullest and unique potential
\item  Esteem - need for self-esteem, acheivement, competence, and independence; need for recognition and respect from others
\item Belongingness  and love - need to love and be loved, to belong and be accepted; need to avoid loneliness and separation
\item Safety - need to feel that the world is organized and predictable; need to feel safe
\item physiological - need to satisfy hunger and thirst
\end{enumerate}
This is sometimes a bit arbitrary, people can shuffle the order of needs when they feel pasionate about things (starving for political statements). The order of needs is often influenced but cultural norms as well.

\part*{Module 37: Hunger}
An experiment (Ancel Keys) that consisted of restricting the amount of food consumed by males to half that needed to conserve their body weight showed them start to conserve energy, become obsessed with food to the point of all other things lossing meaning and purpose.
\section*{The Physiology of Hunger}
Stomaches contract when you are hungry, but when the stomache is removed hunger still exists.
\subsection*{Body Chemistry and the Brain}
We automatically montor caloric intake to maintain a stable body weight. An indicator of caloric intake is glucose levels, secreting insulin from the pancreas decreases this by converting it to stored fat. When glucose drops too low the brain feels hungry, it receives the low glucose signal from the liver or intestines. The lateral hypothalamus is what causes hunger, when glucose gets too low this secretes orexin. The ventromedial hypothalamus depresses hunger, the loss of this results in over eating and getting fat. The hypothalamus also monitors:
\begin{itemize}
\item ghrelin - a hunger arousing hormone secreted by empty stomache
\item obestatin - produced by same gene as ghrelin, sends out fullness signal from stomach to supress hunger
\item PYY - a hormone from the digestive tract, 
\item leptin - a protein secreted by fat cells to diminish the rewarding pleasure of food
\end{itemize}
Animals have a set point that is their stable weight, where they like to be resulting in no feeling of hunger or fullness. The rate at which we take in food and output energy is our \textbf{basal metabolic rate} which tells the amount of energy to maintain the basic body functions at rest. The set point can be changed, especially in the prescence of chocolate.
\section*{The Psychology of Hunger}
People with amnesia will eat again despite having a mean a short while ago simply because they dont remember eating, even though they are not hungy.
\subsection*{Taste Prefrences: Biology and Culture}
We often have cravings for food that will fix other needs as well, for example we eat Carbohydrates when tense or depressed because they boost serotonin. Culture also influterences taste, we dont like to eat food that is unfamiliar to us. Taste can also be adaptive (people is warm climates prefer spices that will keep food from growing).
\subsection*{The Ecology of Eating}
We tend to eat more when around others who are eating. When given the option for a larger portion (larger bowl, or larger scoop) people will always put away more calories, even when not conscious of it.
\subsection*{Eating Disorders}
When ever we fuck with the way we eat our bodies pay for it:
\begin{itemize}
\item Anorexia Nervosa - begins as diet, usually adolescent females, drop 15\% below weight, many also display binge-purge-depression cycle
\item Bulimia Nervosa - begins as diet, eat a large amount called a binge, then they feel guilty for breaking their diet and throw up all of the food in a purge. Weight tends to fluctuate above and below normal weight so its hard to spot
\item Binge-eating - people who eat a large amount of unhealthy foods in a short amount of time, usually results in being overweight
\end{itemize}
Eating disorders are usually caused by environmental factors. Mothers that focus too much on their weight result in anorexic children, obese families breed bulimics. Genetics play a role in the correlation between the likey hood of sets of twins having eating disorders.  The most predominant influence is culture. 

\part*{Module 39: Motivation at Work}
Mihaly Csikszenmikalyi found that there is a certain level of work (not too hard, but not idle) in which we are our most happy. There we experience \textbf{flow}, giving all of our attention to the task at hand to the point of not noticing time passing. Industrial-organizational psychology is how machines and enviornments can be optimally designed to fit human abilites, it has two subfields, personal psychology (the chosing and evaluating of workers) and organizational psychology (how environments and management influence workers). 
\section{Personnel Psychology}
\subsection*{Harnessing Strengths}
Companies want to hire the most suitable employees, some do this by setting interveiw questions and seeing what answers the most successful employees gave and hiring based on that. Gallup researchers tried examining traits of successful employees and comparing them to traits of unsuccessful employees to see what they should look for in the hiring process.
\paragraph*{Do Interviews Predict Performance} Not really. Aptitude tests are much better judges, despite how sure the interviewer is of their judgement.
\paragraph*{The Interviewer Illusion} Interviewers tend to have rediculous, unfounded confidence in their judgements of people: 
\begin{itemize}
\item People are on their best behavior in an interview, a better judge of character is the person we were
\item Interviewers can only track the careers of those they hired, the dont see any success from people they rejected
\item Our behavior is very situational
\item Interviewers can be influenced by mood or priming just like anyone else
\end{itemize}	
\paragraph*{Structured Interviews} This is giving exactly the same questions and script to each interviewee to prevent interviewer bias. Instead of asking them to describe their own traits, ask them to give examples of the skills you are looking for. They also take notes and make judgements during the interview so as to not forget.
\subsection*{Appraising Performance}
Some companies employ a checklist method, just going through a series of specific behaviors and checking which one belongs to the employee. Other companies prefer graphic rating scales which check the perfomance on a five point scale. The last method id behavior rating scales which the supervisor scales which behaviors each employee has. Feed back can come from anywhere in a company. 360 degree feedback  is rating yourself, manager and coworkers. The halo effect is when your rating of a emplyees personality effects your rating of their ability. Lenience and severity errors are the evaluators tendency to be too kind or harsh. Recency errors are when the evaluator focuses on easy to remember current events.
\section{Organizational Psychology: Motivating Achievement}
Motivating employees is hard. Some are \textbf{achievement motivated} where they are motivated simply by wanting to be the best (like no one ever was). The time old proverb a genious will never beat a hard working man comes into play here. To be an expert in a field you need to invest 10 years of hard work
\subsection*{Satisfaction and Engagement}
Satisfaction and productivity are definitely correlated, but psychologist are sure if they are causily related. There have been lots of surveys, but they all just show that the most productive employees are ones that know what is expected of them, have what they need, and  other fairly obvious stuff. Companies with happier employees do make more income.
\subsection*{Managing Well}
\paragraph*{Harnessing Job-Relevant Strengths} Many people think that successful managers should first select the right people for the job, then tailor the job to fit that person's natural abilities. Don’t bother trying to teach people skills they don’t have, instead grow on the skills they do have. The best method is operand conditioning, find a good behaviour and reward it.
\paragraph*{Setting Specific, Challenging Goals} When people layout goals, subgoaslm and implementation intentions they become much more focused and productive. 
\paragraph*{Choosing an Appropriate Leadership Style} There are different types of leading and their use varies based on the situation and individual. \textbf{Task Leadership} is a goal focused type of leadership, it involves setting standards, organizing work, focusing on goals. This works well if the leader is intelligent enough to give good orders. the \textbf{Social Leadership} is all about team work, explaining decisions, mediating conflicts, and building high-achieving teams. This is a more democratic system that allows workers to have a say in things, it often results in good morale. \textbf{Transformational Leadership} encourages the individual to transcend their personal goals in favor of those of the collective. These are very inspiring leaders created more engaged and trusting employees. Women tend to be better at this than men. The best leaders incorporate all of these things. The voice-effect is that if you give employees a say in a decision they respond more positively and are more loyal to that decision.
\part*{Module 40: Introduction to Emotion}
\section*{Theories of Emotion}
Psychologists are currently under debate over the direction of the causality relationship between physical sensation of emotions and mental (do we cry because we are sad or vice versa). the \textbf{James-Lange Theory} is that we first have a physiological response before we experience emotions. This doesn’t really work because our physiological responses arent numbered enough to elicit the vast range of emotions we can feel. So the \textbf{Cannon-Bard Theory} finds that physiological arousal and emotional experience occur simultaneously, they do not cause each other. The final theory is \textbf{two-factor theory} by Schachter and Singer proposes that physical arousal and a cognitive label trigger emotions. 
\section*{Embodied Emotion} 
\subsection*{Emotions and the Autonomic Nervous System}
Our body reacts to a situation without us cognitively causing it. The sympathetic division basically shifts everything to top gear to prepare to fight or flee. Once the crisis is gone the parasympathetic system calms everything the fuck down. 
\subsection*{Physiological Similarities Among Specific Emotions}
There is no way to identify a specific emotion based solely on physiological response, they all physically are the same, nevertheless we still feel each emotion very differently. 
\subsection*{Physiological Differences Among Specific Emotions}
Most of the physical differences between the emotions go on inside the brain. The amygdala's activity level can be an indicator (fear and anger light it up more). Different areas of the cortex can also be for different emotions. Negative emotions tend to light up the left side of the brain while positive tend to come from the right. This also shows in personality. This may be explained by the left side of the frontal lobe containing more dopamine receptors. People with spinal injuries do have less emotional response to things which supports the James-Lange theory
\subsection*{Cognition and Emotion} 
\paragraph*{Cognition Can Define Emotion} The spill over effect is when our arousal to to event spills over to the next event. An experiment was run where participants were injected with epinephrine and then put in a room with someone acting euphoric or irritated. The participants who were told that the drug would have a physical effect felt no emotion, but the participants that were told the drug would have no efect found that their emotions started matching that of the person in the room with them. Arousal can be attributed to any emotion we want it to enfocing the two-factor theory. Arousal fuels emotion; cognition channels it.
\paragraph*{Cognition Does Not Always Precede Emotion} When an emotion is aroused, then you are distracted by something the motion may disappear, but the lable of the emotion (that nagging feeling) may still linger. People are very good at recognizing emotional content subconsciously and we can even be primed by it without knowing. The explanation for this is that some emotions can bypass the cortex from eye or ear to the thalamus to amygdala. This is why our emotions can hijack our brain. 

\part*{Module 45: The Psychoanalytic Perspective of Personality}
\section*{Exploring the Unconscious}
Freud came to his conclusions about the unconscious after observing people with psychological causes for neurological disorders. At first he though hypnosis was the key to getting at the unconscious but when his patients showed uneven responses he started \textbf{free association}. This is the process of getting the patient to speak their mind no matter how trivial the thoughts to try to trace back to the unconscious memories that are causing these problems and release them. This theory of personality and its treatments is called \textbf{psychoanalysis}. Frued was very interested in the mass of unacceptable thoughts we repress. He beleived that everything had a cause. He felt that jokes and dreams were great expressors of the subconscious. 
\subsection*{Personality Structure}
According to Freud personality is the result of trying to resolve the conflict between wanting to act on our impulses without feeling guilt for doing so. Basically three systems acting against eachother:
\paragraph*{ID} this is the part of our subconiscious that tries to satisfy basic urges. Operates on the pleasure principle.
\paragraph*{EGO} ties to satisfy the id's impulses in realistic ways. Operates on the reality principle, tries to get what you want without getting in trouble.
\paragraph*{SUPEREGO} This is the moral compass of a person. tries to do what is right, not what you want.
\subsection*{Personality Development}
Freud layed out a few stages of psychosexual development:
\begin{enumerate}
\item Oral(0-18 months) - pleasure centers on the mouth
\item Anal(18-36 months) - pleasure centers on bowl movement and coping with demands for control
\item Phallic(3-6 years) - pleasure center is genitals; coping with incestuous feelings. here is where the Oedipus complex forms (out of fear of castration by their father children feel guilt over their sexual interest in their mother). Children try to become like their rival parent and repress these feelings, this is how children's superego gains control through adopting their parents values. 
\item Latency(6-puberty) - dormant sexual feelings
\item Genical(puberty-on) - maturation of sexual interests
\end{enumerate}
\subsection*{Defence Mechanisms}
Anxiety formes when we must ignore our inner desires in favor of existing in a society. To avoid this anxiety we develop defence mechanisms:
\begin{itemize}
\item repression - pushes anxiety causing impulses to the unconscious. The is usually incomplete and these anxieties often bubble to the surface
\item regression - retreating to an earlier stage of development 
\item reaction formation - making unacceptable impulses look like their opposites
\item projection - attributes threatening impulses to others
\item rationalization - generate self-justifying explanations to hid the real reason for our actions
\item displacement - diverting wishes (usually sexual or aggressive) toward an object
\item denial - rejecting a fact or its seriousness
\end{itemize}
\section*{The Neo-Freudian and Psychodynamic Theorists}
This is the name for the group of physicians who followed Freud and his theories. They agreed with his theory of id, ego, superego, but they diverged. They placed more emphasis on the conscious' role and they doubted that sex and agression were the only motivations.  

Alfred Adler and Karen Horney believed that childhood revolved around social tensions and not sexual. Adler (founder of inferiority complex) beleived that behavior is driven by efforts of over come childhood feelings of inferiority. Horney disagreed that women have weak superegos and penis envy.

Carl Jung beleived that we have a collective unconscious that contains images derived from our species' universal experiences.

\section*{Assessing Unconscious Processes}
The problem with assessing the unconscious is that you must first access it so objective assessment tools dont work as they only tap the conscious. The main way to do this is through \textbf{projective tests}, getting people to extrapolate from an ambiguous situation. One test is the \textbf{Thematic Apperception Test} which shows an ambiguous situation and asks the patient to make up a story about it. The most famous projective test is the \textbf{Rorschach test} which asks people to describe an ink blot. The problem with these is that they can easily be altered by priming or bias on the researcher's part.  Software has been developed to help remove some of this bias. They are not very reliable and often not used to make any concrete diagnoses. 
\section*{Evaluating the Psychoanalytic Perspective}
\subsection*{Contradictory Evidence from Modern Research}
Many of Freud's theories are off in their details. Developement is life long, sexual identity happens much earlier, etc. Regression is much rarer than Freud thought, often trauma further ingrains a memory rather than wipes it.

\part*{Module: The Humanisitc Perspective of Personality} 
\section*{Abraham Maslow's Self-Actualizing Person}
This is the same guy who came up with hierarchy of needs, the ultimate goal of which is self actualization and self transcendence. It is through the pursuit and definition of this that personality forms.
\section*{Carl Rogers' Person-Centered Perspective}
People are basically good and its only if the environment stopping them that they do stive for self actualization. Our environment must be genuine, accepting, and empathetic:
\section*{Assessing the Self}
One way to measure personality to to get them to fill out a survey to describe themselves. Rogers also asked them to describe their ideal self and measure the discrepancy. Some psychologists feel that questionnaires are depersonalizing and prefer face to face interaction.
\section*{Evaluating the Humanistic Perspective}
Many people feel that the concepts of humanism are too vague and subjective. They also found that the individualism of the humanistic approach can lead to self-indulgance instead of focusing on others. The main objection is that it doesnt take into acount the human capacity for evil. 

\part*{Module 47: Contemporary Research on Personality}
\section*{The Trait Perspective}
Gordon Allport was the first to describe personality in terms of fundamental traits after meeting Freud and concluding that he ignores manifest motives. Isabel Briggs Myers brached off that idea to sort people based on their responses to 126 questions according to Jung;s personality types.
\subsection*{Exploring Traits}
\paragraph*{Factor Analysis} this is a statistical approach used to identify clusters of tests items that tap basic components of intelligence. Psychologists try to reduce these down to a few dimensions that everyone measure differently on. 
\paragraph*{Biology and Personality}
Many personality traits that can be mapped to different parts of the brain. Extroverts seek more stimulation because their normal brain arousal is relatively low, they also are more impulsive because they have less activity in their frontal lobe. Genes also play a large role in our personality which is why dog breeders can breed for aggressiveness or passiveness. 
\subsection*{Assessing Traits}
The most common method of assessing personality traits is a \textbf{personality inventory} which is basically a long questionairre that surveys multiple areas of personality. The classic example of this is the Minnesota Multiphasic Personality Inventory. It is often used to diagnose abnormal personality traits. It was created by Starke Hathaway who likened its development to that of Binet's intelligence test in that all of its items were experimentally derived. From a large pool of itemsd they selected those on which particular diagnostic groups differed. Then they grouped the questions into 10 scales to measure people on. These tests are much more objective than the Roschach inkblots, but people can still lie on their responses so the test often also contains a hidden lie scale to try to catch people giving the answer they think is correct rather than the one they actually fit with. 
\subsection*{The Big Five Factors}
Over the years psychologists have identified 5 basic traits to describe personlaity:
\begin{itemize}
\item Conscientiousness
	\begin{itemize}
	\item organized $\longleftrightarrow$ disorganized
	\item careful $\longleftrightarrow$ carelessness
	\item disciplined $\longleftrightarrow$ imulsiveness
	\end{itemize}
\item Agreeableness
	\begin{itemize}
	\item soft-hearted $\longleftrightarrow$ ruthless
	\item trusting $\longleftrightarrow$ suspicious
	\item helpful $\longleftrightarrow$ uncooperative
	\end{itemize}
\item Neuroticism
	\begin{itemize}
	\item calm $\longleftrightarrow$ anxious
	\item secure $\longleftrightarrow$ insecure
	\item self-satisfied $\longleftrightarrow$ self-pitying
	\end{itemize}
\item Openness
	\begin{itemize}
	\item imaginative $\longleftrightarrow$ practical
	\item variety $\longleftrightarrow$ routine
	\item independent $\longleftrightarrow$ conforming
	\end{itemize}
\item Extraversion
	\begin{itemize}
	\item sociable $\longleftrightarrow$ retiring
	\item fun-loving $\longleftrightarrow$ sober
	\item affectionate $\longleftrightarrow$ reserved
	\end{itemize}
\end{itemize}
These traits then to be quite stable during adulthood (occasionally wander during midlife). Conscientiousness ruses during 20's and agreeableness rises during 30's. These traits are genetically driven about 50\%. These traits can be good indicators fro other personal attributes. 
\subsection*{Evaluating the Trait Perspective}
Some people find that personality traits are stable and others find that they vary situation to situation.
\paragraph*{The Person-Situation Controversy}
People's personality is usually determined by the interaction between their inner traits and the environment that they are in. Because of this it is hard to find traits that are enduring and always present. Our personality tends to stablize more as we age. While these traits are fairly stable our individual behaviors are not and can very situation to situation despite our personality traits. This is why the results of a personality test are terrible predictors for behavior. Personality is a reflection of the average behavior nto the specific. We often hide our personality when in new or formal situation so when we are in a familiar one it appears and is very consistent which is why people can identify personality after a very small chunk of time. Some people are naturally more expressive than others and vice versa and its very hard to get them to act the opposite. 
\section*{The Social-Cognitive Perspective}
This is a perspective on personality proposed by Albert Bandura which emphasizes the interaction of our traits with out situations. We learn our behavior by observing and mimicing others. 
\subsection*{Reciprocal Influences}
The person-environment interaction is \textbf{reciprocal determinism} meaning that it combines past behavior, internal factors, and environmental factors. Different people choose different environements, our personalities shape how we interpret situations, and our personalities help create situations to react to. 
\subsection*{Personal Control}
This is whether people feel they are controlling or be contolled by, their environment. 
\paragraph*{Internal Versus External Locus of Control} 
People tend to have wither a external locus of control (chance or outside forces determine their fate) or an internal locus of control (we control our own destiny). People with an internal locus of control tend to be better off.
\paragraph*{Depleting and Strengthening Self-Control}
Willpower works a lot like exercise and can even make you more tired and less active after prolonged use. It can be trained so that you have more of it, just like muscles. 
\paragraph*{Learned Helplessness Versus Personal Control}
When people feel helpless and oppressed they have a external locus of control which often makes them feel more helpless and less likely to act in the future. this passive resignation is called \textbf{learned helplessness}. People in unfamiliar or uncontrolable situations (new culture, prison, etc) tend to have very low morale, but giving them some opportunity to exert control quickly boosts that. On the other hand too much freedom can lead to indicision and depression as well. When given too many options people experience the \textbf{tyranny of choice} making them much more likely to regret their choice.
\paragraph*{Optimism Versus Pessimism}
People who feel helplessness often have a pessimistic personality type. Too much pessimism negatively effect aspect that requires effort (school grades, work satisfaction). It can also neatively effect their health and relationships. Optimistic people expect the best from others and usually get it. Too much optimism can be bad as well. It often makes people overconfident and prone to failure, it also blinds us to risk causing disasterous risk taking personalities. Near dead lines this over abundance of optimism vanishes and we feel as much doubt as anyone else. The unfortunate thing is that incompetant people tend to feel the most optimistic about their results, this mostly caused by being too incompetent to know you are incompetent. 
\subsection*{Assess Behavior in Situations}
Psychologists like to try to predict behavior by obseving behavior in similar situations. This method of evaluation is much better at assessing behavior and skill than writen tests or oral interveiws. A prime example of this is the intern system  used by many companies. The overall best way to assess a person's behavior is to look at their past behavior.
\subsection*{Evaluating the Social-Cognitive Perspective}
Some people find that the social-cognitive perspective focuses on the situation too much and ignores the person. 
\section*{Exploring the Self}
For a very long time exploring the self has been a large component of psychology. Some researchers have formed the threory of possible selves which is that you mentally store multiple versions of yourself in the future, some you stive towards and others you fear. Those with clear visions of a possible self to strive towards often do better than those without. This self focusing persepective can result in the \textbf{spotlight effect} which makes us think that others are noticing us far more than they actually are.  
\subsection*{The Benefits of Self-Esteem}
People with high self-esteem tend to be better off and achieve more. Its hard to measure the effect of high self-esteem because the direction of causality (self-esteem to success) can run both direction. Its much easier to measure the effect of low self-esteem. If you lower a persons self-esteem they are more likely to put down others or be overly judgemental.
\subsection*{Self-Serving Bias}
Very few people have no self-esteem, most of us like ourselves at least a little. This is a natural \textbf{self-serving bias} which is our readiness to perveive ourselves more favorably. We like to accept credit for good things and externalize credit for bad things.  We also tend to veiw ourselves as better than average, this is due to an overestimation of ourselves not an underestimation of others. We rationalize self serving bias by:
\begin{itemize}
\item remember and justify past in self-enhancing ways
\item overconfidence in beliefs and judgements
\item overestimate how we would ideally behave in difficult situations
\item seek favorable information
\item quicker to beleive flattering descriptions
\item overestimate the commonality of our faults and rareness of our skills
\item group pride
\end{itemize} 
This self-serving bias often results if very bad things (like holocaust bad) where why blame others and then start to find fault in them after blaming them for our faults. When someone with high self-esteem get it deflated they are much more aggressive and violent to others. Over recent years there has been a increase in narcisim, not wholey for the better. 

\part*{Module 48: introduction to Psychological Disorders}
\section*{Defining Psychological Disorders}
The basic definition is patterns of thoughts, feelings, or behaviors that are deviant, distressfull and dysfunctional. The definition of deviant often varies with context and culture. The vagueness of these definitions lead to many 'psychological disorders' being removed from that list, the prime example of this is homosexuality. Usually being deviant only matters if it causes that person distress, usually termed harmful dysfunction. 
\section*{Understanding Psychological Disorders}
\subsection*{The Medical Model}
For a very long time mental health patients were treated inhumanly (trephination, castraton, removing organs, etc) because they thought the cause of their 'insanity' was demons or other such supernatural things. Philippe Pinel help change that and got france to treat their patients much better. After the discovery that syphilis infects the brain hospitals started understanding that the mind could get sick and therefore could also be cured. Now mad people were sent to hospitals instead of asylums and scientists stared looking for causes of mental disorders. They began to use the \textbf{medical model}: A mental illness (psychopathology) needs to be diagnosed based on its symptoms and cured through therapy.
\subsection*{The Biopsychosocial Approach}
Since all behavior is a mixture of internal physiological factors and external environmental factors a person's environment can also be fixed to cure their mental illness. Some illnesses are global and others are cultural (the causes are global, but the symptoms are culturally based). So we must assess the whole situation before giving a diagnosis or treatment.
\section*{Classifying Psychological Disorders}
A large portion of psychology is devoted to the classification of disorders as each classification can tell alot about what symptoms to expect and how to cure them. The current manual for diagnosis and classification is the Diasnostic and Statistical Manual of Mental Distorders (fourth edition, text revise) or DSM-IV-TR. Most insurance companies required a International Classification of Disease (based on DSM-IV-TR) before they will pay for treatment. It just lists symptoms and classifies them into diagnosis and treatments, it does not discuss the possible causes. It basically standardizes the procedure, even giving the exact questions to ask. The problem is it casts a very wide net leading to over diagnosis.
\section*{Labeling Psychological Disorders}
An experiment was done where people checked into a hospital complaining of hearing voices, but otherwise being completely normal. Naturally they were diagnosed and given treatment. At the end of this they stopped pretending to the have the symptoms, but they psychologist started finding causes for this condition and even started misinterpreting their actions as symptoms. Similarly people judged psychiatric patients much harsher than regular people even in an experiment where they responded to questions exactly the same.
\section*{Rates of Psychological Disorders}
Americans seems to have a higher prevalence of mental disorders, even for immigrants from other countries, it seems being in the country increases your likelihood. Poverty is also a good predictor of mental disorders (not sure which way the causation goes). Most disorders appear by early adulthood. 

\part*{Module 49: Anxiety Disorders}
\section*{Generalized Anxiety Disorder}
A generalized anxiety disorder is characterized by unfocused, out-of-control, negative feelings. Basically its an ever present sense of worry that influences a person's life. Most peoplw with this cannot identify (or avoid or deal with) the source of the anxiety. This can lead to high blood pressure. This is often caused by being maltreated or inhibited as a child, it mellows with age. 
\section*{Panic Disorder}
An anxiety tornado. In more precise words its an very quick and alarming event. For a few minutes the person is filled with the intense fear that something horrible is about to happen causing their heart rate to increase, shortness of breath, and other physical symptoms. 
\section*{Phobias}
This is an anxiety disorder centered around an inrrational fear. Some phobias are very specific (fear of fried pickles) and others very general (social fear). People who have had multiple panic attacks may begin to fear fear itself and avoind places where those panic attacks happened. If it gets all-encompasing it may turn into agoraphobia. 
\section*{Obsessive-Compulsive Disorder}
This is defined as obssessive thoughts or complulsive beavhiors athat intergere with everday living and cause distress to that person.
\section*{Post-Traumatic Stress Disorder}
When a traumatic event creates recurring memories, flashbacks, nightmares, withdrawl from the social, anxiety, insomnia, etc. The more emotionally connected ti ab event you are or the more often it happens the worse the PTSD you get from it is. Some people beleive that it is over diagnosed based to the vagyeness of the word trauma. Getting patients to relive their trama is a bad idea and often only makes things worse. The people that do survive a traumatic event without PTSD often react in the opposite way by trying to correct the things that were done to them (holocaust survivors being very altruistic), often called \textbf{post-traumatic growth}.
\section*{Understanding Anxiety Disorders}
Freud beleived that from childhood onward people repress intolerable impulses, ideas and feelings these attempting to surface is what causes anxiety.
\subsection*{The Learning Perspective}
\paragraph*{Fear Conditioning}
We can create anxieties using classical conditioning. Often anxiety comes from expecting something bad so a predictable pain will cause it as the cause of that pain approaches. \textbf{Stimulus generalization} is when a specific event creates a phobia towards all similiar events. This coupled with \textbf{reinforcement}, the act of feeling better by avoiding a source of anxiety, creates anxiety disorders.
\paragraph*{Observational Learning}
This is when a phobia is learning be observing others' fears.
\subsection*{The Biological Perspective}
\paragraph*{Natural Selection}
Humans are more likely have phobias toward things that would have been a threat to our ancestors. These fears are much harder to extinguish. The opposite is true in that we get phobias from things that are a threat to us, but not toward our ancestors (air raids in WW2, muggers, etc).
\paragraph*{Genes}
Some people are much more anxious than others implying that there is a genetic factor. People who are related are more likely to have the same level of anxiety. Genes influence anxiety by controlling neurotransmitters serotonin and glutamate. 
\paragraph*{The Brain}
Anxiety disorders are caused by over-arousal in the areas of the brain involved in impulsive behaviour and habitual behaviours. The area of the brain associated with error checking (the anterior cingulate cortex) is hyperactive for people with OCD. Similarly fear circuits can form in the amygdala as a reult of trama. This is why some anti depressants that damped the amygdala can help with OCD.

\part*{Module 50: Dissociative, Personality, and Somatoform Disorders}
\section*{Dissociative Disorders}
These are when a person experiences a sudden loss of memory or change in identity, basically any dissociation between conscious and unconscious.
\subsection*{Dissociative Identity Disorder}
This is a massive dissociation of self from ordinary consciousnesses causing multiple identities that swap control. 
\subsection*{Understanding Dissociative Identity Disorder}
A researcher got his students to pretend they were on trial for murder and put them under hypnosis and most of them came up with an alternate personality. He theorized that DID was an extreme version of our ability to have multiple selves for each situation. People with DID are very susceptible to hypnosis so it is possible that their diagnosis is actually just triggering fantasy role-playing on one of those selves in fantasy prone people. This make diagnosis of this very suspect along with the fact that more DID diagnosis happen in north america than anywhere else where they are extremely rare and that the number of diagnosis went from 2 per decade to 20000 once it was listed as an actual disease. Many beleive it is a cultural phenomemon caused by therapists looking specifically for DID and "finding" it. The proof for DID's exists does like in the physical in that glasses perscription, handedness and brain activity changes between personalities of DID people in a way that it does not for people faking it. Psychoanalysis see DID as the subconscious creating another personality to vent unaccpetable feelings and bahavior through. Learning theoriests see DID as behaviors reinforced by andiety reduction. People with DID often have horrible and traumatic childhoods.
\section*{Personality Disorders}
These are any enduring habior patterns that impair one's social functioning. These tend to come in three groups:
\begin{itemize}
\item expressing anxiety (avoidant PD)
\item expressing eccentric behaviors (schizoid PD)
\item dramatic or impulsive behavior (histronic PD, narcissitic PD)
\end{itemize}
\subsection*{Antisocial Personality Disorder}
Basically this is the absence of a conscience, usually apparent by 15. They usually have little fear if any and often feel little. These people usually cannot function in society and often end up as criminals or in extreme cases serial killers.
\subsection*{Understanding Antisocial PD}
There is a biological aspect as people with a genetic connection to someone with antisocial PD have an increased likelihood of having it as well. People with antisocial PD show little autonomic system reaction when anticipating pain or anxiety when committing a crime. It is spotted very early on in children with less fear and less concern with social rewards. Channelled correctly they can become quite successful as athletes or politicians or even as criminals. Convicted murders had decreased activity in the frontal lobe and even had less frontal lobe tissue, these finding were even more so in people who had commited impulsive murders. People with APD often suck at planning, organization, and inhibition all of which are controlled by the frontal lobe. Unfortunatly it cant be all biological because violent crime rates have risen very quickly to much to be accounted for in a change in genes. Think of Australia which was colonized by criminals but has low crime rate. Environment also plays a factor in creating APD in individuals. 
\section*{Somatoform Disorders}
These are disorders where the symptoms take shape in psychical (somatic) distresses. In Western cultures we are more likely to find an emotional cause for things that cannot be explained physically than our eastern counter parts. \textbf{Conversion Disorder} is a somatic disorder caused by anxiety converted into a physical form. This results in a physical ailment that is impossible to cause physically, like losing feeling in only your right wrist or something similar). \textbf{Hypochondriasis} which is where people interpret normal sensations as the symptom of some horrible disease. 

\part*{Module 51: Mood Disorders}
\section*{Major Depressive Disorder}
Depression is considered the common cold of psychiatry, very common, but not very serious. It is often caused by a past event, unlike anxiety which is caused by a future one. Depression works like physical pain in that it gets us to slow down, calm down and ruminate on life. \textbf{Major Depressive Disorder} is when at least five signs of depression last two or more weeks not caused by something physical (drugs or medication).
\section*{Bipolar Disorder}
This is the fluctuation between major depressive episodes and episodes of mania (euphoric, hyperactive, wild optimism). Often the natural moodiness of teenagers is attributed to bipolar disorder causing it to be over-diagnosed. During mania a person with BD is very inhibited and impulsive and often needs to be protected from themselves. When people are told to read something out loud very quickly they experience a feeling of confidence and arousal which is very similar to mania. People with BD tend to be in the creative fields where they can have lots of productivity in a short amount of time and none during their period of depression. They do not fair well in jobs that require precision and logic. 
\section*{Understanding Mood Disorders}
Any theory of depression must be able to explain the facts that:
\begin{itemize}
\item Many behavioral and cognitive changes accompany depression 
\item Depression is widespread
\item Compared with men, women are nearly twice as vulnerable to major depression
\item Most major depressive episodes self-terminate
\item Stressful events related to work, marriage, and close relationships often precede
depression
\item With each new generation, depression is striking earlier (now often in the late
teens) and affecting more people
\end{itemize}
\subsection*{The Biological Perspective}
\paragraph*{Genetic Influences}
You risk for mood disorders increases if you have family with one. Scientists have yet to isolate the gene or groups of genes that cause mood disorders.
\paragraph*{The Depressed Brain}
Brain activity significantly slows down during a depressed state. The left frontal lobe (incharge of positive emotions) is usually inactive as well. People with severe depression have smaller loves than normal people. They hyppocampus is very vulnerable to stress related damage will often be different for people with severe depression. Norepinephrine (boosts arousal) is usually scarce and serotonin (also used for arousal) is as well. Drugs to releive depression often block the reuptake of these.
\subsection*{The Social-Cognitive Approach}
Depressed people often have self-defeating believes and an negative explanatory style. 
\paragraph*{Negative Thoughts and Negative Moods Interact}
Self-defeating beliefs probably come from learned helplessness. Depression often comes from uncontrollable events. Some theorists believe that women suffer more from depression because they remember such events much more clearly. People with depression respond to bad events with a interal explanation style (blame to me) and good events with an external (credit to them). Depression is less common in communities that focus of the group instead of the individual. The direction for causality with these things is currently unknown.
\paragraph*{Depression's Vicious Cycle}
The problem is that depression leads to symptoms that lead to more depressions causing a positive feedback loop. People who are depressed are more likely to have depressing events (getting fired, divorced, etc).

\part*{Module 52: Schizophrenia}
\section*{Symptoms of Schizophrenia}
\subsection*{Disorganized Thinking}
This is shown as fragmented, bizarre thoughts. Usually this results in false beliefs and delusions. People with paranoid tendencies are prone to delusions. Often patients jumble ideas creating word salad. Disorganization may result from a breakdown in selective attention, people with schizophrenia cannot filter out external stimuli and focus on what they are doing. 
\subsection*{Disturbed Perceptions}
Hallucinations are very common for people with schizophrenia, auditory being the most prevalent. Most often these are usually voices that insult or order the patient. 
\subsection*{Inappropriate Emotions and Actions}
People with schizophrenia have the most inappropriate emotions for that situation or have none at all called the flat affect. Their actions can also be senseless and compulsive or they may just suddenly freeze in place for hours on end. 
\section*{Onset and Development of Schizophrenia}
It usually sets in for teens entering adulthood. It effects all nationalities and genders equally. For some its onset is sudden as if a reaction to stress and for others it develops gradually. Patients with schizophrenia can have positive symptoms which are inapproriate emotions and actions or negative symptoms which are a lack of emotions or actions. \\
Subtypes of schizophrenia:
\begin{itemize}
\item paranoid - delusions of persecution or gradiosity
\item disorganized - disorganized speech or behavior, or flat or inappropriate emotion
\item catatonic - immobility, extreme negativism, repeating others' speech or movement
\item undifferentiated - mix of symptoms that doesnt fit in above catagory
\item residual - recovery after delusions have disappeared
\end{itemize}
People who have sudden onset schizophrenia have a much greater chance of recovery than those with gradually developing (chance is very slim for them). Gradual developers usually have negative symptoms and sudden usually have posistive.
\section*{Understanding Schizophrenia}
\subsection*{Brain Abnormalities}
\paragraph*{Dopamine Overactivity}
The brains of schisophrenia patients have an excess of dopamine receptors by a factor of 6. Scientists think that this is what causes positive symptoms and have found that drugs that negate this do help. These drugs do not help negative symptoms though. Impaired glutamate activity might be the cause of negative symptoms.
\paragraph*{Abnormal Brain Activity and Anatomy}
People with schizophrenia have decreased activity in the frontal lobe and brain waves that imply out of sync neural firing. Hallucinations cause increased activity in parts of the brain, one of which is the thalamus which filters incoming sensory signals and transmits them to the cortex. Their brains also often have large fluid filled ares and shrinkage of the cerebral tissue nearby. This is also found in the brains of some people who would later on get schizophrenia. Basically schizophrenia is not just a problem with one portion of the brain, but a combination of problems with several regions and their connections. Two causes that are known for schizophrenia are low birth weight and reduced ozygen deprivation during delivery. 
\paragraph*{Maternal Virus During Pregnancy}
Viruses contracted by the mother during pregnancy can increase the risk of schizophrenia. A flue epidemic can increase schizophrenia as can: being born in a densly populated are, where viral diseases spread more readily, being born during flue season, and other virus infection related things. 
\subsection*{Genetic Factors}
The odds of you getting schizophrenia greatly increase if a family member has it so there must be a genetic correlation. This correlation increases among twins, then identical twins, then twins that shared the same placenta (get the same viruses). We currently havent identified the specific genes that cause it and nature plays a large part in these genes being activated.
\subsection*{Psychological Factors}
Its very hard to isolate an environmental cause of schizophrenia it doesnt seem to stem much from the environment whilst developing. Research continues here.

\part*{Module 53: The Psychological Therapies}
Most psychotherapists use the eclectic approach which blends different therapies.
\section*{Psychoanalysis}
Very few therapists practice the same therapy as Freud, but they do use some of his techniques and assumptions.
\subsection*{Aims}
The goal is to brain the repressed impulses to the patients conscious awareness. Get the patient to investigate their past to attempt to find causes for their problems. Healthy living comes from releasing the energy used to maintain the id-ego-superego conflict.
\subsection*{Methods}
Focuses on the formative power of childhood experiences. At first Freud tried hypnosis and found it too unreliable. Then he started using free association. You lie on a couch and say whatever comes to your mind. Patients still edit their thoughts as they go pausing before saying something shameful or embarrassing. These blocks in the flow of your free association indicate resistance to that thought belaying the underlying anxiety. Dreams were another method of analysis as their content suggested some meaning. 
\subsection*{Therapy}
Therapists attempt to understand patients current symptoms by focusing on themes across important relationships. this therapy is done face to face, once a week for a few months. Interpersonal psychotherapy is focused on curing the symptoms instead of removing the root cause.
\section*{Humanistic Therapies}
These attempt to boost self fulfillment by helping people grow in self awareness and self acceotance. This is done by trying to reduce inner conflicts so in some ways humanistic therapies are grouped with psychoanalytic as insight therapies. They differ in that humanistic focuses on the present and future (instead of past), the conscious (instead of hte subconscious), taking responsibility for one’s feelings (instead of uncovering hidden determinants), and promoting growth (instead of curing illness). Carl Rogers developed \textbf{client centered therapy} which is a non-directive (therapist makes no judgements or interpretations and tries to keep from directing the client in anyway) therapy focusing on the patients' conscious self perceptions. Therapists need to be genuine, accepting, and empathetic. Another technique of Rogers' is \textbf{active listening} which involves echoing, restating and seeking clarification when the patient speaks.  He calls this creating a environment of \textbf{unconditions positive regard}. 
\section*{Behavior Therapies}
These therapists view the behavior in itself as the problem and seek onlu=y to rid the patient of their learned behavior and not look closely at the cause.
\subsection*{Classical Conditioning Techniques}
The name is pretty explanatory, applying classical condition techniques to get a patient to stop and undesirable behavior. To get rid of phobias they use \textbf{counter conditioning} where they try to associate the fearfull stituation with something more relaxing. This is done through exposure therapy and aversive conditioning.
\paragraph*{Exposure therapies}
This is gradually exposing a person to the source of their phobias while in a pleasurable situation. A subset of exposure therapies is \textbf{systematic desensitization} in which you are given progressive relaxation (relaxing each muscle in your body untill you reach a dowsy state) and then asked to imagine various fear inducing images and anytime you feel anxiety you back off back to a relaxed state. This is repeated with increasing levels of fear untill your phobia is resolved. Eventually they move to real world tests each time in a relaxed state to desensitize you to your fear. Sometimes they use virtual reality to create the anxious situations if they are too expensive or embarassing. 
\paragraph*{Aversive Conditioning}
This is conditioning to get rid of a positive response to a negative situation (get you to stop doing bad things). It associates unwanted behavior with unpleasent feelings. This isnt as effect as you'd think because people know that the cause of the unpleasentness only exists int the therapists office and that they can do what they want outside of it.
\subsection*{Operant Conditioning}
Behavior modification is used to reinforce desired behaviors and withold reinforcement for undesirable behavior. This reinforcement might come as just praise while others will require concrete rewards. Some institutions create a token economy, good behavior is rewarded with a token that can be used to buy things. One concern for this is the lasting of the behavior modification after they are removed from the source of rewards. This is often not a problem because the reward become the ability to function in society and being better off than they were. Another is over how ethical this is. The arguments are that its authoritative and controlling and you are just programminga another human, the opposing says that they are helping people who voluteered to join.
\section*{Cognitive Theories}
These therapies opperate on the idea that thinking colors our feelings. Therapists work to alter the way people think about events to gain a more positive reaction.
\subsection*{Beck's Theory for Depression}
Aaron Beck was training in Freudian techniques. He found recuring themes when analyzing dreams and conversations and it led him to a new method of treatment. He poses gentle questions to attempt to point our irrational thinking and then tries to persuade them of the reality of the situation. A specific example of this is stress inoculation training which is teaching people to restructure their thinking in stressful situations. Basically think more positively about the situation you are in to make the outcome better.
\subsection*{Cognitive-Behavior Theory}
This is the act of making people aware of their irrational negative thinking and replace it with better thoughts.
\section*{Group and Family Therapies}
Group therapy works similarly to individual therapy with the same rate of success. Its often good for people to find out there are others with the same problem as them. Family therapy is the act of getting the family involved in the process of helping someone.






















































\end{document}