\documentclass{article}
\usepackage{parskip}
\usepackage[margin=.6in]{geometry}
\title{Lecture 9: Social Influence}
\begin{document}
\section*{Intro}
The study of psychology as a individual in a group. How the group effects people's psyche.
\subsection*{Persuasion}
\paragraph*{Central Route} going directly through the rational, mind, influencing attitudes with evidence and logic. Give facts. This only really works when your customers are knowledgeable enough to care about these facts. This is the cognitive route.
\paragraph*{Peripheral Route} changing attitudes by going around the rational mind and appealing to fears, desires, associations, etc. Pairing product with something that elicits positive response. A good example is the Joe Canadian. This is classical conditioning.
\section*{Conformity}
\paragraph*{Informational}Muzafer Sherif (1906-1988) thought that we conform to be right, to know what to do in foreign situations. We watch what other people are doing in the same environment and imitating so that our ignorance isn’t known. Anytime that there is ambiguity in a situation, imitate your surroundings. People were told it was a study on judgement, person is sat facing a wall with their head strapped in. The lights go out and a pin point of light starts to move and people are to yell out how its moving. At the end of the study they calculate the average length the light moved. They bring people back in, but this time three at a time. Overtime people start to report that the light is moving exactly the same distance as the other people in the room. The trick is that the light isnt actually moving. When the individual is separated again, they continue to report that normal results.

\paragraph*{Normal} Soloman Asch (1907-1996), the Holocaust happened and found that people knew what was going on and what was wrong, but they did it anyway. This is where the situation wasnt ambiguous, but conformed to something that isnt right. This started the idea that people conform to get along with each other. 5 people in a room together, 4 people are actually research assistants in disguise. Soloman tells you to sit at the front, and the other four make sure that you sit on the end. Person needs to tell which of three lines is the same length as another. Soloman asks people in order so that the participant is always last. Every time, the answer is obvious, but overtime all of the assistants start to all say the wrong answer. When people say the wrong answer this wakes up the participant and makes him start to think. For a long time the participant continuous to report the correct answer, but he starts to get uncomfortable and eventually 2/3 cave and say the same answer as everyone else. When they start doing this they are no longer uncomfortable. Afterwards he interveiwed people and 1/3 still thought they had the right answer. Most people knew that they were saying the wrong answer. In order to break this feeling of discomfort all they need is for one person to say something different, and then the participant will say the correct answer.

\subsection*{Compliance} 
Robert Cialdini (1945-present) his career kicks off when a little girl asks for door to door for a donation. During her speech he decided not to donate, but did anyway. He decides to take jobs in various sales fields (door to door salesman for dictionaries, used car salesman, busboy). His theory is of 6 principles of persuasion:
\begin{itemize}
\item authority - people defer to credible experts
\item liking - people respond more affirmatively to those they like
\item social proof - people allow the example of others to validate how to think, feel, and act
\item reciprocity - people feel obliged to repay in kind what they've received
\item consistency - people tend to honor their public commitments
\item scarcity - people prize what is scarce (get everyone to show up at the same time, and tell people that other people are gonna buy it)
\end{itemize}
\paragraph*{Floor in the Door Technique} Make a small request and then ask for something larger along the same lines. You can continue to build on this so long that the increments are difficult to refuse. This was seen during the Korean. There were a large number of people who would speak out (defect) against America, far more than in WW2 despite brutal torture. This was done by asking a small request which was hard to deny especially in difficult conditions. Then they get people to sign something confirming that little request, then they have to read it to people (building on the rewards). Basically inch toward a goal by slowly increasing commitments and rewards.
\paragraph*{Low-Ball Technique} Start with an initial deal that is hard to resist, then come back with a larger deal that you would probably have rejected at the beginning. Finally they come in with a concession deal that was what they originally wanted to get. They like to try to get you to sign something if they can ,even if it isn’t anything committing. Also they will leave you alone for a while so that you convince yourself to buy it. We try to get ourselves to chose this one by emphasizing the positives of this deal and the negatives of the other deals, 
\subsection*{Obediance to Authority}
Stanley Milgram (1933-1984). This experiment was also spawned by the Holocaust. It was a demonstration to see if people would act againt their nature if told to. Obedience to authority as opposed to inflicting harm again another person. If that authority has no power we should chose to disobey, but as that authority gains power we might chose to act against our nature. You sit in a waiting room with a assistant in disguise. You are told its a study on the effects of punishment on learning, one person is teacher and one is student. You are told to be the teacher and the other person is set up to be a student. The student is strapped into a chair with their hand strapped to an electric plate. You are told to read 10 words over the intercom and the student learns them and reads them back, for every mistake you shock them. They even let you feel the shock at 15V. At this point the student is replaced by a recording. You are told to increase the voltage by 15V for each mistake, on this board the buttons (for increasing shocks) have lables of increasing pain and even danger. People are allowed to leave anytime they want to, but everytime you want to leave or stop you are given little directives (not orders). Before actually running this experiment he asked 20 experts to predict the highest voltage people will obey to. 2/3 of people went all the way to the highest voltage, but people anguish and feel real pain over their decisions. He starts to try to get people to disobey. He changes it to being conducted out of office, or have a normal person running it, uses person conducting over the phone, uses actor in same room so that they can see it, even made people restrain victims. Over each condition at lease 30\% of people went all the way. If no commands are given only 3\% go all the way. If multiple people argue against him obedience is broken.



\end{document}