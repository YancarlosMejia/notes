\documentclass{article}
\usepackage{parskip}
\usepackage{pdfpages}
\usepackage[margin=.6in]{geometry}
\begin{document}
\includepdf[page=1-11]{slides.pdf}
Example: thermostat
\begin{itemize}
	\item R: Want to keep the air temperature at or above the set temperature.
	\item Phenomena: sensor to detect the actual temperature and turn the furnace on or off to change the actual temperature
	\item S:( sensed temp $\gte$ input desired temp) turn off furnace (else) turn on furnace
	\item A: sensed temp = actual temp
	\item A: furnace is working and powerful enough
\end{itemize}

Example: traffic light
\begin{itemize}
	\item R: Allow car traffic to cross an intersection safely, without colliding with traffic travelling in other directions
	\item S1: only one direction has green light at a time. Change which direction has green light, yellow light long enough to stop safely
	\item A: Drivers follow the correct behavior at lights
\end{itemize}

Example: patient monitor
\begin{itemize}
	\item R: Retain records of patient's vital signs
	\item R: Warn nurses if the patient's readings exeed safe ranges
	\item Phenomena: sensors of patient's vital signs, input of safe ranges, sound alarm
	\item S: if sensed vital signs outside safe ranges, sound alarm
	\item A: sensors are working properly
	\item A: nurse is paying attention
\end{itemize}

\end{document}