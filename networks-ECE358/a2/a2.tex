\documentclass[12pt]{article}
\usepackage{parskip}
\usepackage{amsmath}
\usepackage{pdfpages}
\usepackage{listings}
\usepackage{color}
\usepackage[margin=.6in]{geometry}

\definecolor{dkgreen}{rgb}{0,0.6,0}
\definecolor{gray}{rgb}{0.5,0.5,0.5}
\definecolor{mauve}{rgb}{0.58,0,0.82}

\lstset{frame=tb,
  language=C++,
  aboveskip=3mm,
  belowskip=3mm,
  showstringspaces=false,
  columns=flexible,
  basicstyle={\small\ttfamily},
  numbers=none,
  numberstyle=\tiny\color{gray},
  keywordstyle=\color{blue},
  commentstyle=\color{dkgreen},
  stringstyle=\color{mauve},
  breaklines=true,
  breakatwhitespace=true,
  tabsize=3
}

\begin{document}
\section{Question 1} % (fold)
\label{sec:question_1}
For any network with m we can always set p = 0. From there we can build a counter example where q = $2^{m-1}$. If we put no nodes between p and q (q-p = 0) every entry in p's lookup table will be q. This way we guarantee that q will be the next hop from p. If we set k to some value greater than q, k-p=1. So any value of m can generate a network with p = 0, q = $2^{m-1}$, and k = $2^m-1$. This will always result in q-p=0 and k-p=1 which is a counter example to the generic case. Therefore this holds for all values of m. 


Another counter example can if you always have p = 0 , q = 2, no node between p and q, and k = 3 it doesn't matter how many nodes you add to this network or how you shift the value of m the equations of q-p=0 and k-p=1 will still hold which violates our claim. 
% section question_1 (end)


\section{Question 2} % (fold)
\label{sec:question_2}
\subsection{a)} % (fold)
\label{sub:a_}
NO IDEA WHAT TO DO
% subsection a_ (end)



\subsection{b)} % (fold)
\label{sub:b_}
When we add a node n into this network only one table will need to be updated. The node right before n will have to update its table to now have n as its successor. Everyone else will remain the same.

% subsection b_ (end)


% section question_2 (end)

\section{Question 3} % (fold)
\label{sec:question_3}
\subsection{a)} % (fold)
\label{sub:a_}
We know that the expected number of hops for a network is $O(\log n)$. We know that the finger table must have enough entries to accommodate n so $n\le 2^{m-1}$. So we can sub in this value for n and still maintain this worse case relation, so $O(\log 2^{m-1})$ if we simplify this we get that number of expected hops is $O(m)$. Based on this we can assume that there exists some case where the number of hops is less than linear. 
% subsection a_ (end)

\subsection{b)} % (fold)
\label{sub:b_}
NO IDEA WHAT TO DO

% subsection b_ (end)


% section question_3 (end)












\end{document}