\documentclass{article}
\usepackage{parskip}
\usepackage{amsmath}
\usepackage[margin=.6in]{geometry}
\begin{document}
\section{1) RPF Pruning} % (fold)
\label{sec:rpf_pruning}
We want to prune off non-host routers if they do not have integral children (basically if their adjacent edges are not red). We can tell who this is because non-forwarded paths return data. So a node knows which of its children are integral.
% section rpf_pruning (end)

\section{2) Multiplexing} % (fold)
\label{sec:2_multiplexing}
In addr any just picks an available address. If we specify an address not in in addr any we are restricting what can be received. When the host receives a packet on some interface and we have a application bound to something, will it get the packet? If an application has specified in bind in addr any. When the application sends something which interface should the packet be sent on. Note this question will encompase multihoming and multiplexing (two separate slide sets).
% section 2_multiplexing (end)

\section{3) Reliability} % (fold)
\label{sec:3_reliability}
look at rdt 3.0 (we will be given the sender FSM). This question will not be about the bugs in the slides. 
% section 3_reliability (end)

\section{4) Congestion} % (fold)
\label{sec:4_congestion}
Look at FSM on slide 104. The units of congestion window is number of segments (like 5 MSS) or you can think of it as bytes. We tend to want to think of it as segments because thats how we can count acknowledgments. He'll give us some formula


% section 4_congestion (end)







\end{document}