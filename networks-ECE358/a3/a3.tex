\documentclass[12pt]{article}
\usepackage{parskip}
\usepackage{amsmath}
\usepackage{pdfpages}
\usepackage{listings}
\usepackage{color}
\usepackage[margin=.6in]{geometry}

\definecolor{dkgreen}{rgb}{0,0.6,0}
\definecolor{gray}{rgb}{0.5,0.5,0.5}
\definecolor{mauve}{rgb}{0.58,0,0.82}

\lstset{frame=tb,
  language=C++,
  aboveskip=3mm,
  belowskip=3mm,
  showstringspaces=false,
  columns=flexible,
  basicstyle={\small\ttfamily},
  numbers=none,
  numberstyle=\tiny\color{gray},
  keywordstyle=\color{blue},
  commentstyle=\color{dkgreen},
  stringstyle=\color{mauve},
  breaklines=true,
  breakatwhitespace=true,
  tabsize=4
}

\begin{document}

\title{Computer Networks: ECE 358 Assignment 3}
\date{Lara Janecka, Kevin Carruthers, Clarisse Schneider}
\maketitle

\section*{Question 1} % (fold)
\label{sec:question_1}
Given its a connected graph there must exist a minimum spanning tree. If we assign IP addresses to the nodes in that minimum spanning tree with respect to topology then we can aggregate IPs so that we can infer the next hop from a node given only the destination IP address. This means that the size of each node's routing table is equal to the number of incident edges to that node. This means that the minimum largest routing table in this graph will be on the node with the most number of incident edges and its routing table will be that size.

It is impossible to tighten this minimum bound because each node must at the very least know of its children, otherwise there could be a destination that does not exist on any routing table making it unreachable.


% section question_1 (end)


\section*{Question 2} % (fold)
\label{sec:question_2}
If you were to first sort the set of prefixes in their binary form, which costs $O(n\log n)$, you can pass through the list once and merge as you go to build the largest possible prefixes. Once these maximal prefixes are found you have the set P'. The creation of P' will take $O(n)$ since we check each node once. The final pass through P' counts how many entries there are and if this number is less than k we return ``yes'' else return ``no''. We know that this will create the optimal P' because we first sort the prefixes and check for full sets to create prefixes that will encompass the maximal number of prefixes resulting in a minimal P'.

The above algorithm is just one example of a polynomial time solution for this problem, since it occurs in $O(n\log n + n) = O(n \log n)$. 

% section question_2 (end)

\section*{Question 3} % (fold)
\label{sec:question_3}
The PA address of the site matches the prefix for ISP P1 so the address gets aggregated into its network mask which remains 12/8. The address does not match the prefix for ISP P2 so it must register with it making it advertised on ISP P2. Packets with the given destination will go through ISP P2 because the PA address registered to it matches more closely to the destination address of the packets. In this case the longest matching prefix algorithm will favor ISP P2 due to it have the full PA address.

A solution to this would be for the site to chose to use PI address space instead of PA. This limits aggregation but ensures a unique address. Another solution is to explicitly not advertise PA addresses on ISP P2 that would be aggregated into ISP P1.

% section question_3 (end)

\section*{Question 4} % (fold)
\label{sec:question_4}
Since the node only contained one incident edge it must have been a leaf node. This edge only connected this one node to the rest of the graph. When we remove this node it had no other connections so the only thing effected is that one edge is now missing its end. Every other edge in that tree remains the same so the paths between nodes are still minimal. None of the minimal paths to other nodes incorporated this missing node (since it only had one incident edge nothing could go through it) so they remain unaffected.
% section question_4 (end)

\section*{Question 5} % (fold)
\label{sec:question_5}
\subsection*{a)} % (fold)
\label{sub:a_}
We assigned MAC addresses as letters from left to right, top to bottom

Nodes with MAC addresses B and C both match the query for IP address 1.2.3.10 since they have a 16-bit mask on IP addresses 1.2.3.10 and 1.2.1.1.

\begin{tabular}{|c|c|c|c|c|}
\hline
\textbf{Request/Response} & \textbf{Sender IP} & \textbf{Sender MAC} & \textbf{Receiver IP} & \textbf{Receiver MAC} \\ \hline
request & 1.2.3.4 & A & 1.2.3.10 & 0 \\
response & 1.2.3.10 & B & 1.2.3.4 & A \\
response & 1.2.1.1 & C & 1.2.3.4 & A \\
\hline
\end{tabular}
% subsection a_ (end)



\subsection*{b)} % (fold)
\label{sub:b_}
\label{sub:a_}
Since the node at MAC address A and IP address 1.2.3.4 only has access to nodes B and C, node C acts as a proxy ARP for resolving nodes E and F, which match the IP request of 15.16.17.18 with 15.16.17.25/8 and 15.16.17.18/8.

\begin{tabular}{|c|c|c|c|c|c|}
\hline
\textbf{Request/Response} & \textbf{Sender IP} & \textbf{Sender MAC} & \textbf{Receiver IP} & \textbf{Receiver MAC} \\ \hline
request & 1.2.3.4 & A & 15.16.17.18 & 0 \\
request & 10.11.12.1 & C & 15.16.17.18 & 0 \\
request & 15.16.17.25 & E & 15.16.17.18 & 0 \\
response & 15.16.17.18 & F & 15.16.17.25 & E \\
response & 10.11.12.25 & E & 10.11.12.1 & C \\
response & 15.16.17.25 & C & 1.2.3.4 & A \\
response & 15.16.17.18 & C & 1.2.3.4 & A \\
\hline
\end{tabular}
% subsection b_ (end)

% section question_5 (end)
  
\end{document}
