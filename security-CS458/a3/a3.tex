\documentclass{article}
\usepackage{parskip}
\usepackage{listings}
\usepackage{amsmath}
\usepackage[margin=.6in]{geometry}
\begin{document}

\section{Question 1} % (fold)
\label{sec:question_1}
\subsection{a)} % (fold)
\label{sub:a_}
The world interacts with a surface router. This router directs packets  to an internal router which redirects those packets to the closest geographic location. At this location there is a webserver, database, and file server. The webserver just serves the user interface for the website. The file server is attached to the database to authenticate the user's credentials and if they have access to the content that they have requested. This way the webserver can authenticate the user with access to the database, but databases and file servers have redundancy by residing at each geographical location.

INSERT PICTURE HERE	

% subsection a_ (end)
\subsection{b)} % (fold)
\label{sub:b_}
The firewall should go right before the internal router. this way the webserver can sit in the demilitarized zone and not be slowed down by the firewall. Users will see their interface as responsive and possible. A this location the firewall can still filter packets to all sensitive locations even though we only have one firewall.

This firewall should be a simple packet filtering gate way. This is because little is known about the packet's final destination application so an application proxy won't work. Literally every packet served by the application will go through this firewall so it is important that it be as fast as possible because any cost will have massive repercussions so a stateful firewall will cost too much. Because of this only a simple packet filtering firewall will do.

% subsection b_ (end)
\subsection{c)} % (fold)
\label{sub:c_}
If we have a firewall for each geographical location we can put a more specific firewall at the nexus for each of those locations (right before traffic divides into database and fileserver requests). This firewall can be stateful because a much smaller subset of traffic is going through here. We could even use an application firewall with configurations for database access and file server accesses.
% subsection c_ (end)
% section question_1 (end)

\section{Question 2} % (fold)
\label{sec:question_2}
\subsection{a)} % (fold)
\label{sub:a_}
\paragraph{Assumption} % (fold)
\label{par:assumption}
This attack assumes that usernames are evenly distributed across the alphabet (i.e. there are the same number of usernames starting with a as there are starting with x).
% paragraph assumption (end)
\paragraph{Tracker} % (fold)
\label{par:tracker}
Get roughly half the usernames, this works out to be usernames greater than m. This relies on the above stated assumption since this wouldn't work if every username started with x. Since $k = \frac{N}{8}$ we can easily see that $2 \frac{N}{8} < \frac{N}{2} < N - 2 \frac{N}{8}$
% paragraph tracker (end)
\paragraph{Queries} % (fold)
\label{par:queries}
\begin{itemize}
	\item $username == XdarksephirothX || username > m$
	\item $username != XdarksephirothX && username \leq m$
	\item true
\end{itemize}
% paragraph queries (end)
% subsection a_ (end)
\subsection{b)} % (fold)
\label{sub:b_}
First run the query $username < m$. This gives us a baseline to work with. Then we can run $username < m || (username == XdarksephirothX && gold <= x)$ and watch for when the value returned is one greater than the established base line. You can now narrow down on the correct value of x using a binary search within the bounds of zero and 20 billion.  This should always return roughly half or half plus one values (again, assuming that names are evenly distributed) so we get around the restrictions on queries. 
% subsection b_ (end)

% section question_2 (end)
\section{Question 3} % (fold)
\label{sec:question_3}
\subsection{a)} % (fold)
\label{sub:a_}
The first major issue is that it is often very hard to determine which district a hacker's computer resides, so law enforcers cannot tell which judge should sign the warrant. This makes getting warants and prosecuting these criminals much harder. With these changes the police just have to get a warrant from the nearest judge without having to know the physical location of the hacker. 

The second issue is that many hackers use bot nets which include thousands of devices in many different districts. Thats a lot of warrants from a lot of different judges. With these bill changes only one warrant (and by extension one judge) are required to shut down bot nets.
% subsection a_ (end)
\subsection{b)} % (fold)
\label{sub:b_}
Peter Carr is referring to how crazy good hackers are at hiding their identity and location. Without knowledge of who to serve the warrant to and which district needs to approve of it, it is impossible to get a warrant. 
% subsection b_ (end)
\subsection{c)} % (fold)
\label{sub:c_}
The government could freely hack anyone who has ever torrented (or any form of peer to peer sharing) anything. Dropbox (and similar services) would also be wrecked. If any user uploads copyrighted content to their dropbox the government now has the ability to hack Dropbox (from there they can access everyone who has copyrighted content on their dropbox). VPNs would also be hit pretty hard because one bad apple could get then entire network hacked by the police.
% subsection c_ (end)
\subsection{d)} % (fold)
\label{sub:d_}
This policy change could allow investigators to search many different computers with a single warrant, including computers of people who were just victims and didn't even know they were involved. It would also allow judges to approve of warrants in districts that are not their own, allowing investigators to chose the judge they ask.
% subsection d_ (end)
\subsection{e)} % (fold)
\label{sub:e_}
Judge shopping is filling a bunch of stuff in various locations and hoping that one of the many different judges is sympathetic. An example of this would be if you petitioned a judge for a warrant and if you're rejected going to a different judge by claiming your warrant is somehow related to something in their district so they might sign it.

% subsection e_ (end)

% section question_3 (end)









\end{document}	