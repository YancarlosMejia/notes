\documentclass{article}
\usepackage{parskip}
\usepackage{pdfpages}
\usepackage[margin=.6in]{geometry}
\begin{document}
\title{Columbus}
\maketitle
In most history books christopher columbus is portrayed as a great man/christian/explorer. Columbus day was invented in 1906 during a surge in national pride when America was trying to define its national identity. The scandinavians intially protested this because the vikings came first (roughly got here 500 years first). They tried to make a small colony there (it didnt last long). It collapsed due to a breakdown in relations with natives and fragile supply routes.

Up until columbus europeans looked east along the silk road. This snaked through venice, byzantium, india, and finally to china. Wealthy europeans (and upper middle class) were looking for luxury items like spices (cinamin, saffron, nutmeg), silks, dyes, perfumes, jewlry, silver and gold, and anything else that could be sold to the elite. The search for these items was one of the main purposes of columbus' journey.

He was born in genoa 1451 which was an italian city-state. It was a city that lived by trade, surrounded by mountains so trade by sea. Based on an image from that time it had a large safe harbor to dock in with many customs houses and warfs. A very large cathedral dominates the image with many churches outside which shows that it was very religious. So we can concluded that colombus grew up in a city centered around sea trade and catholisism.

Columbus' mum kept a tavern which was frequented by many of the traders that entered the city. Here is where he first heard rumors of a westward path to asia. He was very well educated.

By 1492 he:
\begin{itemize}
	\item experienced salor
	\item mapmaker
	\item navigator
	\item experienced merchant and trader
	\item knew of wealth to be made on slavery
\end{itemize}

He worked as a backroom agent for traders so he knew of the many facets of trade (see above).

People were not afraid of sailing off the edge of the earth (they were not stupid), they were affraid of running out of supplies.

On the first voyage he left with three ships from palos. He first stoppped at the canaries for supplies and sailed directly west to the bahamas (ish).

The people he first contacted spoke an variation of arrowak (spelling?).

Interpretation of image:
\begin{itemize}
	\item the natives are handing over valuables that look like european items
	\item the natives look very reverent of columbus
\end{itemize}

Bartolome de las Casas was an author and anti-racism activist. He tells a different story of the colonization of the indies. He originally was like everyone, he had slaves and prospered from them. But he became a huge ciritic of columbus and the effects of colonization. He never questioned the rightness of colonization (bringing christian values to heathens), he believed that they were bringing god here. Instead he pointed out that the version of christian europe that came to the colonies was a bastardization of civilization.

Hes version explains how the natives were actually treated by the christians and their response of hiding and running from them. A soldier raped the chief's wife so they started to fight back, but the spaniards could easily defeat them due to technological advances. He describes how they forced their way into settlements and killed everyone (women and children). He was actually on the second voyage so his recount is as valid as columbus'.

Inpart he was afraid of god's punishment on spain, but hes was primarily an activist. A friend of columbus (michele da cuneo) also came on the second voyage and his recount backs up casas' version. He describes kidnapping some natives and raping one of them.

Columbus brought many different animals and plants with them. He then took various plants and riches from the americas. This created a massive exchange of flora and fauna. This resulted in a massive rise in the caloric intake for europeans but also other diseases. Siphilis came from the women raped by the colonizers and spread quickly.

By 1600 the population in the americas greatly declined (even counting all the immigrants and slaves brought over) mostly due to violence, disease, and environmental stress (feral animals and rats destroyed crop fields).

Many people thought columbus could not appreciated the new world (lacked the metal equipment). For instance he thought he encountered really ugly mermaids they were actually manatees.






























\end{document}
