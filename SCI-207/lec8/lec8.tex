\documentclass{article}
\usepackage{parskip}
\usepackage{pdfpages}
\usepackage[margin=.6in]{geometry}
\begin{document}
Quantum mechanics shows that there are no waves or particles in nature, there is instead a third particle that combines the nature of these two things.

To understand it lets just look at light, the simplest thing we can. Light behaves like a wave. It has an amplitude and frequency, Waves can interfere with each other. Both the crest and the troph have energy. If you take the amplitude and square it you get its intensity. Mirrors are awesome they reflect light. The half silvered mirror has a layer missing (denoted with dashed line) so that it both reflects and refracts.

Eventually we found that light behaves like a particle. If you have a detector (denoted with D shape) like the one in your camera. Detectors see that light continuously comes through, but if you slow things down the detector sees bursts of energy, called \textbf{photons}. This means that light is a shower of particles. If we combine detectors an an half silvered mirror we see a full photon in one or the other, you never see half a photon. Its a 50\% probability of which detector the photon will go to. This is the same percentage of the wave intensity that arrives there as well. We need to see how the photon decides which detector to go to and this is basically fundamentally random. This was a big change from newtonian physics which had the world as a clockwork universe that was completely deterministic.

%slide 7
We would expect this experiment to show that half the particles go into each detector. Instead what we get is that all of the photons end up in A. So in this case we cannot treat light like a particle. Instead what happens is that the light beams interefere with each other which results in the two waves combining and going into b cancel each other out and the two waves combining and going into A double. This is seen as nothing going into B and everything going into A. This experiment is depended on the length of the sides of the square.

\paragraph{Many Worlds}
\label{par:many_worlds}
There is this idea that at every random choice the photon makes there is another universe in which it did take the other one. This results in a stupid number of universes.

To explain this experiment you have to say that a particle can be in two places at once. It still knows what it is passing through and reacts accordingly.

Because particles behave like they are in multiple places at once they can interfere with each other.

This blew people's minds.

% slide 28
if light is a wave and a particle be looked at how particles could behave like waves. This wave-particle duality is boss.

\paragraph{Ultraviolet Catastrophe}
\label{par:ultraviolet_catastripo}
Our best understanding of the world says people at a campfire should be vaporized by ultraviolet rays. Clearly our theory of nature is very wrong. Plank wanted to explain this, he tried to fudge the data because he couldnt believe his own theory.

\paragraph{Aomic Catastrophe}
\label{par:aomic_catastrophe}
Our best understanding of an atom is the classic core with electrons orbiting it. If electrons orbited around the nucleus the way we thought it would be continuously giving off radio waves and would spiral into the nucleus. All atoms would just collapse. When we heat up elements they did not give off the full spectrum of light, this was confusion. The fact that electrons could behave like a wave explained this, and also showed that atoms could exit. Electrons could be in multiple places at once. Instead of an electrons orbiting it was actually just a ring of charge spinning. This does not give off radiation and thus won't collapse.

\paragraph{Schrodinger's Cat}
\label{par:schrodinger_s_cat}
In the box there is a radioactive element that will decay at a random time. This means the atom is both decayed and not decayed. This is explained through the metaphor of a cat. To explain it he introduced quantum cards. Imagine a card balacing on its edge. As time goes on you have a higher probability of seeing the card fallen over rather than on its edge. Quantum mechanics says that that car will fall both ways. By observing we make it make it stop following the deterministic schrodinger's equation and puts it in the nondeterminsitic situation of making a decision.

People were real confused, what makes the observer special. Its not that the observer is special, they just see the card in one state in each world, there are just many different worlds.

%slide 49
The diagram on the left is the real world, it adds bumps where the two waves interfere with each other. In classical we dont have this interference. Decoherence is this observation happening. It is the thing interacting with the object very quickly. This is why we can observe the univese in a classical sense








\end{document}
