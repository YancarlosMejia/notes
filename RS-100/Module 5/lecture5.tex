\documentclass{article}
\usepackage{parskip}
\usepackage{csquotes}
\usepackage{color,soul}
\usepackage[margin=.6in]{geometry}
\begin{document}
\title{The Path of the Jains}
\maketitle
\section*{Jain Sacred Story}
\label{sec:jain_sacred_story}
The jains believe that everything in the universe has a soul. Even inanimate objects. Some objects even have more than one soul. The universe is alive.

The sacred story begins with \textbf{Mahavira}. He was so commited to ahimsa he didn't kick in the womb. He made the great renunciation as a young man. He pulled out all of his hair with his hands. For 12 years he practiced asceticism and ahimsa. After this time Mahavira attained liberation, called \emph{Kevela}.

\paragraph{Ascetism}
\label{par:ascetism}
jain ascetics refrain from food, movements, over indulgences of all kinds.

\paragraph{Ahimsa}
\label{par:ahimsa}
non violence, towards all living beings. Remember that the universe is full of living beings.

Jains cover their mouths with masks to prevent the accidental ingestion of life in the air, it must be worn at all times. Even the statues have this mask. The shvetambaras's wear simple white robes, but the digambaras's do not, they are \emph{sky clad}. The monks carry whisks to sweep the ground before they walk to move all earth beings out of the way. They also carry a staff for walking and support and a bag/begging bowl. shvetambaras's are allowed a begging bowl, digambaras's are not.

\section*{Jain Sects}
\label{sec:jain_sects}
\paragraph{Shvetambaras}
\label{par:shvetambaras}
norther group, wear white robes (for protection from the elements), allow women in because they believe that women are capable of attaining enlightenment as women

\paragraph{Digambaras}
\label{par:digambaras}
\emph{sky clad}, southern group, do not allow women in they must be reborn as men before they can attain enlightenment, they have no bowl and must accept alms with their hands. The enlightened engage in no worldly activity, they enter a spiritual realm completely

The evolution of these groups ocurred in 300 BC. There was a famine and several jains moved to the south. When they returned north they found the norther monks had deviated from the correct path (according to them). This results in some variation between the texts of these groups.

\section*{The Story of the Man and the Well}
\label{sec:the_story_of_the_man_and_the_well}
This story is told all throughout india. A certain man oppressed by poverty left his home to another country. He passed through the land and got lost after a few days. Eventually he found a forest in which there was a mad elephant charging him. A evil demoness also appeared before him. He ran towards a nearby banyon tree. Unfortunately it was too tall and he could not reach the nearest branch. He saw a well nearby which he dove into. He clung to some reeds in the wall, since at the bottom of the well were a bunch of snakes including a mighty black python. He saw two large mice, black and white, gnawing on the reed clump. The elephant arrived and kept charging the banyon tree. A honeycomb fell from the tree and fell into the well causing him to be stung by a bunch of bees. Some honey fell on his head and dripped into his lips. This lead to him only want more honey and rid his mind of all the bad things in his life currently.

This parable is used to clear the minds of those on the way to liberation. The man is the soul, his wandering through the forest represents the four types of existence. The elephant is death, and the demoness is old age. The banyon tree is salvation where there is no more fears, but no sensual man can climb it. The well his human life, the snakes are passions that overcome people to the point that they dont know what to do. The tuft of reed his mands alloted time on earth, while the soul is embodied. The mice are dark and light karma. The bees are manifold diseases that torment a man. The python is hell. The drops of honey are trivial pleasures.

This shows a negative view of life in samsara and show how serious the situation is and how slight the chance of salvation is. We have to seize control of our situation and climb the banyon tree. A wise man would know that he cannot climb the tree so he would ignore the honey and focus on the path up the tree.

\section*{Jain Worlds of Meaning}
\label{sec:jain_worlds_of_meaning}
Atheistic, no creator of the universe. They believe that the universe had no beginning and goes through endless cycles. Move from unhappy to happy (progressive) and back (regressive). In these cycles there is a brief moment in the middle which is the only place you can get enlightenment. In each half cycle 24 Jinas (conquerers, liberated ones) will appear. Mahavira was the last one in our cycle. This means that no one can become enlightened during this cycle.

Mahavira followed Parshva the 23rd Jina and built on his teachings.

You should still follow the religious path even though you have no change to become enlightened since you might slip back.

\paragraph{Dualism}
\label{par:dualism}
they accept that there are two real principles. Consciousness (jiva) and matter (ajiva). These are real and exist on their own. They are always separate.

At some point in time, for unknown reasons, consciousness got caught up in matter. This caused the creation of everything. Consciousness is trapped in matter which is what causes suffering.

\paragraph{Karma}
\label{par:karma}
is a subtle matter that coats and sicks to the consciousness. It changes the soul (shape), causes confusion and obstruction. Karma also determines the type of embodiment you might suffer.

There are infinite ways that a soul can become embodied. There are infinite souls in infinite configurations. Statistically enlightenment is rare.

\subsection*{Path to Liberation}
\label{sec:path_to_liberation}
We need to separate our consciousness from matter. This is done through asceticism and ahimsa. All the ascetic practices burns off karma from the consciousness. Ahimsa prevents the accumulation of more karma. If this is all done you get \emph{Kevela} where the consciousness is freed of all matter and floats to the top of the universe.

The soul tends towards improvement. There is always hope as your soul always seeks improvement. It needs a transforming event like viewing a jain statue or hearing a jain talk. This event redirects the soul towards kevela.

The path is seen as a 14 runged ladder. Everyone begins in ignorance. During a middle cycle you can experience a transforming event. The fourth rung gives you insight to reality. Souls climb and fall, but  with insight we know that the path will eventually be successful.

The path to liberation is the same for monks and laity. This causes closeness between the community and the faithful. This is partially responsible for the lasting power of Jainism.

Monks/Nuns take 5 restraints and everyone else takes 12 vows. The laypeople have to have a good livelihood. Agriculture and animal husbandry is not allowed. Jains are known for their animal shelters, particularly for the stray cows that wander india. They collect these animals an provide a shelter for them.

Monks and laity both engage in fasting. Saints commit suicide by fasting for 3 months when they are ready to ascend.

5 Vows (monastics take this much more seriously):
\begin{itemize}
	\item non-violence - most important
	\item truthfulness - not only not lying, but also not using speech that would hurt others
	\item non-stealing - people should also not do anything to harm others
	\item chastity
	\item non-attachment
\end{itemize}
3 Merit Vows:
\begin{itemize}
	\item limit area of activity - being careful about things that might harm other living beings
	\item limited use of items
	\item no carelessness in daily thoughts, words, or deeds
\end{itemize}
4 Disciplinary Vows:
\begin{itemize}
	\item meditation
	\item limiting space
	\item ascetic's life
	\item vow for charity
\end{itemize}

\subsection*{Ritual and the Good Life}
\label{sec:ritual_and_the_good_life}
Once the soul rises to the top of the universe, it is no longer acessible, they cannot assist us. You can only rely on yourself for enlightenment.

Worshiping the Jinas is not asking them for help, instead it is reflection on their merits and vowing to follow in their footsteps.

Fasting is the most important ritual for jains.

The laity may worship other gods and goddesses as well, since jains are surrounded by hindus.

There are some festivals and life cycle rituals.

\subsection*{Jain Contributions to the World}
\label{sec:jain_contributions_to_the_world}
One of the biggest contributions of the jains is the concept of ahimsa. This greatly influenced the hindu culture around it, it encourage vegetarianism. It was used by Gandhi to drive the independence movement. Gandhi then influenced Tolstoy and MLK and brought about a world wide movement towards nonviolence.

Jains have also contributed much to philosophy. They brought about the idea of relativity. When you look at a problem you look at it from one perspective. There are many perspectives to every problem. To get close to the truth you need to look at all perspectives.

The jains also did great work towards preserving knowledge from all perspectives even different religions.

They were also awesome architects and artists.




\end{document}
