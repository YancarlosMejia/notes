\documentclass{article}
\usepackage{parskip}
\usepackage{csquotes}
\usepackage{color,soul}
\usepackage[margin=.6in]{geometry}
\begin{document}
\title{The Path of the Jains}
\maketitle
\section*{Jain Sacred Story}
\label{sec:jain_sacred_story}
The jains believe that everything in the universe has a soul. Even inanimate objects. Some objects even have more than one soul. The universe is alive.

The sacred story begins with \textbf{Mahavira}. He was so commited to ahimsa he didn't kick in the womb. He made the great renunciation as a young man. He pulled out all of his hair with his hands. For 12 years he practiced asceticism and ahimsa. After this time Mahavira attained liberation, called \emph{Kevela}.

\paragraph{Ascetism}
\label{par:ascetism}
jain ascetics refrain from food, movements, over indulgences of all kinds.

\paragraph{Ahimsa}
\label{par:ahimsa}
non violence, towards all living beings. Remember that the universe is full of living beings.

Jains cover their mouths with masks to prevent the accidental ingestion of life in the air, it must be worn at all times. Even the statues have this mask. The shvetambaras's wear simple white robes, but the digambaras's do not, they are \emph{sky clad}. The monks carry whisks to sweep the ground before they walk to move all earth beings out of the way. They also carry a staff for walking and support and a bag/begging bowl. shvetambaras's are allowed a begging bowl, digambaras's are not.

\section*{Jain Sects}
\label{sec:jain_sects}
\paragraph{Shvetambaras}
\label{par:shvetambaras}
norther group, wear white robes (for protection from the elements), allow women in because they believe that women are capable of attaining enlightenment as women

\paragraph{Digambaras}
\label{par:digambaras}
\emph{sky clad}, southern group, do not allow women in they must be reborn as men before they can attain enlightenment, they have no bowl and must accept alms with their hands. The enlightened engage in no worldly activity, they enter a spiritual realm completely

The evolution of these groups ocurred in 300 BC. There was a famine and several jains moved to the south. When they returned north they found the norther monks had deviated from the correct path (according to them). This results in some variation between the texts of these groups.

\section*{The Story of the Man and the Well}
\label{sec:the_story_of_the_man_and_the_well}
This story is told all throughout india. A certain man oppressed by poverty left his home to another country. He passed through the land and got lost after a few days. Eventually he found a forest in which there was a mad elephant charging him. A evil demoness also appeared before him. He ran towards a nearby banyon tree. Unfortunately it was too tall and he could not reach the nearest branch. He saw a well nearby which he dove into. He clung to some reeds in the wall, since at the bottom of the well were a bunch of snakes including a mighty black python. He saw two large mice, black and white, gnawing on the reed clump. The elephant arrived and kept charging the banyon tree. A honeycomb fell from the tree and fell into the well causing him to be stung by a bunch of bees. Some honey fell on his head and dripped into his lips. This lead to him only want more honey and rid his mind of all the bad things in his life currently.

This parable is used to clear the minds of those on the way to liberation. The man is the soul, his wandering through the forest represents the four types of existence. The elephant is death, and the demoness is old age. The banyon tree is salvation where there is no more fears, but no sensual man can climb it. The well his human life, the snakes are passions that overcome people to the point that they dont know what to do. The tuft of reed his mands alloted time on earth, while the soul is embodied. The mice are dark and light karma. The bees are manifold diseases that torment a man. The python is hell. The drops of honey are trivial pleasures.

This shows a negative view of life in samsara and show how serious the situation is and how slight the chance of salvation is. We have to seize control of our situation and climb the banyon tree. A wise man would know that he cannot climb the tree so he would ignore the honey and focus on the path up the tree.























\end{document}
