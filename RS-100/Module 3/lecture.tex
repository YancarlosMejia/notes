\documentclass{article}
\usepackage{parskip}
\usepackage{pdfpages}
\usepackage[margin=.6in]{geometry}
\begin{document}
\title{Hinduism II: Classical, Medieval, and Modern Periods}
\maketitle
\section*{Classical Period}
\label{sec:classical_period}

\subsection*{Yogasutras and Dharmasutras}
\label{sec:yogasutras_and_dharmasutras}
\subsubsection*{Classical Period}
\label{par:classical_period}
lasted 400 BCE - 400 AD showed a bunch of new texts come into play.

Up to this point the collection of texts used were called the \textbf{shruti} (meaning what is heard) and consisted of:
\begin{itemize}
	\item rig veda
	\item sama veda
	\item yajur veda
	\item atharva veda
	\item brahmanas
	\item aranyakas
	\item upanishads
\end{itemize}

The new texts to form during the classical period are called the \textbf{smriti} (meaning what is remembered). These have less authority than the veda but are far more accessible. They develope on earlier ideas and form new ones. Here there develop more paths to moksha. They consist of:
\begin{itemize}
	\item yogasutras
	\item dharmasutras
	\item epics
	\begin{itemize}
		\item bhagavad-gita
		\item ramayana
	\end{itemize}
	\item puranas
	\item tantras
\end{itemize}

\subsubsection*{Yogasutras}
\label{sub:yogasutras}
These were compiled by \textbf{Paranjali}. He organizes and systemizes the existing material about the medative tradition. They emphasize withdrawing from society, meditation, asceticism, and the value of knowledge when ataining moksha.

\subsubsection*{Dharmasutras}
\label{sub:dharmasutras}
Dharma means duty, these texts talk about the duties of each class in society. The social order is based on eternal order and layed out in the purusha sukta. The dharmasutras codify the duties of each class mentioned in the purusha sukta. Birth is added to class which evolves into the caste system. The most famouse of the dharmasutras is \textbf{The Law Code of Manu}.


\subsection*{Ramayana}
\label{sec:ramayana}
Most important of the new literature are the epics. These are collections of stores with religious meaning making them the most accessible texts for all. The heroes provide models for life and devotion.

\paragraph{Ramayana}
 \label{par:ramayana}
is the story of Rama a righteous ruler and Sita the perfect wife. Rama gets banished to the forest due to politics. His brother and wife go with him. Sita doesn't have to go but insists. Sita gets kidnapped by Ravana (a demon). Eventually Rama wins her back. He returns to take back his kingship. His subjects speculate about how she was captured by Ravana and if she had be faithful to Rama. Sita undergoes a test of her virtue. She makes a vow of truth and stands in the middle of a fire and doesn't get burned. Eventually people start speculating again. Rama needs the loyalty of his people but they are questioning his honor due to the rumors around his wife. He knows that she has been loyal, but his duty comes first so he banishes her.

The moral of this is duty before desire.


\subsection*{Bagavad-Gita}
\label{sec:bagavad_gita}
The second part of the epics is called the \textbf{Mahabharata} and this core of this is called the \textbf{Bhagavad-Gita}. The main moral of this is that action (namely selfless action) is better than no action. It lays out three paths for action.

In it \textbf{Arjuna} is faced with a dillemma. He is the great hero for the pandavas which are at war against the koravas about who should rule the country. Before the largest battle of the war Arjuna asks to go to the middle of the field to see the battle lines and armies on both sides. His charioteer is Krishna (an avatara of Vishnu responsible for moral order). He looks at his enemies and sees all these people on the battle field (teachers, sons, friends, so on). Seeing this he decides that he does not wish to fight. He realizes these people are very important members of society and wonders what will happen to society if these people are all killed.  Due to this he refuses to do his duty as a warrior.

The main content of the Bhavagad-Gita is a dialog between Arjuna and Krishna. Krishna corrects Arjuna's misunderstandings:
\begin{itemize}
	\item Arjuna mistakingly believes that anyone can kill or be killed, he says that the atma is the eternal soul and cannot be killed
	\item Arjuna wonders what will happen to society once these people are killed, he says the lord set up this social arrangement of society so Arjuna should not question it.
	\item Arjuna is a warrior so he should do his duty (to fight), if Arjuna performs his duty he can make spiritual progress (do so without thought of reward, selflessly)
	\item Arjuna's duty can be considered a gift to god, an act of devotion
\end{itemize}

This lays out the path of action and devotion. Scholars believe this is a response to a historical event. Many young people were effected by the rise of other religious groups (like Janes and Buddhists) that advocated withdrawing from society. Too many people were withdrawing from society which was threatening the social order. By layout the paths of action and devotion it allows people to make spiritual progress while still maintaining society.

\subsubsection*{The Three Paths}
\label{sub:the_three_paths}
The path of knowledge, this is the path described in the Upanishads and Yogasutras. We retire to an isolated place to do yoga and meditate in order to gain knowledge on the oneness between Atma and Bhraman.

The path of action, it is better to stay in society and perform your duty selflessly than to withdraw and follow the path of knowledge (only few can take that).

The path of devotion (this is the one that the Bhagavad-Gita prefers), is total devotion to ones god. This is another way of selflessness, everything in your life become submerged in love of god. Devotion to god can raise someone's karma.


\subsection*{Summary}
 \label{sec:summary}
Each group of texts focus on one aspect of hindu identity:
\begin{itemize}
	\item Yogasutras: withdrawl, knowledge, individual fulfilment
	\item Dharmasutras: duty, social order, cosmic order
	\item Epic: something for the human heart, devotion, love of god (reciprocal love)
\end{itemize}

The main message of the classical period of hinduism is that it provides a path for people of all abilites and classes.

\section*{Medieval Hindusim}
\label{sec:medieval_hindusim}
This is 40-1800 CE. This shows the rise of \textbf{bhaki}, meaning devotion. This comes from the Bhagavad-Gita. A new set of texts also come forth here, called the Puranas which provide a history and proper methods of worship for specific gods. The most notible are the Vishnue and Krishna puranas. The most popular are the stories of Rada and Krishna. They are exemplars of a loving relationship with god.

\paragraph{Devotional Groups}
\label{par:devotional_groups}
tend to fall into two categories. The \textbf{Vaishnavites} worship the avataras of Vishnu. The \textbf{Shaivites} worship the avatars of Shiva. A subgroup of shaivites are the \textbf{lingayats} (called because they wear a lingu, icon for shiva's lingu on their necks) also known as the Virashiva and they reject the caste system. This period also has a rise for the goddess Shakti (Shiva's powerhouse). She is the female aspect to Shiva, but is also a goddess in her own right. She has manifestations known as Devi, Kali, and Durga. People devoted to Shakti are called the \textbf{Shaktas}.

\paragraph{Chaitanya}
\label{par:chaitanya}
was a notible person during this period (1486-1533). He was responsible for a major revival of Krishna worship and began the religious practice of publicly singing and dancing, called \textbf{Kirtan}. This got carried to the west in the 1960's called the Hare-Krishnas, their correct name was International Society for Krishna Consciousness (ISKCON).

\subsection*{Medieval Philosophy}
\label{sub:medieval_philosophy}
6 schools emerged during this time to rationalize and systematize the earlier teachings.

\paragraph{Samkhya}
\label{par:samkhya}
was a school and provided the basis for yoga and the Jane religion. It is a philosophy of dualism, there are two equally real entities (matter and spirit). The spirit is trapped in the matter so this school looks to liberate the consciousness from the matter. In order to stop the wheel of samsara you need to liberate the soul from matter.

\paragraph{Advaita Vedanta}
\label{par:advaita_vedanta}
is the most influential school. A famous philosopher from this school was Shankara (788-820) who derived the equation A=B. He also distinguished between two types of bhraman. Nirguna Brahman meaning Brahman without attributes, the true Brahman (the highest manifestation). The secondary form of Brahman was Saguna Brahman meaning Brahman with attributes. This would be the manifestations of the Vishnu, Shiva, and Shakti, and their avaratas. The other gods were aspects of Bhraman, but these gods ARE Brahman. Liberation is found when one dies and the Atman joins Brahman.

\paragraph{Visitadvaita}
\label{par:visitadvaita}
also contained a very influential philosopher named Ramanuja(1017-1137). He disagreed with Shankara and thought that Saguna Brahman was the highest manifestation or Brahman. This provided the foundation for the devotional movement. He also believed that was a slight distinction between god and atma after death so that the soul could continue in eternal adoration to god. For Ramanujan, Brahman was a huge body made up of souls allowing the dead to continue in adoration of god.

\paragraph{Dvaita Vedanta}
\label{par:dvaita_vedanta}
was founded by Madva. He believed that there was an eternal distinction between the self and God. He expanded on Ramanuja's distinction. For him the self and god had always and would always be separate.

\paragraph{Mimamsa}
\label{par:mimamsa}
is a school very popular in india and not the west. This argues for the eternality of the Vedas and provide the philosophical foundations for the importance of ritual and duty for liberation.

\paragraph{Nyaya}
\label{par:nyaya}
school of logic.

\paragraph{Vaisheshika}
\label{par:vaisheshika}
was an atomistic school. Believing that the world was created by atoms.

The above two schools saw logic and cosmology as a way to liberation.

\subsection*{Tantra}
\label{sec:tantra}
This focuses on Shakti and evolved in the medieval age. Tantra believes that moksha can be attained through ritual and yoga centered on shakti's power. All tantra is based on a corespondance between the macro and microcosms. Macrocosm is the sacred universe and the microcosm is the human body. The body is a sacred geography with certain cosmic centers known as chakras. The purpose of tantra is to unite the macro and microcosm. Doing this unites the devine and human, male and female, and unites shiva and shakti. The practice of this focuses on activating the sacred forces in the body through ritual and yoga. Ultimately we want to unite intelligence and creative energy.

\paragraph{Mandelas}
\label{par:mandelas}
Another way to unite the micro and macrocosm is through \textbf{mandelas}. These are drawings of the sacred universe. They can also be considered as the temple or palace as the home of a god. Drawing a mandela is a religious practice. We want to unite the drawing of the universe (mandela) with our inner drawing of the universe.

\paragraph{Mantras}
\label{par:mantras}
These are sacred sounds. Some of them are words with meaning or verses and some are meaningless sounds. They do not have to make gramatical or intellecual sense. The sounds made when reciting mantras if said in the right tone can unite with the sound of the cosmos. Aum is the most famous mantra representign the primordeal sound.

\paragraph{Rituals}
\label{par:rituals}
Right handed practice includes mandelas, mantras, and yogas. The left handed practice is only for a restricted number of practicioners as it is very dangerous. This is because it uses items that are considered poluted in hindu rituall. The five m's (all words begin with m in sanskrit):
\begin{itemize}
	\item meat
	\item fish
	\item wine
	\item parched grain
	\item sex
\end{itemize}

\section*{Modern Period}
\label{sec:modern_period}
During this period india was under colonial rule and hinduism is being attacked by christian missionaries. There is a dual response of reform and revival against christianity. Each branch focused on social reform, primarily on the caste system and place of women.

\paragraph{Brahmo Samaj}
\label{par:brahmo_samaj}
Stared around 1828. This was a variation on hinduism that had no ritual or custom and instead was more rational and humanistic. The founder was Ram Mohan Roy (?-1833) believed that the vedas and upanishads taught primitive monotheism. The second leader was Debendranath Tagore (1817-1905) was a great poet. He extended Roy's throughts to focus more on reason and conscious. The vedas should be tested by the inner light of reason. This meant the Vedas were not authoritative until the meshed with human reason.

\paragraph{Arya Samaj}
\label{par:arya_samaj}
Founded in 1875 by Swami Ddayananda Saraswati (1825-1833). This focused on social reform

\subsection*{Independance Movement}
\label{sub:independance_movement}
The biggest name in the independance movement is \textbf{Ghandi} (1869-1948). Another well known figure of the independence movement wasd \textbf{Aurobindo} (1872-1950). Both were imprisoned. Ghandi is most notible for founding non-violent resistance, known as \textbf{Satyagraha} or "grasping the truth". Aurobindo withdrew from society after being imprisoned. He wanted to reform society by reforming the individual. It was his thought that society should be transformed so that the individual could make spiritual progress.

\subsection*{East Comes West}
\label{sub:east_comes_west}
\textbf{Maharishi Mahesh Yogi}(1917-present) was a famous teacher of hinduism in the west. He taught the beatles and other celebrities. He wanted to develope eastern spirituality and present it in form more available to westerners. This resulted in the invention of transcendental meditation which is the practice of taking the hindu devotional aspects out of the context of meditation. This secularized hinduism making much far more accessible leading to a large boom in its practice. \textbf{Swami Bhaktivedanta Prabhupada} (1886-1977) brought Chaitanya's teachings (Krishna worship) west resulting in the ISOKC reaching north america.



\end{document}
