\documentclass{article}
\usepackage{parskip}
\usepackage{csquotes}
\usepackage{color,soul}
\usepackage[margin=.6in]{geometry}
\begin{document}
\title{Chinese Religions 2}
\maketitle
\section*{Buddhism in China}
\label{sec:buddhism_in_china}
Buddhism is very focused on otherworldy things while chinese culture is very focused on this world, but for a while it was quite popular.

\paragraph{Arguments Against}
\label{par:arguments_against}
Many people were against the rooting of budhism in china. It was a foreign religion and china was very against anything foreign as barbarian. They received many different buddhist texts that contradicted each other which made it hard to sort it all. The language barrier was another problem. The chinese also felt that buddhism had too much emphasis on non-worldly concepts. They also disliked the notion of karma because it put animals and humans on the same level, they heavily rejected vegetarianism. The heaviest resistance came against the monastic system since chinese culture put so much emphasis on family and children. The chinese felt that the buddhist monks were parasites since they did no work for their food.

\paragraph{Arguments For}
\label{par:arguments_for}
The introduction of buddhism came at a time of great upheaval. Barbarians were common and were quite impressed with buddhism, the felt it could unite their territories. The chinese were also very impressed by the bodhisattva. Life of the peasentry was hard in china so the bodhisattva were attractive. They also like the notion that no matter how low your position you still possessed the buddha nature. Karma was also a welcome idea because it meshed well with confucian ideals because it helped strengthen social order. The translation of buddhist texts gave them more insight into it. Confucianism did not explain life after death so they used the buddhist concepts. Daoist philosophy was already very similar to buddhist ideas so the translation often used daoist term and was used to expand daoist philosophy. They adapted the monastic life as the movement to the monastery as a movement to a new family as the abbot was like a father. The monks started working and producing, this is what allowed the zen tradition to survive persecution(the monestaries were often in isolated areas and self sustaining). Finally buddhist monks in china agreed to bow to the emperor (as opposed to in india where monks were higher).

\section*{Buddhist Schools: The Tiantai and Huayan Schools}
\label{sec:buddhist_schools_the_tiantai_and_huayan_schools}
The chinese wanted to explain the contradictions within the various buddhist texts they had.

\paragraph{The Tiantai School}
\label{par:the_tiantai_school}
Established by \emph{Zhiyi}(538-597). It categorized the texts to account for contradictions between them. It had that the buddha taught in stages so he had to keep them simple for a bit.
\begin{enumerate}
	\item theravada
	\item simple mahayana
	\item advanced mahayana (emptiness)
	\item lotus sutra
\end{enumerate}

\paragraph{Huayan School}
\label{par:huayan_school}
They also felt that the buddha had to teach in stages which lead to the contradictions. They thought he started with the full knowledge and had to dumb things down as he went, so he then jumped down to theravada. They also believd that the absolute and the temporary were infused with each other. Emptiness is that from which all things come and it is the absolute. The temporary is the form that issues from emptiness. These are infused. Everything is a manifestation of the absolute and this explains the relationship between all living beings.
\begin{enumerate}
	\item theravada
	\item mahayana
	\item emptiness
	\item garland sutra
\end{enumerate}

\section*{Pureland Buddhism}
\label{sec:pureland_buddhism}
This focuses on the concept of the \emph{degenerate age}. There is an idea that all things decay over time, even dharma. Enlightenment becomes more difficult because individuals become less competent. The narrative focuses on a specific bodhisattva, \textbf{Dharmakara}.

\paragraph{Dharmakara}
\label{par:dharmakara}
He was a monk that heard a particular speech that led him to take the bodhicitta. He vowed that he would create a pureland where it would be easier to attain enlightenment. The goal is now to be reborn in the pureland since full enlightenment is not easy here. Supposidly Dharmakara fulfilled his vows and becomes the buddha \textbf{Amitabha} and creates the pure land \textbf{Sukhavati}. Everything there is designed to make us think about dharma to help us attain enlightenment. It is described as very lush.

\paragraph{Ritual}
\label{par:ritual}
You recite a homage to the buddha Amitabha 10 times and fill your mind with this image. This recitation purifies the mind and allows rebirth into the pure land.

\paragraph{New Developments}
\label{par:new_developments}
Less emphasis on the recitations and more emphasis on the grace of Amitabha. The idea was that the recitation alone was not enough, you needed an act of grace as well. We now have the notion of faith ``in'' someone instead of faith ``that'' the buddha's words are true. We also see a shift to a more simplified religious practice. No longer were people focusing on meditation and instead recitation and homage. This makes it much easier for people to practice. The view of enlightenment became much more inclusive as well. The people who more traditional sects would say cannot attain enlightenment are allowed into the pure land by the grace of Amitabha.

\paragraph{Chinese Development}
\label{par:chinese_development}
Pure land was developed in India but it flourished in china due to the help of three patriarchs. \emph{Tanluan}(476-542CE) was the first and most notable. He move away from many forms of practice to put more emphasis in the recitation and your faith in Amitabha. Instead of complicated rituals you should put most of your effort into your faith and recitation.

\section*{Buddhist Schools: Chan Buddhism}
\label{sec:buddhist_schools_chan_buddhism}
This was instituted in china in the 5th century by \emph{Bodhidharma}. There are many stories about this guy. One is that he was summoned by the emperor who complained that he has helped buddhism alot and wants to know how much merit he has accrued. Bodhidharma said ``none''. He was banished to the north where he meditated for nine years until his legs fell off. This shows that in chan buddhism one must be totally dedicated to meditation above all.

Chan buddhism believes that you can attain enlightenment without assistance. Meditation is the best path to this.

\paragraph{Dharma Assembly}
\label{par:dharma_assembly}
(732CE) This event was convened to discuss the views on enlightenment and succession between the northern and southern schools. The northern school held that enlightenment was gradual, the souther school believed that it was sudden. The southern school was taken by the sixth patriarch \emph{Huineng}. The southern school won. Enlightenment is sudden and the same in all beings. The same buddha nature dwells in everyone.

\paragraph{Key Points}
\label{par:key_points}
Meditation is the most important ritual. There are stories about zen masters destroying texts, many sutras were very important and these master had already committed the texts to memory. Do not take these stories literally, instead these show the idea that we can become to attached to scholarly knowledge and this can halt our progress on the path. Chan buddhism also has the notion of mind to mind transmission. A chan master has a duty to produce a dharma successor, one who has the same mastery of dharma as the master. The mind of the disciple and that of the teacher become as one. \textbf{Gongon/Koan} are stories and riddles that help us explore dharma. There is no simple answer, reason cannot really be applied, they are meant to push the mediator beyond this.

\section*{Buddhist Schools: Persecution 845}
\label{sec:buddhist_schools_persecution_845}
There were many schools that developed in china and were supported. In 845 a daoist emperor came into the throne and instituted a persecution of buddhism. It only lasted about a year but it devastated all but the pure land and chan schools. Pure land survived because it had become a mainstream religion, it had too many followers to just kill them all. Too much of society had it in them. Chan was a more elite practice. It was spared because their monasteries were in remote areas and they were self sufficient. The main stream population also greatly respected chan monks and monasteries so they worked to protect them

\section*{Worlds of Meaning: Religious Concepts}
\label{sec:worlds_of_meaning_religious_concepts}

\paragraph{Shang Di}
\label{par:shang_di}
The concept of heaven comes from the Shang dynasty, it was a divine personality that probably evolved from the ancestor gods of the ruling family. He controlled nature and brought good and evil to humanity.

\paragraph{Tian}
\label{par:tian}
The Zhou dynasty added the idea that heaven was a universal moral order. It was not a personality, but a cosmic power. Tian gives the right to rule to a specific group, the \textbf{Mandate of Heaven}. The Zhou use this as an argument for them taking the throne from the Shang.

\paragraph{Personal or Impersonal}
\label{par:personal_or_impersonal}
This has been a running debate about heaven. \emph{Maozi} said it was like a caring father. \emph{Xunzi} said it was just the operation of the universe. \emph{Liu Yuxi} explains both positions and concludes that it is heaven that produces and reproduces, and it is humans that create moral order and regulation.

\paragraph{The Cosmos}
\label{par:the_cosmos}
The world is self evolved, there is no creator go. Prior to all categories there was chaos. The cosmos is represented as an egg, outside the egg is chaos, the white is heaven, and the yolk is earth. Chaos is sometimes identifies with the dao. From the dao we get yin and yang, heaven and earth, everything.

\paragraph{The Dao}
 \label{par:the_dao}
This creates everything. It cycles yin and yang to rise and fall. It also operates the five elements which are always moving and changing.

\paragraph{Humans}
\label{par:humans}
Humans have two souls, the yang or heavenly soul, and the yin or earthly soul. The yin soul is burried in the earth and the yang soul goes upwards to heaven.

\paragraph{Path of Transformation}
\label{par:path_of_transformation}
For daoist this is a path of withdrawl, meditation, and wuwei

\section*{Worlds of Meaning: Religious Daoism}
\label{sec:worlds_of_meaning_religious_daoism}
This operate on two levels, that of the expert/priest and that of ordinary people.

\paragraph{Expert/Priest}
\label{par:expert_priest}
They must learn the religious registry and all the rituals. They must nourish the life forces within, spirit, breath, and spermatic essence. These are concentrated in the cinnabar fields, the head, the heart, and the belly. These are also the places where the \emph{Three Pure Ones} reside. These are the cosmic gods. They must preserve and draw down the lord or heaven, the lord of earth, and the lord of humans. The lord of heaven is the representative of the primordial breath, the lord of earth of the spirit, and the lord of humans of the seminal essence.

\begin{itemize}
	\item head - lord of heaven - primordial breath
	\item heart - lord of earth - spirit
	\item belly - lord of humans - seminal essence
\end{itemize}

The expert must renew the yang and learns the techniques for calling down the three pure ones. Restoring yang forces is also done through diet, exercise, and meditation. The expert can also act on behalf of the community, all cosmic energy moves through the expert into the community. \textbf{Jiao} is the rite of cosmic renewal which involves offerings to the pure ones (the offerings must be pure).
\begin{itemize}
	\item wine
	\item tea
	\item cakes
	\item fruits
\end{itemize}
This is different from the meat offerings that are often made to the more popular gods. The worship of the three pure ones is done by the priest alone, away from the people. The power of the ritual goes through the export into the community.

\paragraph{Ordinary People}
\label{par:ordinary_people}
Use the experts to renew yang energy. They worship at temples and they worship their ancestors. Evil and suffering occur because we are out of balance so ordinary people must seek balance.

\section*{Worlds of Meaning: Popular Religion}
\label{sec:worlds_of_meaning_popular_religion}
Popular religion focuses on the immediate and practical concerns of human life. These are rituals that:
\begin{itemize}
	\item expel evil spirits
	\item protect property
	\item bring prosperity, health, longevity
	\item respect the ancestors
	\item favor the gods
\end{itemize}

Many gods of popular religion were once human and became elevated. The relationship is reciprocal. You make a donation and the god provides the service. If the god does not act the person can appeal to another god.

The gods have a hierarchy, likened to a heavenly court. At the top is the \emph{Jade Emporer}, also known as the lord of humans. In the second layer are the \emph{city gods}, the \emph{tai shan}, and the \emph{houshold gods}. Under the city gods are the \emph{tudigong}, local gods. Under the houshold gods are the \emph{Zaojun}, god of the cooking stove, he keeps a ledger of the family's deeds and reports them to the Jade Emperor.

\paragraph{Divine Figures}
\label{par:divine_figures}
The most revered figure is \textbf{Guanyin}, the bodhisattva that you can come and request for help in all things. \textbf{Mazu} was once a young girl that saved her brother but not her father, so she is associated as a goddess of the sea.

Mediums, writers (communicates with ancestors), and healers are also common in popular religion. Feng shui is another aspect

\paragraph{Ritual Life}
\label{par:ritual_lif}
\textbf{Qing Ming} is a day of honoring the ancestors. You clean their graves and renew family ties with a feast. It also ushers in spring. \textbf{Ullambana}, the feast of souls. You give offerings and burn spirit money. This month is where the gates of purgatory are left open to allow souls to wander about. Buddhist monks are busy during this time. \textbf{New Year} is a major festival. The \textbf{Dragon Boat Festival} is also very popular is southern china.

\end{document}
