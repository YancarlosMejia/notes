\documentclass{article}
\usepackage{parskip}
\usepackage{pdfpages}
\usepackage[margin=.6in]{geometry}
\begin{document}
\section*{What is Religion}
\label{sec:what_is_religion}

Religion is a response to:
\begin{itemize}
	\item the human condition, namely \textbf{anomie} (meaning a lack of meaning)
	\item chaos
	\item death 
\end{itemize}

Religion responds to it through putting us in touch with the sacred:
\begin{itemize}
	\item ultimate in what is real
	\item ultimate value
	\item ultimate meaning
	\item ultimate wholeness
\end{itemize}

It explains the relationship from my life, to others, to the world.

Religion connects us to what is sacred through:
\begin{itemize}
	\item myth
	\item symbol
	\item ritual
\end{itemize}

\textbf{Myths} are narratives or stories that touch the core of our being (meanings and values in our lives).

\textbf{Symbols} are a shorthand way of communicating profound information. Connecting complex ideas with an image for communication. 

\textbf{Ritual} many kinds of rituals to connect with the sacred, recreate things, rights of passage, 


\section*{What Makes a Religion}
\label{sec:what_makes_a_religion}

\begin{itemize}
 	\item Theoretical - what is said or written
 	\begin{itemize}
 		\item doctrine - what people believe about the sacred
 		\item philosophy - applies logic and consequences to beliefs
 		\item myths - provide narrative to those beliefs
 	\end{itemize}
 	\item Practical - our action in response to our beliefs
 	\begin{itemize}
 		\item ritual - things we do to commemorate/aknowledge/recreate the sacred and our connection to it
 		\item ethical - how we behave in connection to our religion
 	\end{itemize}
 	\item Sociological
 	\begin{itemize}
 	 	\item community - provides support and the way that tradition is passed on
 	 \end{itemize} 
\end{itemize}

\subsection*{Methodology}
\label{sub:methodology}
We approach religion as a human phenomenon.

Religion can be:
\begin{itemize}
	\item study
	\item describe
	\item analyze
\end{itemize}

We can provide a picture of the main elements of a tradition such that believers of it will recognize it.

We can examine how religions can remain living.


\end{document}