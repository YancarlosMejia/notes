\documentclass{article}
\usepackage{parskip}
\usepackage[margin=.6in]{geometry}
\begin{document}
\title{Conclusions and Reflections}
\maketitle

\section*{Challenges to Religion}
\label{sec:challenges_to_religion}
\paragraph{Modernism}
\label{par:modernism}
This is associated with the beginning of the enlightenment. It increased secularism. Religion no longer dominates our view of the world, it has be supplanted by science, medicine, and politics. Religion is kept separate from everything else which makes it challenging for religion to keep its relevance.

\paragraph{Globalization}
\label{par:globalization}
The world is shrinking. We have more contact with other religions. This challenges our religious view of our world.

\paragraph{Rapid Change}
\label{par:rapid_change}
Modern society changes far quicker than religion can. We always feel insecure due to this constant change. This causes us to retreat and retrench. Often times this is religion that we retreat to, causing conservativism.

\paragraph{Violence}
\label{par:violence}
Conflicts are now much more publicized. Religion still plays a large role in many modern conflicts. Adding religion to a conflict deepens it and escalates it. Religion engages in our core values and this causes a pattern of escalation then we get a holy war.

\paragraph{Holy War}
\label{par:holy_war}
There are three factors that are common cross culturally to a holy war:
\begin{itemize}
	\item success is thought of as inevitable, it is fought on behalf of god and god will always be victorious
	\item long timelines, people are willing to fight for a long time because it is on a cosmic level
	\item individual sacrifice, people are much more wiling to sacrifice in a holy war because they will be rewarded after death
\end{itemize}

\section*{Future of Religion}
\label{sec:future_of_religion}
Religion will continue to be until \textbf{Anomie}, \textbf{Chaos}, and \textbf{fear of death} are removed from the human experience. Religion can play a positive role in the world:
\begin{itemize}
	\item mutual understanding and respect - try to understand others, all religions say the human view of the sacred is limited for us to achieve mutual understanding
	\item vision of the good life for all
	\item religious commitment supersedes commitment to the state or other groups - religion can be subversive and cause people to reject immoral governments
	\item inter religious cooperation - global concerns requires global cooperation
\end{itemize}

\section*{Dialog: Approaches}
\label{sec:dialog_approaches}
\paragraph{All Religions are the Same}
\label{par:all_religions_are_the_same}
We might want to develop a super religion, or global theology. This option is rejected because many religions are already multifaceted and contradictory. When we look at new religious movement or personal religions we cannot expand this global theology to encompass them. We would be ignoring the integrity of the individual religions. The context is what makes religious symbols.

\paragraph{Dialog is Impossible}
\label{par:dialog_is_impossible}
We cannot get people to agree on an exclusive truth. Dialog needs to involve listening and understanding instead of trying to convert people. Here we are just trying to find the exclusive truth. Linguistic problems are a big hindrance here. Much can be lost in translation. We can try to communicate using basic human qualities which can get points across much better. Dialog between different faiths does happen, we can understand each other.

\section*{Process of Dialog}
\label{sec:process_of_dialogue}
How can we make religious dialog more effective.

\paragraph{Being Self Aware}
\label{par:being_self_aware}
bracket your preconceived notions. Put your ideas on the shelf for a while when trying to understand someone.

\paragraph{Walking in their moccasins}
\label{par:walking_in_their_moccasins}
by bracketing our own notions we try to enter the universe of symbols of someone else. Try to walk in their moccasins to see their religion from the inside to the best of our abilities.

\paragraph{Return Home}
\label{par:return_home}
examine the new ideas that we learned from walking in their moccasins. Make meaningful comparisons with our own religion

\paragraph{Integrate}
\label{par:integrate}
Use our knew knowledge to see our own ideas clearer and try to understand the other better.

Guide to dialog:
\begin{enumerate}
	\item grasp of ones own tradition - we need to understand ourselves before engaging in dialog
	\item goal is not conversion - we just want to understand, not win
	\item respect for others
	\item willingness to share - be vulnerable, show what is most valuable to us
	\item willingness to learn - try to grasp their ideas, willingness to grow and change
\end{enumerate}


\end{document}
