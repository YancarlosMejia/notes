\documentclass{article}
\usepackage{parskip}
\usepackage{csquotes}
\usepackage{color,soul}
\usepackage[margin=.6in]{geometry}
\begin{document}
\title{Textbook}
\maketitle
\section{Perspectives on the Religious Path}
\label{sec:perspectives_on_the_religious_path}

\paragraph{The Sacred}
`'is the focal point of religion, the ground of ultimate vitality, value, and meaning. The various modes of experiencing the sacred and the responses to this experience is what makes up the religious traditions of the world.

\subsection{Studying Religious Experience}
\label{sub:studying_religious_experience}
Some scholars approach the study of religion using scientific methodology to develop theories that explain religious activities like any other human activity. Others believe that scholars should just describe the phenomenon of religion as it actually exists and as it is interpreted by its practitioners.

\subsection{Some Dimensions of Religion}
\label{sub:some_dimensions_of_religion}
The word religion comes from the latin \emph{religio} which means bond or link.

Key dimensions of religion:
\begin{itemize}
	\item human involvement in the realm of the sacred
	\item expressed in thought, action, and social forms
	\item a total system of symbols with deep meaning
	\item path of ultimate transformation
\end{itemize}

\subsubsection{Human Involvement in Realm of Sacred}
\label{ssub:human_involvement_in_relm_of_sacred}
This suggests a relationship between two levels of experience, that of the limited human level and that of the sacred level. Basic to human experience is a deep sense of the \emph{numinous} (a word stemming from latin for holy/sacred). The numinous has an unlimited, primordial, overpowering quality which leads to us to say that the sacred is ultimate, the basis of everything else, and nothing can supersede or encompass it. The human reaction to experiencing the sacred is called \emph{mysterium tremendum} which means terrifying mystery because it evokes awe in us. Since the sacred is the source of ultimate value, the deepest need of fhuman life is to have a relationship with it.

\subsubsection{Expression in Thought, Action, and Social Forms}
\label{ssub:expression_in_thought_action_and_social_forms}
\paragraph{Religious Tradition}
\label{par:religious_tradition}
It is suggested that there are three modes of religious expression that combined make up religious tradition.
\begin{itemize}
	\item theoretical: thinking or speaking
	\item practical: doing or acting
	\item social: fellowship or community
\end{itemize}

\paragraph{Theoretical Mode}
\label{par:theoretical_mode}
sets forth a way of thinking about the most important issues of life. You can talk about it through narrative (story or myth) or doctrine (theoretical  statements)

\paragraph{Practical Mode}
\label{par:practical_mode}
is the expression that has to do with religions visible and performed side.

\paragraph{Social Forms}
\label{par:social_forms}
is religion as it is experienced by the group. The commumity is what carries religious tradition forward.

\subsubsection{Total System of Symbols}
\label{ssub:total_system_of_symbols}
Religion guides and gives meaning by presenting a whole view of the world through the use of \textbf{symbols}. If we imagine all that a human knows as a circle of the symbols important to them, then all the symbols will have a centeral vision which is colored by their religion. The symbols closest to the center are the \textbf{primary symbols} which are essential to those of that religious path. Towards the outside of the circle are the \textbf{secondary symbols} which change easily with new experiences. Religious tradition is not static. When we consider a symbol we must view it in its total context as the same symbol may have many meanings.

\subsubsection{Path of Ultimate Transformation}
\label{ssub:path_of_ultimate_transformation}
Religion is a way of life, a path of transformation to ultimate meaning. Basically we recognize our fractured relationship with the sacred and use religion to make our way back to it.

\subsection{Basic Human Concerns and Religious Responses}
\label{sub:basic_human_concerns_and_religious_responses}
We explore religious responses through three areas of questions:
\begin{itemize}
	\item Sacred Story
	\item Worlds of Meaning
	\item Ritual Practices and the Good life
\end{itemize}

\subsubsection{Sacred Story and Historical Context}
\label{ssub:sacred_story_and_historical_context}
Individual tend to connect their own story with the sacred story of their religion. Frequently the story of the founding or the revealing of the religious path is important as it frequently ties divine authority to one's religious identity.

\paragraph{Myths}
\label{par:myths}
are stories that form the central focus by which people express their religious identity. There is always one \emph{master story} more important than the other.

Our understanding of religious tradition comes form it being passed down between generations, which gives it a dynamic quality as each generation puts their own spin on it.

\subsubsection{Worlds of Meaning}
\label{ssub:worlds_of_meaning}
Often the power encountered in the world gets personified as gods or spirits resulting in \emph{polytheism}. In most of these religions there is a supreme god (a creator or one who weild absolute authority) that delegates power to other gods. If there is only one god its called \emph{monotheism}. Some religions are \textbf{nondualism} which means that there is no difference between the two realities (real, and sacred). \textbf{Monism} is the view that all reality is one unified divine reality. Within monism there might be gods but they are facets of the one sacred reality

Many religions try to deal with the big questions through \textbf{cosmogonic stories} about the creation and maintenance of the world. Frequently the world is created as the result of a divine beings actions and it is humanity's role to assist this being in their work. Through this we start to get the concept of and \emph{ideal human existence}. By having this ideal to work towards humans can evaluate their actions and make changes to move toward it.

Many religions have a path that leans you to a transformation way from the "badness" of the world and towards a more perfect life. A religious path presents methods of interaction with the realm of the sacred so that this sacred power can transform your life. Some religions emphasize that humans cannot transform themselves alone and must rely on outside forces to help them and others emphasize the exact opposite. One of the distinctive characteistics of a religion is its particular vision of the interaction between human practice and sacred gift usually through the modes of religious expression. The path to follow usually involves distancing yourself from badness and moving toward goodness.

The path to transformation si a means and an end. You want to reach the ideal state but you also are experiencing the sacred just by being on the path.

\subsubsection{Ritual Practices and the Good Life}
\label{ssub:ritual_practices_and_the_good_life}

Most cultures allocate \textbf{sacred time} in which you practice ritual. It is thought that humans could not tollerate the meaningless chaos that would evolve from a life without special or ``strong'' times to provide centers of meaning. This is often called ``re-creation'' because we are recreating our connection with the sacred. It is at these time that we feel we are contemporaries with the spirits and heros of our myths. Rituals also reorient our life back in line with the sacred path. They also connect the sacred to average life to remind us that its always there.

Theres needs to be an emptying out called \textbf{kenosis} of the power of life. This allows renewal to take place. This is followed by a filling up called \textbf{plerosis} where the renewing power of the sacred can be felt. Often there is an inbetween state called \textbf{liminal} where you return to a sort of prebirth existence.

Rituals are often repeating on patterns of varying length but they can also occur at important life cycle changes and only happen once (usually on an individual basis).

Most religions also have rituals to bring about healing and wholeness to a person. This often entails the creation of sacred time in which the individual can tap the powers of the sacred to heal. Frequently the community helps.

Religion is most often expressed through art because interaction with the sacred is based on perception (aka \emph{aesthesis}). Art can point beyond itself to the sacred dimension and this act of symbolizing the sacred also works to share it somehow. Sometimes are is there to represent the sacred and other times it just presents it (presenting the sacred allows you to experience it while representing it is more informative). Art can evoke experience of the sacred much better than rational and logical attempts because it is so much more expressive. Often times religious architecture tries to embody \emph{axis mundi} which is the center of the world when planning their orientation. Sometimes are can be too evocative and be considered idolatry (worship of an idol) which is not ok in some religions. Religious art of the east emphasizes intuitional, meditative, aesthetic experience while religious art of the west emphasizes word, intelligence, and logic.

Most religious societies gain their structure from their master story. This can result in religious leaders who's power is derived from their office or religious leaders that derive power from their own charisma.

Almost all of our understanding of religion is from a male centric point of view. Very rarely are there predominant females in myths or in places of power, and if there were little is known about them

Many religions have a sacred place usually establised in their master story as the center of the world. A sacred land provides a feeling of rootedness which is why is so very important to most religions.

All religions have a \emph{sacred history} usually outlining what their gods did and such. It is in these texts that the good life is usualy defined. The ethical life is how we should be and usually based on the religious vision of creation and human nature. This is usually decreed by some higher being outside oneself that we then internalize and make part of our life. Ethical life centers around controlling your desires and transforming them into something good to best serve your community. Many religions view their ethics as the truth and thus feel a responsibility to spread it to others so that they many live the good life as well.

\subsection{Key Terms}
\label{sub:key_terms}
\paragraph{Aestetic}
\label{par:aestetic}
concerning beauty or artistic perception, important for religious expression.

\paragraph{Cosmogenic Myth}
\label{par:cosmogenic_myth}
sacred story that tells of the creation or founding of the world and of basic human realities.

\paragraph{Ethics}
\label{par:ethics}
thought and study about moral decisions on the basis of traditions of right and wrong.

\paragraph{Healing Rituals}
\label{par:healing_rituals}
religious rituals devoted to promoting health, often complementary to modern medical practices.

\paragraph{Kenosis}
\label{par:kenosis}
``emptying out''; in ritual, the movement of separation or doing away with the old state

\paragraph{liminal}
\label{par:liminal}
in ritual, the state between separation and restoration

\paragraph{Monism}
\label{par:monism}
view that all reality is one unified divine reality

\paragraph{Monotheism}
\label{par:monotheism}
belief in one almighty god, separate from the world

\paragraph{Myth}
\label{par:myth}
story about sacred beings in the beginning time, telling how existence came to bas as it is and providing the pattern for authentic life

\paragraph{Nondualsim}
\label{par:nondualsim}
view that ultimate reality and the phenomemonal world are not different

\paragraph{Path of Transformation}
\label{par:path_of_transformation}
practice in a religious tradition that changes one from the wrong state to the right state

\paragraph{Plerosis}
\label{par:plerosis}
``filling up''; fulfillment or restoration movement of ritual

\paragraph{Polytheism}
\label{par:polytheism}
belief that many divine powers share in the world's operation

\paragraph{Religious Traditions}
\label{par:religious_traditions}
sacred stories and basic ideas and practices that religious communities hand over from generation to generation; that which is handed over (Latin tradio) is thought to maintain a recognizable unity even while changing over time

\paragraph{Rites of Passage}
\label{par:rites_of_passage}
rtuals connected with the critical chagnes or passages in a person's life

\paragraph{Rituals}
\label{par:rituals}
activities of many kinds that connect people with sacred realities

\paragraph{Sacred Space}
\label{par:sacred_space}
space that is made special by connection with the sacred, providing orientation for a people

\paragraph{Sacred Story}
\label{par:sacred_story}
master story of a religion, providing identity for the adherents

\paragraph{Sacred Time}
\label{par:sacred_time}
special time of ritual and festival, when mythic events are made present ocne more

\paragraph{The Sacred}
`'what is experienced as ultimate reality that is the ground of ultimate value and meaning

\paragraph{Symbols}
\label{par:symbols}
words, pircures, ideas, rituals, and so on that evoke deep meanings by connecting with sacred reality

\paragraph{Transformation}
\label{par:transformation}
the act of reaching the ideal state

\paragraph{Understanding}
\label{par:understanding}
``standing under'' another's way of thought and life, comprehending it by reference to ones own experience

\paragraph{Worship}
\label{par:worship}
respectful ritual activity in special times, directed toward sacred beings or realities of ultimate value

\section{Hindu Sacred Story and Historical Context}
\label{sec:hindu_sacred_story_and_historical_context}
The word hindu stems from the persian perd meaning indian. The only people of india that wouldn't be classified as hindus are those that do not repectect the Vedic scriptures and social class structure they describe.

\subsection{Foundations of the Hindu Tradition}
\label{sub:foundations_of_the_hindu_tradition}
Hinduism does not have a specific founder. This faith places little emphasis on historical events or people, most things happen in a transcendental time frame. The vedic scriptures are considered timeless and eternal. The events described in their epics are said to have happened hundreds of thousands of years ago during various mythic ages. They do not attempt to fit their religious events and people into human time frames, givving them a eternal religious truth.

\subsubsection{The Formative Period}
\label{ssub:the_formative_period}
The hindu story probably started with the Aryans in the indus valley, as their sacred text was the Vedas. The aryans migrated to india around 2000 BCE and absorbed some of the indigenous peoples religions into the formative hinduism they practiced. The civiliation of the indus valley was very advanced (enough to rival the egyptians) with its own writing system and advance agricultural techniques. It had two major cities, Mohenjo-daro and Haooa taht were major centers for religion and government. There was a uniformity of culture throughout this civilization (all cities were laid out with the same plan) which implies it was controlled by powerful rulers.

The IVC (indus valley civilization) had a great reverence for water, each house had a bathroom and the major city centers had large water tanks for ritual bathing. The concept of water purification is also prevalent in hinduism. The region also greatly worships fertility and cycles (as a argiculturally based people this makes sense) which may have influenced the concept of samsara in hinduism. They also worshiped the male fertility power with many phalic symbols. Most interesting of all is a recurring figure of what appears to be a man doing yoga. It has been suggested that this figure grew into Shiva, a great yogi that is often symbolized in a lingam (a phallic symbol). The IVC started declining around 1900 BCE due to geological changes (extreme dryness).

After the IVC died out the Aryans moved into the area. The aryans were pasturalists and splintered into tribes with cheiftans. They had a three caste society (priests, warriors, and herders, sound familiar). They domesticated horses and invented the chariot and were quite good at metal working. The aryans didnt have a writing system for  a long time, so the Vedas were ritual hymns and shared through word of mouth, supposidly it was heard by the poet-prophets of old. This is how hinduism was formed. The four Smhitas were collections of the original hymns, also called the Vedas.

The \textbf{rig-veda} is the oldest and most important collection. The \textbf{sama-veda} contains the verses to be sung to music during the sacrifice. The \textbf{yajur veda} had forumals to be spoken by the preist who performed the physical component of the ritual. the \textbf{atharva veda} caontained incantations for priests to use during various other events (childbirth, illness, etc), kind of like spells.

The worship of the gods centered around sacrifice, petition, and praise. They resided in three realms, sky, atmosphere, and earth. The sanskrit word for god \emph{deva} means shining. The gods were the powers that create life and growth in the many things around us (wind, fire, speech, consciousness, etc). There are many gods so we'll look at the four most commonly mentioned in the rig-veda.

\paragraph{Varuna}
\label{par:varuna}
is the god of the vault of the sky. He guards the cosmic order (called \emph{rita}). He does this by watching humans and punishing their wrongdoings with disease (he catches them with his noose) so you must petition him to remove these effects.

\paragraph{Indra}
\label{par:indra}
is the god of the atmosphere (storm, lightnight, thunder, and so on). He is a boistrous god that like to drink and lead people into battle (think of Thor). He is famous for defeating the demon vritra what had shut up the waters and the sun.

\paragraph{Agni}
\label{par:agni}
is the god of fire, he resides in the third realm (earth). He is most important as the god of the sacrificial fire, he accepts the sacrificial offerings and transports them to the real of the gods. This makes him the priest of the gods.

\paragraph{Soma}
\label{par:soma}
god of soma, an intoxicating drink, also resides on earth. This drink is often offered to the gods as a sacrifice. He represents exstacy and the power of those that have experience the divine.

The aryans believed that life centered around the \emph{atman} or breath/should of a person. This lived on after death and by means of a funeral fire sacrifice could be transported to the heavenly realm (depended on their deeds in life). Most devotion to the gods came in the form of fire sacrifice. It would be done daily by the head of the house and in more complex ceremonies by the priests. It was believed that the gods would come and sit with you to enjoy the sacrifice and listen to your petitions. It was believed that these sacrifices were necessary to help sustain the gods so that they could continue their work on earth.

Hinduism entered a speculative period around 1000 BCE which resulted in the \textbf{Bramanas} and the \textbf{Upanishads}. The sages start to think that some power prior to the gods is the origin of everything and they start looking for the sacred center. They try to describe it and even personify it has a giant man called \emph{purusha}. It is when Purusha is sacrificed that the universe comes into existence. This means that when a priest is enacting a sacrifice he is mimicing the primordeal sacrifice. Purusha also becomes the classes.
\begin{itemize}
	\item mouth - priest class
	\item arms - warrior class
	\item thighs - merchant class
	\item feet - servant class
\end{itemize}

This is the first time the hindu caste system is mentioned in the vedas. This means that the caste system is of cosmic order. It also marks a transition where the priest class is now at the top where the warrior class used to be.

This period also showed an increase in the importance placed on knowledge. It is not just the sacrifice that is important but the inner knowledge that powers the blessing.

There isnt a clear break between the brahmanas and the upanishads. These combined with the vedic hymns are collectively known as the vedas. The brahmanas outlined new rituals that had exploded in complexity. The upanishads looked more into the inner truth of reality and the cause of human problems. Upanishad may come from the notion of teachers passing down knowledge (upa = near, ni = down, sad = sit).

This is the point at which ``repeated death'' is introduced. This is the concept of a death that ends all future lives. This is something to be feared but it grows into the concept of samsara and our need to escape it. The upanishad concludes that samsara is the cause of human suffering. The suffering you go through is caused by the karma attached to your atma from a previous life.

The upanishads continue to speculate on the one source of all and eventuall name it \textbf{Brahman} which is prior to all that is know, including samsara. From this they derive that knowledge of brahman could bring you liberation from samsara, called \textbf{moksha}. This meant that it is through inner knowledge that you can reach the divine and not external sacrifice.

The vedic hymns, brahmanas, and upanishads are collectively called the \textbf{shruti}, that which was ``heard'' by the sages, and form the basis of hindu belief.

\subsubsection{The Many Faces of Hindu Tradition}
\label{ssub:the_many_faces_of_hindu_tradition}
Many powerful nations had started to form in the area and these gave rise to challenging faiths to Hinduism. Two such movements where Jainism and Buddhism, both of which strongly opposed the sacrifices used in hinduism. Jainists focused on asceticism (self control) and buddhists emphasized discipline and knowledge. In the midst of all this unrest, Alexander the Great invaded in 400 BCE and the Maryuna family created a dynasty in northern india that expanded greatly. This dynasty was not very loyal to the vedic faith and frequently converted to buddhism. Eventually unity was refound around 500 CE.

During this period of great upheaval many different religious traditions were formed. They tended to focuse in a number of directions, later denoted \emph{margas} (paths). This period was known as the epic period due to all these changes. Many new writings were composed and they are referred to as the \textbf{smriti} or what was remembered. They include the yogas sutras (yoga sutras of patanjali), the dharmasutras (the law-code of marnu), and the epics(ramayana and mahabharata).

\paragraph{Yoga}
\label{par:yoga}
A common concept in vedic ritual is the withdrawal from society and practicing ascecitism as you should have the utmost control of yourself. The image given is that the body is the chariot and the soul is the charioteer. Yoga takes withdrawal one step farther and aims to withdraw even from mental activity. The Yoga Sutras of Patanjali outline an eight-step method for discipline and control which is sometimes called royal yoga. This writing is often considered scripture.

\paragraph{Dharma}
\label{par:dharma}
Dharma is another name for the cosmic order (the thing guarded by Varuna) and many sages spent much time trying to outline what that is. In the Dharma Sutras they attempt to picture an idealized society. This often lead to tension with the persuit of moksha that required withdrawing for society, but this is considered just a facet of the hindu faith. The Lawn of Manu is considered the laws of the cosmic order as passed down by Manu, the originator of man. It outlines the rights and responsibilities of the four classes (also called varnas) in hindu society. This was probably an attempt to order society after a time of great chaos and upheaval but came to be very influential and still practiced today.

\paragraph{Ramayana}
\label{par:ramayana}
These epics outline the ideals of hindu values in a perfect history. It does not reflect actual events though. The Ramayana is the story of Vishnu and his fights with demonic forces. The hero of the story is \textbf{Rama} and prince that was born as an \textbf{avatara} (incarnation) of vishnu. The villian is Ravana the demon king of the island Lanka. Rama fights a great war with the demon to save his wife Sita with the help of some monkey people lead by their cheif Hanuman. During this time he sets a model for the ideal ruler. He accepts 14 years of exile instead of questioning his father's rash decision and even exiles his beloved wife because his subjects question her chastity and it was causing unrest. Rama is still considered a great hero and has a festival Ramlila every autumn.

\paragraph{Mahabharata}
\label{par:mahabharata}
The Mahabharata outlines the conflict between two clans in northern india. The heroes are the five Pamdava brothers who have a conflict with their cousins about who will rule. It is a long and rambling tale, but the most famous part is the Bhagavad-Gita (song of the beloved one) in which one brother, Arjuna is overcome with grief when he sees that his enemies contain many good people. He says that he would rather die than inflict such pain. His charioteer is Krishna (an avatara of Vishnu) and he explains that he must do his duty as a warrior. He explains that although he will be killing their bodies, their atman will continue much like shedding clothes. He also explains that action done without desire (for reward) is higher than not acting. So you can seek liberation by taking selfless action which helps ease the tension between seeking moksha through withdrawal and the need to maintain society. Krishna also explains that liberation can be achieved through \textbf{bhakti} (the love of the gods), he says that the highest path is selfless action and wisdom through loving and surrendering to the gods. In this way the Gita summarizes the three paths to liberation.

\subsection{Continuing Transformation in the Hindu Story}
\label{sub:continuing_transformation_in_the_hindu_story}
Unlike many other religions, hinduism doesn't have a clear end to its scripture, it can continue to evolve limitlessly.

\subsubsection{Shaping the Sacred Ways: Puranas and Tantras}
\label{ssub:shaping_the_sacred_ways_puranas_and_tantras}
An explosion of bhakti resulted in many devotional cults to various gods. The new line of kings united northern india and gave strong worship to Vishnu. A series of stories called the \textbf{Puranas} were published about the various gods and devpotion to them. Many stories focus on Vishnu, his wife Lakshmi, and his avatara Krishnu.

A great revival of Vishnu bhakti swept through india as lead by \emph{Chaitanya}. It introduce parades and singing and dancing as shows of devotion. He also synthesizes Krishna worship and built many temples. These pracitces are still carried on today even in north america through the International Society for Krishna Consciousness (aka Hare Krishna).

Shiva is the god of dualities so his stories have a different tone. He is creation and destruction, male and female, so on. Shiva is a great yogi so many of his followers were ascetics and yogis. Worship of him became very prevalent in southern india. The largest group was called the \emph{Shaiva Siddhanta} and they had their own sacred texts equal to the vedas called the \emph{Agamas}. Another group called the \emph{Virashaiva} protested the caste system and refused to go through the intiation ceremony. They also were called the \emph{Lingayats} because they wore the lingam (Shiva's symbols).

In conjunction with the growing worship of Shiva, a group stated worshiping his wife Shakti. Since Shiva embodies all dualities it was believed that Shakti was his power personified. Sometimes she was called Pavarti or Uma. Shakti herself is also worshiped as a separate goddess through her three avataras Durga, Kali, and Devi. The relationship between these three is muddied, but Kali and Durga are vicious and destructive. Durga is famous for saving the world from a demon. Devi is worshiped as the divine mother. Kali is said to have emerged from when Durga became furious while fighting demons. She's scary as shit DO NOT CROSS KALI. Weirdly enough there are many people who worship Kali, the destructive mother, specifically.

\paragraph{Tantrism}
\label{par:tantrism}
This is the persuit of moksha through elaborate rituals including yoga and the worship of the great Goddess. It focused on uniting the dualities, specifically that of Shiva (intelligence) and Shakti (creativity). The right handed path involved matras, mandalas, and yoga to activate Shakti to unite it with Shiva. The left handed path was only for advanced practitioners, it involved a special ritual called circle-worship, and using forbidden elements. The forbidden elements (called the  M's since all their names in sanskrit start with M) are wine, meat, fish, parched grain, and sexual intercourse. This is often frowned upon by traditional hindus.

\subsubsection{Rethinging the Vedic Truth: Philosophical Systems}
\label{ssub:rethinging_the_vedic_truth_philosophical_systems}
Many people spent a lot of time and thought on the philosophical underpinnings of the Hindu tradition. Through this extensive questioning 6 schools or darshanas (meaning view points) of evolved.
\begin{itemize}
	\item Samkhya
	\item Yoga
	\item Nyaya
	\item Vaisheshika
	\item Mimamsa
	\item Vedanta
\end{itemize}

\paragraph{Nyaya}
\label{par:nyaya}
(school of logic) used logic as a means to reach liberation.

\paragraph{Vaisheshika}
\label{par:vaisheshika}
(atomistic school) used cosmology as a means to reach liberation.

\paragraph{Mimamsa}
\label{par:mimamsa}
focused on the eternality of the vedas and thus how important vedic rituals are to dharma.

\paragraph{Samkhya-Yoga}
\label{par:samkhya_yoga}
focused highly on the 8 stage path taught by Patanjali. This was combined with the samkhya's map of the cosmost and the causes of bondage. It talks about the duality of pure matter (a person's body and mind) and pure spirit (the transcendent consciousness). Bondage stems from the mind mistaking itself for the pure spirit and we must pull these apart to reach moksha.

\paragraph{Vedanta}
\label{par:vedanta}
focused on the teachings of the Upanishads concerning Brahman. They focus on the union of atman and brahman and the correct knowledge (jnana). \textbf{Shankara} was a famous thinker in this school that wrote many very influential commentaries on the upanishads and the bhagavad-gita. He focused on the fact that brahmna does not have a duality and thus is identical to the atman. He believes that our problem is that we view the world as changing and the way to liberation is through meditation to break this illusion.

\paragraph{Advakta}
\label{par:advakta}
believed in the nondualist vedantic view, founded by Shankara. There are no permanent individual selves and no samsara, these are all illusions. This implied that devotion to a god will not bring liberation because even the gods are just an illusion. There are different levels of truth and for those at a lower level, devotion to a god can lead you to spritual understanding.

\paragraph{Vishishtadvaita}
\label{par:vishishtadvaita}
founded by \textbf{Ramanuja} beleived that withing reality there is a distinction between the self and god.

\subsubsection{Musil Presence and Impact on India}
\label{ssub:musil_presence_and_impact_on_india}
A great expansion of Islam spread throughout southern asia. By the thirteenth century muslim was dominant in northern india. Many clashes occurred between the two very different religions resulting many temples being replaced by mosques and heavy religious restrictions. The two religions heavily influenced each other and their ideas.

\subsection{The Modern Era: Renaissance and Response to the West}
\label{sec:the_modern_era_remaossamce_amd_res}
When england arrived in india a bunch of stuff changed, they introduced christianity and english and many other european things.

\subsubsection{Reform Movement and Thinkers}
\label{ssub:reform_movement_and_thinkers}
Western ideas brought to india lead to people questioning the caste system. A man named \emph{Ram Mohan} spoke out against abuses in the hindu system (ploytheism, neglect of womens education, and burning a window alive at her husbands cremation). He founded the \emph{Brahmo Samaj} to advocate a rational hamistic religion without the hindu rituals and customs. This society later broke away from the vedas by arguing that reason and conscience are the authority of religion. This group still advocated for ethical reform until it collapsed due to internal arguing.

The \emph{Arya Samaj} was founded by Swami Dayanada Sarasvati also wanting to restore hindu purity. He rejected the puranas and all the popular gods of hindu as well as the caste system. In his book he argued that hindus should only rely on the original vedas and all other religions had perverted this truth, it should be the source of all science as well. Anyone can study the vedas and women should have more rights. In its work for social reform the arya samaj was very similar to the brahmo samaj, but they strongly rejected western influence.

\paragraph{Ramakrishna Paramahamsa}
\label{par:ramakrishna_paramahamsa}
he was a temple priest for Kali, but the worshiped many other gods. He practiced worship through a trace like state. Weirdly enough he also followed western disciplines from Christianity to islam. He used these experiences to expand hindu to incorperate any tradition of worship. This was based on his belief that the ultimate reality could be experienced by any one of any religion. His disciple Swami Vivekanada traveled the world spreading hindu.

\subsubsection{Independence and New Visions: Gandhi and Aurobindo}
\label{ssub:independence_and_new_visions_gandhi_and_aurobindo}
Gandhi was the leader of the independance movement in india to get away from britan and remove muslim influence. He was heavily influenced by the bhagavad-git and the sermon of the mount and he used these to develop the philosophy if \emph{satyagraha} (holding the truth) which is an act of nonviolent resistance that awakens guilt in your opponent. He lived an ascetic life with his wife and his spinning wheel which became the symbol of his movement.

Aurobindo was a contemporary of ghandi that retreated from the world to practice tantric yoga and rewrite his books that described that spiritual reality can be found through the practice of an all-encompassing discipline of yoga.

\subsection{Hindu Worlds of Meaning}
\label{sec:hindu_worlds_of_meaning}
\subsubsection{The Ultimate Real: Brahman}
\label{sub:the_ultimate_real_brahman}
Hindus often refer to two levels of truth when referring to brahman. There is the `formed' brahman which is personified for worship and the `unformed' brahman which is more a force of nature. The brahman without attributes (unformed) is called \emph{nirguna} and is often described by saying what it is not, this is the highest truth. The brahman with attributes is called \emph{saguna} and is the creative power of the universe. Usually this is for experiencing brahman in a personal sense called \emph{isvara}. When you become enlightened you gain knowledge that the physical world is an illusion, called \emph{maya} superimposed on the one reality, brahman. It is through closer inspection that we can come to know this.

The above concept of two levels of truth is called \emph{dualism} and its main proponent is \emph{Shankara}.

\subsubsection{God as Supreme Sacred Reality}
\label{sub:god_as_supreme_sacred_reality}
Many philosophers argued the opposite of Shankara and said that the personal experience of brahman (saguna) is the highest truth. This movement was lead by \emph{Ramanuja}. He was a dualist that was very devoted to Vishnu. He argued that saguna brahman was like your souls soul. God is the ultimate reality and so worshiping him is the highest truth.

The bhagavad-gita presents Vishnu as the highest god and in essence brahman. He is the creator and preserver of the world and his wife Laksmi is the goddess of wealth and abundance. When the world is threatened by demons he embodies one of his avataras to go fight (as seen in the ramayana epic). The vaishnavites worship him.

Many worship Shiva as the ultimate reality as they feel that the duality of his nature better represents the ultimate reality. A story in the puranas shows brahman and vishnu arguing and a large pole separating them from which the sacred om emerges. It is revealed that this pillar is shivas lingum and they worship him. He represents all dualities, yoga, and divine dancing. His lingum is the axis of the universe. He is worshiped bu the shaivites.

Some people worshipt Shakti (shiva's wife) as supreme. She has many different forms and is usually associated with shiva. Durga is often worshiped for her power in fighting demons and protecting the world. A demond was set to attack the world and he was impervious to everything except a woman so the gods pooled all their power and formed durga, thus she is the embodiment of the strength of all the male gods. Her fury becomes Kali. She can also use her power to nourish the world. Some hindus believe there is a unified goddess Devi (Goddess) or Mahadevi (great goddess). Shakti (power) is applied to her because is is the creative force.


\subsection{Existence in the World: Dharma and Samsara}
\label{sub:existence_in_the_world_dharma_and_samsara}
\subsubsection{The World and Human Nature in Hindu Thought}
\label{ssub:the_world_and_human_nature_in_hindu_thought}
The penultimate truth of the world is that \emph{the worlds is here, it is real, it functions by the eternal Dharma, the gods work to keep it going, and humans have the duty to contribute to its welfare}.

Dharma is the main source of truth when you are trying to understand how you should live. It is the eternal order of things so this makes sense, The word Dharma means to ``sustain'' or ``support'' and in hindu it kinda means the essential foundation. Originally it was similar to the vedic word rita which was the univeral harmony of everything.

\paragraph{Creation Stories}
\label{par:creation_stories}
The original version of creation is that the world was formed from the sacrifice of Purusha. A story in the upanishads has the first man get so lonely he falls into two people and they breed and shapechange to create all life. A similar story has the primordial one desire to become many and in the heat of this evolve into the entirety of creation. The last creation story tells of Vishnu who is Brahman unified (he has spirit, matter, and time united in him). He decides to play and creates the whole universe throught uniting those three things and creates a vast egg resting on the cosmic waters. He enters this egg as Brahman and creates the three worlds (earth, sky, atmosphere) and populates them. Finally he becomes the preserver of the world. When the world becomes exhausted he kills it and takes a nap until it is time to create a new one.

\paragraph{Time}
\label{par:time}
The last of those stories fits the concept of cycles that is prevalent in the hindu belief. It has time broken up into smaller cycles, \emph{yugas}, set within larger cycles, \emph{kalpas} which denote times between the world creation and destruction. Each kalpa is 1000 mahayugas which are 4 320 000 years. Each mahayuga has four lesser yugas which degenerate until a renewal happens in the new mahayuga. Each kalpa is one day of brahman and the resting time between kalpas is one night. They belief that Brahman will live for 100 years of 360 days. Then the process reverses itself until there is just vishu ... until he decides to play again.

\paragraph{Humans Selves}
\label{par:humans}
Hindus believe that there is a \emph{real} self and an \emph{empirical} self. The empirical self is made of the elements (earth, water, light, wind, and ether) and includes the subtle body made of vital breaths, the organs of action, and the organs of knowledge. The real self is the atman

\subsubsection{Samsara and the Problem of Existence}
\label{ssub:samsara_and_the_problem_of_existence}
Hindus view the cycle of rebirth as bad, but they still view the value of happiness in life.

This section is just a rehash of samsara, karma, and moksha.

\subsection{Three Paths to Transformation and Liberation}
\label{sub:three_paths_to_transformation_and_liberation}
\subsubsection{The Path of Knowledge}
\label{ssub:the_path_of_knowledge}
Also known as \textbf{jnana-marga}

The end goal is to understand that you are brahman and thus earn release from samsara. You must know this in more than your mind since your subtle self is still tied to samsara. It is because we don't know this we view our material self as the main concern and thus are trapped by worldly desires. If we understood that we are brahman we would have no worldly desires because we are all.  Knowing this breaks all desires and thus all karma and thus achieves moksha.

This path requires education and spritual perfection so only the elite can really follow it. The first step is to become a \textbf{samnyasin}, a renouncer,as someone who renounces worldly possessions (including ties to other humans). Some people cannot fully renounce everything so they take partial steps. Yoga involves moral restraints and withdrawing the senses inward. Meditation is the most common activity in the path of knowledge. This is usually the process of withdrawing your consciousness inward, away from reality, to allow yourself to experience brahman. OM or AUM is often used in meditation. A stands ofr the waking state with consciousness turned outward, U stands for the dreaming state with consciouness turned inward, M stands for the deep-sleep state which is the blissful unified consciousness, and the silence between  is the end goal of experiencing brahman.

\subsubsection{Path of Action}
\label{ssub:path_of_action}
Also known as \textbf{karma-marga}

The path of knowledge is the only real way to achieve moksha, but only a few can start on that path. When we act out of desire it generates hot karma which negatively effects our rebirth. Action done without desire creates cool karma and possitively effects our rebirth. After enough good karma has been accumulated you will be reborn as someone with enough spiritual purity to follow the path of knowledge. One way to act without desire is to fulfull your dharma.

\subsubsection{Path of Devotion}
\label{ssub:path_of_devotion}
Also known as \textbf{bhakti-marga}

Not everyone can follow the path of knowledge and our own desires may get in the way of us following the path of action so a third path is needed. This is spiritual purity through love of the gods. Many believe that specific gods are actually brahman and by devoting yourself to them so fully as to forget all worldly desires (self-surrender) they can help you achieve liberation (it works the same way as the path of knowledge in that you become one with brahman through love). Frequently groups of people were banned from reading the vedas and following the path of knowledge, but anyone can follow the path of devotion.

\paragraph{Vishu}
\label{par:vishu}
Worshippers of Vishnu (vaishnavites) can be feel him throughout all of creation and its continued existence or through his avataras. A common theme in stories of Vishnu (in particular Krishna) is the lovable nature of god as he is reborn amongst humans. God invites us with port and pay to share in the divine self-delight. Krishna frequently goes to earth and falls in love with women leading some to believe that the highest form of loving a god is to be their lover and surrender to them in ecstasy.

\paragraph{Shiva}
\label{par:shiva}
Worshippers of Shiva (shaivites) are all about experiencing him in the duality of the world. He is very wild and destructive. He also emodies the tension between erotic powers and ascetic powers, he is both the great ascetic but also wildly attracted to pavarti. In one store he destroys \textbf{Kama} the god of desire and uses his desire for pavarti to bring him back ass all life needs kama for fertility. His worshipers often practice yoga and are ascetics, they feel great guilt for being separated from him. Usually they experience him through his immense holiness.

\paragraph{Shakti}
\label{par:shakti}
Worshippers of Shakti use many different names for her but all are worshipping the great goddess. They believe that she represents the energy active at the heart of the world. Some tantri practicioners even worship Kali and confront her head on as the source of fear which must be overcome, others view her as a mother that they must cling to like children. It is through her cruetly that Kali taught her children to abandon worldly desires.

\section{Hindu Ritual and the Good Life}
\label{sec:hindu_ritual_and_the_good_life}
\subsection{Rituals and Sacred Times of Hinduism}
\label{sub:rituals_and_sacred_times_of_hinduism}
\subsubsection{Vedic Rituals and Puja}
\label{ssub:vedic_rituals_and_puja}
Hinduism centers on spiritual purity where interaction with the sacred powers is essential but obstacles in human life can limit it. Many bodily functions are considered polluting (for instance menstrating) so the morning devotions should happen after a purifying bath.

\paragraph{Vedic Rituals and Sacrifice}
\label{par:vedic_rituals_and_sacrifice}
The most common ritual is the sacrifice  of food and drink to a sacred fire which then takes the sacrifice to the gods. This can be done simply in the home (for instance to the house's sacred fire) or in large scale events that brahamns lead. Frequently in times of disaster people will turn to brahmans to try to get the grace of the gods back through ritual. The most important verse is the \emph{Gayatri} which is the first words of veda and said to hold the whole power of the vedas. When this is recited in the morning and at night it destroys the effect of bad deeds, drives away darkness, and harmonizes your mind.

\paragraph{Puja: Celebrating the Powerful Presence of God}
\label{par:puja_celebrating_the_powerful_presence_of_god}
Vedic rituals are usually limited in who can practice them (women and the slave class were banned) but the act of worshipping a god, called \textbf{puja} is not. These are usually rituals performed in the presence of the devine image at a temple or domestic shrine. The image is usually created by sacred craftsmen who build it according to special iconography and then a ritual is used to call down the presence of the god. The large icons at temples are usually very carefully tended to as they treat it like a visiting dignitary. Puja at a temple begins with purification, calling the god, and greet them with respect. The god is honored with \textbf{kirtana} which is devotional worship (could be in the form of singing, music, dancing, offerings, marking your self, or any number of other acts). The worshiper then holds their hands over a sacred fire and touches their forehead. The aim is to see the god, called \textbf{darshana}, and thus known them better. The gods have food prepared for them and some of it is given to worshipers and it is called \textbf{prasad}. The worshipers eat this food and thus accept the blessing of the god.

\subsubsection*{Ascetic and Meditational Practices: Yoga}
\label{ssub:ascetic_and_meditational_practices_yoga}
Sacrificial fire has always been very important in hindu tradition and meditation frequently looks to internalize the heat and power from sacrifice. Many forms of meditation are spoken of as yoga. Most spiritual practices are called a yoga. There are many different variations and practices from the well known western version to sitting in a circle of fire. All of these stem from the classic eight-stage system from Patanjali's yogasutras. He aims to find discriminative knowledge to tell the difference between the physical/mental reality (prakriti) and the true nature (purusha).

Ethical values
\begin{itemize}
	\item moral self restraints(yama) - getting you life in moral order through nonviolence, truthfulness, not stealing, sensual restraint, and not being greedy (dont be an asshole)
	\item moral commitments(niyama) - laying the moral foundation of purity, contentment, self-discipline, self-education, and dedication to god
\end{itemize}

Physical Practices
\begin{itemize}
	\item Postures(asana) - involving steadiness, endurance, equilibrium, relatxation, and meditation on the infinite
	\item regulation of vital energy through breath(pranayama) - controlling the energy of the mind-body system by breath exercises
	\item withdrawal of senses(pratyahara) -  eliminating mental distraction from the sense-organs so that all powers of consciousness can be focused inward on the source of being
\end{itemize}

Cultivation of consciousness
\begin{itemize}
	\item concentation(dharana) - one pointed mental concentration
	\item meditation(dhyana) - sustained attention to the object
	\item meditative trance(samadhi) - the culminating experience in which the object of meditation vanishes and the mind wells to encompas a limitless reality
\end{itemize}

\subsubsection*{Ceremonies, Festivals, and Pilgrimages}
\label{ssub:ceremonies_festivals_and_pilgrimages}
\paragraph{Daily Rituals}
\label{par:daily_rituals}
Many small ceremonies happen on a daily basis for the devout. They rise before the sun to meditate, bathe, and offer puja before breakfast. This sequence may be performed again before sunset for brahmans. Some very orthodox houses give five sacrifices daily (study the vedas, food to the gods, water to the ancestors, food to the brahmans and students, and food to all beings). The kitchen is a sacred place as the ceremony of serving food is one most hindus have time to follow. The food is often encircled with water and mantras are chanted which turn the food into prasad.

\paragraph{Festival Calendar}
\label{par:festival_calendar}
Villiage wide ceromonies are a bigger deal but the typical villiage will have over 40 yearly occasions. A famous one at the end of the rainy season is Durga Puja celebrating durgas triumph over the demon mahisha. This festival lasts 10 days. Divali is another festival where they light shit tons of lamps. Lakshmi is the main patron and symbols of prosperity are all over the place. The most popular festival is Holi where people burn a bunch of stuff under a full moon. It usually devolves into a rave where everyone gets high and parties.

\paragraph{Sacred Geography}
\label{par:sacred_geography}
Pilgrimages start with rituals of separation (shaving the head, special clothes, and leaving your house). The actual journey is a liminal state where the ordinary structures of social life are removed. Caste is ignored and people sing. When they get to the site they gain sacred power through rituals and receiving darshana. Once thats all done they return home to rejoin society. Some of these locations are visited at specific times (ex Kumbha Mela is held every 12 years), others are very remote (ex Mt. Kailasa in the himalayas is the home of Shiva and Parvati). Some cities are sacred (ex Banaras the city of light, another home for shiva and parvati).

\subsubsection*{Ritualizing the Passages of Life}
\label{ssub:ritualizing_the_passages_of_life}
Hindus have a series of \textbf{samskaras} to ritualize an individual's life events.

Before birth the husband does the ritual parting of the hair in which we parts his wife's hair and applies a red mark to ward off demonds.

Ten days after birth is the naming ceremony.

At 3 years there is the first hair cut which leaves only a small tuft that some high-caste hindus never cut.

\textbf{Upanayana} is the initiation ceremony performed when a male is 8-12. This is when the boy in introduced to his vedic guru that will teach him. He drapes a sacred thread over the boy's shoulder that will mark him as twice-born for the rest of his life. He is then taught the gayatri which he recites daily thereafter and he is shown how to perform the ritual fire sacrifice. He has now died in the world of childhood and been reborn as an adult so he can now study the vedas.

There are some puberty rights for girls (not recognized by orthodox brahman tradition). These tend to be localized and not really standard.

Marriage is a very important time. The law of manu implies that this is the female equivalent of upanayana, serving her husband is similar to studying the vedas and doing household chores is similar to the ritual sacrifice. Parents arrange the pairing and do some premarriage ceromonies. The day of the wedding is chosen by an astrologist. On the day of the families gather at the bride's house where a priest performs sacrifice and mantras. The groom says some stuff, the bride steps on a grinding stone to symbolize firmness and their clothes are knotted together. They take seven steps around a fire and the groom says some more stuff. Finally the husband puts a mark on her forhead and they go outside to look at the northern star.

The last samskara happens at death. The body is washed and clothed before the family carries it to the cremation grounds led by the eldest son. The eldest son then lights the funeral pyre and says a prayer to agni. The mourners say prayers to help the soul reunite with its ancestors leaving behind sin and avoiding yamas dogs. After the cremation the mourners leave without turning around and take a purifying bath before returning home.  Three days later the bones are gathered and brought to some place with sacred water. For the 10 days after death many sacrifices of food and water are done to build the next body (the 10 days symbolize the 10 lunar months of pregnancy).

For those born of high spiritual purity there is another ceremony to become and \textbf{samnyasin} in which they renounce the world. They give away everything they own and shave their heads. They then perform the household rituals for the last time and a guru may cut the sacred string releasing them from caste.

\subsubsection*{Healing and Medicinal Practices: Ayurveda}
\label{ssub:healing_and_medicinal_practices_ayurveda}
\paragraph{Aryuveda}
\label{par:aryuveda}
is a holistic system of medicine. It has seen a recent surge in popularity and the government of india has a department to promote it. There are as many practitioners of aryuveda as there are western medicine. Everything in the world has its particular characteristics through the interaction of three universal qualities called \emph{gunas}.
\begin{itemize}
	\item Sattva - white/bright bringing clarity of perception
	\item Rajas - red bringing passion, emotion, senses, and movement
	\item Tamas - dark brings heaviness, inertia and confusion
\end{itemize}
Each person's body is made of five elements:
\begin{itemize}
	\item Space - permits the flow of intelligence between cells
	\item Air - vital force that governs all sensory stimuli and mortor responses
	\item Fire - regulates body temperature and is responsible for digestion and absorbtion of food
	\item Water - the body fluids, carrying energy from one cell to another
	\item Earth - present in all solid structures and tissues
\end{itemize}
The actual functioning of the body is goverened by three energies \emph{doshas}:
\begin{itemize}
	\item Vata(space and air) - energy of movement, is dry, light, cold, active, astringent
	\item Pitta(fire and water) - energy of digestion, is hot, sharp, light, sour, oily, and bitter
	\item Kapha(earth) - energy that forms the body's structures, is heavy, slow, cool, dense, static, sweet, and salty
\end{itemize}
Having perfect balance of the 5 elements and 3 energies is healthy and sickness stems from imbalance.

\subsubsection*{Hindus and Sacred Art}
\label{ssub:hindus_and_sacred_art}
The hindus believe that the sacred is everywhere so all of your senses are involved in knowing the sacred. Due to this art of all forms is by default sacred. there are many poems in various dialects from various times about the gods. A famous one is about the love between Krishna and Radha, called \emph{Gitagovinda}. Chaitanya is famous for introducing singing as a form of worship so there are also many songs. Statues and religious carvings are also very common. The mere act of creating one of these is a yoga or devotional act. The artist must see into the divine to make the image (called \emph{murti}) so that others might take darshana from it.

The iconography used in this art is on page 82 of the textbook. I dont feel like writing it out.

Drama is another sacred art. Frequently ceremonies wil invovle reenacting various stories and myths. These dramas usually contain mostly dancing. By watching the play (called \emph{lila}) the audience can experience \emph{rasa}, the flavour of the sacred presence on earth.

Architecture is also very important as temples are the homes of the gods and fix the center of the world. Temples are modeled after squares to reflect divine perfection. The temple will have a holy center called \emph{garbha} (womb) which houses the divine image form with power radiates. This is thought of as the atman of the temple and the rest of it around the garbha is like the mortal body. Some temples have towers representing the mountain one must ascend and others have deep underground areas representing the navel of the cosmos from which the gods send their power.

\subsection*{Society and Wholesome Life}
\label{sub:society_and_wholesome_life}
\subsubsection*{Structure of Hindu Society}
\label{ssub:structure_of_hindu_society}
Modern hindus still follow the caste system very closely. Caste discrimination is outlawed but it still has its hold over many people. The caste system is actually two systems meshed together:
\begin{itemize}
	\item varna - color
	\item jati - birth
\end{itemize}
There are a couple varna classes and thousands of jati classes. The castes are orginally outlined in the vedas and draw their origin fron the cosmic order Dharma. The five varnas are:
\begin{itemize}
	\item Brahmans - study and teach the vedas, oversee ceremonies
	\item Kshatriya - protect the people and run the government
	\item Vaishya - provide for the economic needs of the community
	\item Shudra - serve the upper three casts, not twice born, cannot study the vedas
	\item Untouchables - have no caste, do very polluting duties, have to avoid other classes to not pollute them, not twice born, cannot study the vedas
\end{itemize}

Jati are usually divided by three restrictions on its members:
\begin{itemize}
	\item endogamy (who they can marry)
	\item commensality (who they can eat with)
	\item occupational exclusivity (who they can live with)
\end{itemize}
Almost always jati can only marry within their own jati. Similarly they tend to only eat and live with members of their group. Jati are very concerned with mainting their ritual purity so they frequently avoid interactions with other classes. Villages are laid out to maintain this and when they do have to meet seating is arrange to prevent this.

\subsubsection*{Living According to Dharma}
\label{ssub:living_according_to_dharma}
Dharma more focuses on your day to day duties as it assumes that certain things are known to be evil (killing, stealing, etc) and others are known to be good (respecting your parents, helping others, etc).

\paragraph{Four Stages of Life}
\label{par:four_stages_of_life}
There are four stages of life, called \emph{ashramas}, and what your dharma is varies based on what stage of life you are in. Traditionally Shudras, Untouchables, and women do not go through the stages of life. Women kinda go through the stages of life vicariously through their husbands.

The four stages are:
\begin{itemize}
	\item student
	\begin{itemize}
		\item study the vedas
		\item respect of teacher
		\item develop self control
		\item learn to contribute to society
	\end{itemize}
	\item householder
	\begin{itemize}
		\item women are obliged to serve their husband (this obedience lead to child brides and widow suicides, called \emph{satu})
		\item men are obliged to protect and honor their wife
	\end{itemize}
	\item forest-dweller
	\begin{itemize}
		\item starts when your son transitions to being a householder (has his own son)
		\item basically retirement to focus on spirituality
	\end{itemize}
	\item renouncer(samnyasin)
	\begin{itemize}
		\item
	\end{itemize}
\end{itemize}

\paragraph{Four Aims of Life}
\label{par:four_aims_of_life}
Hindus summed up the good life in four aims (called \emph{purusharthas}):
\begin{itemize}
	\item material prosperity (artha)
	\item pleasure (kama)
	\item liberation (moksha)
	\item dharma
\end{itemize}
Different aims are more important at different stages in life.

\paragraph{Leadership}
\label{par:leadership}
Brahmans are very important leaders in the community. Even though some castes can never hear or see vedic rituals they still reap the benifits. Some men and women that are very learned in spirituality and thus called \emph{gurus}. These people do nothing for society, but they provide a role model to aim for in the quest for moksha. There are many women in religious leadership roles, frequently as gurus.


\subsection*{Key Terms}
\label{sub:key_terms}
\paragraph{Advaita}
\label{par:advaita}
nondualism, shankara's school of Vedanta, emphasizing the all-encompassing one ultimate reality
\paragraph{Agni}
\label{par:agni}
Vedic god of fire
\paragraph{Aryans}
\label{par:aryans}
peopls related to the Indo-Europeans, who migrated into India in ancient times
\paragraph{ashramas}
\label{par:ashramas}
the cour stages of life for higher=class males in hinduism: student, householder, forest-dweller, and renouncer; also a hermitage or place for meditation
\paragraph{atman}
\label{par:atman}
the sould or self, considered eternal
\paragraph{avatara}
\label{par:avatara}
descent or incarnation of a god
\paragraph{Ayurveda}
\label{par:ayurveda}
traditional system of medicine in india
\paragraph{Bhagavad-Gita}
\label{par:bhagavad_gita}
important scripture from the Mahabharata summing up the fundamental ideas of hindism
\paragraph{bhakti}
\label{par:bhakti}
devotion, self surrender to one's god
\paragraph{brahma}
\label{par:brahma}
designation for the creator god
\paragraph{brahman}
\label{par:brahman}
ultimate reality
\paragraph{brahmanas}
\label{par:brahmanas}
ritual commentaries, part of the Vedas
\paragraph{brahmans}
\label{par:brahmans}
highest-ranked caste, priests
\paragraph{darshana}
\label{par:darshana}
ritual act of being granted the seeing of a sacred thing; also the six viewpoints/schools of philosophy
\paragraph{devi}
\label{par:devi}
goddess, sometimes refers to the Great Goddess
\paragraph{Dharma}
\label{par:dharma}
cosmic order, social duty, proper behavior
\paragraph{divali}
\label{par:divali}
autumn festival of lights and good fortune
\paragraph{durga}
\label{par:durga}
fierce goddess, a form of devi
\paragraph{gandhi}
\label{par:gandhi}
leader of hindu independence movement emphasizing spiritual preparation and nonviolent resistance
\paragraph{ganesha}
\label{par:ganesha}
son of shiva, elephant headed god who overcomes obstacles and brings good fortune
\paragraph{guru}
\label{par:guru}
spiritual guide and master
\paragraph{holi}
\label{par:holi}
popular festival in northern india with a carnival atmosphere
\paragraph{indra}
\label{par:indra}
vedic storm warrior god
\paragraph{indus valley civilization}
\label{par:indus_valley_civilization}
urban-agricultural civilization that flourished in the third millennium bce, influenced hinduism
\paragraph{international society for krishna consciousness}
\label{par:international_society_for_krishna_consciousness}
new movement worshiping krishna as the supreme manifestation of the divine, drew many western devotees
\paragraph{jati}
\label{par:jati}
ones caste or closed social group as determined by birth
\paragraph{kali}
\label{par:kali}
goddess of death and destruction, a form of devi
\paragraph{karma}
\label{par:karma}
law that all deeds and thoughts will have set consequences on your rebirth
\paragraph{kirtana}
\label{par:kirtana}
devotional group worship through song and dance
\paragraph{krishna}
\label{par:krishna}
avatara of vishnu, hero of bhagavad-gita
\paragraph{kshatriyas}
\label{par:kshatriyas}
warrior caste
\paragraph{lingam}
\label{par:lingam}
phallic pillar that sumbolizes the great god shiva
\paragraph{mahabharata}
\label{par:mahabharata}
one of the two great epics
\paragraph{mantra}
\label{par:mantra}
powerful sacred words, formula, or verse chanted as a focus for meditation and devotion
\paragraph{maya}
\label{par:maya}
appearence/illusion, term to indicate that which prevents one from seeing truly
\paragraph{moksha}
\label{par:moksha}
lineration from bondage to samsara and karma
\paragraph{path of action}
\label{par:path_of_action}
path toward liberation based on acting according to dharma, without desire of the fruits of actions
\paragraph{path of devotion}
\label{par:path_of_devotion}
path toward liberation based on devotional practices directed toward ones god
\paragraph{path of knowledge}
\label{par:path_of_knowledge}
path toward liberation based on knowledge, emphasizing meditation
\paragraph{prasad}
\label{par:prasad}
gif from the deity consecrated in ritual, expecially food, shared by the devotee
\paragraph{puja}
\label{par:puja}
ritual worship of the image of a god by offering food, flowers, music, and prayer
\paragraph{puranas}
\label{par:puranas}
late scriptures that developed from popular theistic devotional movements
\paragraph{rama}
\label{par:rama}
avatara of vishnu, hero of ramayana
\paragraph{ramakrishna}
\label{par:ramakrishna}
modern holy man whose teachings were brought to america by his diciple vivekanada, who established the ramakrishna mission
\paragraph{ramanuja}
\label{par:ramanuja}
philosopher and avocate of the vaishnavite bhakti tradition
\paragraph{ramayana}
\label{par:ramayana}
story of rama, one of the two great epics
\paragraph{rebirth}
\label{par:rebirth}
belief that after the death of its body the sould takes on another body, determined by karma
\paragraph{rig veda}
\label{par:rig_veda}
earliest and most important collection of vedic hymns
\paragraph{samhitas}
\label{par:samhitas}
collections of early vedic hymns and verses, four collections are rig veda, sama veda, yajur veda, and atharva veda
\paragraph{samkhya}
\label{par:samkhya}
one of the classical schools of philosophy stressing an absolute distinction between matter and spirit
\paragraph{samnyasin}
\label{par:samnyasin}
one who renounces the cares and concerns of the world, the fourth stage of life
\paragraph{samsara}
\label{par:samsara}
the rebirth cycle of existence
\paragraph{samskaras}
\label{par:samskaras}
rituals performed at the critical changes and passages of life
\paragraph{sati}
\label{par:sati}
self sacrifice of a widown on her husbands funeral pyre; outlawed now
\paragraph{shakti}
\label{par:shakti}
devine energy, personified as a goddess; female aspect of a god, especially shiva
\paragraph{shruti}
\label{par:shruti}
``heard'' the eternal truth, the vedas
\paragraph{shudras}
\label{par:shudras}
servant caste
\paragraph{smriti}
\label{par:smriti}
``remembered'' the scriptural writings after the vedas
\paragraph{tantrism}
\label{par:tantrism}
movement using initiation, rituals, imagination, and sexual symbolism as spiritual practices leading towards liberation
\paragraph{transcendental meditation}
\label{par:transcendental_meditation}
meditation movement founded in americal by maharishi mahesh yogi, emphasizing simple miditation techniques for practical benefits
\paragraph{upanayana}
\label{par:upanayana}
initiation of high-class boys into student stage of life; he is given a sacred cord to wear over his left shoulder and taught the appropriate mantras
\paragraph{upanishads}
\label{par:upanishads}
collection of teachings about the self and ultimate reality that makes up the last part of the veda (shruti)
\paragraph{vaishyas}
\label{par:vaishyas}
producer caste
\paragraph{varna}
\label{par:varna}
``color'' classes
\paragraph{varuna}
\label{par:varuna}
vedic god of heavens
\paragraph{vedanta societies}
\label{par:vedanta_societies}
groups in america and europe following the teaching of swami vivekananda and sri ramakrishna
\paragraph{vedas}
\label{par:vedas}
most important scriptures, the shruti, consist of samhitas, brahmanas, aranyakas, and upanishads
\paragraph{vishnu}
\label{par:vishnu}
on of the two great gods, worshiped in avataras rama and krishna as well

\section*{The Path of the Jains}
\label{sec:the_path_of_the_jains}
\subsection*{Sacred Story and Historical Context}
\label{sub:sacred_story_and_historical_context}
Based on the teaching and model of the \textbf{Jinas}, the ``conquerers'', they are also sometimes called the ``Fort Builders'' or \textbf{Tirthankara}. They conquered by reaching liberation from the wheel of existence and show the way across the ocean of suffering. The most recent Jina is \textbf{Mahavira} who lived 2.5k years ago.

\subsubsection*{Mahavira Becomes the Jina for Our Age}
\label{ssub:mahavira_becomes_the_jina_for_our_age}
The universe is a vast structure subject to endless cycles, half of each is progressive and half is regressive. 24 Jinas will arise in each half cycle. The first in our cycle was \emph{Rshabha} who created civilization and the Jain path.The 23rd was \textbf{Parshva} who lived in the middle of the 19th century b.c.e. and established the order of the merchants. Mahavira is the last of our cycle (no more for thousands of years) and brought much reform that structured the jain community as it is today.

Mahavira's story varies between the two sects of jainism, the \textbf{shvetambaras} (white clad) and the \textbf{digambaras} (sky clothed, aka naked). Both secs share the same five key events that all jains celebrate:
\begin{itemize}
	\item conception
	\item birth
	\item renunciation
	\item enlightenment
	\item death
\end{itemize}
He was born around 540 BCE in northern Bihar, India to parents of the kshatriya caste who were followers of Parshva. His father Siddhartha was a warrior cheiftan and his mother Trishala was the sister of a ruler. Around his birth were many signs that he would be great. His mother had 14 dreams involving auspicious things (white elephants, rising suns, jewels, etc). These dreams are still celebrated in art. In the womb he did not kick which showed his dedication to non-violence, called \textbf{ahimsa}. His birth name was \emph{Vardhamana} and he brought his parents much prosperity. He had great spiritual power as a child. He married a princess and had a daughter according to the shvem sect, the digam say he was a bachelor. He fulfilled his duties as a householder until his parents died and he then became a renouncer.

The gods themselves came down and urged him to become a renouncer and attended his ceremony personally. He even pulled out his hair by hand instead of shaving it. This ceremony is celebrated by many jains. He joined a group of hermits for a while but then felt he needed more renouncing. He then spent 12 years being hella ascetic. He practiced very extreme ahimsa, even allowing insects and parasites to attach themselves to him, he didnt travel during the four rainy months to keep from stepping on bugs washed up. He chose to go to cold places in the winter and hot in the summer to meditate.

After these 12.5 years of meditation he reached the highest englightenment, \textbf{kevela}, an extreme omniscient state. And this is how he became the 24th jina.

The digmabara say that after his englightenment he was pure of all defects and sat in omniscient meditation in a hall made for him by the gods. People were attracted to him and thus the jain community was formed. The shvetambara say he preached to the gods and even converted some vedic priests that were about to sacrifice. Mahavira made one of the first structured religious communities with four orders:
\begin{itemize}
	\item monks
	\item nuns
	\item laymen
	\item laywomen
\end{itemize}
His teachings about how to get liberation is very similar to hinduism. He says the sould is born pure, with infinite knowledge, and it is only through actions that it becomes enmeshed with karmic matter. You must find the knowledge required to reach liberation.

At 72 Mahavira reached nirvana and thus set the role model for all jains. His 11 disciples all reached enlightenment. One of his disciples was too attached to him and could not gain enlightenment until he scolded him and died. The stories say that when he preached Mahavira's words became divine and was translated into the scriptures by his disciples who passed them on. The scriptures were called the \textbf{agamas}.

\subsubsection*{Historical Transformations in Jainism}
\label{ssub:historical_transformations_in_jainism}
After Mahavira's death the Jain community spread south and west, often supported by kings. Overtime, differences lead to a schism between the two sects. This started when one group flead south to avoid a famine and while they were gone the northern group renounced parts of the holy text and started wearing clothes. They also did not agree with allowing women to become monks. Th shventambara felt that wearing a loin cloth is a requirement of the medicant life and thus not a possession and they allowed women in as they felt that they could also reach enlightenment. To the shventambara the 19th jain was a woman.

The Jains had many interactions with the Hindus that lived in the same area as them, they frequently interwove hindu faith into their religious context. There were many professions that Jains cannot hold, and they had a reputation for honesty so they gained much commercial success. They also do much charity work and give much to the community. Its also clear how their practice of ahimsa influenced the independence movement in india.

\subsection*{Jain Worlds of Meaning}
\label{sub:jain_worlds_of_meaning}
\subsubsection*{Ultimate Reality: Eternal Universe, Liberation}
\label{ssub:ultimate_reality_eternal_universe_liberation}
Jains do not have a supreme being that created the universe, the universe just operates itself without beginning and end. The only permanent thing are the laws around how the universe flows through its cycles. Since all 24 jinas have been revealed we know that no one will be able to attain enlightenment and thus things will decline.

Jains still talk about god, but they are referring to god as representing the totality of all the jinas and other perfected souls. So its ok to worship the jinas as conquerors whose souls have been liberated.

God is worshiped under many names:
\begin{itemize}
	\item ishvara
	\item vishnu
	\item brahma
	\item mahadeva
\end{itemize}

\subsubsection*{Karmic Mattern and Eternal Souls}
\label{ssub:karmic_mattern_and_eternal_souls}
There is no beginning or end to the process of the universe. The eternal principles by which the universe operates are congruent with the findings of science. For instance they believe that karma operates through karmic particles that could be examined by science.

The world is an eternal three dimensional structure. It is frequently envisioned as a human (man or woman) with arms and legs apart showing the three levels.
\begin{itemize}
	\item lower level containing many hells
	\item middle level - humans and animals home
	\item upper level - gods home
\end{itemize}
Beyond these levels is the relm of the liberated souls. Outside of this is nothingness.

The middle level is subject to the cycles of time. It has the above described half cycles each of which have six stages. Only during the middle stages of each half cycle allows for movement toward enlightenment.

Everything can be divided into souls(jiva) and nonsouls(ajiva). Nonsouls are things like space, motion, and time. Space has infinite souls which are embodied in matter which is what leads to karma. You can be reborn as anything, with four main categories:
\begin{itemize}
	\item gods
	\item humans
	\item hell beings
	\item animals and plants
\end{itemize}

These four categories were initially symbolized in the swastika.

Jains see karma as a subtle form of matter. The universe is filled with tiny particles of material karma attracted to an embodied soul. The soul's inherent energy creates vibrations that attract karmic particles to it due to its impurities. Due to worldly desires the soul is moistened allowing karma particles to stick to it, clouding its pureness and creating more desire. Karma causes a change to the soul (likened to wine on your body's chemistry) depending on the type of karma. Different acts attract different types of karma that have different effects on the soul.

The human form is the only one that can reach enlightenment so all classes of jains are very focused on following the path of transformation.

\subsubsection*{Path of Liberation}
\label{ssub:path_of_liberation}
The jain goal is to stop the influx of karma and expel what karma has been accumulated so that the soul can move towards enlightenment. We are currently in a declining part of the cycle so enlightenment is impossible, but being human is so rare that you cannot pass up this opportunity to advance.

Jains reject the notion of bhakti, that some outside force can help you reach enlightenment, they also reject the idea that castes are solid. The soul has an innate tendency towards upward movement. You can have great advancement by having a soul in a relatively pure state an an external force that activates its energies (like coming into contact with a jinas). This transforms the soul in such a way that it cannot drop in purity.

There are 14 stages of purification and you only jump between stages through great shifts. The first awakening is likened to a blind man gaining vision. The soul achieves certain attainments of knowledge and energy that eliminates masses of accumulated karmas. The fourth stage is when the soul gains true insight. At this point the soul cannot fall backward, it is now irreversibly on the path. Even when the soul falls down it retains its true insight. At this point the soul needs to have voluntary restrictions to continue advancing.

The monk and nun classes have a series of restraints and other disciplines. The take the five \emph{Great Vows}, called \textbf{Mahavrata}:
\begin{itemize}
   	\item injuring life
   	\item false speach
   	\item taking what is not given
   	\item unchastity
   	\item possessions
\end{itemize}
Just before enlightenment one enters a trance that stops all activities of the body and mind so that at the moment of death they jump toward liberation.

Laypeople take 12 partial vows, the first five are similar:
\begin{itemize}
	\item violence
	\item lying
	\item stealing
	\item illicit sex
	\item attachment to possessions
\end{itemize}
Laypeople can continue to advance until they are as pios as nuns and monks.

\subsection*{Ritual and the Good Life}
\label{sub:ritual_and_the_good_life}
\subsubsection*{Ritual and Worship Among the Jains}
\label{ssub:ritual_and_worship_among_the_jains}
Jains traditionally are not big on ceremonies and rituals but they live amongst the hinu people and some of it rubbed off on them. They worship the jinas and perform the rites of the life cycle. Most importantly they were encouraged to participate in prayer and meditation.

The jains are technically atheist  in that they did not have a creator god. So when they worship they worship humans who have attained perfection (because their gods cannot attain perfection). Statues of the Jinas are installed in temples and remind the jains of their foal. They are worshiped not because they hear prayer or grant wished but because it elevates the soul to dwell on the perfection that they represent.

The laypeople may worship the gods and goddesses who control and protect aspects of the human existance. There are some sects that outlaw this.

They have daily worship involves a bath and repairing to the temple to stand incorn of the statue of the jina, reciting sacred formulas and making offerings. Tehy many say the rosary of 108 beads and spend a few minutes studying the scripture. If more time is available they may bathe the image in pure water and make more elaborate offerings.

Most holy places are temples built around the image of a jinas where people can come to worship.

Rituals of the passages of life are often observed. There are many throughout your life. Like hinudism there is an initiation ritual at 8 where they become a student. Marriage also works similarly. Death works a bit differently where the dying person ideally has some dying prayers and repetance and then dying in meditation. The ashes are thrown in a river and the family meditates on the trasitory nature of life. There are no rituals to honor the dead,

Jains contributed to the Aryuveda medical system, but most jains follow modern medicine and many are doctors. Ahimsa greatly influences their approach to medicine. They argue that the natural diet of humans is vegetarian and point out how meat and fat can bring negative effects. They do not condone any procedure that involves violence (ex abortions) and they try to end animal testing. They also frequently engage in discussion about how ahimsa applies to their practice.

\subsubsection*{Society and Ethics}
\label{ssub:society_and_ethics}
The jain community has never been large but it is very old and long lasting. This is mainly due to the monks and nuns remaining close to the laypeople and just practice a more extreme form. Women are also very involved in the faith, more so than other contemporary religions. Today there are more than twice as many nuns as monks. For shvetembaras nuns are equal to monks, but for digambaras they cannot do certain things required for enlightenment.  Many nuns are widows that enter later in life.

Some jains have adopted some portions of the hindu caste system. They consider the caste system not as cosmically ordained but as necessitated by events. The first jinas Rshabha was a layman that had to take up arms to deal with lawlessness, thus making the kshatriya caste. He invented new means of livelihood and thus created the Vaishya and Shudra castes. His son honored holy people and gave them a a sacred thread, thus creating the brahman caste. They say that the shudra caste cant be come a full medicant but they are allowed to participate in all ceremonies.

The main distinction between monk/nun and layperson is the taking of the great vows, called \textbf{diksha} ceremony. The initiate renounces all possessions and takes a new name. They renounce everything, receive a wiskbroom (for gently removing insects) and pull their hair out (in five handfuls). The day after the monk/nun goes begging for the first time and whoever gives them their first alms gains great merit.

The most important principle for jains is ahimsa, the practice of nonviolence. They do make a distinction between intent. It is ok for laypeople to harm one-sense organisms for reasons such as food but they are almost all vegetarian. Monks and nuns often go to great lengths to prevent harming living creatures. Some say that nonviolence is the highest form of bliss. It means showing benevolence toward other beings, feeling joy at the sight of virtuous beings, showing compassion toward the suffering, and displaying tolerance toward the ill-behaved.

The restraint of false speech is related to the restraint of violence because lying is motivated by the passions and damages of the soul. You should not have speech acts that might cause harm as well. When this leads to a conflict of vows you should remain silent.

Stealing always brings a rise in greed and causes violence which is why it is a restraint. You should do nothing that causes you to gain at another's expense, such as substituting inferior goods, using underhanded measures, accepting stolen goods, and so on. Even finding and keeping something that has been lost is wrong.

For laypeople the fourth restraint means to have proper sexual behavior only within marriage. Laypeople may take a vow of celibacy later in life. Monks and nuns go so far as to avoid any form of sexual feeling  since that leads to passion.

Monks and nuns clearly embody the fifth restraint as they renounce all possessions. Merely thinking about possessions can be damaging. For laypeople possessions are necessary but you should not be overly attached to them, you should put checks on yourself to keep from becoming too greedy.

\subsection*{Key Terms}
\label{sub:key_terms}
\paragraph{agamas}
\label{par:agamas}
main scripture
\paragraph{ahimsa}
\label{par:ahimsa}
``nonviolence'' important principle
\paragraph{digambara}
\label{par:digambara}
``sky-clad'' renoucing the use of clothing; one of the two main sects
\paragraph{diksa}
\label{par:diksa}
iniitiation ceremony for monks and nuns
\paragraph{jina}
\label{par:jina}
``conqueror'' one who has reached total liberation, also tirthankara
\paragraph{karma}
\label{par:karma}
subtle form of matter that clings to the soul because of the soul's passion and desire, causing rebirths
\paragraph{kevela}
\label{par:kevela}
highest state of enlightenment
\paragraph{mahavira}
\label{par:mahavira}
last jina of the present half cycle
\paragraph{parshva}
\label{par:parshva}
second to last jina
\paragraph{rebirth}
\label{par:rebirth}
belief that after the death of its body, the accumulated karma causes the sould to take on another body from the almost infinite range of possibilities, including one-sense bodies
\paragraph{restraints}
\label{par:restraints}
vows of nonviolence, not lying, not stealing, refraining from wrong sex, and nonposessio or nonattachment
\paragraph{shivetambara}
\label{par:shivetambara}
``white-clad'' accepting the use of clothing, one of the main sects
\paragraph{tirthankara}
\label{par:tirthankara}
``ford builder'' one who has reached total liberation and shows the way across the ocean of suffering

\section*{Buddhist Sacred Story and Historical Context}
\label{sec:buddhist_sacred_story_and_historical_context}
\subsection*{Story of the Buddha}
\label{sub:story_of_the_buddha}
All buddhists stem from the story of the \textbf{Buddha}, Siddhartha Guatama. This means that Buddhism is a \emph{founded} religion. While he is the most important, there have been many other buddhas. The first scriptures were passed down orally by the buddha's disciples after his \textbf{parinirvana} (passing away). Most records of the buddha were written 500 years after his death. In this aspect we cannot find the historical details of his life (much like we cannot find jesus, moses, or muhammad).

\subsubsection*{Sacred Biography of the Buddha}
\label{ssub:sacred_biography_of_the_buddha}
Buddha was born in northern india in the 6th century following the composing of the upanishads. This was a time of great social and political instablility with 16 different states that would eventually form the empire of the Magadha. The main religious leader of this time where wandering ascetics that discussed the nature of reality and human existence. There was alot of religious discussion and experimentation. Buddha was one of these ascetics.

Buddha was born to a tribe called the shakyas (which is why he is sometimes called \textbf{shakyamuni}, wise one of the shakyas). His parents were the king and queen of kapilavastu. He grew up, married, had a child. At 29 he decided to become a wandering ascetic. Buddhists believe that his birth came after eons of preparation and his teachings have ramifications everywhere. Some stories tell of his previous lives. One such story is of a time he was a prince named Mahasattva who sacrificed himself as a meal for a mother tiger exhausted from giving birth since his body was doomed to perish anyway. After much time in heaven he decides to be reborn one last time. His mother dreamed of being taken away by four guardian angels who purify her in the himalayan mountains. While laying there she saw a white elephant enter her body. When he was born his mother was standing in a beautiful grove, four guardian kings placed him on a golden net, the heaven and earth shook. Buddha then stood, took seven steps and said ``I am born enlightened for the wel being of the world; this is my last birth''.

His mother died seven days later and his father married her sister who raised him as a foster mother. He was well educated in order to inherit the throne and he excelled. He was given three palaces and a lovely wife that bore him a son. The gods knew his time of enlightenment was drawing near so they sent him \textbf{Four Sights}.

\paragraph{Four Sights}
The first was an old man he encountered that had to explain old age to him. Siddhartha asks if the ``evil'' of old age will come for him. The old man says yes. Siddhartha thinks about how all humans are robbed of their beauty and strength by time but still pursue selfish desires. The second is a diseased man, the third a dead man. Both lead him to question why people focus on selfish desires. The final sight is a hermit that is above all these evils through his withdrawal from the world. This finally leads Siddhartha to become a hermit.

He wanted to become a ascetic, but he still had a wife, son, and kingdom to take care of. His father tries to persuade him to fulfill his duty as a householder first. He counters with ``It is not right to hold by force a man who is anxious to escape from a burning house''. The king tries to convince him to stay by providing him with dancing girls as entertainment, but he is just disgusted by them and leaves after a silent goodbye to his wife and son. His servant and horse go with him. The gods have to intervene to get the palace gates open after his father ordered them sealed. He then cut off his hair and beard and told his servant to return all of his belongings to his father. This is called the \emph{Great Renunciation}.

For 6 years he wandered, begging for food and learning yoga and knowledge from two teachers. He came to understand that this was not the correct path and switched to extreme deprivation like the Jains. Right before he dies of starvation he realizes that this cannot be the correct path either. He reasons that someone who is so exhausted and ill at ease cannot possibly achieve enlightenment. Weakening the body weakens the mind which halts the path. He takes a middle path. At this point the gods send a cowherd named Sujata to bring him food. The disciples he had gathered left him because they believed he had given up the search.

He then sat in meditation under a wisdom(bodhi) tree at \textbf{Bodh Gaya} vowing not to move form that spot untill he reached enlightenment. The god of passion and death \textbf{Mara} tried to tempt him with his three daughters, then his army, and the earth saw the strength of his meditation and destroyed Mara's hosts. He continued his meditaiotn and eventually reached buddhahood. After this he reviewed his past lives and thought of the suffering of all beings. On the second night he gained the divine eye that allowed him to see all beings impelled by their deeds to continue to be reborn and he gained more compassion.

He sat there for a while anbd looked at his own mind and new freedom. Brahma then came and pleaded with him to share his teachings with the world. He went out and found the 5 disciples that left him earlier and taught them of the middle path and the \textbf{Four Noble Truths}.
\begin{itemize}
	\item Sorrow: All components of individuality are sorrow
	\item Arising of Sorrow: sorrow arises from craving
	\item stopping of sorrow: removing your craving stops sorrow
	\item the way: the eightfild path
	\begin{itemize}
		\item right views
		\item right resolve
		\item right speech
		\item right conduct
		\item right livelihood
		\item right effort
		\item right mindfulness
		\item right concentration
	\end{itemize}
\end{itemize}

He basically embodied alot of the beliefs of india at the time (karma, samsara, moksha, etc) but called liberation \textbf{nirvana} and outlined the \textbf{Eighfold Path} designed to remove craving.

His five disciples affception the Buddha's Dharma and became the first monks and helped create the \textbf{sangha} (the community of people seeking enlightenment). Buddha supposidly even went back and taught his father, and even went to heaven to teach his mother and the gods. Basically he spent the rest of his life traveling and teaching. When the community had enough enlightened monks (called \emph{arhats}, worthy ones), Buddha sent them out as missionarries. Sangha accepted anyone of any caste or gender to study. There were extra rules for women that put them subservient to men though. The community was mostly wanderers but they established retreats during the monsoon season in caves.

It got to be that the community grew too large and Buddha gave monks permission to ordain other monks and so it spread as a republican society. The Buddha also taught people that they could attain merit by supporting the community without becoming a monk.

Eventually the Buddha had to die, he said his goodbyes and went out meditating. This is when he achieved liberation, supposedly the earth and heavens shook. The people of the area grieved and gave him a funeral pyre. Some bones and other relics were taken in golden jars to be venerated for seven days. These relics were divided into eight parts and taken by nearby kings who built stupas (memorial mounds) over them.

\subsubsection*{Word of the Buddha}
\label{ssub:word_of_the_buddha}
The first retreat after the buddha's death 500 arhats gathered in the \textbf{First Council of Buddhism} and collected the sayings and teachings of buddha. Ananda was his primary disciple so they had him recite the Buddha's sermons and dialogues. Another disciple Upani recited the various rules to regulate the lives of monks and nuns. Thus formed the two main sections of scripture the \emph{Sutra Pitaka} and \emph{Vinaya Pitaka}. Later disciples would created the Abhidharma a set of scholarly treaties derived from the word of buddha. These three works are the \textbf{Tripitaka} or three baskets.

Scholars think this might have been a more drawn out process than tradition says. The sayings of buddha were passed down orally with interpretations and additions from the people passing them

\subsection*{Historical Transformations: Shaping the Buddhist Way}
\label{sub:historical_transformations_shaping_the_buddhist_way}
\subsubsection*{Ashoka, the Second Founder of Buddhism}
\label{ssub:ashoka_the_second_founder_of_buddhism}
\textbf{Ashoka} was a great king that helped buddhism spread out of the ganges river valley southward to Sri Lanka. He was the third maurya king and a devout buddhist. He finished the last area through much bloodshed and destructure which gave him great feelings of remorse and led him to study buddhism. After around 5 years of study he passed 14 edicts that were carve in stone around the country that everyone should live buy buddhist percepts.
\begin{itemize}
	\item outlawed animal sacrifice
	\item regulated animal slaughter for food
	\item provided welfare for commoners
	\item built many stupas
	\item supported monastic communities
	\item worked with different buddhist sects to tolerate each other
\end{itemize}

He also sent out many emissaries all over the world to spread buddhism and managed to convert the royal court of Sri Lanka.

\subsection*{Theravada and Mahayana}
\label{sub:theravada_and_mahayana}
Buddhism developed into two branches, \textbf{Theravada}(path of elders) and \textbf{Mahayana}(great vehicle). Theravada became dominant in india and Mahayana became dominant in the orient. A third branch called \textbf{Vajryana} (or tantric) dominated in Tibet.

For the first few centuries after his death many worked to define buddha's teachings resulting in 18 different schools. There wasnt a schism since they all followed the core beliefs, they would just debate alot. Theravada claimed to most closely follow the teachings of buddha. \emph{Mahasanghikas} admitted lay followers and taught that a transfigured buddha existed in endless life beyond to show up sometimes. Aside from Theravada all these schools died out.

In the sanha the arhats would for elite guilds and claim that only they knew true Dharma and thus alienating many people. The hindu bhakti movement started to influence buddhism and more people wanted to be considered equals on the path. People also wanted more ways of expressing devotion and worship. Thinkers started to approach the nature of reality in a more philosophical way. These all together brough about Mahayana Buddhism. They wrote new scriptures in sanskrit.

A big movement in mahayana is to include all people on the path. They still practiced monastic discipline, but they felt anyone could reach enlightenment. They called traditional buddhism \textbf{Hinayana} meaning lesser vehicle. The Theravada claimed that the Mahayana were inventing new scripture since they had their own scripture that they claimed were secret teachings from buddha. The mahayana taught a progressive truth, for basic learners were the tripitaka and for more senior learners the mahayana sutras.

Mahayana is actually the great course of the \textbf{bodhisattva} (one becoming a buddha). They taught that mahayana leads to buddhahood and theravada only leads to arhatship. The theravada has always associated buddhahood to very special individuals. The mahayana looked to a being higher than the arhats in the bodhisattva, being that reached enlightenment but chose to stay in samsara to teach others. Some of these are in the heavens to be praued to for help (sound familiar to bhakti?). For instance mearly hearing the name of Manjusri (appears in your dreams) could subtract your time spent in samsara. \emph{Avalokiteshvara} is described in the lotus sutra as the onmipresent savior who comes down to help living beings.

The mahayana emphasized that the buddha is really the eternal power of dharma and his body is just transcendent and universal. This Dharma body is the only real body of buddha. This body has many manifestations to help people, som are human like Siddhartha Guatama and others are the heavenly buddhas.

\paragraph{Madhyamika}
\label{par:madhyamika}
The mahayana way started to spawn new schools of thought, such as the Madhyamika school founded by \textbf{Nagarjuna}. They thought that all things come into this world as a result of causes and conditions and do not have an independent existence making them ``empty'' (\emph{shunya}). From this they said that the entirety of samsara is characterized by empiness, called \textbf{shunyata}. They said that nirvana was also empty since it was devoid of all definitions. Since both worlds are empty they are equivalent, enlightenment comes from realizing this. Nagarjuna would reduct an object down until he found a contradiction and thus prove that it was empty.

\paragraph{Yogacara}
\label{par:yogacara}
This school of mahayana focuses on how the mind creates and experiences the illusionary world as real. They had a concept called the ``storehouse consciousmess'' which they later equated with the womb of buddha, the place from which all buddhas are born. All beings posses buddhahood and we just have to realize that meaning there are many different paths to buddhahood.

\paragraph{Vahrayana}
\label{par:vahrayana}
This is a trantric from of buddhism taht was heavily influenced by hinduism. It uses mantras, mandalas, ritual sexual intercorse, and such to reach buddhahood. They made new scripture supposidly from buddha to ouline these rituals. These were created by buddha and passed to only his highest students in a coded language that lessers could not understand. This school also expanded the pantheon to include female consorts, gods, and goddesses  as well as many new bodhisattvas. These all reside in us and we must discover them.

\subsubsection*{Great Expansion of Buddhism into Asia}
\label{ssub:great_expansion_of_buddhism_into_asia}
One important duty of buddhists was to spread the word about Dharma. It spread all over the place and adapted to the people that adopted it. Within india it was strong until the 7th century after which it started to decline as the government sponsored monestaries were unresponsive to the needs of the people so they converted more to hinduism since they could practice bhakti easily. Huns also invaded and destroyed monestaries, as did muslim invaders. Hinduism survived since it was based on villiage brahmans, but buddism needed monestaries to continue so it weakened.

Sri Lanka has always had a strong buddhist (specifically Theravada) population, supported by the government. They even have a tooth of buddha and a tree grown from the branch of the Bodh Gaya.

Burma got heavily influenced by indian traders that used it as a gateway. It was Mahayana for awhile but Sri Lanka converted it to Theravada. They adapted it a bit to introduce local spirits, called \emph{nats} that would posses shamans to help them.

\paragraph{East Asian Buddhism}
\label{par:east_asian_buddhism}
Buddhism traveled along the silk road from central asia where it was dominant until muslims took over. Mahayana rose to one of the three great religions of china (next to confucianism and daoism). It adopted some chinese attributes and formed linages of schools of buddhism.
\begin{itemize}
	\item \textbf{Tiantai} - focused on lotus sutra
	\item \textbf{Huayan} - focused on Garland sutra
	\item \textbf{Pure Land} - one of two biggest, focused on compassionate buddha \emph{Amitabha} and promised rebirth in pure land
	\item \textbf{Chan} - one of the two biggest, focused on methods of awakening to ones innate buddhahood through meditation
\end{itemize}

From china these schools spread to vietnam, korea, and japan. Vietnam had been influenced by Theravada from india, but mahayana from china kicked it out.

In korea in intermingled with this shaman based religion. Korea came to heavily influence buddhism and printed its scripture.

From korea it traveled to japan when korean kings sent gifts of sutras and images to the emperor. The court adopted it as a way of unifying the country and then trickled down to the people. The \textbf{Tendai}(tiantai) school was formed as well as the \textbf{Shingon} (tantric) and became the basis around japan. During fuedal era with political upheaval the Pure Land and \textbf{Zen} (Chan) schools took over.

\paragraph{Tibetan Buddhism}
\label{par:tibetan_buddhism}
Missionaries from India initially brought buddhism to Tibet. Its indigeonous religion was shaman based. During the 7th centry Vajrayana buddhism came from india and Chan buddhism came from china and both were present.  The king staged a debate between monks from both sects which lead the country to Vajrayana buddhism. The king supported missionaries coming in to teach and translate. A monk named \emph{Pamasambhava} used his Tantric powers to subdue all the local demons and turned them into protectors of the buddhist faith. With the support of the government all the sutras were translated into tibetan and many monestaries were created. They called their experts \textbf{lamas} and they would be living gods that taught people.

A number of schools were created in tibet, the three biggest being Nyingma-pa, Sakya-pa, and Kagyu-pa. The first of these mostly closely follows the teachings of Pamasambhava in its emphasis on rituals and magic. \textbf{Atisha} was a monk from india that formed the monastic order \emph{Kadam} which focused on scholarship and strict discipline.

\textbf{Drok-mi} founded the \emph{sakya}-pa monestary  as be combined kadam with tantric practices. This was dominant in the government for a while.

\textbf{Marpa} studied tantric practices in india and when he returned he would teach a select few  which rose into the \emph{kagyu-pa}. It featured very complex yoga systems and secret teachings.

\textbf{Tsongkhapa} founded the \emph{Geluk-pa} school which developed a comprehensive system of teaching all of buddism equally. This school also enjoyed political power, even converting a Khan who bestowed on him the title of \textbf{Dalai Lama} (ocean of wisdom teacher).

\paragraph{Dalia Lama}
\label{par:dalia_lama}
This created the idea that a lama could chose to be reincarnated into a child who would become the next lama. The dalai lama was believed to the reincarnation of the bodhisattva \textbf{Avalokiteshvara}. The fifth dalai lama became the leader of tibet which put the geluk-pa school in power for a long time. After the death of a dalai lama a search party would be dispatched to find the next one in an infant born 49 days later who would be taken and educated to be the next ruler.























\end{document}
