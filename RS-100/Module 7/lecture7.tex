\documentclass{article}
\usepackage{parskip}
\usepackage{csquotes}
\usepackage{color,soul}
\usepackage[margin=.6in]{geometry}
\begin{document}
\title{Buddhist Transformations}
\maketitle
\section*{Historical Transformations: Ashoka}
\label{sec:ashoka}
The reign of king ashoka is the second moist important event after the death of buddha. He was a world conquerer that united most of the subcontinent of india together. He was heavily influenced by budhism.

The last major battle of his was against the \emph{Kalingas}. This was a viscious battle with 150k deaths. This amount of violence disgusted him so he vowed to be a \emph{dhammaraha}, a righteous ruler.

His reign was marked by many innovations. Many people traveled the roads a that time so he built way stations along routes. He ended animal sacrifice in many festivals. He cut down the number of animals consumed in the palace. Those sentenced to death to have some days to settle their affairs before being execute. All over the country pillars were erected with edicts on them teaching people how they should live their lives.

He was a great supporter of buddhism and aided its root in india.

\section*{Historical Transformations: Early Schools}
\label{sec:early_schools}
After the buddha's death the sangha got dispersed widely. Variations in practice occurred.

There were about 18 early schools and new thinking emerged. A group was identified as separate from the 18 schools. Today only one of the 18 schools remains, the \textbf{Theravada}. Often scholars refer to the 18 schools collectively as the theravada, but this is not quite correct. Another group, separate from the early schools was the \textbf{Mahasamghika}, or large assembly. It developed new ideas.

\paragraph{Mahasamghika}
\label{par:mahasamghika}
these people introduced the idea that the buddha was trascendental. This is different from the theravada who believed that he was just a historical figure. The Mahasamghika sect also did not include lay people in their decision making (this is an error in the textbook).

\paragraph{Mahayana}
\label{par:mahayana}
There has been discussion about a new form of buddhism, or the second turning of the wheel. This stemmed from the mahasamghika. This is not fully proven, but the two share ideas.

\section*{Rise of Mahayana}
\label{sec:rise_of_mahayana}
This is first visible to us in the production of new scriptures. They claim to be the word of the buddha. This is odd since buddha has been dead for a while at this point. They have this notion of a cosmic buddha, that hes a principle rather than historical person. This means that revelation was always available through visions or dreams, or other such methods. They also have many buddhas existing in many buddha-fields.

They believed in a more inclusive practice. They had more faith in the ability of lay people to make spiritual people. Mahayana means large/great vehicle. It saw itself as greater than the traditional ways.

\paragraph{Aspiration}
\label{par:aspiration}
They saw themselves as greater in terms of aspiration. Their texts have new religious heroes. Previously the heroes were the arhant, now we have the boddhisattva. This a person that has taken a vow that they will attain enlightenment for others. This vow minimizes the contributions of the arhant since they are just people who have achieved enlightenment. The aspiration of enlightenment for selfless reasons is seen as a higher aspiration than achieving it for yourself.

\paragraph{Compassion}
\label{par:compassion}
They also saw themselves as greater than the previous tradition in compassion as well since they seek salvation for all beings rather than just the self. This means the arhant's compassion is limited and some texts go as far as to call them selfish.

\paragraph{People}
\label{par:people}
They also saw themselves as greater since they have more people (greater vehicle holds more people). They include a wider range of practice, so more people can practice.

\paragraph{Three Bodies}
\label{par:three_bodies}
Theravada held that there were two bodies of buddha, human where he taught, and dharma, the embodiment of truth. The mahayana have their own version of this. They too have the notion of the dharma body as the trascendant body of the truth. They differ in that they add a third body that is the bliss body that is the type of spiritual body this is the body that boddhisattva have. It allows them to manifest their bodies in many ways and can travel to the pure land to listen to the teachings of a buddha. For the theravada the form body was the form of the historical body, but for the mahayana the form body is the many forms buddha can take (since they believe there are many buddhas)

\begin{itemize}
	\item form
	\item dharma
	\item bliss
\end{itemize}


\section*{Buddhist Schools: The Madhyamika School}
\label{sec:the_madhyamika_school}
This is a school from the mahayana sect. It was founded by \textbf{Nagarjuna} that introduced the notion of \textbf{shunyata}, or emptiness. They use language as an example.

The terms nirvana and samsara mean nothing on their own, they need to be placed in opposition to each other. All language is based on binary opposites. Emptiness refers to how all things are relational, they are empty of essence.

There is no point at which analysis of something would end. When we looked at the concept of dependent co-arising there was this idea that there were base pairings that all things could be broken down into, emptiness does not have this. Everything is empty of own-being/substance.

Buddhist doctrine are meant to help us find the truth, not representations of the truth.

\paragraph{Prasanga Method}
\label{par:prasanga_method}
This was a method designed to demonstrate emptiness and provide the opponent with the feeling of emptiness. It was a negative dialectic. Every argument of the opponent is demolished but no opposing argument is put in place. The do not take the opposite stance and instead disprove both sides of the debate. Nothing is put in place, there is no conclusion.

\section*{Buddhist Schools: Yogacara}
\label{sec:yogacara}
Some people felt the notion of emptiness was frightening. In response the yogacara school was formed. It is often called the mind-only school because it gives dominance to the mind when it comes to perception.

We never see things they way they are in themselves, we only see our perception of them. Everything we see is filtered through our perception and thus distorted. The have three basic levels of consciousness,

\paragraph{Level of the Senses}
\label{par:level_of_the_senses}
In indian philosophy there are six senses since the mind is considered a sense. Everything comes in through the senses.

\paragraph{The Level of the I}
\label{par:the_level_of_the_i}
We don't just perceive we add ourselves to it.

\paragraph{Storehouse of the Conscousness}
\label{par:storehouse_of_the_conscousness}
This stores all our memories, some karmic some physical. This, unlike the above levels is passive. Below this is the level of emptiness, \emph{tathagatagarbha}. It is the level of pristine consciousness.

So yogacara agrees with the concept that everything is empty, but they have the storehouse consiousness which is empty of defilements, but does infact exist. Their argument is that consciousness exists

\section*{Buddhist Schools: Comparison of Theravada and Mahayana}
\label{sec:comparison_of_theravada_and_mahayana}

\begin{tabular}{c|c}
\textbf{Theravada} & \textbf{Mahayana}\\
\hline
pali cannon as scripture & accept pali but add new texts\\
\hline
see buddha as human & see buddha as cosmic principle, there are many\\
\hline
two bodies of buddha, dharma and form & three bodies, dharma, form, and bliss\\
\hline
arhant as example & boddhisattva as example\\
\hline
an individual must find salvation on their own & you can gain assistance from boddhisattvas through worship\\
\hline
south and south east asia & east asia\\
\end{tabular}

\section*{Buddhist Schools: Tantric Buddhism}
\label{sec:tantric_buddhism}
There is a third turning of the dharma wheel, tantric buddhism. Its very different from the other traditions and very complicated. Also called \textbf{Vajrayana} in tibet, meaning \emph{thunderbolt} or \emph{diamond}. It could be called diamond because it is industructable.

It talks about the indestructible union of pure wisdom and passion. It is also esoteric in that there are secrets you need to have you teacher pass on to you.

Also called \textbf{shingon} in japan, it is an effective means to liberation but is considered dangerous. Instead of suppressing desire and cutting it off at its roots, it tries to channel that energy and purify it.

This also created new texts, but kept them secret. It used many methods of purification.
\begin{itemize}
	\item rituals
	\item mantras
	\item mandalas
	\item visualization
	\item ritual sex
\end{itemize}

They feel that the sacred beings of mahayana text are within us. We all have the potential to become one. When we worship a deity we are worshiping that quality in ourselves.

\section*{Worlds of Meaning: Dependent Co-Arising}
\label{sec:dependent_co_arising}
\paragraph{Dharma}
\label{par:dharma}
Dharma refers to the truth, about reality, understanding of this gives liberation. This also refers to the buddha's teachings.

\paragraph{Dependent Co-arising}
\label{par:dependent_co_arising}
Dependent Co-arising means that everything has a cause and all causes are multiple. This is how they say that there is no self-existent thing. Everything is the product of something else.

\paragraph{Nature of the Self}
\label{par:nature_of_the_self}
The self is also the effect of dependent co-arising. It is the combination of the five aggregates, matter, feeling, mental formations, perception/cognition, and consciousness (at the center). We cannot describe ourself using only one of the aggregates. `Me' is just a handy way of referring to an ever changing collection of aspects.

\paragraph{Arising of Suffering}
\label{par:arising_of_suffering}
Ignorance causes psychic constructions, meaning thoughts and ideas and mental constructs. Out of ignorance we developed these ideas and act upon them. These actions cause us to accumulate karma and cause us to be reborn. Ignorance and ideas are the past causes for our coming into existence. Our physical being's interactions with the world are the present effects. This then leads to desire which leads us to further acts and more karma causing rebirth. The fact that we exist is the present causes for our rebirth. Its a big ass cycle.

\begin{itemize}
	\item ignorance and psychic constructions $\rightarrow$ past causes
	\item consciousness, names and forms, sense organse, contact, and sensations $\rightarrow$ present effects
	\item desire, attachment, and existence $\rightarrow$ present causes
	\item birth, old age, death $\rightarrow$ future effects
\end{itemize}

\paragraph{Example}
\label{par:example}
A dude wants an ice cream, but has no money. Desire has arisen, he now has thoughts and the will to do something. He then sees a woman with a purse so he decides to steal it. We now have karmic action with this. Because of these action he gets reborn. Thus the cycle begins again.

\section*{Worlds of Meaning: Nirvana, Emptiness, and Tathagatagarbha}
\label{sec:worlds_of_meaning_nirvana_emptiness_and_tathagatagarbha}
\paragraph{Nirvana}
\label{par:nirvana}
The ultimate goal is nirvana, either with or without support. Nirvana with support refers to support of a living body. When this person dies we don't know what happens, the buddha refused to answer that question.

\paragraph{Emptiness}
\label{par:emptiness}
This is very similar to dependent co-arising. Nirvana is considered absolute and unconditioned. Nirvana is the same as samsara in this emptiness concept. Both are empty because they make no sense without each other. This means that nirvana is not self existent, it has causes and these causes are multiple. All things are equal in their emptiness meaning there should be no distinctions between them.

\paragraph{Tathagatagarbha}
\label{par:tathagatagarbha}
Garbha means embryo or womb. Tathagata is another term for buddha. Gata means to go and tatha means suchness. So its the womb of the one who has gone to suchness. This idea influenced all mahayana. The concept of suchness is used alot by the yogacara. Its a positive description of emptiness. The buddha nature is empty of all defilement. This means that it truely exists, the yogacara equates it with the storehouse consciousness. This buddha nature is within each of us.

\section*{Worlds of Meaning: The Eight Fold Noble Path}
\label{sec:worlds_of_meaning_the_eight_fold_noble_path}
This is the fourth noble truth. It describes how to reach liberation.

\paragraph{Wisdom}
\label{par:wisdom}
The entails \emph{right understanding} of the four noble truths. This leads us to \emph{right intention} which is to follow the path to enlightenment.

\paragraph{Moral Conduct}
\label{par:moral_conduct}
\emph{Right speech} entails no useless or abusive speech. \emph{Right act} means acting morally. \emph{Right livelihood} means doing a job that doesn't harm others.

\paragraph{Meditation}
\label{par:meditation}
This is where wisdom and morality are integrated into us. This includes \emph{right effort} which is the effort it takes to meditate fully. \emph{Right mindfulness} is putting you mind to meditating on the right content. Finally \emph{right concentation} is focusing fully.

\paragraph{Path of the Bodhisattva}
\label{par:path_of_the_bodhisattva}
You become a boddhisattva by taking the vow to bodhicitta which is to become fully enlightened for the benefit of all living beings. All the rules of how the boddhisattva should behave is gathered in a 10 stage path. After the 8th stage you are irreversible (you are going to become a boddhisattva). The description of a boddhisattva are the same as the buddha, unlike the buddha that continue to be reborn to help everyone else.

\section*{Worlds of Meaning: Four Sublime States}
\label{sec:worlds_of_meaning_four_sublime_states}
These are meditative states where one begins with the self and extends the states to all others. We start meditating on a topic as it pertains to the self and then grow it to include more and more living beings until it applies to the whole of creation. This eventually lets you lose the distinction between the self and the rest of the world.
\begin{itemize}
	\item boundless love
	\item boundless compassion
	\item sympathetic joy
	\item equanimity
\end{itemize}

\section*{Ritual and the Good Life: The Five Precepts}
\label{sec:ritual_and_the_good_life_the_five_precepts}
The five precepts are the basic moral guidlines for all buddhists, lay people and monastics alike.

\begin{itemize}
	\item refrain from murder $\rightarrow$ compassion
	\item refrain from theft, also includes manipulation $\rightarrow$ generosity
	\item refrain from inappropriate sex, means harming others or self centered sex
	\item refrain from abusive speech, includes saying mean things as well as lying
	\item refrain from intoxicants, these make you more prone to break other precepts and can lessen your mental focus
\end{itemize}

As you refrain from an activity you will expand the opposite in your consciousness.


\section*{Ritual and the Good Life: Ritual Times and Festivals}
\label{sec:ritual_and_the_good_life_ritual_times_and_festivals}
\paragraph{Uposatha}
\label{par:uposatha}
This is often referred to as the buddhist holy day. Comes on half and full moon. Monks and nuns join the community and chants the monastic rules together. This renews their vows and unites the community in purity. This is also when the laity can come to the monastery.

\paragraph{Vesakha}
\label{par:vesakha}
Celebrates buddha's life and death.

\paragraph{Rains Retreat}
\label{par:rain_retreat}
Monks and nuns take a rains retreat for 3 months. They do not travel. This was traditionally the time of the monsoon. Travel was dangerous and the monastics did not want to harm the rain spirits. So they all stayed in and focused on their personal development. At the end of this there is a large ceremony where the laity bring the monks and nuns the things they will need for the next year.

\paragraph{Ordination}
\label{par:ordination}
When one decides to become a monk/nun. This is a major ceremony in which you receive your robe and bowl from a preceptor. They are then questioned about their will to join, then they are instructed, and accepted.


\section*{Ritual and the Good Life: Buddhist Nuns}
\label{sec:ritual_and_the_good_life_buddhist_nuns}
This order was founded 5 years after the order of munks. This is recorded in the vinaya. The nuns order was debated and the buddha was reluctant about it at first, but he did say that women are equally capable of attaining enlightenment. The initial reluctance of the buddha might show evidence of social context or it might have been added later during translation or re-dictation by monks.

The order of nuns died out in theravada countries. It lasted in sri lanka and burma until the 13th century, never reach thailand or tibet. The monk order in sri lanka died out once and almost another time. In those cases the monastic order was reestablished by monks from thailand and burma but this did not happen to the nuns. There is no recorded reason for this. Many in the theravada community believe that women must wait until the next buddha comes to get a new nun order. They can take the 10 basic precepts of novice initiation. There are not 10 fully ordained nuns so no new ones can be ordained.

The nun order has continued in china since the 4th century. In the last 20 years many nuns from theravada countries have taken full ordination in china. There is also a very active female sangha in thailand that also ordinates women from other countries. These ordinations were not accepted by their home countries. The argument is that chinese buddhism is mahayana buddhism, not theravada. The chinese vinaya comes from one of the 18 schools that are very similar to theravada. There is still great debate on the issue, but its moving towards a reestablishment of the female sangha.

The nuns get trained in social service work and work very hard on their own spirituality.

\paragraph{Sakyadhita}
\label{par:sakyadhita}
This organization has worked hard on the bukkhuni sangha. The name means \emph{daughters of the buddha}. It was founded in 1987 after a conference of buddhist nuns where they discussed their difficults and to work together. One of its major mandates was the reestablishment of the bukkhuni sangha where it no longer existed.












\end{document}

