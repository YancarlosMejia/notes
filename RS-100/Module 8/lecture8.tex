\documentclass{article}
\usepackage{parskip}
\usepackage{pdfpages}
\usepackage[margin=.6in]{geometry}
\begin{document}
\title{The Way of the Sikhs}
\maketitle
Sikh is the disciple of the guru, where guru can mean one of the 10 historical gurus, or the sacred scripture (adi granth/granth sahib), or god.

The sikh traditions formed during the devotional movement in india. Kabir is a devotional poet, many of his poems are in the sacred text.

\paragraph{Guru Nanak}
 \label{par:guru_nanak}
As a child exposed to islam and hinduism. He often asked holy men about their beliefs. When he was 30 he had a transformative experience. One morning he dissappeared for 3 days. He had been carried to gods presence where he had been given a cup of nectar and the mission to rejoice in gods name and teach others to do so. For a day after he emerged he said nothing, then he said that "there is neither hindu nor muslin, I shall follow god's path, he is neither". He then traveled, accompanied by \textbf{Mardana} that played an instrument along with his hymns. He settled in Kartapur. In the homes of disciples there would be a room for them to gather and sing, called the \emph{sangat}. The singing is called \emph{Kirtan}.

teachings:
\begin{itemize}
	\item one god
	\item experience god through love and devotion
	\item ignorance and self centeredness keeps us from loving god (we believe that the world is separate from god)
	\item we become dominated by passions
	\item samsara is because we are separated from god
\end{itemize}
 the soul can rejoin with god only the guru knows the way, it involves inner preparation of the heart to receive gods grace. To do this required kirtan and meditation on gods name. There should be no inequality since all were devine sparks from gods love and we must all be equal. He reject ritual as well since it did not bring us closer to god/

\paragraph{Guru}
\label{par:guru}
one who drives away darkness and teaches enlightenment, giving voice to god the true guru.

\section*{10 Gurus}
\label{sec:10_gurus}
Guru nanak transfered his mission to the second guru, called \textbf{Lenha}, meant to be his successor. He changed his name to guru \textbf{Angad}, which means part of me. This means that angad was part of him and inherited his charisma and mission.

Guru \textbf{Amar Das} started to collect the hyms and poems to write the sacred text. He also created festivals.

Guru \textbf{Ram Das} had a relationship with the current muslim ruler, Akbar. This ruler was very open and interested in religion and invited holy men to come talk to him. Akbar loved Ram Das and gave him some land at Amritsar where the Golden Temple (holiest place) was constructed. Ram Das also created the \emph{Langar} or free kitchen which would serve anyone from anywhere could receive a free meal.

At this point the line of gurus became hereditary.

Guru \textbf{Arjan} started the golden temple building and finished the sacred text. He was the first martyr when muslim rulers wanted to get rid of other religions.

Guru \textbf{Hargobind} created the two swords of authority.

Guru \textbf{Har Rai}

Guru \textbf{Har Krishnan}

Guru \textbf{Tegh Bahadur}, also a martyr

Guru \textbf{Gobind Singh}, the community was under considerable stress so he created a military defence force, called the Khalsa (a community of the pure defenders of the community if necessary). He did this by explaining to the community and asked for volunteer. 5 men volunteered and they started the khalsa, they are called the \emph{beloved ones}. The markings of a khalsa are called the 5 k's,
\begin{itemize}
	\item uncut hair (kes)
	\item comb (kanga)
	\item dagger (kirpan)
	\item wristguard (kara)
	\item short pants (kachera)
\end{itemize}
They also had a string code of conduct. The khalsa are open to women. Men take the name Singh (lion) and women take the name Kaur (princess). Not all sikhs are members of the khalsa, they must be initiated.

Guru Gobin Singh says that after himself the sacred text will become the guru thus making himself the last guru. He also creates the \textbf{Dasam Granth} a collection of works.

After the gurus death the military problems did not stop so \textbf{Ranjit Singh} took over and created a small kingdom that was eventually taken over by the british colonization.

\section*{Communities and Modern Challenges}
\label{sec:communities_and_modern_challenges}
Three main groups:
\begin{itemize}
	\item Singh - largest group, not all are members of the khalsa, not all agree with the idea of creating a sikh state.
	\begin{itemize}
		\item contemplative community - devoted to scholarship
		\item military - more aggressive in the goal of creating a sikh state
	\end{itemize}
	\item Udasi - ascetic
	\item Saha-dhari - split from the main group because they rejected the idea of a military community
\end{itemize}

There is conflict within the community about how to create the sikh state and how to acheive it. They want a piece of the punjab region which was gained when india gained its independence. Around 2.5 million sikhs were relocated into india when the punjab region was divided.

They also have problems with keeping young people tied to the tradition in the modern world. They are also pressured to join the hindu religion and they have many practicioners spreading throughout the world.

\section*{Sikh Worlds of Meaning}
\label{sec:sikh_worlds_of_meaning}
\paragraph{God}
\label{par:god}
is both transcendent and immanent. God is everything and there is nothing outside him, but god can be known by us in a personal way. He is revealed in his name \emph{Nam} and Sati Nam (God is truth) is an object of meditation.

\paragraph{The Universe}
\label{par:the_universe}
is real but not eternal and has no independent existence outside of God. Overall the world is good and beneficial, so the tradition is world affirming.

\paragraph{Soul}
\label{par:soul}
is spark from the devine fire, recepticals for gods divine love. The migration of souls by karma is controlled by god. Nirvana is union with god.

\paragraph{The Human Predicament}
\label{par:the_human_predicament}
Out soul is good and could join with god, but our ego centric nature stops that. This causes us to turn to the desires of the world instead of God.

\paragraph{The Path}
\label{par:the_path}
Hear the word of the guru which reverberates with god's name. Repeating the divine name, meditating on it, and kirtan.


\section*{Ritual and the Good Life}
\label{sec:ritual_and_the_good_life}
Daily worship at home or at the Gurdwara (meeting place). After singing praises and hyms at the gurdwara you receive karah prasad (which is food made of gi and sugar made into a ball).

\paragraph{Baisakhi}
\label{par:baisakhi}
celebration of the new year and formation of the khalsa

\paragraph{Divali}
\label{par:divali}
also celebrated it by sikhs (usually associated with hindus), this is celebrated as the release of guru hargobind from prison.

Celebrations for the birthdays of gurus, and sacred times for the martyrdom of gurus. Initiation into the khalsa is a major life event.

The langar is a sacred duty













\end{document}
