\documentclass{article}
\usepackage{parskip}
\usepackage{csquotes}
\usepackage{color,soul}
\usepackage[margin=.6in]{geometry}
\begin{document}
\title{Chinese Religions 1}
\maketitle
\section*{Background: Common Elements}
\label{sec:background_common_elements}
\paragraph{Common Elements}
\label{par:common_elements}
Chinese religions are rooted in concrete experiences, practical in orientation, and syncretic (people practice many religions based on their need or inclination, its not exclusive).

\paragraph{Common Values}
\label{par:common_values}
Prosperity, longevity, and posterity (leaving something behind to be remembered by)

Most religions are rooted in \emph{agrarian}, farming images. There is a center on the mean (middle or harmony) and family (primary social unit).

\paragraph{Four Main Sources}
\label{par:four_main_sources}
The \textbf{ancient religions} draw from animism, ancestor worship, and fold religion. Animism is based on belief that there is one living reality, there is spirit in everything. Here is where we get divination. \textbf{Confucianism} is another source of religion founded by \emph{Kong-zi} and focuses on the five classics. \textbf{Daoism} was founded by \emph{Lao-zi} in the \emph{Dao de Jing}. The final source is \textbf{Buddhism}, specifically the pureland school or ch'an school.

\section*{Background: Early History}
\label{sec:background_early_history}
\paragraph{Period of Cultural Heros}
\label{par:period_of_cultural_heros}
(3000-2200BCE) This period emphasizes the respect for people who made practical contributions to society. Here we have praise for those that domesticated animals. This is also the era where the institution of family life was formed (this included extended family). Agriculture was also instituted during this time. This era had a \textbf{Five Emporers Period}. \emph{Huang Di}, the yellow emperor, the symbolic ancestor of all people. He instituted all arts (writing, mulic, etc).

\paragraph{Xia Dynasty}
\label{par:xia_dynasty}
(2200-1750BCE) Here we have the development of silk. The wheel and bronze work are also invented.

\paragraph{Shang Period}
\label{par:shang_period}
(1751-1111BCE) This is the start of the fuedal era. The feudal system is very important to the Chinese culture. Confucian thought says that this is the perfect system. There is a strict hierarchy all the way down to the family unity. This was also the time of a theocracy where the ruler was the political and religious ruler. \emph{Shang Di} was the high god of the time, ``The Lord Above''. There were other gods, but he was most important, ancestor of ruling clan. Only the king worshiped Shang Di. This period had the institution of ancestor worship and divination.

\paragraph{Zhou Dynasty}
\label{par:zhou_dynasty}
(1123-221BCE) This was the golden era of Chinese history, it was the model of an ideal Chinese society. The notion of heaven(called \emph{Tian}), a righteous rule over all replaces Shang Di as ``the sacred''. It is much less personalized. The Duke of Zhou establishes the king as the son of heaven. He is ruling on earth on behalf of heaven. They developed the \emph{Mandate of Heaven}, heaven gives the right to rule to a righteous ruler and if he strays heaven will take it from him. This is the reason given for the change in dynasty, the Shang had lost the mandate of heaven so the Zhou took over. \textbf{Li}, the concept of proper reverential ceremonies (an expression of cosmic order), grew in importance.

\section*{Background: Ancient Religions}
\label{sec:background_ancient_religions}
\paragraph{Yin and Yang}
\label{par:yin_and_yang}
These are complimentary opposites and play heavily into the Chinese notion of balance and harmony. Both are necessary and must be balanced. A fundamental tenant of Chinese medicine is to try to balance yin and yang.

\paragraph{Five Elements}
\label{par:five_elements}
Wood, fire, water, metal, and earth. These elements are causal, they make things come into existence and determine their nature.

\paragraph{Yijing}
\label{par:yijing}
This is a method of divination. It started with the casting of stones and developed into a complicated method of casting sticks. You ask a question and do a complicated patter of casting sticks.

\paragraph{Confucianism: Beginnings}
\label{par:confucianism_beginnings}
He was probably a poor aristocrat, no metion of his parents or wife. We think that he had a son and daughter. Little is known about him. He wanted to restore the culture and religion of china to the Zhou ideal.

\paragraph{Human Dilemma}
\label{par:human_dilemma}
He believed that the cause of the chaos of his era was due to a breakdown in morality and values. He believed in a trickle down effect where the break down in morality of the rulers lead to a break down in morality of the common folk. He felt that things were so bad that the leaders oppressed the people and caused them to respond with violence. We could solve the social anarchy by returning to the ideals.

\paragraph{Teachings}
\label{par:teachings}
He focused on restoring the old ideal and not establishing anything new.

\paragraph{Mandate of Heaven}
\label{par:mandate_of_heaven}
He expanded on this and established this Zhou idea. He removed all notion of anthropomorphism of the concept of heaven. Heaven was a righteous principle that required rulers to be righteous to keep their mandate to rule on behalf of heave. Part of being righteous is that rulers should be concerned about the people that they rule. This idea was surprisingly revolutionary at the time. Righteous rule led to peace and harmony. Righteous rulers created righteous subjects.

\section*{Confucianism: The Social Order}
\label{sec:confucianism_the_social_order}
\paragraph{The Rectification of Names}
\label{par:the_rectification_of_names}
Every title/role comes with a set of rights and responsibilities. People should fulfill only their own duties.

\paragraph{Ideal Social Order}
\label{par:ideal_social_order}
Very hierarchical and patriarchal.

\paragraph{The Five Relationships}
\label{par:the_five_relationships}
These are all understood to be reciprocal.
\begin{itemize}
 	\item emperor - subjects $\rightarrow$  being benevolent is rewarded with loyalty
 	\item father - son $\rightarrow$ being loving is rewarded with reverence
 	\item husband - wife $\rightarrow$ being good is rewarded with obedience
 	\item elder brother - younger brother $\rightarrow$  being gentle is rewarded with respect
 	\item elder citizen - younger citizen $\rightarrow$ being considerate is rewarded with deference
 \end{itemize}

\section*{Confucianism: The Five Classics}
\label{sec:confucianism_the_five_classics}
We can learn how to return to the ideals of the Zhou by looking at five classic texts.
\begin{itemize}
	\item history
	\item changes
	\item poetry
	\item rites
	\item spring and autumn annals
\end{itemize}
His disciples compiled the \emph{Analects}, his teachings, after his death. It is often considered the sixth classic.

Through studying these we can learn the virtues of an ideal society and become people of \emph{ren}.

\paragraph{Ren}
\label{par:ren}
This is the term humane goodness. It indicates conscientiousness and reciprocity. Its characters are those values. Conscientiousness is doing one's duty and reciprocity is behaving properly based on the relationships.

\paragraph{Li}
\label{par:li}
This is the notion of propriety and respectful ritual. You must perform the ancient rights properly. Confucius added inner li. The external ritual reforms us internally into a respectful person. Confucius did not believe that ritual worked, but he felt that li was important in building a person that is respectful internally and solidifying our relationships. He also added the notion of filial piety, respect to your elders.

\section*{Confucianism: Later Confucians}
\label{sec:confucianism_later_confucians}
The primary interpreter of Confucius was \textbf{Menzi} (372-289BCE). He believed that humans are essentially good.

\paragraph{Four Essential Beginnings}
\label{par:four_essential_beginnings}
Menzi believed we have four values that we are born with. We are all born with a sense of \textbf{humanity}, mercy. He describes a child that has fallen into a well, everyone goes to help the child before asking anything about the child. We are all born with \textbf{righteousness}, we all have sense of shame. We know when we have done something wrong. We are all born with \textbf{propriety}, we know what respect is. Finally we are born with a \textbf{wisdom}, we have an innate sense of right and wrong.
\begin{itemize}
	\item humanity
	\item righteousness
	\item propriety
	\item wisdom
\end{itemize}

\paragraph{Social and Political Institutions}
\label{par:social_and_political_institutions}
Menzi also felt that institutions exist for the benefit of people. If those institutions nurtured the goodness that we are born with, then we would have an ideal society. Evil arises due to these institutions not nourishing it. The scholar fulfills a priestly role, they must teach the classics and by doing so enables people to nourish their own goodness.

\paragraph{Xunzi}
\label{par:xunzi}
(300-238BCE) Differed greatly in his ideas. Human nature is essentially selfish. If left alone we would do only what is best for us. He believes that goodness can be acquired through education and law. The classics put restraint on our natural selfishness. Through study we learn goodness. Rewards and punishments from the law can help people become good.

\paragraph{Legalists}
\label{par:legalists}
This group adopted Xunzi thought and eventually overtook the Zhou Dynasty that had been declining. The brought in the \textbf{Qin Dynasty} (221BCE). This dynasty united china through an unprecedented level of violence. This violence is speculated to be what caused the dynasty to fall quickly. They used Xunzi ideas to implement their violent rule. All non-legalist thought was persecuted causing confucian thought to go underground.

\paragraph{Han Dynasty}
\label{par:han_dynasty}
Replaced the Qin dynasty when it fell. They were still violent but less so. The looked to find other ways to keep the country united. Many scholars were brought in to figure it out. \emph{Emperor Wu} was impressed by an confucian scholar, \emph{Dong Zhongshu}, who argued that confucian thought should be used to govern the state. This led to a revitalization of confucianism. The emperor adopted a new role of channeling cosmic forces to provide harmony for the state. This turned china back into a theocracy again. Confucianism became the new religion of the state. Confucius become a god during this time, temples are build and rituals for him.

\section*{Confucianism: Neo-Confucians}
\label{sec:confucianism_neo_confucians}
This arose during the \textbf{Song Era}(960-1280CE), became dominant in the \textbf{Ming Dynasty}(1368-1644CE), and controlled thought fom the Qing dynasty to the revolition.

\paragraph{Literati}
\label{par:literati}
The government held examinations to become a civil servant giving individuals roles in the government. This established a bureaucracy that ran the empire. The \textbf{literati} where those that had passed the exam. These became the dominant group during the song era. Some delved into politics, but others attempted to recover the ideal of a righteous person, ren. This is what Confucius wanted for us.

\paragraph{Sages}
\label{par:sages}
These are the literati seeking ren. Often they presented themselves as the sages of old. They worked to identify with the nature of thing and to seek the heavenly principle. A sage brought harmony to everyone around them. A metaphysical aspect starts to encroach on confucianism. The heavenly principle is universal and found in all things which is why sages sought to identify with all things.

\paragraph{Chou Tun-i}
\label{par:chou_tun_i}
The founder of neo-confucian philosophy. He and a group of thinkers worked out cosmology and metaphysics based on confucian thought. This was called the \textbf{great ultimate}. This is what generated yin and yan, the elements, and all creation. All of creation is united under this principle. Sages wanted to identify with this principle.

\paragraph{Li}
\label{par:li2}
This started as the proper performance of ritual, confucius added a moral dimension, and the neo-confucians turned it into a principle. Li is one, its manifestations are many. We discover the one by investigating internal and external things.

\paragraph{Zhu Xi}
\label{par:zhu_xi}
He organized all these thoughts into a system. At its head was the great ultimate. Every phenomenon has its own defining principle.

\section*{Daoism: The Dao}
\label{sec:daoism_the_dao}
\paragraph{Laozi}
\label{par:laozi}
(640-?BCE) There are many legends about him, but scholars thing he is a compilation of scholars at the time that rebelled against confucianism as it was too artificial and structured. They felt that true harmony came without structure through getting in touch with the natural rhythms of the universe. Loazi means old master, it comes from the legend that his mother was pregnant for 60 years. At 160yo he decided to go west as he was dishusted with the corruption of his land. The boarder guard refused to let him pass before he recorded his wisdom. So he wrote the \emph{Daode Jing}.

\paragraph{Daodi Jing}
\label{par:daodi_jing}
One of the most translated books in the world. It describes the mysterious flow that goes through the universe,

All chinese religions use the term dao, but they all interpret it differently:
\begin{itemize}
	\item confucian $\rightarrow$ the way of tian
	\item buddhiam $\rightarrow$ karma
	\item daoism $\rightarrow$ the sacred principle of nature, the flow of life
\end{itemize}

The dao is the source of all existence, it contains all opposites. It is unchanging.

\paragraph{De}
\label{par:de}
This is any particular form of the dao, it is constantly changing.

\section*{Daoism: Wuwei}
\label{sec:daoism_wuwei}
The highest good is to be in harmony with the dao. We do this through \textbf{Wuwei}, it means literally non-action. It actually refers to no non-natural action. Confucius believed that human fulfillment is through rules, daoism seeks fulfillment through non-action. There is emphasis of weakness and passivity. It is explained through trees in a storm, they bend in the wind, straightening when the wind is done. Willows bend, oaks break. Be like the willow.

\paragraph{Social Order}
\label{par:social_order}
Action provokes reaction, this is what causes problems in society.

\paragraph{Natural Action}
\label{par:natural_action}
This is action that comes from within. It is likened to the female since they represent passivity. It is also compared to uncarved blocks since they have limitless potential. Society is regrettable necessity. You may resist if someone is trying to interfere with your wuwei. Only defend yourself.

\paragraph{Virues of the Wise}
\label{par:virues_of_the_wise}
You can live in harmony with other citizens through three principles.
\begin{itemize}
	\item mercy, being merciful means you will receive mercy
	\item frugality, you will never be the object of envy
	\item never be first, you will never be the object of jealousy
\end{itemize}

\section*{Daoism: Political Thought}
\label{sec:daoism_political_thought}
Government is inversely proportional to order. Every time you pass a law you create more law breakers. If you keep people ignorant they will not develop expectations, and thus be easier to fulfill. Give people the opportunity to find balance, do not try to lead them.

\section*{Daoism: Later Daoism}
\label{sec:daoism_later_daoism}
\paragraph{Zhuangzi}
\label{par:zhuangzi}
(369-286BCE) Is the primary interpreter of Laozi's thought. The \emph{Zhuangshi} is his writings. It contains all sorts of odd stories about non-conventional saints. We distinguish between things but all things are the product of the dao which makes them nature. The sage must be in tune with the dao. Being in harmony with the dao you rise above the conventional. The sage does not distinguish between things (beauty and ugly, joy and sorrow, etc) and instead accepts all as a natural process.

\section*{Daoism: Religious Daoism}
\label{sec:daoism_religious_daoism}
Religious daoism built upon the philosophical daoism. It incorporated techniques and rituals from them. The primary goal is long life and immortality. Sages rise above the conventional, they interpret this as immortality. They often did yoga and consuming weird things. The 5 elements were very valued. They also believed there was a plant on the \emph{Isle of the Blessed} that could give immortality.

\paragraph{Qi}
\label{par:qi}
This is the primordial breath. It is present in gods, astral bodies, and nature. Religious daoism focused heavily on qi. We want to renew our qi, this is the source of immortality. You can call down the gods through ritual and they actualize cosmic forces in your body to renew your qi. Ingesting things is called \emph{alchemy}. They often consumed mercury and arsenic to transform the human body into an immortal. Elaborate breaking and gymnastics, as well as diet and ritual sex were also used to reach immortality.

\paragraph{Isle of the Blessed}
\label{par:isle_of_the_blessed}
There were frequent expeditions to reach the isle of the blessed.

\paragraph{Gods and Goddesses}
\label{par:gods_and_goddesses}
The \emph{Queen Mother of the West} is a very important goddess. This resulted in the movement of carrying amulets of the queen mother to protect yourself from danger.

\section*{Daoism: Sects}
\label{sec:daoism_sects}
\paragraph{Way of Heavenly Peace}
\label{par:way_of_heavenly_peace}
Their leaders were priests and military officers. They worshiped laozi as a god, creator of the world. Their sacred book was the \emph{Scripture of Great Peace}, it focused on how to restore the government, bring peace, and restore the health of people. Also known as the ``yellow turbans'', they instituted a rebellion to being an end to the Han dynasty.

\paragraph{Way of the Celestial Masters}
\label{par:way_of_the_celestial_masters}
Developed later in the han period. It is still influential today.

\section*{Orthodox Unity Sect}
\label{sec:orthodox_unity_sect}
Also is active today, centered in Taiwan.

\paragraph{Maoshan}
\label{par:maoshan}
Most important during the tuang dynasty. They built many temples and trained many new priests. They practiced many rituals to bring down cosmic powers and meditated. Their leader was very prevalent.

\paragraph{Total Perfection Sec}
\label{par:total_perfection_sec}
Rose up during the Sung dynasty, founded by \emph{Wang Ze}. Studied many other religions like buddhism and daoism. He had some interactions with daoist immortals that lead him to live in a grave for a while. After that he lived in a thatched hut before he burned it down. This sect was a group of eccentric ascetics, but they attracted many followers. This was the first sect with a monastic system, they wanted a simple life. They practiced inner alchemy and self perfection. They incorporated Confucian and Buddhist doctrine as well.















\end{document}
