\documentclass{article}
\usepackage{parskip}
\usepackage{csquotes}
\usepackage{color,soul}
\usepackage[margin=.6in]{geometry}
\begin{document}
\title{Buddhist Foundations}
\maketitle


\section*{Indian Background}
\label{sec:indian_background}
Buddhism arose in india during a time of great change:
\begin{itemize}
	\item technological: invention of the iron plow revolutionized agriculture, allowing someone to produce more with less people, invention of iron weapons increased damage in war
	\item social: kinship system broke down (oligarchy ended) a single tribe of many families exercised control over land and was ruled by the head of these clans distributing profits as needed, the iron plow broke this system as individual clans could have the land and many people were not needed for the work displacing them so they moved into cities, this lead to urbanization
\end{itemize}

\section*{Life of Buddha: Birth and Early Life}
\label{sec:life_of_buddha_birth_and_early_life}
There are few historical details, most of what we know is religious in text. There is no life story in early texts. In recent discussion there has been debate about the dates of buddha (it was 566-486, but now we think its more like 490-410). There is about a hundred years of difference so if we accept these dates we need to question the history of that era.

The first biography is from 200 bc, called the \emph{buddhacarita} (deeds of the buddha). It was written by a famous poet called \emph{asvaghosa}.

The buddha was born in \emph{Lumbini} in southern napal. His mother was on her way to her home villiage to have her baby. She became tired and stopped in lumbini for a rest. She bathed and leaned against a tree under which she had her baby. This tree is a modern relic. The king ashoka established this as the birth place of buddha. He erected a pillar to mark the place.

At his conception his mother had a dream about a white elephant entering her side. Her husband called the sooth sayers who said that the she would give birth to an exceptional son. He had only two possible careers, one was to become world conquerer, or a great religious leader. His father (a king) was concerned that his heir would take the religious path. The sages told him that if we wanted buddha to become a world conquerer he should keep the baby from seeing any forms of suffering. So his father was ultra careful to keep him from experiencing any unpleasantness.

So buddha followed in his fathers footsteps. He married a woman named \emph{Yashodhara} was intelligent, beautiful, and spiritual. They had a son \emph{Rahula}.

\section*{Life of the Buddha: The Four Sights}
\label{sec:life_of_the_buddha_the_four_sights}
Siddhartha was driven from his world conquering destiny by The Four Sights. The king tried to protect him from seeing anything unpleasant so he would have the palace parks cleared before he was allowed to enter them, since they were public parks.

While he was out he saw an old man. He had never seen anyone not young before. Siddhartha wondered what it was and his charioteer replied that it was old age. Siddhartha was not a fan. The servant had to break the news to him that old ages happens to everyone.

His father did not like that so he moved Siddhartha to the summer palace and invited some dancers and singers over for a party. While Siddhartha was out and about he saw a diseased man (probably leprosy). Once again he wondered what it was and his servant had to explain that disease happens to everyone.

On the third occasion they saw a funeral procession where the family was weeping. This also shocked Siddhartha. His servant had to explain yet again that death comes for us all.

The fourth sight was that of a mendicant. Siddarta thought the man looked very calm and peaceful. His charioteer explained what a mendicant was. When Siddhartha returned to the palace he had decided to become a mendicant.

He saw his family once while they were sleeping and left the palace with the charioteer. He gave the charioteer everything he had and walked into the forest to become spiritual.

\section*{Life of the Buddha: Ascetic Life}
\label{sec:life_of_the_buddha_ascetic_life}
He followed the traditional way of asceticism and meditation. This first teacher was \textbf{Arada Kalama} who worked with him to develop the a sophisticated spiritual stage called the \emph{sphere of nothingness}. Kalama was so impressed that he wanted to share leadership with Siddhartha, but he knew he hadn't found the truth yet.

He then studied with \emph{Udraka Ramaputra} who helped him attain an even higher state, the \emph{sohere of neither perception no non-perception}. The teacher was so impressed that he offered to become Siddhartha's disciple and give him all of his disciples. Siddhartha still knew he didn't have the answer so he left.

He lived on his own in a life of great austerity. Sever food deprivation, he became so thin that when he touched his stomach he could feel the vertibrea in his back. He perfected the technique of breathing suspension. This breathing suspension was supposed to bring him alter state of consciousness. All that happened was that he got a headache that interefered with his concentration on the truth so he stopped.

Finally he was offered some food by a lay woman so he took some. His disciples thought he was giving up on his quest and left. He hadn't given up. He had instead found the Middle Path. The spiritual life is hard and difficult and needs strength. This means that it requires nourishment. Control of the senses is what is important, not just pain.

He wandered around thining about this until he finds the \textbf{Bodhi Tree} where he sat down. He vowed not to move until he had reached his goal. There he sat and meditated, he recalled a spontaneous trance he had entered once as a child when he sat and watched a farmer working.

At this point great concern was raised in the temple of \textbf{Mara}, the god of death, obstacles, and temptation. He did not want Siddhartha to reach enlightenment so he tried to get him to stop. First he sent a hoard of demons to frighten him. Then he sent beautiful women to tempt him. Finally he confronted Siddhartha. Nothing he did could stop the man and he he reached enlightenment by the end of that night.

\paragraph{Nirvana}
\label{par:nirvana}
the state beyond birth and death, the ultimate goal for all buddhists.

\section*{The Life of the Buddha: Enlightenment, Three Knowledges}
\label{sec:the_life_of_the_buddha_enlightenment_three_knowledges}
After attaining enlightenment Buddhad gained three knowledges:
\begin{itemize}
	\item knowledge of his past lives, he could look back on his past lives to see the karmic connections
	\item ability to see the karmic chain of others, to understand why people had been born as they had, also why an enlightened teacher can help you to progress
	\item knowledge that for him rebirth had ended, the knowledge of the four noble truths, the knowledge of reality
	\begin{itemize}
		\item reality is impermanent
		\item reality is without essence
		\item reality is ultimately unsatisfactory
	\end{itemize}
\end{itemize}

\section*{The Four Noble Truths}
\label{sec:the_four_nobel_truths}
After englightenment the buddha gained the 3 knowledges:
\begin{itemize}
	\item his karmic history
	\item other's karmic history
	\item the four noble Truths
\end{itemize}

Buddha wanted to share enlightenment with someone. When he looked with his 3rd eye he saw that both of his teachers had died. His 5 former disciples were at the \textbf{Deer Park at Sarnath}. His disciples wanted to give him a rough time, but as he approached they got him a seat. He seemed different to them but they still gave him a hard time. Buddha preached enlightenment. They were skeptical, but he pointed out that he was a changed man. They recognized the authority that comes from pure knowledge.

During his preachings that day one of this disciples gained insight. After a few days all 5 of his disciples gained enlightenment. This formed the \textbf{sangha}, the community.

The main alter at the temple in deer park has a gate that you can enter and \textbf{circumabulate} the alter. This means you walk around keeping the religious artifact/person to your \textbf{right}. The walls of the main temple has many drawings of the events in Siddhartha's life.

The five noble truths bringing enlightenment to the disciples of buddha is known as the \emph{first turning of the Dharma Wheel}. The first teaching.

\section*{The First Noble Truth}
\label{sec:the_first_noble_truth}
There is suffering.

There are three types of suffering:
\begin{itemize}
	\item ordinary suffering - generic physical discomfort
	\item impermanence - losing something, all things leave
	\item component parts - everything is made of parts so these parts will definitely come apart
	\begin{itemize}
		\item the soul is made of parts, called \textbf{the five aggregates}
		\item this entails inherit suffering
	\end{itemize}
\end{itemize}

You cannot avoid suffering.

Everyone suffers.

Things are ultimately unsatisfactory, because all things in the end lead to suffering.

\paragraph{The Mustard Seed Story}
\label{par:the_mustard_seed_story}
There was a widow without family, all she had was her son. Her son died. She could not accept it and wouldn't leave the body of her dead son. She wandered around the country side asking others for help. Eventually she found a village where a buddha lived that might be able to help. The buddha told her to collect 1 mustard seed from every home in the village that has not experienced death. She went from village to village but she could not find a single home that had not experienced death. Finally she understood that death is natural and had her son cremated. She returned to the buddha and thanked him. Buddha made her into a nun.

\section*{The Second Noble Truth}
\label{sec:the_second_noble_truth}
Suffering is caused by ignorance and craving.

\paragraph{Craving}
\label{par:craving}
The best definition for craving is excessive desire, obsession.  Craving is described as a fire of lust, hatred, and delusion. These cause suffering.

\paragraph{Ignorance}
\label{par:ignorance}
Ignorance about the nature of reality and the nature of the self.
\begin{itemize}
	\item Annica: impermanent
	\item Anatta: without essence
	\item Dukha: ultimately suffering
\end{itemize}
The primary core of this ignorance is about the nature of the self. We believe there is an essence to the self, the soul. We think that is our true nature. This is notion leads to problems. It means that we view the world from our perspective instead of seeing things as they are. Our view of reality is distorted by our perspective. The presence of lust, hatred, and delusion is what causes this.

We strengthen our sense of self by acquiring material possessions because they make us feel important. This cuts us off from other people, we make divisions. It also changes how we see things, outside of our ownership of it. When we look at ``my'' thing it imposes characteristics on the object that might not be true.

This distortion of reality causes attachment and clinging to things. This then causes suffering when the impermanence of the world takes that from us.

According to Buddhism we are not a soul. This is a very harmful view. The nature of the self is ever-changing. We are made of 5 components:
\begin{itemize}
	\item matter
	\item feeling
	\item consciousness
	\item mental formations
	\item perception
\end{itemize}

This idea of the everchanging self is \textbf{Anatman} or no-soul.

\section*{The Third Noble Truth}
\label{sec:the_third_noble_truth}
There is an end to suffering, \textbf{Nirvana}

The buddha is often compared to a physician. He diagnosis the condition, suffering. He looks for the cause of the condition, ignorance and desire. He administers a treatment, the noble truths.

You can attain nirvana while still living. This is the ultimate goal of all buddhists. It is inexpressible. It is the state of perfect freedom and grasp of the nature of reality. Freedom brings compassion.

Nirvana literally means ``blown out'' this refers to extinguishing the fire of desire.

Two kinds of Nirvana:
\begin{itemize}
	\item nirvana with support - meaning support of the body
	\item nirvana without support
\end{itemize}

The effects of karma are always experienced, the body being one of those effects. If you experience nirvana while still in your body, when you die its called \textbf{parinirvana}.

There are 10 questions that the buddha never answered. When someone asked him them the refused to respond. One question is what happens to the arhat(one who has attained nirvana) after death. This puzzled \textbf{Ananda} the buddha's cousin/attendant for many years. The buddha explained it as, there is no way to explain the answer to that question accurately, whatever he would have said would have been misinterpreted. You cannot explain nirvana to one who had not experienced it.

\section*{The Fourth Noble Truth}
\label{sec:the_fourth_noble_truth}
There is a path to the end of suffering.

The path is the \textbf{Noble Eight Fold Path} it is summarized into three categories:
\begin{itemize}
	\item morality
	\item meditation
	\item wisdom
\end{itemize}
These are compared to a three legged stool, they support each other. If any of them are missing it won't work.

\section*{The Founding of the Sangha}
\label{sec:the_founding_of_the_sangha}
After the buddha and his disciples became enlightened we have the starting of the buddhist monastic community. Once they had 60 arhants the buddha sent them out into the world to spread word. There is a community of monks, called \textbf{Bikkhus} and nuns, called \textbf{Bhikkhunis}. The community of nuns was formed about 5 years after that of the monks. This is an intentional community. This means that the bonds between them are chosen and not just given at birth. Everyone here is united in their intention to attain nirvana and help others to do so.

When a person becomes a buddhist they take the three refuges:
\begin{itemize}
	\item buddha
	\item dharma
	\item sangha
\end{itemize}

\section*{Death of the Buddha and the First Council}
\label{sec:death_of_the_buddha_and_the_first_council}
After the buddha died his remains were divided into groups and given to communities that supported the buddhism. These relics were encase in a \emph{stupa}. These are based on ancient burial mounds for heroes. That have a summit that contains a relic of the person they represent. Here is where people can meditate on the excellences of the buddha.

\paragraph{The First Council}
\label{par:the_first_council}
This is found in the pali cannon of the Theravada sect of buddhism. People were concerned about maintaining orthodoxy (correct belief) and orthopraxy (correct practice) now that the buddha was dead. They wanted to preserve the buddha's teachings. A council was convened with 500 arhants. These people represent a pure vessel into which the teachings and practice of buddhism could be poured. This resulted in the 3 baskets.

\paragraph{Three Baskets}
\label{par:three_baskets}
This is the scriptural corpus of the theravada teachings. The frist is the dharma basket which held the teachings of the buddha. He gave a number of talks on a variety of topics and they are recorded here. It begins with ``thus I have heard''. Its usually in the form of question and answer. The second basket is the vinaya basket and it outlines the monastic discipline. It outlines the list of rules for monks and the rest of the community. This basket is incredibly important. As long as the vinaya is preserved buddhism will last. The third basket is the Abhidarma. These were not added at the first council, but at the third council. This is the philosophical texts that bring together the buddha's teachings.











\end{document}
