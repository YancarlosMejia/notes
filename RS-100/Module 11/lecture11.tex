\documentclass{article}
\usepackage{parskip}
\usepackage[margin=.6in]{geometry}
\begin{document}
\title{Japanese Religions}
\maketitle

\section*{Shinto}
\label{sec:shinto}
Japanese religion is \emph{syncretic} meaning they practice rituals from various religions. They do not maintain a separation between the human and spiritual worlds.

\paragraph{Shinto}
\label{par:shinto}
This is an umbrella term for a variety of practices. Usually it refers to religion before buddhism arrived in japan. \textbf{Kami} are spiritual beings that shinto practices focus on. These can be natural phenomenon (like mt fuji), or human (like after a momentus even occurrs in someone's life), or buddha/boddhisattva (buddhism became integrated with local religions). The most important kami is \emph{Amaterasu}, the sun kami. This is the circle on their flag.

\paragraph{Amaterasu}
\label{par:amaterasu}
Became important in the Jomon period. Her sacred objects were curved beads, a sword, and a mirror.

Each clan originally had a representative kami and the leader of the clan was that kami's priest. During the Kofun period one clan became dominant, they became the imperial line. Their kami was amaterasu, she became the kami of all of the people of japan. The emperor was her preist and he descended from amaterasu. He, \emph{Jimmu} was the great grandson of \emph{Ninigi} who was the grandson of Amaterasu.

\paragraph{Texts}
\label{par:texts}
We know about the kami from stories in the \emph{Kojiki} and \emph{Nihon Shoki}. The creation of the land, imperial family, and japanese people was from \emph{Izanagi} and \emph{Izanami}. They churned the waters to create the lands. They gave birth of all of the other kami. When Izanami gave birth to the kami of fire was burned and had to go to the underworld. Izanagi wanted to follow her but she chased him out. He had to cleanse himself and in doing so he cried a single tear which became Amaterasu.

\section*{Buddhism: History}
\label{sec:buddhism_history}
Buddhism was introduced in 552CE from korea. Images and texts were sent as a gift to the imperial court. The emperor and shinto priests were concerned. He gave the image to a general \emph{Soga no Iname} that was very supportive of buddhism. He was very pleased and practiced with his family. A pestilence swept over japan and buddhism was blamed. All evidence of buddhism was purged from the country. At this time a huge fire destroyed the great hall in the imperial palace. It returned again 50 years later. This time monks and nuns came and taught the texts.

\paragraph{Prince Shotoku}
\label{par:prince_shotoku}
(573-622)One of the main drivers for the adoption of buddhism. He was very well learned on buddhism, confucianism, and shinto tradition. He formulated the first constitution(604), the \emph{17 point constitution0}. The three religions were recomended turning them into state religions.

\paragraph{Nara Period}
\label{par:nara_period}
(710-784) This is the era of the first buddhist schools and the temple to the sun buddha \emph{Dainichi}. They were concerned that this temple would offend amaterasu, an emissary was sent to consult her and she said that she was one with the sun buddha. This provoked a convergence of shinto and buddhism, they were practiced interchangeably until the 19th century

\section*{Buddhism: Developments}
\label{sec:buddhism_developments}
\paragraph{Saicho}
\label{par:saicho}
(767-822) He introduced the Tendai system of buddhism. Buddhism had become very lax and involved in the imperial court. This school was very strict and required a 12 year training period before you could become a monk. This ensured that people joined the faith for the right reasons and not just for political gains.

\paragraph{Kukai}
\label{par:kukai}
(773-835) He introduced \textbf{Shingon} buddhism, a form of tantric buddhism. It had much ritual and pagentry which made it very popular. It influence the arts and culture greatly.

\paragraph{Kamakura Period}
\label{par:kamakura_period}
(1192-13333) along with the \textbf{Muromachi Period}(1333-1600) were times of great turmoil. There was a lot of uncertaintly and violence. The sages declared that a degenerate age ha begun. Rule was by military power from the Shogunate. The samurai flourished here. The were great participants of the arts that came from buddhist monistaries. The code of bushido was also established here. This disciplined them to avoid unecessary chaos. It included elements from the three major religions:
\begin{itemize}
 	\item confucian - loyalty to his lord unto death
 	\item buddhist - meditation (usually focused on death) allowed the samurai to become fearless, they no longer feared death
 	\item daoism - martial arts, practice a movement until it is natural
 \end{itemize}

\section*{Buddhism: Rinzai/Hakuin}
\label{sec:buddhism_rinzai_hakuin}
\paragraph{Eisai}
\label{par:eisai}
(1141-1215) He was a Tendai monk that traveled to china and brought back Rinzai(Sudden) Zen. He wanted to establish it as a second school but tendai monks resisted it. He was supported and protected by the samurai. He also introduced tea to japan and recommended it for its medicinal properties and as a break from meditation. It is from this that we associated samurai with rinzai zen.

\paragraph{Hakuin}
\label{par:hakuin}
(1686-1768) He reformed rinzai zen. This happened during the tokugawa period. In china the sudden school of zen (the Chan) had developed koans, riddles, to help the disciple think beyond language and reason. Hakuin developed a training system around koans. They became the focus of rinzai zen study. Today it takes 10 years to go through the list of koans.

You begin the koans with great hope. You start to apply rational thought to it. After bringing your insight to your master he rejects it, this cycles for a long time until the student develops the \emph{great doubt}. This is doubt that you can ever solve the koan. This puts you in a bind and causes your mind to seize up. Hakuin likens it to sheet of ice, brittle and hard, susceptible to cracks. Something comes and cracks your mind which provokes the \emph{great death}, death of the ego. This is immediately followed by enlightenment.

\section*{Buddhism: Soto Zen}
\label{sec:biddhism_soto_zen}
This was the gradual school.

\paragraph{Dogen}
\label{par:dogen}
(1200-1253) He was a tendai monk. He provided the intellectual foundation for all zen. He is recongnized as one of the great masters of buddhism in japan. He had a fundamental question: \emph{if we all have buddha nature, then in some senses we are already enlightened, why do we have to seek enlightenment?} He concluded that we dont just have to sit, \textbf{za zen} just sitting. This is a pointless meditation. This is a demonstration of the enlightenment we already have.

\section*{Buddhism: Nichiren}
\label{sec:buddhism_nichiren}
\paragraph{Nichiren}
\label{par:nichiren}
(1222-1282) He established his namesake school. He was very critical of other buddhist schools and felt nothing wrong with persecuting them. He also criticized the country and its leadership. He felt that japan had abandoned the true buddhist path and it needed to return or suffer great defeat. These thoughts got him exiled twice. This ended when he predicted a mongol invasion properly. Nichiren believed he was a bodhisattva that would establish the lotus sutra as the full and final revelation of the buddha. Japan needed to follow the lotus sutra to maintain itself as a great nation.

\paragraph{Soka Gakkai}
\label{par:soka_gakkai}
This was a lay organization developed after the second world war. It was associated with nichiren buddhism. It became an international organization and became very popular in america.

\section*{Buddhism: Pureland}
\label{sec:buddhism_pureland}
This had similar popularity in japan as it did in china.

\paragraph{Honen}
\label{par:honen}
(1133-1212) Established pureland as a separate school called \emph{Jodo Shu}. It emphasized recitation of the \emph{nembutsu}, homage to the buddha amitabha, called \emph{Amida} in japan. It focused on Amida's vow to reassure you of salvation in the pure land.

\paragraph{Shinran}
\label{par:shinran}
(1173-1262) Honen's prime disciple esablished the \emph{Jodo Shin Shu}, the True Pureland School. There are likenesses drawn to MLK. People are totally incapable of any self-power in achieving salvation. He had a deep sense of our inability and our need for grace. This lead him to leave the monastic life as useless, so he disrobed and married. This established the tradition of married monks in japanese buddhism. We have to rely compleley on Amida, all we have to do is accept the gift from Amida. When we chant the nembutsu we are not reassuring ourselves since we cannot do anything, we are just thanking Amida.

\section*{Buddhism: Meiji Restoration and Post War}
\label{sec:buddhism_meiji_restoration_and_post_war}
\paragraph{Meiji Restoration}
\label{par:meiji_restoration}
(1867-1945) This is the time in which imperial rule was restored in japan. The borders had also been closed for nearly 200 years. When its borders were opened everyone was fascinated with western goods and technology, many scholars feared for the loss of japanese traditions. At this time the tea ceremony was integrated into the school system so that the practice would not be lost. The fascination with wester stuff lead to a resurgence in shinto, called \textbf{Nationalist Shinto}.

\paragraph{Nationalist Shinto}
\label{par:nationalist_shinto}
There was a move to separate buddhism from shinto. This was done by making shinto the cultural heritage of all japan regarless of religious affiliation. Now shinto was not just one of their religions.

\paragraph{Manifest Destiny}
\label{par:manifest_destiny}
This was the idea that the japanese were a people born of the gods which gave them the destiny to rule all of asia. They wanted to form an asian block against western interference. This lead to imperialistic exercises including the occupation of korea. This ended with the atomic bombs.

\paragraph{Post War}
\label{par:p}
The idea of a nationalistic shinto was disestablished. A separation of church and state was placed in the constitution. The emperor gave up his divine status.

\section*{Worlds of Meaning: Shinto}
\label{sec:worlds_of_meaning_shinto}
\paragraph{Shinto}
\label{par:shinto}
The religious practices in japan from before buddhist movements. It means the way of the kami. Kami are the inner power of nature. The world is good, pure, and beautiful, it is moving towards a good end. Humans are children of the kami. We were originally pure but we are imperfect and limited. Pollution obstructs the positive flow of the kami and their blessings. Shinto focuses on the notion of pollution and purity rather than right and wrong. Negative things are discouraged as they block the flow. So shinto is a pat of purification.

\paragraph{The Four Affirmations}
\label{par:the_four_affirmations}
\textbf{Tradition} is extremely important, a ritual must be very precise. We must pass the method of ritual down through family lines, we must try to do it exactly the same way as our ancestors. Often people do not know why a ritual is the way it is but we keep it going. \textbf{Purification} is very important, and most rituals start with water purification. \textbf{Life in the World}, shinto is very affirming of living your life here. All of you interactions here are important. \textbf{Festival} is important because it is where the people and the kami meet. There are four elements to shinto festival.

\begin{itemize}
	\item tradition
	\item purity
	\item life in the world
	\item festival
	\begin{itemize}
		\item purification
		\item offerings to the kami
		\item prayers for help and honoring the kami
		\item participation
	\end{itemize}
\end{itemize}

\section*{Worlds of Meaning: Buddhism}
\label{sec:worlds_of_meaning_buddhism}
\paragraph{Shingon}
\label{par:shingon}
tantric, founded by \emph{Kukai}. Focused on sun buddha \emph{Dainichi}, all of the universe is his body. Meditation and ritual can allow us to connect to his power. The three mysteries of the buddha:
\begin{itemize}
	\item body
	\item speech
	\item mind
\end{itemize}

\paragraph{Pureland}
\label{par:pureland}
One of the largest, relies on grace and faith in Amida. Practice focuses on nembutsu, \emph{Namu Amida Butstsu} (honor to amida the buddha). Honen says recitation of this releases the power of amida to us. Shinran says we are already saved so we must rely totally on amida, the nembutstu is showing gratitude.

\paragraph{Zen}
 \label{par:zen}
path of self transformation, you must use your own power to attain enlightenment in any age. Dogen provided the foundation and the concept of zazen, just sitting. Hakuin provided rinzai zen, he brough the koan riddles to push you through the great doubt, great death, and into the great enlightenment.

\section*{Ritual and the Good Life}
\label{sec:ritual_and_the_good_life}
\paragraph{Home Worship}
\label{par:home_worship}
Shines to kami and/or buddha/boshisattva. Kami shrine = kamidon, buddha shrine = bustsudon

\paragraph{Torii}
\label{par:torii}
These are the cross bars that mark the line between the sacred and profane (think of the beautiful arches with roofs). This shows that you are entering a sacred space.

\paragraph{Meditation}
\label{par:meditation}
Very important in the buddhist tradition, along with chanting.

\paragraph{Rights of Passage}
\label{par:rights_of_passage}
The \emph{cherry blossom festival} is well celebrated. The \emph{obon festival} honors the ancestors, people go to graves and give offerings and thanks. The \emph{buddha's birthday} is another widely celebrated occasion.











\end{document}
