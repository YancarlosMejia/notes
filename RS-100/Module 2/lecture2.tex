\documentclass{article}
\usepackage{parskip}
\usepackage{pdfpages}
\usepackage[margin=.6in]{geometry}
\begin{document}
\title{Hinduism I: Early History and Basic Concepts}
\maketitle

\section*{Hinduism: The Term}
\label{sec:hinduism_the_term}
Hinduism is an umbrella term. Some are elite others are devotional. Many practices are regional. Hinduism is a very flexible religious tradtion, adapting and incorperating many traditions, and tries to provide a path for everyone.

A \textbf{hindu} is someone who:
\begin{itemize}
	\item accepts the vedas as authoritative
	\item accepts the caste system (not applicable outside india)
	\item self defined
\end{itemize}

\section*{The Indus Valley Civilization}
\label{sec:the_indus_valley_civilization}
Centered around harrapa and mohenjo-daro
\begin{itemize}
	\item Active 3000-1500BCE
	\item ubran centers
	\item streets in grid pattern
	\item indoor plumbing
	\item governed by theocracy
\end{itemize}

Symbols are found here that will later be incorporated
\begin{itemize}
	\item Water - there are large tanks throughout the city meaning water was used in religious practices
	\item Firtility - female, male, animal, and nature symbols
	\item Proto-Shiva - the god of all dualities, a very thin figure seated in meditation with a helmet of horns in between the horns are vines, it has an erect phallus
\end{itemize}

\section*{Indo-Europeans}
\label{sec:indo_europeans}
The term for the mass migration into india around 1500BCE. At this time the indus valley civilization was in decline and were probably overcome by the incomers through war. Some scholars think they conquered through assimilation instead of war.
\begin{itemize}
	\item about 12 tribes
	\item cattle herders (pastralists)
	\item three classes, warriors, priests, herders
	\item the \textbf{Arryans} are the tribe that settled into india
\end{itemize}

The religion of the indo-europeans was called the vedic religion because they are based on the Four Vedas or the \textbf{Veda} (this is the four Vedas and the three commentaries attached to each).

\section*{The Four Vedas}
\label{sec:the_four_vedas}
The Vedas were heard by the elders.

\textbf{Rig}
\begin{itemize}
	\item hymns of praise to gods and goddesses
	\item oldest and most important
\end{itemize}

\textbf{Sama}
\begin{itemize}
	\item liturgical arrangement, organizes the order in which the hymns should be sung
\end{itemize}

\textbf{Yajur}
\begin{itemize}
	\item outlines the formulas used during sacrifice
	\item outlines where the formulas are to be used
	\item describes how a sacrifice should be performed
\end{itemize}

\textbf{Atharva}
\begin{itemize}
	\item The most recent
	\item day to day needs of people
	\item spells and charms
	\item information on medicine
\end{itemize}

\section*{Vedic Religion}
\label{sec:vedic_religion}
Vedic religion focused on worship of the gods through sacrifice, petition, and praise, in order to reap benefits both now and in future life.

They believed that the breath (or \textbf{atma}) lived on after the body so they believed in heaven after death.

Most vedic gods were nature figures:
\begin{itemize}
	\item \textbf{Ushas} the dawn
	\item \textbf{Varuna} cosmic order (brough good or evil to humans)
	\item \textbf{Agni} fire
\end{itemize}

Sacrifice was very important. Sacrifice is what brings order out of chaos. Human order is based on the sacrificial order so human and cosmic order are eternal and define.

The creation of the world and human order is outlined in the \textbf{Hymn of Parusha} from the rig veda. There is more than one creation myth, but this is the biggest. Purusha was a great being where all living and non living things are a quarter of him  and the heaves are three quarters. The gods sacrificed him out of which the veda was formed along with all animals.
\begin{itemize}
	\item mouth = brahman
	\item arms = kshatriya
	\item thighs = vaishyas
	\item feet = shudra
	\item mind = moom
	\item eye = sun
	\item mouth = indra and agni
	\item breath = wind
	\item navel = atmosphere
	\item head = heaven
	\item feet = earth
	\item ears = quarters
\end{itemize}

Through sacrifice:
\begin{itemize}
	\item the world was made
	\item the world is sustained
\end{itemize}

Gods and humans work together to maintain order.

\section*{The Three Commentaries}
\label{sec:the_three_commentaries}
Each Veda has three commentaries on it.
\begin{itemize}
	\item Barhamana
	\item Aranyaka
	\item Upanishad
\end{itemize}

The commentaries comment on the meaning of the text. They keep the tradition living because the understanding of the text can change throughout history.

\paragraph{Brahmanas}
\label{par:brahmanas}
These were written in 1000-800 BCE. These were ritual manuals and layed out the powers of the priest, but they also contain speculations on the nature of the religion. It asks if there are many gods or only one and what is it that is important about the sacrifice.

\paragraph{Aranyakas}
\label{par:aranyakas}
These written in 1000-8000 BCE. They work as a bridge between the brahmanas and the upanishads. Commonly they are referred to as the forest books because they speculate on things on the part of hermits that went to live in the forest to think about these things.

\paragraph{Upanishads}
\label{par:upanishads}
Most important of the commentaries. Often called the Vedenta (end of the Vedas). They contain the culimation of the speculation about humans, the cosmos, the sacrifice and their relation to each other. Here we find many ideas that are not found in other texts. For instance sacrifice is internalized and cosmologized. This means that the sacrifice is transformed into inner assetic meditation. Mediation creates a fire inside thus the sacrifice is internal. What happens within a person has a consequence in the cosmos connecting the 2.

\section*{Samsara and Karma}
\label{sec:samsara_and_karma}
The Upanishasds represent the culimnation of brahmanical thought. Over time a notion developed that life and death are not a permanent thing. We eventually fall from heaven to be reborn. This is called \textbf{samsara}. In the west reinarnation is considered a positive thing (we can go back and do better this time). With samsara the repeated birth/death is a bad thing. Ideally we want to stay in heaven instead of being reborn. Think of having the same thing over and over, no matter how good that thing is you'd eventually hate it.

Karma propells the wheel of samsara. \textbf{Karma} means act or deed or performance. It has different meanings for the various religions.
\begin{itemize}
	\item Vedic - the act of sacrifice
	\item Hindu - following your role or duty, called \textbf{dharma}
	\item Buddhists, Janes and Hindu - being a moral person
\end{itemize}

Karma is treated as a law of cause and effect. The effects for karma is consistent to the deeds you did in your life. If you were a scrouge in your life then in the future you will be in need of charity.

Karma doesn't determine your life or rebirth, it conditions it. Think of it as setting the parameters for your existence. You can still do what you want in life, but Karma makes it easier or harder to do so. It limits your potential (think of a short person trying to play basketball). Karma controls where we begin (physical, social, and character conditions). Karma also controls your luck throughout life.

Basically we are all trapped on a wheel (samsara) powered by our past actions (karma). We are reborn into three spheres:
\begin{itemize}
	\item superhuman relm: where the gods live
	\item the relm of humans and animals
	\item the relm of ghosts and demons
\end{itemize}

The gods can never get off the wheel. They are simply taking advantage of their good karma at this moment to be reborn as gods. Animals also cannot get off the wheel, they are suffering the effects of bad karma. Only humans can get off the wheel. This is why human life is considered rare and precious.

A popular story is that a human rebirth is as rare as a turtle that lives on the bottom of the ocean that only surfaces once every thousands years surfaced at the exact moment a yoke dropped down putting it on the turtle randomly.

We should focus on doing actions that will improve our karma. A special kind of knowledge can free us from samsara.

\section*{Brahman, Atman, and Moksha}
\label{sec:brahman_atman_and_moksha}
Atman = Brahman.

\paragraph{Brahman}
\label{par:brahman}
originally referred to the words used in the sacrifice. Through the speculative period it came to mean the one reality. This referrs to the power that stands behind everything, and the reality that permated everythibg.

\paragraph{Atman}
\label{par:atman}
originally meant breath. The atman is what goes to heaven after death. Evtually it came to be understood as the soul or essence of a person, their true self.

Brahman is the cosmic soul and atman is the individual soul, thus these two are the same,

\textbf{Brahman is the one power}; the source of all.

\paragraph{Story}
\label{par:story}
A boy went away to school and when he comes home he believes he knows more than his father. His father asks him if he has learned an instruction containing many paradoxes (where the unlearned becomes learned, unheard hear, and so on). His father explains that through a lump of clay knowledge of all clay may be learned. The modification is only through speach (basically its only the name we give things, we must learn its true self). In the beginning this world was just being, one only with a second. Some say that in the beginning there was nonbeing but how could being come from nothing? The being thought to self, may I be many and may I procreate and thus made fire. Fire did the same thing and made water. Water did the same and made food. Then being wanted to make names and forms (let everything be tripartite). When you keep breaking things to their essence we find the essence that they come from and it is this that we are all made of. This is the one, the brahman.

\textbf{Monism} is the concept of one reality

\textbf{Tat Tvam Asi} - that you are

\textbf{Moksha} - knowing that atman and brahman are one a the core of your being brings you freedom from the wheel.

\textbf{Jivamukti} - a freed soul, one that has attained moksha, when this soul dies atman merges with brahman

\section*{Summary}
\label{sec:summary}
The speculative period of the brahmanas and upanishads changes ideas about sacrifice, the nature of reality, and the goal of life. Sacrifice is internalized and cosmologized. Samsara-Karma-Moksha, new ideas about reality are introduced. Brahman the one reality (the cosmic soul) is introduced. The cosmic soul is the human soul (Atman = Brahman). Understanding this unity brings moksha and breaks you out of samsara. Proper actions (gaining good karma) can bring us closer to this but will not get us fully to moksha.

\section*{Defintions}
\label{sec:defintions}
\textbf{Veda}

\textbf{samsara}

\textbf{karma}

\textbf{moksha}

\textbf{Brahman}

\textbf{Atman}

\textbf{Rig}

\textbf{Sama}

\textbf{Atharva}

\textbf{Ushas}

\textbf{varuna}

\textbf{Agni}

\textbf{Aryan}

\textbf{Purusha Sukta}

\textbf{Brahmana}

\textbf{Aranyaka}

\textbf{Upanishad}

\section*{TEXTBOOK CH3}
\label{sec:textbook_ch3}





\end{document}
