\documentclass{article}
\usepackage{parskip}
\usepackage{csquotes}
\usepackage[margin=.6in]{geometry}
\begin{document}
\title{Hindu Worlds of Meaning}
\maketitle

\section*{Ultimate Reality}
\label{sec:ultimate_reality}
\paragraph{Brahman}
\label{par:brahman}
is the sacred power that pervades and maintains all things. There are two levels of knowing brahman. \textbf{Nirguna} is knowing brahman without attributes and \textbf{Saguna} is knowing brahman with attributes. Nirguna Brahman is formless and impersonal so its usually described by stating what it is \emph{not}. Basically this is known intuitively with the equation A=B. For \textbf{Shankara} this is the highest expression of Brahman. Saguna Brahman is the creative power of brahman embodied in Vishnu, Shiva, Devi, and their avatars. \textbf{Ramanuja} was a proponent of this form of Brahman.

\paragraph{Devotional Groups}
\label{par:devotional_groups}
Vaishnavites worship Vishnu as the highest as he is concerned with cosmic stability. When cosmic stability is threatened he produces an avatara, two well known ones are Krishna (Bhagavad-Gita) and Rama (Ramayana). Shaivites worship Shiva as the highest as he is the lord of all dualities. A subgroup are the Virashivas (or Lingayats). The Shaktas worship Shakti and her manifestations (Devi, Durga, and Kali) as the power behind Shiva. She has the power of creation and by association destruction.

\paragraph{Gods and Goddesses}
 \label{par:gods_and_goddesses}
These are all aspects of Brahman. The major gods mentioned above are Brahman. Its estimated that hinduism has 330 million gods. This is because everyone responds to the devine differently and associate with it differently.

\section*{Dharma and Samsara}
\label{sec:dharma_and_samsara}

\paragraph{Dharma}
\label{par:dharma}
This is the eternal unchanging order and everything has a proper place and function. Expanded into each person has a given place and function to fulfill to maintain the eternal order (built into caste system). So it outlines the eternal order and the duties we must follow to maintain it.

\paragraph{Karma}
\label{par:karma}
Literally this is "act" or "deed". Its not a reward or punishment, but more of a law. Our actions determine the circumstances of our rebirth. Karma is our Dharma basically.

\paragraph{Moksha}
\label{par:moksha}
This is liberation from the wheel of samsara which pulls us away from the world (usually for meditation and yoga) which contrasts with Dharma, our duty to fulfill in the world. One requires us to be away from the world and the other requires us to be an active participant in it. This paradox is solved with the three ways (there are three ways to acheive moksha while still fulfilling dharma).

\paragraph{Samsara}
 \label{par:samsara}
This is the eternal wheel of birth and rebirth (creation, and destruction).

\section*{The Three Paths to Liberation}
\label{sec:the_three_paths_to_liberation}
These are described in the Bhagavad-Gita.

\paragraph{Path of Knowledge}
\label{par:path_of_knowledge}
Inana Marga includes mantras. Aum, A for the waking state, U for the dreaming state, M for the deep state, the silence in between is the Atman. It also includes Yoga (the 8 stages of Patanjali).

\paragraph{Path of Action}
\label{par:path_of_action}
Karma Marga describes action without desire, meaning we act without concern of the rewards for it. When we act without external desires it developes "cool" karma which is less likely to keep us in samsara. Actions done with desire produce "hot" karma which keeps us on samsara more tightly. The path of action is best described in Arjuna's hesitation to fight in the war. The Karma Marga on its own does \emph{not} provide moksha.

\paragraph{Path of Devotion}
\label{par:path_of_devotion}
Bhakti Marga is the worship and love of the great gods. Similar to the path of action this is an act of selflessness. We abandon our selves in our love of the gods. The path of knowledge is only for those with the time and leisure to withdraw from the world, so women and sudras cannot follow it (they are specifically excluded). Anyone is able to practice the path of devotion though. We should be able to feel the love of god in a multifaceted way (as a mother or child and so on). quote
\begin{displayquote}
``United with the great God of the universe, who is also the inner soul of one's heart, one rises above the bondage of karma and finds joy, peace, and ecstasy in the power of the Beloved One.''
\end{displayquote}

\section*{Hindu Ritual and the Good Life}
\label{sec:hindu_ritual_and_the_good_life}
The vedic rituals and sacrifice are still performed today. The \textbf{Agnihotra}, or fire ceremony is the most important one. The \textbf{Puja} is a ceremony celebrating the gods (either at home or a temple) where offerings are places and icons are worshiped. Going to the temple allows you to see and be seen by the god, called \textbf{Darshan}, and it confers blessings.

\paragraph{Festivals}
\label{par:festivals}
\textbf{Durga Putra} is a festival celebrating Durga's defeat of the demon. \textbf{Divalid} is the festival of lights (in the fall) celebrates Lakshmi (the goddess of good fortune and light). \textbf{Holi} is a time when fires are set and games are played.

\paragraph{Geography}
\label{par:geography}
India is a sacred land. The \textbf{Kumba Mela} is a ceremony held every 8 years or so where the two great rivers meet. \textbf{Mt. Kailasa} is considered the center of the universe. \textbf{Banaras/Varanasi/Kasi} is the most sacred city.

\paragraph{Samskaras}
\label{par:samskaras}
These are life cycle rituals. The most important is the sacred thread ceremony where a young man is presented with the sacred thread that marks him as "twice born". This is his religeous birth where he takes on his rights and responsibilities as an adult hindu. Marriage is another one of these rituals, as it death. \textbf{Samnyasin} is someone who renounces the social order (there is a ceremony for this) to become an individual. This is following the path of knowledge.

\paragraph{Art}
\label{par:art}
Its a thing.

\section*{Society and the Ethical Life}
\label{sec:society_and_the_ethical_life}
Hinduism has a very specific social order, the caste system. This combines the class system with birth.

\begin{itemize}
	\item Brahman = Priests, studies and teaches the sacred texts
	\item Kshatriya = Warriors, protector of the people and administration of government
	\item Vaishyas = Producers, provides economic needs of society
	\item Shudras = Servants, serve upper class
\end{itemize}

All castes have specific duties and responsibilities and are vital to maintaining the social order of society and the cosmos.

The servant class and women are not allowed to hear or study the veda. A caste has been developed below the shudras called the outcaste where people who dont fit in the other castes go, its officially known as the \textbf{Scheduled Classes}.

\paragraph{Caste}
\label{par:caste}
Jati refers to birth. Each of the four classes have a caste attached to them. You are born in to a caste. People marry within their own group and eat together. There are certain occupations that only that caste can do. The caste you are born into due to your karma. So you are born where you belong. You cannot progress into other castes in this life, you must be reborn into them. The caste system should provide a place for everyone (provides a strong sense of identity) and everyone is in their place (provides a strong sense of stability). By fulfilling your role and by doing so supports each other. Brahmans are at the top due to their ritual purity.

Some believe that castes have no place in modern society. Violent caste reform is very common in india.





\end{document}
